\documentclass[10pt,a4j]{jarticle}
\usepackage{graphicx,wrapfig}
\setlength{\topmargin}{-1.5cm}
%\setlength{\textwidth}{15.5cm}
\setlength{\textheight}{25.2cm}
\newlength{\minitwocolumn}
\setlength{\minitwocolumn}{0.5\textwidth}
\addtolength{\minitwocolumn}{-\columnsep}
%\addtolength{\baselineskip}{-0.1\baselineskip}
%
\def\Mmaru#1{{\ooalign{\hfil#1\/\hfil\crcr
\raise.167ex\hbox{\mathhexbox 20D}}}}
%
\begin{document}
\newcommand{\fat}[1]{\mbox{\boldmath $#1$}}
\newcommand{\D}{\partial}
\newcommand{\w}{\omega}
\newcommand{\ga}{\alpha}
\newcommand{\gb}{\beta}
\newcommand{\gx}{\xi}
\newcommand{\gz}{\zeta}
\newcommand{\vhat}[1]{\hat{\fat{#1}}}
\newcommand{\spc}{\vspace{0.7\baselineskip}}
\newcommand{\halfspc}{\vspace{0.3\baselineskip}}
\bibliographystyle{unsrt}
\pagestyle{empty}
\newcommand{\twofig}[2]
 {
   \begin{figure}[h]
     \begin{minipage}[t]{\minitwocolumn}
         \begin{center}   #1
         \end{center}
     \end{minipage}
         \hspace{\columnsep}
     \begin{minipage}[t]{\minitwocolumn}
         \begin{center} #2
         \end{center}
     \end{minipage}
   \end{figure}
 }
%%%%%%%%%%%%%%%%%%%%%%%%%%%%%%%%%
%\vspace*{\baselineskip}
\begin{center}
{\Large \bf 2020年度 構造力学I及び演習B 演習問題6 解答} \\
\end{center}
%%%%%%%%%%%%%%%%%%%%%%%%%%%%%%%%%%%%%%%%%%%%%%%%%%%%%%%%%%%%%%%%
\subsubsection*{問題1.}
図\ref{fig:fig1}に示すように$x$座標をとり,$x=a$の位置に単位荷重が加えられたときに,
支点Bに発生する曲げモーメント$M_B$と鉛直反力$R_B$を求める.
そのために,荷重項を$q(x)=\tilde P \delta (x-a)$とし, たわみ$v$の微分方程式を指定された
両端固定条件の元で解くと,
\begin{equation}
	v=\frac{\tilde Pl^3}{6EI}\left\{
		\left< \xi-\alpha \right>^3 +\beta^2(2\beta -3)\xi^3 +3\beta^2(1-\beta) \xi^2 
	 \right\}
	\label{eqn:vx}
\end{equation}
となる.ただし,
\begin{equation}
	\alpha=a/l,\, \beta=b/l, \, \xi=x/l
\end{equation}
とした.式(\ref{eqn:vx})より,曲げモーメント$M$とせん断力$Q$は
\begin{eqnarray}
	M(x) &=& 
	-\tilde Pl
	\left\{
		\left< \xi-\alpha \right> +\beta^2(2\beta -3)\xi+\beta^2(1-\beta)
 	\right\}
	\\
	M(x) &=& 
	-\tilde P
	\left\{
		H(\xi-\alpha) +\beta^2(2\beta -3)
 	\right\}
\end{eqnarray}
と得られるため,支点Aの曲げモーメントと鉛直反力は次のようになる.
\begin{equation}
	M_A=M(0)=-\tilde P l \beta^2(1-\beta), \ \ 
	R_A=Q(0)=-\tilde P \beta^2 ( 2\beta-3)
\end{equation}
この結果を$a$の関数として図示すれば,$M_A$および$R_A$の影響線が図\ref{fig:fig2}のようになることが分かる.
%%--------------------
\begin{figure}[h]
	\begin{center}
	\includegraphics[width=0.55\linewidth]{fig1ans.eps}
	\end{center}
	\caption{問題1の解答に用いた系.}
	\label{fig:fig1}
\end{figure}
%--------------------
\begin{figure}[h]
	\begin{center}
	\includegraphics[width=0.6\linewidth]{fig2ans.eps} 
	\end{center}
	\caption{支点部Aにおける曲げモーメントと鉛直反力の影響線.}
	\label{fig:fig2}
\end{figure}
%%%%%%%%%%%%%%%%%%%%%%%%%%%%%%%%%%%%%%%%%%%%%%%%%%%%%%%%%%%%%%%%
%%
%
%
\subsubsection*{問題2.}
支点AとDにおける鉛直反力と,ヒンジBとCで伝達される鉛直力の正方向をそれぞれ図\ref{fig:fig3}-(a)のように定める.
これらの鉛直力は,区間AB,BC,CDそれぞれの区間における力とモーメントの釣り合いを考えることで決定できる.
\begin{enumerate}
\item
はじめに,区間BCにおける釣り合い条件から,
\begin{equation}
	R_B=R_C=\frac{q_0l}{2}
\end{equation}
が言える.
\item
問1の結果を踏まえて区間ABとDEの釣り合い条件を用いれば,両固定端における反力が以下のように求められる.
\begin{equation}
	R_A=\frac{1}{2}q_0l, \ \ M_A=-\frac{1}{4}q_0l^2  
\end{equation}
\begin{equation}
	R_D=q_0l, \ \ M_D=-\frac{3}{8}q_0l^2
\end{equation}
\item
区間AB,BC,CDの単位で見ると,すべて静定構造となっているため,区間毎に釣り合い条件を適用して
曲げモーメントを求めることができる.その結果を曲げモーメント図として示せば,図\ref{fig:fig4}のように
なる.この図には,曲げモーメント分布は連続で,等分布荷重が加わる区間は2次関数的に,
荷重が作用しない区間では直線的に曲げモーメント値が変化することが示されている.
\item
図\ref{fig:fig4}より,曲げモーメントは点D($x=2l$)において最大値をとる.
\begin{equation}
	M_{max}=\left|-\frac{3}{8}q_0l^2\right| = \frac{3}{8}q_0l^2
\end{equation}
\item
	曲げ応力$\sigma=\frac{M}{I}y$は,曲げモーメントが最大となる断面の上縁$y=-\frac{a}{2}$
	あるいは下縁$y=\frac{a}{2}$において最大となる.
	また,一辺の長さが$a$の正方形断面の,中立軸に関する断面2次モーメント$I$は
\begin{equation}
	I=\frac{a^4}{12}	
\end{equation}
だから,
\begin{equation}
	\sigma_{max}=
	\frac{M_{max}}{I}\frac{a}{2}=
	\frac{9}{4} \frac{q_0l^2}{a^3}
\end{equation}
となる.
\item
	上で求めた$\sigma_{max}$を不等式$\sigma_{max}\leq \sigma_a$に代入し,$a$に関して整理すれば,
\begin{equation}
	a^3 \geq \frac{9}{4}\frac{q_0l^2}{\sigma_a} 
\end{equation}
		となる.よって,許容される$a$の最小値は$\sqrt[3]{\frac{9}{4}\frac{q_0l^2}{\sigma_a}}$である.
\end{enumerate}
%--------------------
\begin{figure}[h]
	\begin{center}
	\includegraphics[width=0.7\linewidth]{fig3ans.eps} 
	\end{center}
	\caption{
		支点反力とヒンジ部で伝達される鉛直力の正方向.
		} 
	\label{fig:fig3}
\end{figure}
\begin{figure}[h]
	\begin{center}
	\includegraphics[width=0.7\linewidth]{fig4ans.eps} 
	\end{center}
	\caption{
		曲げモーメント図.} 
	\label{fig:fig4}
\end{figure}
\end{document}

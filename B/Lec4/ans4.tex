\documentclass[10pt,a4j]{jarticle}
\usepackage{graphicx,wrapfig}
\setlength{\topmargin}{-1.5cm}
%\setlength{\textwidth}{15.5cm}
\setlength{\textheight}{25.2cm}
\newlength{\minitwocolumn}
\setlength{\minitwocolumn}{0.5\textwidth}
\addtolength{\minitwocolumn}{-\columnsep}
%\addtolength{\baselineskip}{-0.1\baselineskip}
%
\def\Mmaru#1{{\ooalign{\hfil#1\/\hfil\crcr
\raise.167ex\hbox{\mathhexbox 20D}}}}
%
\begin{document}
\newcommand{\fat}[1]{\mbox{\boldmath $#1$}}
\newcommand{\D}{\partial}
\newcommand{\w}{\omega}
\newcommand{\ga}{\alpha}
\newcommand{\gb}{\beta}
\newcommand{\gx}{\xi}
\newcommand{\gz}{\zeta}
\newcommand{\vhat}[1]{\hat{\fat{#1}}}
\newcommand{\spc}{\vspace{0.7\baselineskip}}
\newcommand{\halfspc}{\vspace{0.3\baselineskip}}
\bibliographystyle{unsrt}
\pagestyle{empty}
\newcommand{\twofig}[2]
 {
   \begin{figure}[h]
     \begin{minipage}[t]{\minitwocolumn}
         \begin{center}   #1
         \end{center}
     \end{minipage}
         \hspace{\columnsep}
     \begin{minipage}[t]{\minitwocolumn}
         \begin{center} #2
         \end{center}
     \end{minipage}
   \end{figure}
 }
%%%%%%%%%%%%%%%%%%%%%%%%%%%%%%%%%
%\vspace*{\baselineskip}
\begin{center}
{\Large \bf 2020年度 構造力学I及び演習B 演習問題4 解答} \\
\end{center}
%%%%%%%%%%%%%%%%%%%%%%%%%%%%%%%%%%%%%%%%%%%%%%%%%%%%%%%%%%%%%%%%
\subsubsection*{問題1}
断面$S$における$y^m\,(m=0,1,2)$の面積積分を
\begin{equation}
	J_m:=\int_S y^m dS, \ \ (m=0,1,2)
\end{equation}
とおく.$k=\frac{\pi}{b}$とすれば,$J_m$は
\begin{eqnarray}
	J_m 
	&=& 
	\int_{x=-\frac{b}{2}}^{\frac{b}{2}} \int_{y=0}^{h\cos(kx)}y^m dydx \\
	&=& 
	2\int_{x=0}^{\frac{b}{2}} \int_{y=0}^{h\cos(kx)}y^m dydx \\
	&=& 
	\frac{2h^{m+1}}{m+1}\int_{x=0}^{\frac{b}{2}} \cos^{m+1}(kx)dx \\
	&=& 
	\frac{2bh^{m+1}}{(m+1)\pi}\int_{u=0}^{\frac{\pi}{2}} \cos^{m+1}u du
\end{eqnarray}
として計算することができる.これを$m=0,1,2$それぞれの場合について具体的に計算すると,
\begin{equation}
	J_0=
	\frac{2bh}{\pi}\int_{u=0}^{\frac{\pi}{2}} \cos u du = \frac{2bh}{\pi}
\end{equation}
\begin{eqnarray}
	J_1
	&=&
	\frac{bh^2}{\pi}\int_{u=0}^{\frac{\pi}{2}} \cos^2 u du \\
	&=&
	\frac{bh^2}{\pi}\int_{u=0}^{\frac{\pi}{2}} \frac{1+\cos(2u)}{2} du \\
	&=& \frac{bh^2}{4}
\end{eqnarray}
\begin{eqnarray}
	J_2
	&=&
	\frac{2bh^3}{3\pi}\int_{u=0}^{\frac{\pi}{2}} \cos^3 u du \\
	&=&
	\frac{2bh^3}{3\pi}\int_{u=0}^{\frac{\pi}{2}} \frac{3\cos u + \cos (3u) }{4} du \\
	&=&
	=\frac{4}{9}\frac{bh^3}{\pi}
\end{eqnarray}
となる.
断面$S$の面積$\left| S \right|$は$J_0$で与えられ,
$x$軸に関する断面1次モーメント$G$は$J_1$で与えられるので,中立軸位置$\bar Y$は,
\begin{equation}
	\bar Y =\frac{G}{\left| S\right|}=\frac{J_1}{J_0}=\frac{\pi}{8}h
\end{equation}
となる.また,$x$軸に関する断面2次モーメント$I_x=J_2$だから,
中立軸に関する断面2次モーメント$I$は
以下のように求められる.
\begin{eqnarray}
	I & =& I_x -\bar Y^2 \left| S\right| \\
	&=& J_2-\bar Y^2 J_0 \\
	&=& \left( \frac{4}{9\pi}-\frac{\pi}{32} \right) bh^3
\end{eqnarray}
%%%%%%%%%%%%%%%%%%%%%%%%%%%%%%%%%%%%%%%%%%%%%%%%%%%%%%%%%%%%%%%%%%%%%%%%%%%%
%%%%%%%%%%%%%%%%%%%%%%%%%%%%%%%%%%%%%%%%%%%%%%%%%%%%%%%%%%%%%%%%%%%%%%%%%%%%
\subsubsection*{問題2}
与えられた断面$S$を図\ref{fig:fig1}-(a)に示す部分断面$S_1,S_2$および$S_3$に分割する.
ここで,部分断面$S_i$の$x$軸からの距離で測った中立軸位置を$\bar Y_i$,各々の断面の中立軸
に関する断面2次モーメントを$I(S_i)$と表すことにする.
$S_1$は幅$b$,高さ$h$の三角形なので,
\begin{equation}
	\bar Y_1 = \frac{h}{3}, \ \ I(S_1)=\frac{bh^3}{36}
\end{equation}
である.$S_2$と$S_3$は,$y$軸について対称なので,中立軸位置や断面2次モーメントは互い一致する.
そこで,$\bar Y_2$と$I(S_2)$を,図\ref{fig:fig1}-(b)に示した座標を使って計算する.
$S_2$に関する$m$次のモーメントは
\begin{eqnarray}
	J_m 
	&:=& 
	\int_{S_2} y^m dS \\
	&=& 
	\int_{x=0}^b\int_{y=\frac{h}{b}x}^{\frac{h}{b}x+h} y^m dydx \\
	&=& 
	\frac{h^{m+1}}{m+1}
	\int_{x=0}^b 
	\left\{ 
	\left( \frac{h}{b}+1\right)^{m+1} -
	\left( \frac{h}{b}\right)^{m+1} \right\} 
	dx  \\
	&=&
	\frac{bh^{m+1}}{m+1} \int_0^1 \left\{ (u+1)^{m+1}-u^{m+1}\right\}  du \\
	&=&
	\frac{bh^{m+1}}{(m+1)(m+2)} \left[ (u+1)^{m+2}-u^{m+2}\right]_0^1 \\
	&=&
	\frac{2^{m+2}-2}{(m+1)(m+2)}bh^{m+1}
\end{eqnarray}
と得られるので,
\begin{equation}
	\bar Y_2= \frac{J_1}{J_0}=\frac{bh^2}{bh}=h
\end{equation}
となる.また,
\begin{equation}
	I(S_2) =  J_2-\bar Y_2^2 \left| S_2 \right| = \frac{7}{6}bh^3 -h^2\times bh = \frac{bh^3}{6}
\end{equation}
である.
以上のことを踏まえれば,表\ref{tbl:tbl2}に示すような計算過程を経て,複合断面:
\begin{equation}
	S=S_1\cup S_2 \cup S_3
\end{equation}	
の中立軸位置$\bar Y$と断面2次モーメント$I(S)$が,次のように求められる.
\begin{equation}
	\bar Y = \sum_{i=1}^3 \frac{\left|S_i\right|}{\left| S \right|}\bar{Y}_i=\frac{13}{15}h
	\label{eqn:Yb2}
\end{equation}
\begin{equation}
	I(S)=\sum_{i=1}^3\left\{
		I(S_i)+\left( \bar Y_i -\bar Y\right)^2\left| S_i \right|
	\right\}
		=\frac{97}{180}bh^3
	\label{eqn:Iz_2}
\end{equation}
\begin{table}
\begin{center}
	\caption{部分断面への分割に基づく断面係数計算の過程}
	\begin{tabular}{c||c|c|c|c|c|c|c}
		&
		$\left| S_i \right|$ & 
		$ \xi_i=\frac{\left| S_i \right|}{\left| S\right|} $  &
		$ \bar{Y}_i $ & 
		$ \xi_i\bar{Y}_i $ & 
		$\bar{Y}_i -\bar Y$ & 
		$ \left(\bar{Y}_i -\bar Y\right)^2\left| S_i \right|$ & 
		$ I(S_i)$  
		\\
		\hline 
		\hline 
		断面1&	
		$\frac{1}{2}bh$ & 
		$\frac{1}{5}$  &
		$\frac{1}{3}h$ & 
		$\frac{1}{15}h$ & 
		$-\frac{8}{15}h$ & 
		$\frac{32}{225}bh^3$ &
		$\frac{1}{36}bh^3$ 
		\\
		\hline
		断面2&	
		$bh$ & 
		$\frac{2}{5}$  &
		$h$ & 
		$\frac{2}{5}h $ & 
		$\frac{2}{15}h$ & 
		$\frac{4}{225}bh^3$ &
		$\frac{1}{6}bh^3$ 
		\\
		\hline 
		断面3&	
		$bh$ & 
		$\frac{2}{5}$  &
		$h$ & 
		$\frac{2}{5}h $ & 
		$\frac{2}{15}h$ & 
		$\frac{4}{225}bh^3$ &
		$\frac{1}{6}bh^3$ 
		\\
		\hline 
		\hline 
		合計&	
		$\frac{5}{2}bh$ & 
		$1$  &
		$-$ & 
		$\bar Y=\frac{13}{15}h $ & 
		$-$ & 
		$\frac{8}{45}bh^3$ &
		$\frac{13}{36}bh^3$ 
	\end{tabular}
\label{tbl:tbl2}
\end{center}
\end{table}
%%%%%%%%%%%%%%%%%%%%%%%%%%%%%%%%%%%%%%%%%%%%%%%%%%%%%%%%%%%%%%%%%%%%%%%%%%%%
%%%%%%%%%%%%%%%%%%%%%%%%%%%%%%%%%%%%%%%%%%%%%%%%%%%%%%%%%%%%%%%%%%%%%%%%%%%%
%--------------------
\begin{figure}[h]
	\begin{center}
	\includegraphics[width=0.8\linewidth]{fig1ans.eps} 
	\end{center}
	\caption{断面2次モーメントの計算に用いた部分断面.} 
	\label{fig:fig1}
\end{figure}
%--------------------
\end{document}
%--------------------

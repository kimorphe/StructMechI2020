\documentclass[10pt,a4j]{jarticle}
\usepackage{graphicx,wrapfig}
\setlength{\topmargin}{-1.5cm}
%\setlength{\textwidth}{15.5cm}
\setlength{\textheight}{25.2cm}
\newlength{\minitwocolumn}
\setlength{\minitwocolumn}{0.5\textwidth}
\addtolength{\minitwocolumn}{-\columnsep}
%\addtolength{\baselineskip}{-0.1\baselineskip}
%
\def\Mmaru#1{{\ooalign{\hfil#1\/\hfil\crcr
\raise.167ex\hbox{\mathhexbox 20D}}}}
%
\begin{document}
\newcommand{\fat}[1]{\mbox{\boldmath $#1$}}
\newcommand{\D}{\partial}
\newcommand{\w}{\omega}
\newcommand{\ga}{\alpha}
\newcommand{\gb}{\beta}
\newcommand{\gx}{\xi}
\newcommand{\gz}{\zeta}
\newcommand{\vhat}[1]{\hat{\fat{#1}}}
\newcommand{\spc}{\vspace{0.7\baselineskip}}
\newcommand{\halfspc}{\vspace{0.3\baselineskip}}
\bibliographystyle{unsrt}
\pagestyle{empty}
\newcommand{\twofig}[2]
 {
   \begin{figure}[h]
     \begin{minipage}[t]{\minitwocolumn}
         \begin{center}   #1
         \end{center}
     \end{minipage}
         \hspace{\columnsep}
     \begin{minipage}[t]{\minitwocolumn}
         \begin{center} #2
         \end{center}
     \end{minipage}
   \end{figure}
 }
%%%%%%%%%%%%%%%%%%%%%%%%%%%%%%%%%
%\vspace*{\baselineskip}
\begin{center}
{\Large \bf 2020年度 構造力学I及び演習B 演習問題4} \\
\end{center}
%%%%%%%%%%%%%%%%%%%%%%%%%%%%%%%%%%%%%%%%%%%%%%%%%%%%%%%%%%%%%%%%
\subsubsection*{問題1}
図\ref{fig:fig1}-(a)に示す断面$S$について,$x$軸に平行な中立軸の位置と,
中立軸に関する断面2次モーメント$I$を求めよ.なお,断面$S$の境界となる
曲線部分$C$は次のような関数で表され,中立軸の位置は$x$軸からの距離$\bar{Y}$で表すこととする.
\begin{equation}
	C: y=h \cos \left( \frac{\pi y}{b}\right) 
	\ \ \left(-\frac{b}{2} < y < \frac{b}{2} \right)
\end{equation}
\subsubsection*{問題2}
図\ref{fig:fig1}-(b)に示す断面$S3$について,$x$軸に平行な中立軸の位置と,
中立軸に関する断面2次モーメント$I$を求めよ.
中立軸の位置は,$x$軸から中立軸までの距離$\bar{Y}$で表すこと.

\vspace{15mm}
%--------------------
\begin{figure}[h]
	\begin{center}
	\includegraphics[width=0.80\linewidth]{fig1.eps} 
	\end{center}
	\caption{断面形状と寸法.} 
	\label{fig:fig1}
\end{figure}
%--------------------
\end{document}

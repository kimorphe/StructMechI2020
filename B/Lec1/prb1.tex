\documentclass[10pt,a4j]{jarticle}
\usepackage{graphicx,wrapfig}
\setlength{\topmargin}{-1.5cm}
\setlength{\textwidth}{15.5cm}
\setlength{\textheight}{25.2cm}
\newlength{\minitwocolumn}
\setlength{\minitwocolumn}{0.5\textwidth}
\addtolength{\minitwocolumn}{-\columnsep}
%\addtolength{\baselineskip}{-0.1\baselineskip}
%
\def\Mmaru#1{{\ooalign{\hfil#1\/\hfil\crcr
\raise.167ex\hbox{\mathhexbox 20D}}}}
%
\begin{document}
\newcommand{\fat}[1]{\mbox{\boldmath $#1$}}
\newcommand{\D}{\partial}
\newcommand{\w}{\omega}
\newcommand{\ga}{\alpha}
\newcommand{\gb}{\beta}
\newcommand{\gx}{\xi}
\newcommand{\gz}{\zeta}
\newcommand{\vhat}[1]{\hat{\fat{#1}}}
\newcommand{\spc}{\vspace{0.7\baselineskip}}
\newcommand{\halfspc}{\vspace{0.3\baselineskip}}
\bibliographystyle{unsrt}
\pagestyle{empty}
\newcommand{\twofig}[2]
 {
   \begin{figure}[h]
     \begin{minipage}[t]{\minitwocolumn}
         \begin{center}   #1
         \end{center}
     \end{minipage}
         \hspace{\columnsep}
     \begin{minipage}[t]{\minitwocolumn}
         \begin{center} #2
         \end{center}
     \end{minipage}
   \end{figure}
 }
%%%%%%%%%%%%%%%%%%%%%%%%%%%%%%%%%
%\vspace*{\baselineskip}
\begin{center}
{\Large \bf 2019年度 構造力学I及び演習B 演習問題1(12月6日)} \\
\end{center}
%%%%%%%%%%%%%%%%%%%%%%%%%%%%%%%%%%%%%%%%%%%%%%%%%%%%%%%%%%%%%%%%
\subsubsection*{問題1. }
図\ref{fig:fig1}に示したような,直線的に変化する分布荷重を受ける梁を考える.
梁の曲げ剛性$EI$は全断面で一定とするとき,以下の問いに答えよ.
\begin{enumerate}
\item
	たわみ$v(x)$を求めよ.
\item
	曲げモーメント図とせん断力図を描け.
\item
	支点Aにおける支点反力を求めよ.
	解答には支点反力の正方向を明記すること.
\end{enumerate}
\begin{figure}[h]
	\begin{center}
	\includegraphics[width=0.5\linewidth]{fig1.eps} 
	\end{center}
	\vspace{-5mm}
	\caption{左半分の区間に等分布荷重を受ける梁.}
	\label{fig:fig1}
\end{figure}
\subsubsection*{問題2. }
図\ref{fig:fig2}に示したような大きさ$P$の集中荷重を受ける梁ACを考える.
梁の曲げ剛性$EI$は全断面で一定とするとき以下の問いに答えよ.
\begin{enumerate}
\item
	たわみ$v(x)$を求めよ.		
\item
	曲げモーメント図とせん断力図を描け.
\item
	支点AとCにおける支点反力を求めよ.解答には支点反力の正方向を明記すること).
\end{enumerate}
\begin{figure}[h]
	\begin{center}
	\includegraphics[width=0.5\linewidth]{fig2.eps} 
	\end{center}
	\vspace{-5mm}
	\caption{大きさ$P$の集中荷重を受ける梁AC.}
	\label{fig:fig2}
\end{figure}
\end{document}

\documentclass[10pt,a4j]{jarticle}
\usepackage{graphicx,wrapfig}
\setlength{\topmargin}{-1.5cm}
\setlength{\textwidth}{15.5cm}
\setlength{\textheight}{25.2cm}
\newlength{\minitwocolumn}
\setlength{\minitwocolumn}{0.5\textwidth}
\addtolength{\minitwocolumn}{-\columnsep}
%\addtolength{\baselineskip}{-0.1\baselineskip}
%
\def\Mmaru#1{{\ooalign{\hfil#1\/\hfil\crcr
\raise.167ex\hbox{\mathhexbox 20D}}}}
%
\begin{document}
\newcommand{\fat}[1]{\mbox{\boldmath $#1$}}
\newcommand{\D}{\partial}
\newcommand{\w}{\omega}
\newcommand{\ga}{\alpha}
\newcommand{\gb}{\beta}
\newcommand{\gx}{\xi}
\newcommand{\gz}{\zeta}
\newcommand{\vhat}[1]{\hat{\fat{#1}}}
\newcommand{\spc}{\vspace{0.7\baselineskip}}
\newcommand{\halfspc}{\vspace{0.3\baselineskip}}
\bibliographystyle{unsrt}
\pagestyle{empty}
\newcommand{\twofig}[2]
 {
   \begin{figure}[h]
     \begin{minipage}[t]{\minitwocolumn}
         \begin{center}   #1
         \end{center}
     \end{minipage}
         \hspace{\columnsep}
     \begin{minipage}[t]{\minitwocolumn}
         \begin{center} #2
         \end{center}
     \end{minipage}
   \end{figure}
 }
%%%%%%%%%%%%%%%%%%%%%%%%%%%%%%%%%
%\vspace*{\baselineskip}
\begin{center}
{\Large \bf 2019年度 構造力学I及び演習B 演習問題3 解答} \\
\end{center}
%%%%%%%%%%%%%%%%%%%%%%%%%%%%%%%%%%%%%%%%%%%%%%%%%%%%%%%%%%%%%%%%
\subsubsection*{問題}
\begin{enumerate}
\item
図\ref{fig:fig1}-(b$_1$)のように$x$座標をとると,分布荷重$q(x)$は
$q(x)=q_0H\left(x-l\right)$と表すことができる.
$q(x)$の4階の不定積分は
\begin{equation}
	\int q(x) dx^4
	= 
	\frac{q_0l^4}{24}
	\left< \frac{x}{l} -1\right> ^4
\end{equation}
と書けるので,梁(b$_1$)のたわみ$v_1(x)$は,$C_1\sim C_4$を積分定数として
次のように表すことができる.
	\begin{equation}
	v_1(x)=\frac{q_0l^4}{24EI}\left\{
		\left< \frac{x}{l} -1 \right> ^4
		+
		C_1
		\left(\frac{x}{l} \right)^3
		+
		C_2
		\left(\frac{x}{l} \right)^2
		+
		C_3
		\left(\frac{x}{l} \right)
		+
		C_4,
	\right\}
	\end{equation}
積分定数$C_3$と$C_4$は,$x=0$における支持条件:$v(0)=0,\, v'(0)=0$より$C_3=C_4=0$と決まる.
一方,$C_1,C_2$は,$x=2l$における支持条件
$v(2l)=0,\, v'(2l)=0$より,$C_1=-\frac{3}{4},\, C_2=\frac{5}{4}$
となる.以上より,たわみは
	\begin{equation}
	v_1(x)=\frac{q_0l^4}{24EI}\left\{
		\left< \frac{x}{l} -1\right> ^4
		-
		\frac{3}{4}
		\left(\frac{x}{l} \right)^3
		+
		\frac{5}{4}
		\left(\frac{x}{l} \right)^2
	\right\}
	\label{eqn:vx1}
	\end{equation}
となる.これより,点Bのたわみは
\begin{equation}
	v_B=v\left( l \right) 
	= \frac{1}{48}\frac{q_0l^4}{EI}
	\label{eqn:vc_q}
\end{equation}
と求められる.
%%%%%%%%%%%%%%
\item
問題で与えられた梁(b)に作用する鉛直方向の支点反力を図\ref{fig:fig1}-(b)に示すようにとる.
この梁を同図(b$_1$)と(b$_2$)のような2つの系の重ね合わせとして表現する.
このとき,梁(b$_1)$のB点におけるたわみは式(\ref{eqn:vc_q})で与えられる.
一方,梁(b$_2$)のたわみ分布$v_2$は,荷重項を
\begin{equation}
	q(x)=-R_B\delta\left(x-l\right)
\end{equation}
とし,たわみの方程式を両端固定条件の元で解くことにより,
\begin{equation}
	v_2(x)=\frac{(-R_B)l^3}{6EI}\left\{
		\left< \frac{x}{l}-1\right>^3
		-
		\frac{1}{2}
		\left(\frac{x}{l} \right)^3
		+
		\frac{3}{4}
		\left(\frac{x}{l} \right)^2
	\right\}
	\label{eqn:vx2}
\end{equation}
と与えられる.よって,点Bにおけるたわみは
\begin{equation}
	v_2\left( l \right) 
	= -\frac{R_Bl^3}{24EI}.
	\label{eqn:vc_P}
\end{equation}
である.$v_B$と$v_2(l)$は相殺すべきものであることから,
\begin{equation}
	v_B+v_2(l)=0 \ \ \Rightarrow \ \ R_B=\frac{1}{2}q_0l
	\label{eqn:RB}
\end{equation}
と,点Bにおける鉛直反力が求められる.\\

一方,支点AとCにおける反力は,曲げモーメントおよびせん断力分布に,式(\ref{eqn:RB})を代入して求めることができる.
図\ref{fig:fig1}-(b)の梁のたわみ$v(x)$は$v(x)=v_1(x)+v_2(x)$と表されるので,
\begin{equation}
	M_1=-EIv_1'', \ \ M_2=-EIv_2''
\end{equation}
および
\begin{equation}
	Q_1=-(EIv_1'')', \ \ Q_2=-(EIv_2'')',
\end{equation}
と書けば,求めるべき曲げモーメント$M$とせん断力$Q$を
\begin{equation}
	M(x)=M_1(x)+M_2(x),  \  \
	Q(x)=Q_1(x)+Q_2(x)
\end{equation}
と表すことができる.これらは,式(\ref{eqn:vx1})と式(\ref{eqn:vx2})を微分して
\begin{eqnarray}
	M_1(x) & =& 
	-\frac{q_0l^2}{24}\left\{ 
	12\left<\frac{x}{l}-1\right>^2 - \frac{9}{2}\left(\frac{x}{l}\right) + \frac{5}{2}
	\right\}
	\label{eqn:M1}
	\\
	Q_1(x) & =& 
	-\frac{q_0l}{8}\left\{ 
	8\left<\frac{x}{l}-1\right> - \frac{3}{2}
	\right\}
	\label{eqn:Q1}
\end{eqnarray}
\begin{eqnarray}
	M_2(x) & =& 
	\frac{R_Bl}{2}\left\{ 
	2\left<\frac{x}{l} - 1\right> -\frac{x}{l} + \frac{1}{2}
	\right\}
	\label{eqn:M2}
	\\
	Q_2(x) & =& 
	\frac{R_B}{2}\left\{ 
	2H\left(\frac{x}{l}-1\right)-1
	\right\}
	\label{eqn:Q2}
\end{eqnarray}
と求められる.これより,
\begin{equation}
	M_1(0)=-\frac{5}{48}q_0l^2, \ \ M_2(0)=\frac{R_Bl}{4}=\frac{q_0l^2}{8}
\end{equation}
\begin{equation}
	M_1(2l)=-\frac{11}{48}q_0l^2, \ \ M_2(2l)=\frac{R_Bl}{4}=\frac{q_0l^2}{8}
\end{equation}
\begin{equation}
	Q_1(0)=\frac{3}{16}q_0l, \ \ Q_2(0)=-\frac{R_B}{2}=-\frac{q_0l}{4}
\end{equation}
\begin{equation}
	Q_1(2l)=-\frac{13}{16}q_0l, \ \ Q_2(2l)=\frac{R_B}{2}=\frac{q_0l}{4}
\end{equation}
となることから,部材端でのせん断力と曲げモーメントが以下のように決まる.
\begin{equation}
	R_A=Q_1(0)+Q_2(0)= -\frac{1}{16}q_0l, \ \ R_C=-Q_1(2l)-Q_2(2l)=-\frac{9}{16}q_0l
\end{equation}
\begin{equation}
	M_A=M_1(0)+M_2(0)= \frac{1}{48}q_0l^2, \ \ M_C=M_1(2l)+M_2(2l)=-\frac{5}{48}q_0l^2
\end{equation}
\item
曲げモーメント$M=M_1+M_2$は,式(\ref{eqn:M1})と式(\ref{eqn:M2})より,
\begin{itemize}
\item
	$x<l$のとき:
	\begin{equation}
		M(x)=\frac{q_0l^2}{48}( -3 \xi +1), \ \ \left( \xi=\frac{x}{l}\right)
		\label{eqn:M_left}
	\end{equation}
\item
	$x>l$のとき:
	\begin{equation}
		M(x)=\frac{q_0l^2}{2}\left( -\eta^2 +\frac{7}{8}\eta - \frac{1}{12}\right), \ \ 
		\left( \eta=\frac{x}{l}-1\right)
		\label{eqn:M_right}
	\end{equation}
\end{itemize}
である.これを曲げモーメント図として表すと,図\ref{fig:fig2}のようになる.
なお,支点反力が求められた時点で,断面力は全て釣り合い条件から決定することができるので,
たわみを微分する以外にも,区間毎に釣り合い式を立てて断面力を求めてもよい.
%--------------------
\begin{figure}[h]
	\begin{center}
	\includegraphics[width=0.6\linewidth]{fig1ans.eps} 
	\end{center}
	\caption{2つの系$(b_1)$と$(b_2)$による梁(b)の表現.}
	\label{fig:fig1}
\end{figure}
%--------------------
%--------------------
\begin{figure}[h]
	\begin{center}
	\includegraphics[width=0.6\linewidth]{fig2ans.eps} 
	\end{center}
	\caption{曲げモーメント図.}
	\label{fig:fig2}
\end{figure}
%--------------------
%%%%%%%%%%%%%%%%%%%%%%%%%%%%%%
\end{enumerate}
\end{document}
x

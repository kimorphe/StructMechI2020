\documentclass[10pt,a4j]{jarticle}
\usepackage{graphicx,wrapfig}
\setlength{\topmargin}{-1.5cm}
%\setlength{\textwidth}{15.5cm}
\setlength{\textheight}{25.2cm}
\newlength{\minitwocolumn}
\setlength{\minitwocolumn}{0.5\textwidth}
\addtolength{\minitwocolumn}{-\columnsep}
%\addtolength{\baselineskip}{-0.1\baselineskip}
%
\def\Mmaru#1{{\ooalign{\hfil#1\/\hfil\crcr
\raise.167ex\hbox{\mathhexbox 20D}}}}
%
\begin{document}
\newcommand{\fat}[1]{\mbox{\boldmath $#1$}}
\newcommand{\D}{\partial}
\newcommand{\w}{\omega}
\newcommand{\ga}{\alpha}
\newcommand{\gb}{\beta}
\newcommand{\gx}{\xi}
\newcommand{\gz}{\zeta}
\newcommand{\vhat}[1]{\hat{\fat{#1}}}
\newcommand{\spc}{\vspace{0.7\baselineskip}}
\newcommand{\halfspc}{\vspace{0.3\baselineskip}}
\bibliographystyle{unsrt}
\pagestyle{empty}
\newcommand{\twofig}[2]
 {
   \begin{figure}[h]
     \begin{minipage}[t]{\minitwocolumn}
         \begin{center}   #1
         \end{center}
     \end{minipage}
         \hspace{\columnsep}
     \begin{minipage}[t]{\minitwocolumn}
         \begin{center} #2
         \end{center}
     \end{minipage}
   \end{figure}
 }
%%%%%%%%%%%%%%%%%%%%%%%%%%%%%%%%%
%\vspace*{\baselineskip}
\begin{center}
{\Large \bf 2020年度 構造力学I及び演習B 演習問題3 解答} \\
\end{center}
%%%%%%%%%%%%%%%%%%%%%%%%%%%%%%%%%%%%%%%%%%%%%%%%%%%%%%%%%%%%%%%%
\subsubsection*{問題}
\begin{enumerate}
\item
図\ref{fig:fig1}-(c$_1$)のように$x$座標をとると,分布荷重$q(x)$は
$q(x)=2q_0H\left(x-l\right)$と表すことができる.
$q(x)$の4階の不定積分は
\begin{equation}
	\int q(x) dx^4
	= 
	\frac{q_0l^4}{12}
	\left< \frac{x}{l} -\frac{1}{2}\right> ^4
\end{equation}
と書けるので,梁(b$_1$)のたわみ$v_1(x)$は,$C_1\sim C_4$を積分定数として
次のように表すことができる.
	\begin{equation}
	v_1(x)=\frac{q_0l^4}{12EI}\left\{
		\left< \xi -\frac{1}{2} \right> ^4
		+
		C_1 \xi^3 + C_2 \xi^2 + C_3 \xi + C_4,
	\right\}
	\  \	
		\left(\xi=\frac{x}{l}\right)
	\end{equation}
積分定数$C_3$と$C_4$は,$x=0$における支持条件:
\begin{equation}
		v(0)=0,\ \  v'(0)=0
\end{equation}
より$C_3=C_4=0$と決まる.一方,$C_1$と$C_2$は,$x=l$における支持条件:
\begin{equation}
	v(l)=0,\ \  M(l)=0
\end{equation}
より,$C_1=-\frac{23}{32},\, C_2=\frac{21}{32}$
となる.以上より,たわみは
	\begin{equation}
	v_1(x)=\frac{q_0l^4}{12EI}\left\{
		\left< \xi -\frac{1}{2}\right> ^4
		-
		\frac{23}{32}
		\xi^3
		+
		\frac{21}{32}
		\xi^2
	\right\}
	\label{eqn:vx1}
	\end{equation}
となる.
%%%%%%%%%%%%%%
\item
たわみを$v_2$,曲げモーメントとせん断力をそれぞれ$M_2,Q_2$とする.
外力は$q(x)=P\delta(x-\frac{l}{2})$だから,$v_2$を
	\begin{equation}
	v_2(x)=\frac{Pl^3}{6EI}\left\{
		\left< \xi -\frac{1}{2} \right> ^4
		+
		C_1 \xi^3 + C_2 \xi^2 + C_3 \xi + C_4,
	\right\}
	\  \	
		\left(\xi=\frac{x}{l}\right)
	\end{equation}
と表すことができる.
ここでも,積分定数$C_3$と$C_4$は,$x=0$における支持条件より$C_3=C_4=0$となる.
$C_1$と$C_2$は,$x=l$における条件から$C_1=-\frac{11}{16},C_2=\frac{9}{16}$
となるので,
	\begin{equation}
	v_2(x)=\frac{Pl^3}{6EI}\left\{
		\left< \xi -\frac{1}{2} \right> ^4
		-\frac{11}{16}\xi^3 + \frac{9}{16}\xi^2 
	\right\}
		\label{eqn:vx2}
	\end{equation}
が得られる.

\item
問題で与えられた梁(c)に作用する鉛直方向の支点反力を,図\ref{fig:fig1}-(c)に示すようにとる.
この梁を同図(c$_1$)と(c$_2$)のような2つの系の重ね合わせで表現する.
このとき,梁(c$_1)$のたわみは,式(\ref{eqn:vx1})で与えられる.
一方,梁(c$_2$)のたわみは,式(\ref{eqn:vx2})において
$P=-R_B$とすることで得られる.
よって,梁(c)の点Bにおけるたわみの適合条件は
\begin{equation}
	v_1\left( \frac{l}{2}\right) 
	+
	\left. v_2\left( \frac{l}{2}\right)  \right|_{P=-R_B}=0
\end{equation}
と表され,これより支点反力$R_B$が
\begin{equation}
	R_B=\frac{19}{28}q_0l
	\label{eqn:RB}
\end{equation}
と求められる.\\

一方,支点AとCにおける反力は,曲げモーメントおよびせん断力分布に,式(\ref{eqn:RB})を代入して求めることができる.
図\ref{fig:fig1}-(b)の梁のたわみ$v(x)$は$v(x)=v_1(x)+v_2(x)$と表されるので,
\begin{equation}
	M_1=-EIv_1'', \ \ M_2=-EIv_2''
\end{equation}
および
\begin{equation}
	Q_1=-(EIv_1'')', \ \ Q_2=-(EIv_2'')',
\end{equation}
と書けば,求めるべき曲げモーメント$M$とせん断力$Q$を
\begin{equation}
	M(x)=M_1(x)+M_2(x),  \  \
	Q(x)=Q_1(x)+Q_2(x)
\end{equation}
と表すことができる.これらは,式(\ref{eqn:vx1})と式(\ref{eqn:vx2})を微分して
	\begin{equation}
	M_1(x)=-\frac{q_0l^2}{4}\left\{
		4\left< \xi -\frac{1}{2}\right> ^2
		-
		\frac{23}{16}
		\xi
		+
		\frac{7}{16}
	\right\}
		\label{eqn:Mx1}
		\end{equation}
	\begin{equation}
	Q_1(x)=-\frac{q_0l}{4}\left\{
		8\left< \xi -\frac{1}{2}\right> 
		-
		\frac{23}{16}
	\right\}
	\label{eqn:Qx1}
	\end{equation}
	\begin{equation}
	M_2(x)=-Pl\left\{
		\left< \xi -\frac{1}{2}\right> 
			-
			\frac{11}{16} \xi
			+
			\frac{3}{16}
		\right\}
		\label{eqn:Mx2}
		\end{equation}
	\begin{equation}
	Q_2(x)=-P\left\{
		H\left( \xi -\frac{1}{2}\right) 
		-
		\frac{11}{16}
	\right\}
	\label{eqn:Qx2}
	\end{equation}
求められる.
これより,
\begin{equation}
	M_1(0)=-\frac{7}{64}q_0l^2, \ \ M_2(0)=\frac{3}{16}R_Bl=\frac{57}{7\cdot 64}q_0l^2
\end{equation}
\begin{equation}
	Q_1(0)=\frac{23}{64}q_0l, \ \ Q_2(0)=-\frac{11}{16}R_B=-\frac{209}{7\cdot 64}q_0l
\end{equation}
\begin{equation}
	Q_1(l)=-\frac{41}{64}q_0l, \ \ Q_2(l)=\frac{5}{16}R_B=\frac{95}{7\cdot 64 }q_0l
\end{equation}
となることから,部材端でのせん断力と曲げモーメントが以下のように決まる.
\begin{equation}
	R_A=Q_1(0)+Q_2(0)= -\frac{3}{28}q_0l, \ \ R_C=-Q_1(l)-Q_2(l)=\frac{3}{7}q_0l
\end{equation}
\begin{equation}
	M_A=M_1(0)+M_2(0)= \frac{1}{56}q_0l^2
\end{equation}
\item
曲げモーメント$M=M_1+M_2$は,式(\ref{eqn:Mx1})と式(\ref{eqn:Mx2})より,
\begin{itemize}
\item
	$x<\frac{l}{2}$のとき:
	\begin{equation}
		M(x)=\frac{q_0l^2}{56}( -6 \xi +1), \ \ \left( \xi=\frac{x}{l}\right)
		\label{eqn:M_left}
	\end{equation}
\item
	$x>\frac{l}{2}$のとき:
	\begin{equation}
		M(x)=-\frac{q_0l^2}{7}(\xi-1)(7\xi-4)
		\label{eqn:M_right}
	\end{equation}
\end{itemize}
である.これを曲げモーメント図として表すと,図\ref{fig:fig2}のようになる.
なお,支点反力が求められた時点で,断面力は全て釣り合い条件から決定することができるので,
たわみを微分する以外にも,区間毎に釣り合い式を立てて断面力を求めてもよい.
%--------------------
\begin{figure}[h]
	\begin{center}
	\includegraphics[width=0.6\linewidth]{fig1ans.eps} 
	\end{center}
	\caption{2つの系$(c_1)$と$(c_2)$による梁(c)の表現.}
	\label{fig:fig1}
\end{figure}
%--------------------
%--------------------
\begin{figure}[h]
	\begin{center}
	\includegraphics[width=0.6\linewidth]{fig2ans.eps} 
	\end{center}
	\caption{曲げモーメント図.}
	\label{fig:fig2}
\end{figure}
%--------------------
%%%%%%%%%%%%%%%%%%%%%%%%%%%%%%
\end{enumerate}
\end{document}
x

\documentclass[10pt,a4j]{jarticle}
\usepackage{graphicx,wrapfig}
\setlength{\topmargin}{-1.5cm}
\setlength{\textwidth}{15.5cm}
\setlength{\textheight}{25.2cm}
\newlength{\minitwocolumn}
\setlength{\minitwocolumn}{0.5\textwidth}
\addtolength{\minitwocolumn}{-\columnsep}
%\addtolength{\baselineskip}{-0.1\baselineskip}
%
\def\Mmaru#1{{\ooalign{\hfil#1\/\hfil\crcr
\raise.167ex\hbox{\mathhexbox 20D}}}}
%
\begin{document}
\newcommand{\fat}[1]{\mbox{\boldmath $#1$}}
\newcommand{\D}{\partial}
\newcommand{\w}{\omega}
\newcommand{\ga}{\alpha}
\newcommand{\gb}{\beta}
\newcommand{\gx}{\xi}
\newcommand{\gz}{\zeta}
\newcommand{\vhat}[1]{\hat{\fat{#1}}}
\newcommand{\spc}{\vspace{0.7\baselineskip}}
\newcommand{\halfspc}{\vspace{0.3\baselineskip}}
\bibliographystyle{unsrt}
\pagestyle{empty}
\newcommand{\twofig}[2]
 {
   \begin{figure}[h]
     \begin{minipage}[t]{\minitwocolumn}
         \begin{center}   #1
         \end{center}
     \end{minipage}
         \hspace{\columnsep}
     \begin{minipage}[t]{\minitwocolumn}
         \begin{center} #2
         \end{center}
     \end{minipage}
   \end{figure}
 }
%%%%%%%%%%%%%%%%%%%%%%%%%%%%%%%%%
%\vspace*{\baselineskip}
\begin{center}
	{\Large \bf 2019年度 構造力学I及び演習B 演習問題7(差替え版) 解答} \\
\end{center}
%%%%%%%%%%%%%%%%%%%%%%%%%%%%%%%%%%%%%%%%%%%%%%%%%%%%%%%%%%%%%%%%
\subsubsection*{問題}
\begin{enumerate}
\item
	部材1の温度上昇によって,部材1には圧縮の軸力が発生し,部材2はそれと
	同じ大きさの水平力を点Bに受ける.部材1の軸力の大きさを$P$とすれば,
	部材1の軸応力$\sigma_1$は
	\begin{equation}
		\sigma_1=-\frac{P}{A_1}
		\label{eqn:sig1}
	\end{equation}
	で与えられる.ここに,$A_1=a_1^2$は部材1の断面積を表す.
	一方,部材2は大きさ$P$の力を,点Bで水平右向きに受け,曲げモーメントと曲げ変形を生ずる.
	曲げモーメントは,点Cで最大値$M_C=Pl$をとることから,最大
	曲げ応力$\sigma_2$は
	\begin{equation}
		\sigma_2=\frac{M_C}{I_2}\times\frac{a_2}{2}
		\label{eqn:sig2}
	\end{equation}
	となる.
	ここに$I_2$は,部材2の断面2次モーメント$I_2=\frac{a_2^4}{12}$
	である.
		$\left| \sigma_1\right|=\sigma_2$とすれば,式(\ref{eqn:sig1})と式(\ref{eqn:sig2})より,
	\begin{equation}
		\frac{a_2}{a_1}=\sqrt{\frac{6l}{a_2}}=10
	\end{equation}
	が得られる.
\item
	部材1の伸び$\Delta l$は,熱による伸びと軸力による伸びの和として,
	\begin{equation}
		\Delta l= \alpha \Delta T l - \frac{Pl}{EA_1}
		\label{eqn:del_l}
	\end{equation}
	と与えられる.
	一方,部材2の点Bにおけるたわみ(水平変位)$\delta_B$は,右方向に
	\begin{equation}
		\delta_B=\frac{Pl^3}{3EI_2}
		\label{eqn:del_B}
	\end{equation}
	である.
	2つの部材は点Bで結合されているため,$\delta_B=\Delta l$
	でなければならず,
	\begin{equation}
		\left| \sigma_1 \right|
		=
		\frac{P}{A_1}
		=
		\frac{\alpha E\Delta T}{1+\frac{4a_1^2l^2}{a_2^4}}
		=\frac{9}{109}\alpha E \Delta T 
	\end{equation}
	となることが示される.よって,
	\begin{equation}
		\Delta T  < \frac{109}{9}\frac{\sigma_a}{\alpha E}		
	\end{equation}
	となる必要があることが言える.
\end{enumerate}
\end{document}
%%%%%%%%%%%%%%%%%%%%%%%%%%%%%%%%%%%%%%%%%%%%%%%%%%%%%%%%%%%%%%%

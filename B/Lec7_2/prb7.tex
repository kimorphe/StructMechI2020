\documentclass[10pt,a4j]{jarticle}
\usepackage{graphicx,wrapfig}
\setlength{\topmargin}{-1.5cm}
\setlength{\textwidth}{15.5cm}
\setlength{\textheight}{25.2cm}
\newlength{\minitwocolumn}
\setlength{\minitwocolumn}{0.5\textwidth}
\addtolength{\minitwocolumn}{-\columnsep}
%\addtolength{\baselineskip}{-0.1\baselineskip}
%
\def\Mmaru#1{{\ooalign{\hfil#1\/\hfil\crcr
\raise.167ex\hbox{\mathhexbox 20D}}}}
%
\begin{document}
\newcommand{\fat}[1]{\mbox{\boldmath $#1$}}
\newcommand{\D}{\partial}
\newcommand{\w}{\omega}
\newcommand{\ga}{\alpha}
\newcommand{\gb}{\beta}
\newcommand{\gx}{\xi}
\newcommand{\gz}{\zeta}
\newcommand{\vhat}[1]{\hat{\fat{#1}}}
\newcommand{\spc}{\vspace{0.7\baselineskip}}
\newcommand{\halfspc}{\vspace{0.3\baselineskip}}
\bibliographystyle{unsrt}
\pagestyle{empty}
\newcommand{\twofig}[2]
 {
   \begin{figure}[h]
     \begin{minipage}[t]{\minitwocolumn}
         \begin{center}   #1
         \end{center}
     \end{minipage}
         \hspace{\columnsep}
     \begin{minipage}[t]{\minitwocolumn}
         \begin{center} #2
         \end{center}
     \end{minipage}
   \end{figure}
 }
%%%%%%%%%%%%%%%%%%%%%%%%%%%%%%%%%
%\vspace*{\baselineskip}
\begin{flushright}
	2020/02/04
\end{flushright}
\begin{center}
	{\Large \bf 2019年度 構造力学I及び演習B (1月31日) 演習問題7 (差替え版)} \\
\end{center}
%%%%%%%%%%%%%%%%%%%%%%%%%%%%%%%%%%%%%%%%%%%%%%%%%%%%%%%%%%%%%%%%
\subsubsection*{問題}
図\ref{fig:fig1}のように, それぞれ,固定壁と床に固定された水平部材ABと
鉛直部材BCが,点Bにおいて互いにピンで結合されている.
2つの部材は同じ材料で作られており,ヤング率は$E$,熱膨張率は$\alpha$とする.
部材の断面形状は正方形で,その一辺の長さは部材1で$a_1$,部材2で$a_2$とする.
いま,部材1の温度だけが一様に$\Delta T(>0)$だけ上昇し,
部材2の長さと断面寸法の比が$\frac{l}{a_2}=\frac{50}{3}$であるとき,以下2つの問に答えよ.
\begin{enumerate}
\item
	部材1の温度上昇に伴い,部材1には軸応力が,部材2には曲げ応力が発生する.
	これら軸応力と曲げ応力の,絶対値としての最大値が一致するよう各々の
	部材断面寸法を決める.このとき,$a_2/a_1$を求めよ.
\item
	部材1と2の許容応力が,ともに$\sigma_a$であるとき,構造ABCが許容しうる最大の上昇温度
	(許される$\Delta T$の最大値)を求めよ.ただし,$a_2/a_1$には前問で求めた値を用いること.
\end{enumerate}
\begin{figure}[h]
	\begin{center}
	\includegraphics[width=0.60\linewidth]{fig1.eps} 
	\end{center}
	\caption{ピンで結合された2つの部材からなる構造ABC.} 
	\label{fig:fig1}
\end{figure}
\end{document}

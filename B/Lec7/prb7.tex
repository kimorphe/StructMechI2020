\documentclass[10pt,a4j]{jarticle}
\usepackage{graphicx,wrapfig}
\setlength{\topmargin}{-1.5cm}
%\setlength{\textwidth}{15.5cm}
\setlength{\textheight}{25.2cm}
\newlength{\minitwocolumn}
\setlength{\minitwocolumn}{0.5\textwidth}
\addtolength{\minitwocolumn}{-\columnsep}
%\addtolength{\baselineskip}{-0.1\baselineskip}
%
\def\Mmaru#1{{\ooalign{\hfil#1\/\hfil\crcr
\raise.167ex\hbox{\mathhexbox 20D}}}}
%
\begin{document}
\newcommand{\fat}[1]{\mbox{\boldmath $#1$}}
\newcommand{\D}{\partial}
\newcommand{\w}{\omega}
\newcommand{\ga}{\alpha}
\newcommand{\gb}{\beta}
\newcommand{\gx}{\xi}
\newcommand{\gz}{\zeta}
\newcommand{\vhat}[1]{\hat{\fat{#1}}}
\newcommand{\spc}{\vspace{0.7\baselineskip}}
\newcommand{\halfspc}{\vspace{0.3\baselineskip}}
\bibliographystyle{unsrt}
\pagestyle{empty}
\newcommand{\twofig}[2]
 {
   \begin{figure}[h]
     \begin{minipage}[t]{\minitwocolumn}
         \begin{center}   #1
         \end{center}
     \end{minipage}
         \hspace{\columnsep}
     \begin{minipage}[t]{\minitwocolumn}
         \begin{center} #2
         \end{center}
     \end{minipage}
   \end{figure}
 }
%%%%%%%%%%%%%%%%%%%%%%%%%%%%%%%%%
%\vspace*{\baselineskip}
\begin{center}
{\Large \bf 2020年度 構造力学I及び演習B 演習問題7 } \\
\end{center}
%%%%%%%%%%%%%%%%%%%%%%%%%%%%%%%%%%%%%%%%%%%%%%%%%%%%%%%%%%%%%%%%
\subsubsection*{問題}
図\ref{fig:fig2}に示すように, 3本の部材ab,cdとefが剛体壁に挟まれて
固定されている.両端の剛体壁1と3は完全に固定,中央の剛体壁2は水平
方向にのみ自由に移動可能で,いずれの剛体壁も回転や転倒は起こさないとする.
部材abは材料1で, 部材cdは材料2, 部材efは3によって作られており,
材料$i(=1,2,3)$のヤング率は$E_i$, 線膨張係数は$\alpha_i$であるとする.
いま,各部材の断面積と長さは,部材abとcdが$A$と$2l$,部材efが$A$と$l$で,
これら3本の部材の温度を一様に$\Delta T(>0)$だけ上昇させる.
このとき,中央の剛体壁が右向きに移動するための条件を導け.
\begin{figure}[h]
	\begin{center}
	\includegraphics[width=0.80\linewidth]{fig2.eps} 
	\end{center}
	\caption{剛体壁に連結された3本の棒部材.} 
	\label{fig:fig2}
\end{figure}
\end{document}

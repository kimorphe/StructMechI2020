\documentclass[10pt,a4j]{jarticle}
\usepackage{graphicx,wrapfig}
\setlength{\topmargin}{-1.5cm}
%\setlength{\textwidth}{15.5cm}
\setlength{\textheight}{25.2cm}
\newlength{\minitwocolumn}
\setlength{\minitwocolumn}{0.5\textwidth}
\addtolength{\minitwocolumn}{-\columnsep}
%\addtolength{\baselineskip}{-0.1\baselineskip}
%
\def\Mmaru#1{{\ooalign{\hfil#1\/\hfil\crcr
\raise.167ex\hbox{\mathhexbox 20D}}}}
%
\begin{document}
\newcommand{\fat}[1]{\mbox{\boldmath $#1$}}
\newcommand{\D}{\partial}
\newcommand{\w}{\omega}
\newcommand{\ga}{\alpha}
\newcommand{\gb}{\beta}
\newcommand{\gx}{\xi}
\newcommand{\gz}{\zeta}
\newcommand{\vhat}[1]{\hat{\fat{#1}}}
\newcommand{\spc}{\vspace{0.7\baselineskip}}
\newcommand{\halfspc}{\vspace{0.3\baselineskip}}
\bibliographystyle{unsrt}
\pagestyle{empty}
\newcommand{\twofig}[2]
 {
   \begin{figure}[h]
     \begin{minipage}[t]{\minitwocolumn}
         \begin{center}   #1
         \end{center}
     \end{minipage}
         \hspace{\columnsep}
     \begin{minipage}[t]{\minitwocolumn}
         \begin{center} #2
         \end{center}
     \end{minipage}
   \end{figure}
 }
%%%%%%%%%%%%%%%%%%%%%%%%%%%%%%%%%
%\vspace*{\baselineskip}
\begin{center}
{\Large \bf 2020年度 構造力学I及び演習B 演習問題7 解答} \\
\end{center}
\vspace{10mm}
%%%%%%%%%%%%%%%%%%%%%%%%%%%%%%%%%%%%%%%%%%%%%%%%%%%%%%%%%%%%%%%%
材料$i(=1,2,3)$に発生するひずみは$\varepsilon_i$は,熱ひずみを$\varepsilon_i^T$
と機械的ひずみ$\varepsilon_i^M$の和として
\begin{equation}
	\varepsilon_i =
	\varepsilon_i^T 
	+
	\varepsilon_i^M, \ \ (i=1,2,3) 
	\label{eqn:eps_sum}
\end{equation}
と表される.熱ひずみは,線膨張係数$\alpha_i$と$\Delta T$を用いて
\begin{equation}
	\varepsilon_i^T= \alpha_i \Delta T, \ \ (i=1,2,3)
\end{equation}
と表される.一方,機械的ひずみは,ヤング率$E_i$と応力$\sigma_i$により,
\begin{equation}
	\varepsilon_i^M= \frac{\sigma_i}{E_i}, \ \ (i=1,2,3)
	\label{eqn:Hooke}
\end{equation}
で与えられる.材料$i$に発生する軸力を$N_i=\sigma_i A_i$とすれば,
中央の剛体壁に作用する軸力は釣り合う必要があることから
\begin{equation}
	-N_1-N_2+N_3=0 
	\label{eqn:equib}
\end{equation}
である.ただし,$A_i$は部材$i$の断面積を表す.一方,材料$i$に発生する伸び$\Delta l_i$は,
部材$i$の長さを$l_i$とすれば,
\begin{equation}
	\Delta l_i = \varepsilon_i l_i, \ \ (i=1,2,3) 
	\label{eqn:Del_l}
\end{equation}
で,適合条件:
\begin{equation}
	\Delta l_1=\Delta l_2 \ \ \Delta l_1+\Delta l_3=0
	\label{eqn:compati}
\end{equation}
を満足しなければならない.よって,式(\ref{eqn:Del_l})と(\ref{eqn:compati})より,
\begin{equation}
	\varepsilon_1=\varepsilon_2, \ \ 2\varepsilon_1+\varepsilon_3=0
\end{equation}
が言える.ここで,式(\ref{eqn:eps_sum})$\sim$式(\ref{eqn:Hooke})より,
\begin{equation}
	N_i=\sigma_i A_i=E_iA_i\left( \varepsilon_i-\alpha_i \Delta T\right), \ \ (i=1,2,3)
	\label{eqn:axial_forces}
\end{equation}
だから,式(\ref{eqn:axial_forces})と式(\ref{eqn:compati})を用いて,式(\ref{eqn:equib})から
$\varepsilon_2$と$\varepsilon_3$を消去すれば,
\begin{equation}
	\varepsilon_1=\frac{\alpha_1E_1+\alpha_2E_2-2\alpha_3 E_3}{E_1+E_2+4E_3} \Delta T
\end{equation}
となることが示される.
$\varepsilon_1>0$のとき,剛体壁2は右方向に移動する.
よって,剛体壁2が右方向に移動するための条件は,$E_i>0$だから
\begin{equation}
	\alpha_1E_1+\alpha_2E_2-2\alpha_3 E_3 >0
\end{equation}
と表すことができる.
\end{document}
%%%%%%%%%%%%%%%%%%%%%%%%%%%%%%%%%%%%%%%%%%%%%%%%%%%%%%%%%%%%%%%

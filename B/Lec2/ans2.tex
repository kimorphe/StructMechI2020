\documentclass[10pt,a4j]{jarticle}
\usepackage{graphicx,wrapfig}
\setlength{\topmargin}{-1.5cm}
%\setlength{\textwidth}{15.5cm}
\setlength{\textheight}{25.2cm}
\newlength{\minitwocolumn}
\setlength{\minitwocolumn}{0.5\textwidth}
\addtolength{\minitwocolumn}{-\columnsep}
%\addtolength{\baselineskip}{-0.1\baselineskip}
%
\def\Mmaru#1{{\ooalign{\hfil#1\/\hfil\crcr
\raise.167ex\hbox{\mathhexbox 20D}}}}
%
\begin{document}
\newcommand{\fat}[1]{\mbox{\boldmath $#1$}}
\newcommand{\D}{\partial}
\newcommand{\w}{\omega}
\newcommand{\ga}{\alpha}
\newcommand{\gb}{\beta}
\newcommand{\gx}{\xi}
\newcommand{\gz}{\zeta}
\newcommand{\vhat}[1]{\hat{\fat{#1}}}
\newcommand{\spc}{\vspace{0.7\baselineskip}}
\newcommand{\halfspc}{\vspace{0.3\baselineskip}}
\bibliographystyle{unsrt}
\pagestyle{empty}
\newcommand{\twofig}[2]
 {
   \begin{figure}[h]
     \begin{minipage}[t]{\minitwocolumn}
         \begin{center}   #1
         \end{center}
     \end{minipage}
         \hspace{\columnsep}
     \begin{minipage}[t]{\minitwocolumn}
         \begin{center} #2
         \end{center}
     \end{minipage}
   \end{figure}
 }
%%%%%%%%%%%%%%%%%%%%%%%%%%%%%%%%%
%\vspace*{\baselineskip}
\begin{center}
{\Large \bf 2020年度 構造力学I及び演習B 演習問題2 解答} \\
\end{center}
%%%%%%%%%%%%%%%%%%%%%%%%%%%%%%%%%%%%%%%%%%%%%%%%%%%%%%%%%%%%
%%%%%%%%%%%%%%%%%%%%%%%%%%%%%%%%%%%%%%%%%%%%%%%%%%%%
%%%%%%%%%%%%%%%%%%%%%%%%%%%%%%%%%%%%%%%%%%%%%%%%%%%%
\subsubsection*{問題(1)}
支点反力の正方向を図\ref{fig:fig1_1}-(a)にあるように定める.はり全体の力とモーメントの
釣り合いから,
\begin{eqnarray}
	R_A &=& \frac{4}{3}P \\
	H_A &=&0 \\
	R_D &=& -\frac{P}{3}
	\label{eqn:Rs_simple}
\end{eqnarray}
となる.ここで,a-a'の位置で梁を切断して自由物体図を描くと,図\ref{fig:fig1_1}-(b)のようになる.
この図に基づき力とモーメントの釣り合い条件を立てると,
\begin{eqnarray}
	Q(s_1) &=& -R_D= \frac{P}{3} \\
	M(s_1) &=& R_D s_1=-\frac{P}{3}s_1
	\label{eqn:eqlbl}
\end{eqnarray}
となる.ただし,$s_1$は図\ref{fig:fig1_1}-(b)に示すように,点Dから切断位置までの距離を表す.
次に,b-b'で梁の切断を行うと,自由物体図は図\ref{fig:fig1_1}-(c)のようになる.この場合,
釣り合い条件から断面力が
\begin{eqnarray}
	Q(s_2) &=& -2P-R_D=-\frac{5}{3}P \\ 
	M(s_2) &=& R_D\left(s_2+\frac{l}{3}\right) +2Ps_2 = \frac{P}{3} \left( 5s_2-\frac{l}{3}\right)
	\label{eqn:eqlbl}
\end{eqnarray}
と求められる.最後に,c-c'の断面における断面力を,図\ref{fig:fig1_1}-(d)を参照して求めると,
\begin{eqnarray}
	Q(x) &=&R_A=\frac{4}{3}P \\ 
	M(x) &=&R_Ax = \frac{4}{3}Px 
	\label{eqn:eqlbl}
\end{eqnarray}
となる.以上の結果を断面力図として示すと,図\ref{fig:fig1_1}-(e)と(f)のようになる.
%--------------------
\begin{figure}[h]
	\begin{center}
	\includegraphics[width=0.50\linewidth]{fig1ans1.eps} 
	\end{center}
	\caption{解答に用いた自由物体図(a)$\sim$(d)と断面力図(e)と(f)(問題1).} 
	\label{fig:fig1_1}
\end{figure}
%--------------------
%%%%%%%%%%%%%%%%%%%%%%%%%%%%%%%%%%%%%%%%%%%%%%%%%%%%
%%%%%%%%%%%%%%%%%%%%%%%%%%%%%%%%%%%%%%%%%%%%%%%%%%%%
\subsubsection*{問題(2)}
支点反力の正方向を図\ref{fig:fig2_1}-(a)のように定める.
鉛直力と,点Aに関するモーメントのつり合い式を立てれば,
これらの支点反力は
\begin{equation}
	H_A=0, \ \
	R_A=\frac{q_0l}{4}+\frac{q_0l}{2}=\frac{3}{4}q_0l, \ \ 
	M_A=- \frac{q_0l}{4}\times \frac{l}{3} -\frac{q_0l}{2}\times \frac{3l}{4} =-\frac{11}{24}q_0l^2
\end{equation}
と求められる.
次に,図\ref{fig:fig2_1}-(a)のa-a'および b-b'で梁を仮想的に
切断し, その結果得られる部分構造について自由物体図を描けば, 
同図(b)と(c)のようになる.また,これらの自由物体図に現れる分布荷重を,
それと等価な集中荷重に置き換えれば,図\ref{fig:fig2_1}-(b')と(c')のよう
になる.この結果を踏まえて釣り合い条件式を立てれば,断面力分布が
区間毎に以下のように求められる.
\begin{itemize}
\item
	区間AB$\left( 0<x<\frac{l}{2}\right)$:\\
	\begin{equation}
		Q(x)=R_A-\frac{x}{2}\times \frac{2q_0}{l}x 
	\end{equation}
	\begin{equation}
		M=M_A+R_Ax-
		\frac{x}{2}\times
		\frac{2q_0}{l}x \times\frac{x}{3}
	\end{equation}
	だから,
	\begin{equation}
		Q(x) =
		q_0l \left\{ 
			\frac{3}{4} -\left( \frac{x}{l} \right)^2
		\right\}
	\end{equation}
	\begin{equation}
		M(x) =
		-\frac{q_0l^2}{24} \left\{ 
			8\left( \frac{x}{l} \right)^3
			-18 \left(\frac{x}{l}\right) 
			+11
		\right\}
	\end{equation}
\item
	区間BC$\left( 0<y<\frac{l}{2}\right)$:\\
	\begin{equation}
		Q(s_)=q_0s
	\end{equation}
	\begin{equation}
		M(s)=-\frac{1}{2}q_0s^2 
	\end{equation}
\end{itemize}
これらの結果を断面力図として示すと,図\ref{fig:fig2_2}のようになる.
%--------------------
\begin{figure}[h]
	\begin{center}
	\includegraphics[width=0.6\linewidth]{fig2ans1.eps} 
	\end{center}
	\caption{問題(2)の解答のための自由物体図.} 
	\label{fig:fig2_1}
\end{figure}
%--------------------
\begin{figure}[h]
	\begin{center}
	\includegraphics[width=0.5\linewidth]{fig2ans2.eps} 
	\end{center}
	\caption{曲げモーメント図及びせん断力図(問題2).} 
	\label{fig:fig2_2}
\end{figure}
%%%%%%%%%%%%%%%%%%%%%%%%%%%%%%%%%%%%%%%%%%%%%%%%%%%%
%%%%%%%%%%%%%%%%%%%%%%%%%%%%%%%%%%%%%%%%%%%%%%%%%%%%
\end{document}

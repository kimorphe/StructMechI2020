\documentclass[10pt,a4j]{jarticle}
\usepackage{graphicx,wrapfig}
\setlength{\topmargin}{-1.5cm}
%\setlength{\textwidth}{15.5cm}
\setlength{\textheight}{25.2cm}
\newlength{\minitwocolumn}
\setlength{\minitwocolumn}{0.5\textwidth}
\addtolength{\minitwocolumn}{-\columnsep}
%\addtolength{\baselineskip}{-0.1\baselineskip}
%
\def\Mmaru#1{{\ooalign{\hfil#1\/\hfil\crcr
\raise.167ex\hbox{\mathhexbox 20D}}}}
%
\begin{document}
\newcommand{\fat}[1]{\mbox{\boldmath $#1$}}
\newcommand{\D}{\partial}
\newcommand{\w}{\omega}
\newcommand{\ga}{\alpha}
\newcommand{\gb}{\beta}
\newcommand{\gx}{\xi}
\newcommand{\gz}{\zeta}
\newcommand{\vhat}[1]{\hat{\fat{#1}}}
\newcommand{\spc}{\vspace{0.7\baselineskip}}
\newcommand{\halfspc}{\vspace{0.3\baselineskip}}
\bibliographystyle{unsrt}
\pagestyle{empty}
\newcommand{\twofig}[2]
 {
   \begin{figure}[h]
     \begin{minipage}[t]{\minitwocolumn}
         \begin{center}   #1
         \end{center}
     \end{minipage}
         \hspace{\columnsep}
     \begin{minipage}[t]{\minitwocolumn}
         \begin{center} #2
         \end{center}
     \end{minipage}
   \end{figure}
 }
%%%%%%%%%%%%%%%%%%%%%%%%%%%%%%%%%
%\vspace*{\baselineskip}
\begin{center}
{\Large \bf 2020年度 構造力学I及び演習B 演習問題5 解答} \\
\end{center}
%%%%%%%%%%%%%%%%%%%%%%%%%%%%%%%%%%%%%%%%%%%%%%%%%%%%%%%%%%%%%%%%
%%%%%%%%%%%%%%%%%%%%%%%%%%%%%%%%%%%%%%%%%%%%%%%%%%%%%%%%%%%%%%%%%%%%%%%%%%%%%%%%%%%%%%%%%%
\subsubsection*{問題1.}
\begin{itemize}
\item
	{\rm 反力:}\\
	支点反力の正方向を図\ref{fig:fig1}-(a)のようにとる.これらの支点反力は,
	力とモーメントの釣り合い式:
	\begin{eqnarray*}
		水平 &:& H_A=0 \\
		鉛直 &:& R_A+R_E-q_0l=0 \\
		モーメント(基準点E) &:& -R_A \times \frac{3l}{2}-q_0l\times \frac{l}{2}=0
	\end{eqnarray*}
	より,
	\[
		H_A=0, \ \ 
		R_A=-\frac{1}{3}q_0l, \ \ R_E=\frac{4}{3}q_0l
	\]	
	と求められる.\\

	以上を元に断面力計算を行うにあたり,部材AB,BC,CDおよびECをそれぞれ部材1,2,3および4と呼ぶ.
	また, 部材$i,\,(i=1\sim 4)$の軸力,せん断力,曲げモーメントをそれぞれ$N_i, \, Q_i,\, M_i$と表す.
	これらの断面力は,図\ref{fig:fig1}-(a)に示したa-a', b-b', c-c'とd-d'において
	構造を仮想的に切断したときの自由物体図について釣り合い式を立てることで決定できる.
\item
	{\rm 部材1:} \\
	図\ref{fig:fig1}-(b)の自由物体図について, 部材軸および部材軸直角方向の力と, 
	モーメントのつり合い式を立てればよい.その結果として,
	\begin{eqnarray}
		N_1(x_1) & =& -\frac{R_A}{\sqrt{2}}=\frac{q_0l}{3\sqrt{2}} \\
		Q_1(x_1) & =& \frac{R_A}{\sqrt{2}}=-\frac{q_0l}{3\sqrt{2}} \\
		M_1(x_1) & =& \frac{R_A}{\sqrt{2}}x_1 =-\frac{q_0l}{3\sqrt{2}}x_1
	\end{eqnarray}
	が得られる.
\item
	{\rm 部材2:} \\
	部材2の断面力計算には図\ref{fig:fig1}-(c)の自由物体図を用いることができる.この自由物体図に対する
	釣り合い条件から, 断面力は
	\begin{eqnarray}
		N_2(x_2) 
			&=& 0\\
		Q_2(x_2) 
			&=&
			R_A= -\frac{q_0l}{3} \\
		M_2(x_2) 
			&=&
			=-R_A(l+x_2)
			=-\frac{q_0l}{3}(x_2+l)
	\end{eqnarray}
	と求められる.
\item
	{\rm 部材3:} \\
	部材3の断面力計算には図\ref{fig:fig1}-(d)の自由物体図を用いることができる.この自由物体図に対する
	釣り合い条件から, 断面力は次のように求められる.
	\begin{eqnarray}
		N_3(s_3) 
			&=&  0 \\
		Q_3(s_3) 
			&=&
			q_0s_3 \\
		M_3(s_3) 
			&=&
			-\frac{q_0}{2}(s_3)^2
	\end{eqnarray}
\item
	{\rm 部材4:} \\
	図\ref{fig:fig1}-(e)に示す自由物体図を用い,部材4の断面力が次のように求められる.
	\begin{eqnarray}
		N_4(x_4) 
			&=&  -R_E=-\frac{4}{3}q_0l \\
		Q_4(x_4) 
			&=&  0 \\
		M_4(x_4) 
			&=&  0 
	\end{eqnarray}
\end{itemize}
以上の結果を断面力図として図示すれば,図\ref{fig:fig2}のようになる.
\begin{figure}[h]
	\begin{center}
	\includegraphics[width=0.7\linewidth]{fig1ans.eps} 
	\end{center}
	\caption{骨組み構造ABCDEの断面力計算に用いた自由物体図.} 
	\label{fig:fig1}
\end{figure}
\begin{figure}[h]
	\begin{center}
	\includegraphics[width=0.5\linewidth]{fig2_2ans.eps} 
	\end{center}
	\caption{骨組み構造ABCDEの断面力図.} 
	\label{fig:fig2}
\end{figure}
%%%%%%%%%%%%%%%%%%%%%%%%%%%%%%%%%%%%%%%%%%%%%%%%%%%%%%%%%%%%%%%%%%%%%%%%%%%%%%%%%%%%%%%%%%
\subsubsection*{問題2.}
支点反力の正方向を図\ref{fig:fig3}-(a)のようにとれば,
トラス全体のつり合い条件:
\begin{eqnarray}
	&& P-H_A=0  \\
	&& R_A+R_A'=0 \\  
	&& R_A'\times 3\sqrt{2}l - P \times \frac{3}{\sqrt{2}}l=0 
\end{eqnarray}
より
\begin{equation}
	H_A=P, \ \ 
	R_A=-\frac{P}{2}, \ \ 
	R_A'=\frac{P}{2}
\end{equation}
となる.\\

次に,a-a'でトラス構造を仮想的に切断し,上側の部分構造について自由物体図を描くと
図\ref{fig:fig3}-(b)のようになる.ここで,点Bに関するモーメントのつり合い式を
立てれば,$N_{EF}$が
\begin{equation}
	-N_{EF}l-P\sqrt{2}l + R_A'\frac{5}{\sqrt{2}}l =0
	\; \Rightarrow \;
	N_{EF}=\frac{\sqrt{2}}{4}P
\end{equation}
と求まる.次に,点Dに関するモーメントのつり合いより,
\begin{equation}
	-N_{EB}\times \sqrt{2}l -N_{EF}l +R_A'\times \frac{3}{\sqrt{2}}l=0
	\; \Rightarrow \;
	N_{EB}=\frac{P}{2}
\end{equation}
が得られる.\\

続いて,b-b'の位置で構造を切断して得られる部分構造について自由物体図を描くと,
図\ref{fig:fig3}-(c))のようになる.
そこで,点C'に関するモーメントの釣り合い式を立てると,
\begin{equation}
	N_{EF'}l-P\times \frac{l}{\sqrt{2}}-H_A\times \sqrt{2}l-R_A\times 2\sqrt{2}l =0 \;
	\Rightarrow \; 
	N_{EF'}=\frac{P}{\sqrt{2}}
\end{equation}
と$N_{EF'}$が求められる.

最後に, 図\ref{fig:fig3}-(d)を参照して節点Eに関する力の釣り合いを考えると,
EC方向の力の釣り合いからは
\begin{equation}
	N_{EC}-N_{EF'}+\frac{N_{EB}}{\sqrt{2}}=0
	\; \Rightarrow \; N_{EC}=\frac{P}{2\sqrt{2}}
\end{equation}
が,EC'方向の釣り合い条件からは
\begin{equation}
	N_{EC'}-N_{EF}-\frac{N_{EB}}{\sqrt{2}}=0
	\; \Rightarrow \;
	N_{EC'}=\frac{P}{\sqrt{2}}
\end{equation}
が得られる.
%--------------------
\begin{figure}[h]
	\begin{center}
	\includegraphics[width=0.7\linewidth]{fig3ans.eps} 
	\end{center}
	\caption{トラス構造の軸力計算に用いた部分構造と自由物体図.} 
	\label{fig:fig3}
\end{figure}
\end{document}

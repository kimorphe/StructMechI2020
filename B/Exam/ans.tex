\documentclass[10pt,a4j]{jarticle}
\usepackage{graphicx,wrapfig}
%,showkeys}
\setlength{\topmargin}{-1.5cm}
%\setlength{\textwidth}{15.5cm}
\setlength{\textheight}{25.2cm}
\newlength{\minitwocolumn}
\setlength{\minitwocolumn}{0.5\textwidth}
\addtolength{\minitwocolumn}{-\columnsep}
%\addtolength{\baselineskip}{-0.1\baselineskip}
%
\def\Mmaru#1{{\ooalign{\hfil#1\/\hfil\crcr
\raise.167ex\hbox{\mathhexbox 20D}}}}
%
\begin{document}
\newcommand{\fat}[1]{\mbox{\boldmath $#1$}}
\newcommand{\D}{\partial}
\newcommand{\w}{\omega}
\newcommand{\ga}{\alpha}
\newcommand{\gb}{\beta}
\newcommand{\gx}{\xi}
\newcommand{\gz}{\zeta}
\newcommand{\vhat}[1]{\hat{\fat{#1}}}
\newcommand{\spc}{\vspace{0.7\baselineskip}}
\newcommand{\halfspc}{\vspace{0.3\baselineskip}}
\bibliographystyle{unsrt}
\pagestyle{empty}
\newcommand{\twofig}[2]
 {
   \begin{figure}[h]
     \begin{minipage}[t]{\minitwocolumn}
         \begin{center}   #1
         \end{center}
     \end{minipage}
         \hspace{\columnsep}
     \begin{minipage}[t]{\minitwocolumn}
         \begin{center} #2
         \end{center}
     \end{minipage}
   \end{figure}
 }
%%%%%%%%%%%%%%%%%%%%%%%%%%%%%%%%%
%\vspace*{\baselineskip}
\begin{center}
{\Large \bf 2020年度 構造力学I及び演習B (2月5日) 期末試験 解答} \\
\end{center}
%%%%%%%%%%%%%%%%%%%%%%%%%%%%%%%%%%%%%%%%%%%%%%%%%%%%%%%%%%%%%%%%
%%%%%%%%%%%%%%%%%%%%%%%%%%%%%%%%%%%%%%%%%%%%%%%%%%%%%%%%%%%%%%%%%%%%%%%%%%%%%%%%%%%%%%%%%%
\subsubsection*{問題1.}
\begin{enumerate}
\item
	支点反力の数が6つであるのに対し,釣り合い条件式は3つであるため.
\item
	$q(x)=\delta\left(x-\frac{2l}{3}\right)$
\item
	梁(a)のたわみ$v(x)$は,たわみの方程式
	\begin{equation}
		EIv''''=F\delta\left(x-\frac{2l}{3}\right)
	\end{equation}
	を,両端固定支持の条件:
	\begin{equation}
		v(0)=0, v'(0)=0, \ \
		v(l)=0, v'(l)=0
	\end{equation}
の元で解くことで,次のように求められる.
\begin{equation}
	v(x)=
	\frac{Fl^3}{6EI}
	\left\{ \left<\xi -\frac{2}{3}\right>^3-\frac{7}{27}\xi^3+\frac{2}{9}2\xi^2 \right\}, 
	\ \ \left(\xi=\frac{x}{l}\right)
	\label{eqn:vx}
\end{equation}
ただし, $\xi=x/l$で$x$はAを原点として右向きを正とする座標を表す.
\item
	支点反力の正方向を図\ref{fig:fig1}-(a)のように定める.
	これらは,
	\begin{equation}
		Q(0)=\frac{7}{27}F, \ \ M(0)=-\frac{2}{27}Fl, \ \ 
		Q(l)=\frac{20}{27}F, \ \ M(l)=-\frac{4}{27}Fl
	\end{equation}
	より
	\begin{equation}
		V_A=\frac{7}{27}F, \ \ M_A=-\frac{2}{27}Fl, \ \ 
		V_D=\frac{20}{27}F, \ \ M_D=-\frac{4}{27}Fl
	\end{equation}
	で,水平反力は外力が作用しないため$H_A=H_D=0$となる.
\item
	\begin{equation}
		Q(\xi)=-\frac{F}{3}
		\left\{ 
			3H\left( \xi-\frac{2}{3}\right) -\frac{7}{9}
		\right\}
	\end{equation}
	より,せん断力図は図\ref{fig:fig1}-(b)のようになる.
\item
	\begin{equation}
		M(\xi)=-\frac{Fl}{3}
		\left\{ 
			3\left< \xi-\frac{2}{3}\right> -\frac{7}{9}\xi+\frac{2}{9}
		\right\}
	\end{equation}
	より,曲げモーメント図は図\ref{fig:fig1}-(c)のようになる.
\item
	点Cに加えられた集中荷重によって生じる曲げモーメントは図\ref{fig:fig1}-(c)
	に示した通りである.一方,点Bに加えられた集中荷重によって生じる曲げモーメント
	は,図\ref{fig:fig1}-(c)のグラフを,部材中央($x=l/2$)で左右反転させた
	ものとして与えられる.よって,求めるべき曲げモーメント図は,それらの重ね合わせ
	により,左右対象なグラフとして図\ref{fig:fig2}-(b)のようになる.
\end{enumerate}
\begin{figure}[h]
	\begin{center}
	\includegraphics[width=0.6\linewidth]{fig1ans.eps} 
	\end{center}
	\caption{支点反力の正方向と断面力図.}
	\label{fig:fig1}
\end{figure}
\begin{figure}[h]
	\begin{center}
	\includegraphics[width=0.6\linewidth]{fig1ans2.eps} 
	\end{center}
	\caption{曲げモーメント図.} 
	\label{fig:fig2}
\end{figure}
%%%%%%%%%%%%%%%%%%%%%%%%%%%%%%%%%%%%%%%%%%%%%%%%%%%%%%%%%%%%%%%%%%%%%%%%%%%%%%%%%%%%%%%%%%
\subsubsection*{問題2.}
\begin{enumerate}
\item
図\ref{fig:fig3}に示すように支点反力の正方向を定めると,
構造系全体のつり合い条件より,これらの反力が次のように求められる.
\begin{equation}
	R_A=-\frac{q_0l}{3}, \ \ 
	H_D=q_0l, \ \ 
	R_D=\frac{q_0l}{3}
\end{equation}
\item
図\ref{fig:fig3}-(a)に示すa-a'断面で構造を切断して,部材1に関する自由物体図を描くと
同図(b)のようになる.これに基づき,つり合い条件から断面力分布を求めれば,
\begin{eqnarray}
	N_1 &=&-R_A=\frac{q_0l}{3} \\
	Q_1 &=& 0 \\ 
	M_1 &=&-\frac{1}{2} q_0x_1^2 
\end{eqnarray}
となる.ここに$x_1$は図\ref{fig:fig3}-(b)に示す座標を意味する.
この結果を断面力図として図示すれば,図\ref{fig:fig4}の区間ABの
部分に示したようになる.
\item
図\ref{fig:fig3}-(a)に示すb-b'断面で構造を切断し, 部材2に関する部分構造の自由物体図を描くと,
同図(c)のようになる.これを参照してつり合い条件から断面力分布を求めれば,
\begin{eqnarray}
	N_2 &=& -q_0l \\
	Q_2 &=& R_A= -\frac{1}{3}q_0l \\
	M_2 &=& R_A\times x_1 -\frac{1}{2}q_0l^2 = -\frac{1}{3}q_0lx_2-\frac{1}{2}q_0l^2
\end{eqnarray}
となる.以上を断面力図として示せば,図\ref{fig:fig4}の区間BCの部分に示したようになる.
\item
図\ref{fig:fig3}-(a)に示すc-c'断面で構造を切断し, 部材3に関する自由物体図を描くと
同図(d)のようになる.これを参照してつり合い条件から断面力分布を求めれば,
\begin{eqnarray}
	N_3 &=&-\frac{1}{3}q_0l\cos \alpha - q_0l \cos \beta = \frac{\sqrt{5}}{3}q_0l \\
	Q_3 &=&-\frac{1}{3}q_0l\sin\alpha +q_0l\sin\beta = \frac{\sqrt{5}}{3}q_0l \\
	M_3 &=&-q_0ls_3\cos\alpha+\frac{1}{3}q_0ls_3\sin\alpha= -\frac{\sqrt{5}}{3}q_0ls_s 
\end{eqnarray}
となる.以上を断面力図として示せば,図\ref{fig:fig3}の区間CDの部分に示したようになる.
なお,上の計算では
	\begin{equation}
		\cos\alpha=\sin\beta = \frac{2}{\sqrt{5}}, \ \ 
		\sin\alpha=\cos\beta = \frac{1}{\sqrt{5}}
	\end{equation}
を用いた.
\end{enumerate}
%--------------------
\begin{figure}[h]
	\begin{center}
	\includegraphics[width=0.8\linewidth]{fig2ans.eps} 
	\end{center}
	\caption{支点反力の正方向と断面力計算のための自由物体図.} 
	\label{fig:fig3}
\end{figure}
\begin{figure}[h]
	\begin{center}
	\includegraphics[width=1.0\linewidth]{fig3ans.eps} 
	\end{center}
	\caption{骨組み構造の断面力図.} 
	\label{fig:fig4}
\end{figure}
\end{document}

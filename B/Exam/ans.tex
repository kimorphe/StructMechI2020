\documentclass[10pt,a4j]{jarticle}
\usepackage{graphicx,wrapfig}
\setlength{\topmargin}{-1.5cm}
\setlength{\textwidth}{15.5cm}
\setlength{\textheight}{25.2cm}
\newlength{\minitwocolumn}
\setlength{\minitwocolumn}{0.5\textwidth}
\addtolength{\minitwocolumn}{-\columnsep}
%\addtolength{\baselineskip}{-0.1\baselineskip}
%
\def\Mmaru#1{{\ooalign{\hfil#1\/\hfil\crcr
\raise.167ex\hbox{\mathhexbox 20D}}}}
%
\begin{document}
\newcommand{\fat}[1]{\mbox{\boldmath $#1$}}
\newcommand{\D}{\partial}
\newcommand{\w}{\omega}
\newcommand{\ga}{\alpha}
\newcommand{\gb}{\beta}
\newcommand{\gx}{\xi}
\newcommand{\gz}{\zeta}
\newcommand{\vhat}[1]{\hat{\fat{#1}}}
\newcommand{\spc}{\vspace{0.7\baselineskip}}
\newcommand{\halfspc}{\vspace{0.3\baselineskip}}
\bibliographystyle{unsrt}
\pagestyle{empty}
\newcommand{\twofig}[2]
 {
   \begin{figure}[h]
     \begin{minipage}[t]{\minitwocolumn}
         \begin{center}   #1
         \end{center}
     \end{minipage}
         \hspace{\columnsep}
     \begin{minipage}[t]{\minitwocolumn}
         \begin{center} #2
         \end{center}
     \end{minipage}
   \end{figure}
 }
%%%%%%%%%%%%%%%%%%%%%%%%%%%%%%%%%
%\vspace*{\baselineskip}
\begin{center}
{\Large \bf 2019年度 構造力学I及び演習B (2月7日) 期末試験 解答} \\
\end{center}
%%%%%%%%%%%%%%%%%%%%%%%%%%%%%%%%%%%%%%%%%%%%%%%%%%%%%%%%%%%%%%%%
%%%%%%%%%%%%%%%%%%%%%%%%%%%%%%%%%%%%%%%%%%%%%%%%%%%%%%%%%%%%%%%%%%%%%%%%%%%%%%%%%%%%%%%%%%
\subsubsection*{問題1.}
\begin{enumerate}
\item
	支点反力の正方向を図\ref{fig:fig0}-(a)のように定める.これらの反力は,力とモーメントの釣り合い条件より
	\begin{equation}
		R_A=F, M_A=-\frac{2}{3}Fl 
		\label{eqn:re_forces}
	\end{equation}
	となる.
\item
	図\ref{fig:fig0}の(b)と(c)に示す通り.
\item
	梁(a)のたわみ$v_a(x)$は,たわみの方程式$EIv_a''''=F\delta\left(x-\frac{2l}{3}\right)$を,
単純支持条件:
\[
	v_a(0)=v_a'(0)=0, \ \ -EIv_a''(l)=0, \ \ (-EIv_a'')'(l)=0
\]
の元で解くことで,次のように求められる.
\begin{equation}
	v_a(x)=
	\frac{Fl^3}{6EI}
	\left\{ \left<\xi -\frac{2}{3}\right>^3-\xi^3+2\xi^2 \right\}, 
	\ \ \left(\xi=\frac{x}{l}\right)
	\label{eqn:va}
\end{equation}
ただし, $\xi=x/l$で$x$はAを原点として右向きを正とする座標を表す.
\item
荷重項を$q(x)=F\delta(x-a)$として,上の問題と同様にしてたわみを求めれば良い.
このときのたわみを$v_b(x)$とすれば,
\begin{equation}
	v_b(x)=
	\frac{Fl^3}{6EI}
	\left\{ \left<\xi -\alpha\right>^3-\xi^3+3\alpha\xi^2 \right\}, 
	\ \ \left(\xi=\frac{x}{l},\, \alpha=\frac{a}{l}\right)
	\label{eqn:vb}
\end{equation}
となる.
\item
梁(c)の支点反力の正方向を図\ref{fig:fig1}-(a)のように定める.
点Dにおけるたわみのうち,外力$F$に起因したものを$v_F$,反力$R_D$によるものを$v_R$とする.
式(\ref{eqn:vb})より,
\begin{equation}
	v_F=\frac{14}{81}\frac{Fl^3}{EI}
	\label{eqn:vF}
\end{equation}
\begin{equation}
	v_R=-\frac{1}{3}\frac{R_Dl^3}{EI}
	\label{eqn:vR}
\end{equation}
と与えられ,また,$v_F+v_R=0$であることから,
\begin{equation}
	R_D=\frac{14}{27}F
	\label{eqn:RB_c}
\end{equation}
と点Dの鉛直反力が決まる.これを踏まえ,梁全体の釣り合い条件を使えば,固定端Aにおける反力が
\begin{equation}
	R_A=\frac{14}{27}F, \ \ M_A=-\frac{4}{27}Fl
	\label{eqn:forces_A_b}
\end{equation}
と求められる.
\item
図\ref{fig:fig1}-(a)に示す通り.
\item
梁(d)の支点反力の正方向を図\ref{fig:fig1}-(b)のように定める.
点Bにおけるたわみのうち,外力$F$に起因したものを$v_F$,反力$R_B$によるものを$v_R$とする.
式(\ref{eqn:vb})より,
\begin{equation}
	v_F=\frac{5}{6\cdot 27}\frac{Fl^3}{EI}
	\label{eqn:vF_d}
\end{equation}
\begin{equation}
	v_R=-\frac{1}{3\cdot 27}\frac{R_Bl^3}{EI}
	\label{eqn:vR_d}
\end{equation}
と与えられ,また,$v_F+v_R=0$であることから,
\begin{equation}
	R_B=\frac{5}{2}F
	\label{eqn:RB_d}
\end{equation}
と点Dの鉛直反力が決まる.これを踏まえ,梁全体の釣り合い条件を使えば,固定端Aにおける反力が
\begin{equation}
	R_A=-\frac{3}{2}F, \ \ M_A=\frac{1}{6}Fl
	\label{eqn:forces_A_d}
\end{equation}
と求められる.
\item
図\ref{fig:fig1}-(b)に示す通り.
\end{enumerate}
\begin{figure}[h]
	\begin{center}
	\includegraphics[width=0.5\linewidth]{fig0ans.eps} 
	\end{center}
	\caption{支点反力の正方向と曲げモーメント図.}
	\label{fig:fig0}
\end{figure}
\begin{figure}[h]
	\begin{center}
	\includegraphics[width=0.8\linewidth]{fig1ans.eps} 
	\end{center}
	\caption{曲げモーメント図.} 
	\label{fig:fig1}
\end{figure}
%%%%%%%%%%%%%%%%%%%%%%%%%%%%%%%%%%%%%%%%%%%%%%%%%%%%%%%%%%%%%%%%%%%%%%%%%%%%%%%%%%%%%%%%%%
\subsubsection*{問題2.}
\begin{enumerate}
\item
図\ref{fig:fig2}に示すように支点反力の正方向を定めると,
構造系全体のつり合い条件より,これらの反力が次のように求められる.
\begin{equation}
	R_A=-\frac{q_0l}{2}, \ \ 
	H_D=q_0l, \ \ 
	R_D=\frac{q_0l}{2}
\end{equation}
\item
図\ref{fig:fig2}-(a)に示すa-a'断面で構造を切断して,部材1に関する自由物体図を描くと
同図(b)のようになる.これに基づき,つり合い条件から断面力分布を求めれば,
\begin{eqnarray}
	N_1 &=&-R_A=\frac{q_0l}{2} \\
	Q_1 &=& -q_0x_1 \\ 
	M_1 &=&-\frac{1}{2} q_0x_1^2 
\end{eqnarray}
となる.ここに$x_1$は図\ref{fig:fig2}-(b)に示す座標を意味する.
この結果を断面力図として図示すれば,図\ref{fig:fig3}の区間ABの
部分に示したようになる.
\item
図\ref{fig:fig2}-(a)に示すb-b'断面で構造を切断し, 部材2に関する部分構造の自由物体図を描くと,
同図(c)のようになる.これを参照してつり合い条件から断面力分布を求めれば,
\begin{eqnarray}
	N_2 &=& \frac{R_D-H_A}{\sqrt{2}}=-\frac{q_0l}{2\sqrt{2}} \\
	Q_2 &=&-\frac{1}{\sqrt{2}}\left(q_0l+\frac{q_0l}{2}\right) = -\frac{3}{2\sqrt{2}}q_0l \\
	M_2 &=& R_D\times \frac{s_2}{\sqrt{2}} -H_A\times \left(2l-\frac{s_2}{\sqrt{2}}\right)
	=
	-q_0l^2 \left( 2-\frac{3}{2\sqrt{2}}\frac{s_2}{l}\right)
\end{eqnarray}
となる.以上を断面力図として示せば,図\ref{fig:fig3}の区間BCの部分に示したようになる.
\item
図\ref{fig:fig2}-(a)に示すc-c'断面で構造を切断し, 部材3に関する自由物体図を描くと
同図(d)のようになる.これを参照してつり合い条件から断面力分布を求めれば,
\begin{eqnarray}
	N_3 &=&-R_D= -\frac{q_0l}{2} \\
	Q_3 &=& H_D=q_0l \\
	M_3 &=& -H_D\times s_3=-q_0ls_3
\end{eqnarray}
となる.以上を断面力図として示せば,図\ref{fig:fig3}の区間CDの部分に示したようになる.
\item
部材断面に発生する直応力$\sigma$のうち,
軸力に起因する応力を$\sigma_N$, 曲げ応力を$\sigma_M$とすると,
$\sigma=\sigma_N+\sigma_M$で,$\sigma_N$と$\sigma_M$は,それぞれ
\begin{equation}
	\sigma_N(x,y)=\frac{N}{A}=\frac{N}{bh}, \ \
	\sigma_M(x,y)=\frac{M(x)}{I}y=\frac{12M(x)}{bh^3}y
	\label{eqn:}
\end{equation}
で与えられる.ただし$y$は,梁断面内の点の中立面から距離を意味する.
この問題では各部材内で軸力が一定のため,$\sigma_N$も部材内で一定値をとる.
$\sigma_M$は$y$に対して単調に変化することから,$\sigma_M$の最大値は
\begin{equation}
	\left| \sigma_M\left(x_{max},\pm \frac{h}{2}\right)\right|=\frac{6M_{max}}{bh^2}
	\label{eqn:eq_lbl}
\end{equation}
と表すことができる.ここに,$M_{max}$と$x_{max}$は,
それぞれ,曲げモーメントの最大値とそれを与える断面の座標を表す.
そこで,
\begin{equation}
	\gamma(N,M_{max}):=\frac{N}{bh}+\frac{6M_{max}}{bh^2}
	\label{eqn:cost}
\end{equation}
を部材毎に求めると,
\begin{eqnarray}
	部材1(節点B) &:& \gamma=\frac{q_0}{b}\left( \frac{1}{2}\frac{l}{h}+3\frac{l^2}{h^2}\right)=305\frac{q_0}{b}\\
	部材2(節点C) &:& \gamma=\frac{q_0}{b}\left( -\frac{1}{2\sqrt{2}}\frac{l}{h}+12\frac{l^2}{h^2}\right)=
	\left(1,200-\frac{5}{\sqrt{2}}\right)\frac{q_0}{b}\\
	部材3(節点C) &:& \gamma=\frac{q_0}{b}\left( -\frac{1}{2}\frac{l}{h}+12\frac{l^2}{h^2}\right)=1,195\frac{q_0}{b}
	\label{eqn:Hvals}
\end{eqnarray}
となる.以上より,最大引張応力は,部材2の節点Cにおいて生じ,その大きさは
\begin{equation}
	\sigma=\left(1,200-\frac{5}{\sqrt{2}}\right)\frac{q_0}{b}
	\label{eqn:sig_max}
\end{equation}
であると分かる.
\end{enumerate}
%--------------------
\begin{figure}[h]
	\begin{center}
	\includegraphics[width=0.7\linewidth]{fig2ans.eps} 
	\end{center}
	\caption{支点反力の正方向と断面力計算のための自由物体図.} 
	\label{fig:fig2}
\end{figure}
\begin{figure}[h]
	\begin{center}
	\includegraphics[width=1.0\linewidth]{fig3ans.eps} 
	\end{center}
	\caption{骨組み構造の断面力図.} 
	\label{fig:fig3}
\end{figure}
\end{document}

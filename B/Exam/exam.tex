\documentclass[10pt,a4j]{jarticle}
\usepackage{graphicx,wrapfig}
\setlength{\topmargin}{-1.5cm}
\setlength{\textwidth}{15.5cm}
\setlength{\textheight}{25.2cm}
\newlength{\minitwocolumn}
\setlength{\minitwocolumn}{0.5\textwidth}
\addtolength{\minitwocolumn}{-\columnsep}
%\addtolength{\baselineskip}{-0.1\baselineskip}
%
\def\Mmaru#1{{\ooalign{\hfil#1\/\hfil\crcr
\raise.167ex\hbox{\mathhexbox 20D}}}}
%
\begin{document}
\newcommand{\fat}[1]{\mbox{\boldmath $#1$}}
\newcommand{\D}{\partial}
\newcommand{\w}{\omega}
\newcommand{\ga}{\alpha}
\newcommand{\gb}{\beta}
\newcommand{\gx}{\xi}
\newcommand{\gz}{\zeta}
\newcommand{\vhat}[1]{\hat{\fat{#1}}}
\newcommand{\spc}{\vspace{0.7\baselineskip}}
\newcommand{\halfspc}{\vspace{0.3\baselineskip}}
\bibliographystyle{unsrt}
%\pagestyle{empty}
\newcommand{\twofig}[2]
 {
   \begin{figure}[h]
     \begin{minipage}[t]{\minitwocolumn}
         \begin{center}   #1
         \end{center}
     \end{minipage}
         \hspace{\columnsep}
     \begin{minipage}[t]{\minitwocolumn}
         \begin{center} #2
         \end{center}
     \end{minipage}
   \end{figure}
 }
%%%%%%%%%%%%%%%%%%%%%%%%%%%%%%%%%
\begin{center}
{\Large \bf 2019年度 構造力学I及び演習B 期末試験} \\
\end{center}
\begin{flushright}
	2019年2月7日(金)
\end{flushright}
%%%%%%%%%%%%%%%%%%%%%%%%%%%%%%%%%%%%%%%%%%%%%%%%%%%%%%%%%%%%%%%%
問題毎に解答用紙を分け,片面のみ使用すること.
下書き用紙, 問題用紙は試験終了後, 持ち帰って下さい.
%
%
%
\subsubsection*{問題1.}
図\ref{fig:fig1}のような4種類の梁について, 以下の問に答えよ.
なお,梁の曲げ剛性$EI$は全断面一定とし,$x$は点Aを原点とし
右方向を正とする座標を表す. 
\begin{enumerate}
\item
	梁(a)の受ける支点反力を全て求めよ.
	解答には支点反力の正方向を明記すること.
\item
	梁(a)のせん断力図と曲げモーメント図を描け.
\item
	梁(a)のたわみ分布を求めよ.
\item
	梁(b)のたわみを求めよ.
\item
	梁(c)の受ける支点反力を全て求めよ.
\item
	梁(c)の曲げモーメント図を描け.
\item
	梁(d)の受ける支点反力を全て求めよ.
\item
	梁(d)の曲げモーメント図を描け.
\end{enumerate}
\begin{figure}[h]
	\begin{center}
	\includegraphics[width=0.5\linewidth]{fig1.eps} 
	\end{center}
	\caption{鉛直下向きの集中荷重を受ける4種類の梁.} 
	\label{fig:fig1}
\end{figure}
%--------------------
\newpage
\subsubsection*{問題2.}
図\ref{fig:fig2}のような骨組み構造ABCDについて以下の問に答えよ.
なお.部材1は水平方向に等分布荷重が作用し,
その大きさは部材の単位長さあたり$q_0$であるとする.
\begin{enumerate}
\item
	支点反力を求めよ.解答には支点反力の正方向を明記すること.
\item	
	部材1の断面力図(軸力,せん断力, および曲げモーメント図)を描け.
\item	
	部材2の断面力図(軸力,せん断力, および曲げモーメント図)を描け.
\item	
	部材3の断面力図(軸力,せん断力, および曲げモーメント図)を描け.
\item	
	部材の断面は, いずれも幅$b$高さ$h$の長方形で$\frac{l}{h}=10$とする.
	このとき,骨組み構造に生じる引張応力の最大値を求めよ.
\end{enumerate}
%--------------------
\begin{figure}[h]
	\begin{center}
	\includegraphics[width=0.45\linewidth]{fig2.eps} 
	\end{center}
	\caption{
		部材1において水平方向に等分布荷重を受ける骨組み構造.
	} 
	\label{fig:fig2}
\end{figure}
%--------------------
%%%%%%%%%%%%%%%%%%%%%%%%%%%%%%%%%%%%%%%%%%%%
%%%%%%%%%%%%%%%%%%%%%%%%%%%%%%%%%%%%%%%%%%%%
\end{document}
\twofig{
	\includegraphics[width=0.9\linewidth]{tri_elem.eps} 
	\caption{微小三角形領域ABCと, 各辺に作用する応力および表面力.} 
	\label{fig:tri_elem}
}
{
	\includegraphics[width=0.9\linewidth]{N_plane.eps} 
	\caption{点$\fat{x}$を通り, $x$軸に対して$\theta$だけ傾いた面とその法線$\fat{N}(\theta)$および接線ベクトル$\fat{T}(\theta)$.} 
	\label{fig:N_plane}
}
\begin{figure}[h]
	\vspace{10mm}
	\begin{center}
	\includegraphics[width=0.5\linewidth]{fig1.eps} 
	\end{center}
	\caption{問題1で用いる座標系.(i)$xy$および$x'y'$座標系, (ii)$xy$座標系と$xy''$座標系. } 
	\label{fig:fig1}
\end{figure}

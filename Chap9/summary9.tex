\documentclass[10pt,a4j]{jbook}
%\usepackage{graphicx,wrapfig}
\usepackage{graphicx,amsmath}
\setlength{\topmargin}{-1.5cm}
%\setlength{\textwidth}{16.5cm}
\setlength{\textheight}{25.2cm}
\newlength{\minitwocolumn}
\setlength{\minitwocolumn}{0.5\textwidth}
\addtolength{\minitwocolumn}{-\columnsep}
%\addtolength{\baselineskip}{-0.1\baselineskip}
%
\def\Mmaru#1{{\ooalign{\hfil#1\/\hfil\crcr
\raise.167ex\hbox{\mathhexbox 20D}}}}
%
\begin{document}
\newcommand{\fat}[1]{\mbox{\boldmath $#1$}}
\newcommand{\D}{\partial}
\newcommand{\w}{\omega}
\newcommand{\ga}{\alpha}
\newcommand{\gb}{\beta}
\newcommand{\gx}{\xi}
\newcommand{\gz}{\zeta}
\newcommand{\vhat}[1]{\hat{\fat{#1}}}
\newcommand{\spc}{\vspace{0.7\baselineskip}}
\newcommand{\halfspc}{\vspace{0.3\baselineskip}}
\bibliographystyle{unsrt}
%\pagestyle{empty}
\newcommand{\twofig}[2]
 {
   \begin{figure}
     \begin{minipage}[t]{\minitwocolumn}
         \begin{center}   #1
         \end{center}
     \end{minipage}
         \hspace{\columnsep}
     \begin{minipage}[t]{\minitwocolumn}
         \begin{center} #2
         \end{center}
     \end{minipage}
   \end{figure}
 }
%%%%%%%%%%%%%%%%%%%%%%%%%%%%%%%%%
%\vspace*{\baselineskip}
%%%%%%%%%%%%%%%%%%%%%%%%%%%%%%%%%%%%%%%%%%%%%%%%%%%%%%%%%%%%%%%%
\setcounter{chapter}{8}
\chapter{静定骨組み構造}
\section{はじめに}
複数の梁部材を平面あるいは立体的に結合して作られた構造を{\bf 骨組み構造}と呼ぶ.
これまでに学んだ軸力問題と曲げ問題に関する知識は,骨組み構造の解析にも適用できる.
骨組み構造では,互いに向きのことなる部材同士が連結されていることや,
部材軸に対して斜め方向に外力が作用することに起因し,曲げモーメントと軸力が
同時に発生する.そのため,変形量や断面力を計算する際, 曲げと軸力を同時に考慮する必要がある.
骨組み構造の変形解析は,部材数の増加に伴い計算が非常に煩雑となり手計算で解くことのできる問題はかなり限定される.
しかしながら,静定構造の断面力計算方法は骨組み構造の場合と同様にして行うことができる.
以下では,その手順を示すために,単径間梁の曲げ-軸力問題についてはじめに述べる.
次に,平面骨組み構造の断面力計算を例題を用いて説明し,最後に,静定トラスの軸力計算方法を示す.
\section{曲げ-軸力問題}
\subsection{例題1}
図\ref{fig:fig12_1}-(a)に示すような単純支持された梁ACが,支間中央の点Bで大きさ$P$の集中荷重を受けるとする.
荷重の向きは梁の長手方向から反時計回りの方向に$\theta$とする.
ここで,支点反力の正方向を図\ref{fig:fig12_1}-(b)のように定め,
構造全体の釣り合い条件式を立てれば,
\begin{eqnarray}
	P\cos\theta-H_A=0 & \Rightarrow & H_A=P\cos\theta \\
	P\sin\theta \times \frac{l}{2}-R_C\times l =0 & \Rightarrow & R_C=\frac{P}{2}\sin\theta \\
	R_A+R_C=P\sin\theta & \Rightarrow & R_A=\frac{P}{2}\sin\theta
\end{eqnarray}
となり,水平反力$H_A$と鉛直反力$R_A,R_C$が求められる.
そこで,図\ref{fig:fig12_1}-(b)のa-a'の位置で梁を切断して
自由物体図を描くと同図(c)のようになる.これより,区間ABにおける
軸力$N(x)$, せん断力$Q(x)$, 曲げモーメント$M(x)$は次のように求められる.
\begin{eqnarray}
	N(x)+P\cos\theta=0 & \Rightarrow & N(x)=-P\cos\theta \\
	Q(x)-\frac{P}{2}\sin\theta=0 & \Rightarrow & Q(x)=\frac{P}{2}\sin\theta \\
	M(x)-\frac{P}{2}\sin\theta\times x =0 & \Rightarrow & M(x)=\frac{P}{2}x\sin\theta
\end{eqnarray}
このように,水平反力$H_A$がゼロでないことに起因し,区間ABにはせん断力と曲げモーメントに加え
軸力が発生する.一方,
図\ref{fig:fig12_1}-(c)の自由物体図をもとに水平力,鉛直力,モーメントの釣り合いを考えることで,
右半分の区間BCには次のような断面力が働くことが分かる.
\begin{eqnarray}
	N(s)&=& 0 \\
	Q(s)&=&-\frac{P}{2}\sin\theta \\
	M(s)&=&\frac{P}{2}s \sin\theta
\end{eqnarray}
以上の結果を断面力図として示すと,図\ref{fig:fig12_1_1}のようになる.
%--------------------
\begin{figure}[h]
	\begin{center}
	\includegraphics[width=0.85\linewidth]{./fig1.eps} 
	\end{center}
	\caption{
		単径間梁に関する曲げ-軸力問題.
	} 
	\label{fig:fig12_1}
\end{figure}
%--------------------
\begin{figure}[h]
	\begin{center}
	\includegraphics[width=1.00\linewidth]{./fig11.eps} 
	\end{center}
	\caption{
		単径間梁の曲げ-軸力問題に対する断面力図(例題1).
	} 
	\label{fig:fig12_1_1}
\end{figure}
\subsection{例題2}
単径間梁における曲げ-軸力問題の別の例として,図\ref{fig:fig12_1}-(d)に示すような
鉛直方向の等布荷重を受ける梁ABを考える.
ここで,等分布荷重の大きさは,梁長手方向の単位長さあたり$q_0$で,
梁は水平方向から$\theta$だけ傾いた状態で単純支持されているとする.
支点Aはピン支点のため,水平,鉛直方向に変位が拘束されることから,
水平反力$H_A$と鉛直反力$R_A$が発生する.一方,支点Bはローラー支点で,
あることから鉛直変位のみが拘束され,鉛直反力$R_B$のみが生じる.
これらの反力と外力について水平力,鉛直力,およびモーメントの釣り合い条件を
立てれば,3つの反力が次のように決まる.
\begin{equation}
	R_A=R_B=\frac{q_0l}{2}, \ \ H_A=0
\end{equation}
と求まる.
次に,図\ref{fig:fig12_1}-(d)のa-a'の位置で梁を長手方向に垂直に切断して
自由物体図を描くと,図\ref{fig:fig12_1}-(e)のようになる.
そこで,梁の長手方向($N$方向),長手直角方向($Q$方向)の力の釣り合い式を立てると,
\begin{eqnarray}
	N(x)+H_A\cos\theta +R_A\sin\theta - q_0x\sin \theta =0 
	& \Rightarrow  & 
	N(x)=q_0\left(x-\frac{l}{2}\right)\sin\theta
	%N(x)=-\frac{q_0l}{2}\sin\theta -q_0x \sin \theta
	\\
	Q(x)+H_A\sin\theta -R_A\cos\theta +q_0x\cos \theta =0 
	& \Rightarrow  &  
	Q(x)=-q_0\left(x-\frac{l}{2}\right) \cos\theta	
	%Q(x)=\frac{q_0l}{2}\cos\theta -q_0x \cos \theta
\end{eqnarray}
と,軸力$N$とせん断力$Q$が求められる.また,a-a'断面に関するモーメントの釣り合い式からは,
\begin{equation}
	M(x)+H_Ax\sin\theta -R_Ax\cos\theta +q_0x\times \frac{x}{2}\cos \theta =0 
	\Rightarrow    
	%M(x)=\frac{q_0lx}{2}\cos\theta -\frac{q_0x^2}{2} \cos \theta
	M(x)=-\frac{q_0x}{2}\left(x-l\right)\cos\theta
\end{equation}
と曲げモーメントが求められる.以上の結果を断面力図として示せば図\ref{fig:fig12_1_2}のようになる.
%--------------------
\begin{figure}[h]
	\begin{center}
	\includegraphics[width=1.00\linewidth]{./fig10.eps} 
	\end{center}
	\caption{
		単径間梁の曲げ-軸力問題に対する断面力図(例題2).
	} 
	\label{fig:fig12_1_2}
\end{figure}
%%%%%%%%%%%%%%%%%%%%%%%%%%%%%%%%%%%%%%%%%%%%%%%%%
\section{静定骨組み構造の断面力計算}
\subsection{例題 1}
図\ref{fig:fig12_2}-(a)に示すような2つの部材ABとBCを直角に剛結して作られた
平面骨組み構造ABCを考える.
以下ではABを部材1, BCを部材2と呼び,骨組み構造ABCは単純支持された状態で
部材1(部材AB)に水平方向の等分布荷重を受けるとする.

はじめに,支点AとCにおける支点反力の正方向を図\ref{fig:fig12_3}-(a)のように定める.
この図を元に,構造全体の力とモーメントの釣り合い式を立てれば,支点反力は次のように求められる.
\begin{eqnarray}
	q_0l-H_A=0 & \Rightarrow & H_A=q_0l \\ 
	R_C\times l -q_0l \times \frac{l}{2}
	 & \Rightarrow & R_C=\frac{q_0l}{2} \\ 
	R_A+R_C=0 &\Rightarrow& R_A=-\frac{q_0l}{2}
\end{eqnarray}
次に,a-a'の位置で部材1を,b-b'の位置で部材2をそれぞれ切断して断面力を求める.
ここで,部材$i(=1,2)$の軸力を$N_i$,せん断力を$Q_i$,曲げモーメント$M_i$と表し,
これらの正方向を図\ref{fig:fig12_3}-(b)および(c)のように定める.
断面力の正方向は,軸力は引張が正となるようにとる.せん断力と曲げモーメント
の正方向は任意に定めてよいが,ここでは,A$\rightarrow$B$\rightarrow$Cの方向を
正断面と考え,正断面における曲げモーメントは反時計周りを正方向とする.
また,正断面におけるせん断力の正方向は,軸力の正方向を時計回りに90度回転させた方向としている.
これら図\ref{fig:fig12_3}-(c)と(d)に示された自由物体図を参照して力とモーメントの
釣り合いを考えれば,部材1と2の断面力は,以下のように求められる.
\begin{eqnarray}
	N_1(x_1) &= & \frac{q_0l}{2}  
	\\
	Q_1(x_1) &= & q_0(l-x_1)
	\\
	M_1(x_1) &= & q_0x_1\left(l-\frac{x_1}{2}\right)
\end{eqnarray}
\begin{eqnarray}
	N_2(s_2) &= & 0
	\\
	Q_2(s_2) &= & -\frac{q_0l}{2}
	\\
	M_2(s_2) &= & \frac{1}{2}q_0ls_2
\end{eqnarray}
図\ref{fig:fig12_3}-(d)から(f)は,以上の結果を断面力図として示したものである.
骨組み構造の断面力計算では,断面力の正方向を各自で定義し,断面力図には
断面力の正負を明記することが必要である.
\begin{figure}[h]
	\begin{center}
	\includegraphics[width=1.0\linewidth]{./fig2.eps} 
	\end{center}
	\caption{
		剛結された2つの部材から構成される静定平面骨組み構造(a)〜(e). 
		分布荷重の大きさは,梁の単位長さあたり$q_0$で,図中の数字1と2は部材番号を表す.
	} 
	\label{fig:fig12_2}
\end{figure}
%%%%%%
%--------------------
\begin{figure}[h]
	\begin{center}
	\includegraphics[width=0.9\linewidth]{./fig3.eps} 
	\end{center}
	\caption{
		図\ref{fig:fig12_2}-(a)の構造の断面力計算に用いる自由物体図(a)-(c)と,
		断面力図(d)-(f).
	} 
	\label{fig:fig12_3}
\end{figure}
%%%%%%%%%%%%%%%%%%%%%%%%%%%%%%%%%%
\subsection{例題 2}
図\ref{fig:fig12_2}-(d)に示した骨組み構造の断面力を求める.
はじめに,構造全体の釣り合い条件から支点反力を求める.
次に,部材毎に構造を切断して自由物体図を描き,部分構造の釣り合い条件から断面力を決定する.
以下では,
図\ref{fig:fig12_4}-(a)に示すように支点反力$R_A,H_A,M_A$の正方向をとり,
部材$i$の断面力を$N_i, Q_i, M_i$と書く.

この問題では,水平方向に作用する外力は存在しないため$H_A=0$は明らかである.
また,鉛直方向の力の釣り合いと点Aに関するモーメントの釣り合いより,
\begin{equation}
	R_A=q_0l, \ \ M_A=-\frac{3}{2}q_0l^2
\end{equation}
となることも分かる.
以上を踏まえ,a-a'の箇所で構造を切断して自由物体図を描けば,図\ref{fig:fig12_4}-(b)のようになる.
そこで,支点反力を部材軸方向と部材軸直角方向の力に分解し,それぞれの方向で釣り合い条件式を立てれば,
部材1の断面力が次のように求められる.
\begin{eqnarray}
	N_1(x_1) &= & -\frac{q_0l}{\sqrt{2}} 
	\\
	Q_1(x_1) &= & \frac{q_0l}{\sqrt{2}} 
	\\
	M_1(x_1) &= & q_0l \left( \frac{x_1}{\sqrt{2}}-\frac{3}{2}l\right) 
\end{eqnarray}
部材2の断面力については,図\ref{fig:fig12_4}-(a)のb-b'において構造を切断して
自由物体図を描く(図\ref{fig:fig12_4}-(c)).この場合,水平に支持された片持梁
と同様にして断面力を得ることができ,その結果は次のようになる.
\begin{eqnarray}
	N_2(s_2) &= & 0
	\\
	Q_2(s_2) &= & q_0s_2
	\\
	M_2(s_2) &= & -\frac{q_0}{2}s_2^2
\end{eqnarray}
最後に,一連の結果を断面力図として示せば図\ref{fig:fig12_5}のようになる.
%--------------------
\begin{figure}[h]
	\begin{center}
	\includegraphics[width=0.9\linewidth]{./fig4.eps} 
	\end{center}
	\caption{
		図\ref{fig:fig12_2}-(d)の構造の断面力計算に用いる自由物体図.
	} 
	\label{fig:fig12_4}
\end{figure}
%--------------------
\begin{figure}[h]
	\begin{center}
	\includegraphics[width=1.0\linewidth]{./fig5.eps} 
	\end{center}
	\caption{
		図\ref{fig:fig12_2}-(d)の構造の断面力図.
	} 
	\label{fig:fig12_5}
\end{figure}
\subsection{問題}
図\ref{fig:fig12_2}-(b),(c),(e)に示された骨組み構造の断面力図を描け.
%%%%%%%%%%%%%%%%%%%%%%%%%%%%%%%%%%%%%%5
\section{静定トラス構造の軸力計算}
\subsection{トラス構造の部材力}
図\ref{fig:fig12_9}-(a)に示すような,三角形の部材組を基本単位とした構造を{\bf トラス構造}と呼ぶ.
トラス構造を構成する部材の中で,構造の上縁側,下縁側に位置する水平材はそれぞれ{\bf 上弦材},
{\bf 下弦材}と呼ばれ,斜めに配置された部材は{\bf 斜材}と呼ばれる.
部材どうしの結合部は{\bf 節点}あるいは格点と呼ばれ,節点では複数の部材が{\bf ピン結合}されていると考える.
トラス構造では部材が外力を直接受けることはなく,トラス構造上に設置された床版や桁を介して
間接的に節点へ集中荷重が加えられる.従って,各トラス部材に作用する力は部材端部にのみ働くために,
単一のトラス部材の自由物体図は図\ref{fig:fig12_9}-(b)のようになる.
ここで,部材端aとbは別の部材や支点にピン結合されているので,部材端に曲げモーメントは加わらない.
これらのことを踏まえてトラス部材abの力とモーメントの釣り合いを考えれば,
\begin{equation}
	N_a=N_b, \ \ Q_a=Q_b=0
\end{equation}
が示される.さらに,軸力が部材内部で一定値をとることも軸力の釣り合い条件から容易に示される.
以上より,ピン結合で作られたトラス構造は,各部材が軸力だけを伝達する構造となっており,
断面力計算では部材毎に一定値をとる軸力を決定すればよいことになる.
%--------------------
\begin{figure}[h]
	\begin{center}
	\includegraphics[width=0.9\linewidth]{./fig9.eps} 
	\end{center}
	\caption{
		(a)トラス構造の一例(ワーレントラス), (b)トラス部材に作用する力. 
	} 
	\label{fig:fig12_9}
\end{figure}
%--------------------
\begin{figure}[h]
	\begin{center}
	\includegraphics[width=0.9\linewidth]{./fig6.eps} 
	\end{center}
	\caption{
		外力を受けるトラス構造.数字は部材番号を表す.
	} 
	\label{fig:fig12_6}
\end{figure}
%--------------------
\subsection{例題1}
図\ref{fig:fig12_6}-(a)に示すような水平荷重を受ける正三角形の部材組について
部材1から3に生じる軸力を求める.はじめに,支点反力の正方向を図
\ref{fig:fig12_7}-(a)のように定める.これらは,構造全体の釣り合い条件から
\begin{equation}
	H_A=F, \ \ 
	R_A=-\frac{\sqrt{3}}{2}F, \ \ 
	R_B= \frac{\sqrt{3}}{2}F
\end{equation}
と求まる.次に,図\ref{fig:fig12_7}-(a)のa-a'にそって構造を切断し,
節点Cに加わる力を描き込むと,同図(b)のようになる.
ここで,$N_i$は部材$i(=1,2,3)$の軸力を意味し,$N_i>0$は引張,
$N_i<0$は圧縮の軸力を表す.$N_2$と$N_3$は,節点Cに作用する力の釣り合いより,
\begin{equation}
	N_2=F, \ \ N_3=-F
\end{equation}
と決まる.  残る$N_1$は節点AあるいはBにおける力の釣り合いから
求めることができる.あるいは,図\ref{fig:fig12_7}-(a)のb-b'の位置で
構造を切断した結果として得られる部分構造(図\ref{fig:fig12_7}-(c))
を利用して$N_1$を決めることもでき,その場合,力だけでなく
モーメントの釣り合い条件も利用できる.
例えば,図\ref{fig:fig12_7}-(c)において点Cに関するモーメントの釣り合い式を立てれば,
\begin{equation}
	N_1\times \frac{\sqrt{3}}{2}l -R_B\times \frac{l}{2}=0 \Rightarrow
	N_1=\frac{F}{2}
\end{equation}
となり,軸力$N_2$が未知でも$N_1$を求めることができる.
\begin{figure}[h]
	\begin{center}
	\includegraphics[width=0.9\linewidth]{./fig7.eps} 
	\end{center}
	\caption{
		図\ref{fig:fig12_6}-(a)の構造に対する軸力計算に用いた自由物体図.
	} 
	\label{fig:fig12_7}
\end{figure}
%--------------------
\subsection{例題2}
図\ref{fig:fig12_6}-(b)のようなワーレントラスに発生する軸力を, 
図\ref{fig:fig12_8}に示す自由物体図を用いて求める.
図\ref{fig:fig12_8}-(a)は支点反力の正方向と,軸力計算に用いる自由物体図(b)$\sim$(e)を
描くための切断位置を示したものである.
はじめに構造全体の釣り合い条件を用いれば
\begin{equation}
	H_A=0, \ \ R_A=R_C=\frac{F}{2}
\end{equation}
と反力が求まる.次に,a-a'で切断を行い,図\ref{fig:fig12_8}-(b)を参照して節点Aに関する
力の釣り合いを考えれば,軸力$N_1,N_3$が
\begin{eqnarray}
	水平方向&:& N_3\sin\frac{\pi}{3}+\frac{F}{2}=0 
	\ \ 
	\Rightarrow 
	\ \ 
	N_3=-\frac{F}{\sqrt{3}} \\
	鉛直方向&:& N_3\cos\frac{\pi}{3}+N_1=0
	\ \ \Rightarrow   \ \ 
	N_1=\frac{F}{2\sqrt{3}} 
\end{eqnarray}
と求められる.c-c'での切断(図\ref{fig:fig12_8}-(c))に基づき節点Cでの力の釣り合いを考えれば,
ほとんど同じ計算で$N_2$と$N_6$が
%\end{document}
\begin{equation}
	N_6=-\frac{F}{\sqrt{3}}, \ \ N_2=\frac{F}{2\sqrt{3}} 
	\label{eqn:}
\end{equation}
と求まる.続いて,切断b-b'によって得られる部分構造(図\ref{fig:fig12_8}-(d))に
着目すれば,節点Bに関するモーメントの釣り合いから
\begin{equation}
	N_7 \times l\sin\frac{\pi}{3}+\frac{F}{2}\times l =0 \Rightarrow 
	N_7=-\frac{F}{\sqrt{3}}
	\label{eqn:N7}
\end{equation}
と$N_7$が求まる.一方,節点Cに関するモーメントの釣り合いを考えると,
\begin{equation}
	N_4 \times l\sin\frac{\pi}{3}- N_7 \times l\sin\frac{\pi}{3} - F\times l =0 \Rightarrow 
	N_4=\frac{F}{\sqrt{3}}
\end{equation}
と$N_4$が求められる.
最後に,d-d'での切断(図\ref{fig:fig12_8}-(d))では,
$N_4$と$N_7$を求めた直前の計算とほとんど同じ計算により,$N_7$と$N_5$が次のように求められる.
\begin{equation}
	N_7=-\frac{F}{\sqrt{3}}
	, \ \ 
	N_5=\frac{F}{\sqrt{3}}
	\label{eqn:}
\end{equation}
勿論,ここで得られた$N_7$は,切断b-b'で既に明らかになった結果と一致する.なお,
力とモーメントの釣り合い条件を利用して静定トラスの軸力を求める際の手順は,
上に示したもの以外にも無数に考えることができる.どのような順序や釣り合い式を
用いて軸力を決定すべきかは問題に依って異なるため,状況に応じて最も少な計算量や
最も確実な方法を見つけるためにはその都度試行錯誤が必要とされる.\\

節点に作用する力(軸力や反力)の釣り合いからは,水平方向と鉛直方向の2つの条件を得ることができる.
従って,その節点に作用する未知の荷重が2つ以下であれば,
力の釣り合いからそれら未知の荷重を決定できる.
このことを踏まえて節点D,EおよびBを見ると,節点DとEには3本の部材が,
節点Bには4本の部材が結合されている.そのため,最初にこれらの節点に関する力の釣り合い式を
立てるだけでは未知の軸力を決定することはできない.
一方,切断b-b'やd-d'の結果として得られる部分構造では,力の釣り合い式に
モーメントの釣り合い式を加えた,合計3つの条件式が得られる.
従って,切断の結果として現れる未知の軸力の数が3以下であればそれらの軸力を決定できる.
さらに,モーメントの基準点を適切に選べば,いくつかの未知の軸力が
つりあい方程式に含まれないようにすることができる.その結果として未知数の合計
を3以下にすることができれば,切断面に現れる軸力の一部が決定できる.
式(\ref{eqn:N7})による軸力$N_7$の求め方は,このような方針に基づいて選択したものである.
\begin{figure}[h]
	\begin{center}
	\includegraphics[width=0.9\linewidth]{./fig8.eps} 
	\end{center}
	\caption{
	図\ref{fig:fig12_6}-(b)の構造に対する軸力計算に用いた自由物体図.
	} 
	\label{fig:fig12_8}
\end{figure}
\subsection{問題}
図\ref{fig:fig12_6}-(c)と(d)のトラス構造において発生する軸力を全てを求めよ.
\end{document}

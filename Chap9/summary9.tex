\documentclass[10pt,a4j]{jbook}
%\usepackage{graphicx,wrapfig}
\usepackage{graphicx,amsmath}
\setlength{\topmargin}{-1.5cm}
\setlength{\textwidth}{16.5cm}
\setlength{\textheight}{25.2cm}
\newlength{\minitwocolumn}
\setlength{\minitwocolumn}{0.5\textwidth}
\addtolength{\minitwocolumn}{-\columnsep}
%\addtolength{\baselineskip}{-0.1\baselineskip}
%
\def\Mmaru#1{{\ooalign{\hfil#1\/\hfil\crcr
\raise.167ex\hbox{\mathhexbox 20D}}}}
%
\begin{document}
\newcommand{\fat}[1]{\mbox{\boldmath $#1$}}
\newcommand{\D}{\partial}
\newcommand{\w}{\omega}
\newcommand{\ga}{\alpha}
\newcommand{\gb}{\beta}
\newcommand{\gx}{\xi}
\newcommand{\gz}{\zeta}
\newcommand{\vhat}[1]{\hat{\fat{#1}}}
\newcommand{\spc}{\vspace{0.7\baselineskip}}
\newcommand{\halfspc}{\vspace{0.3\baselineskip}}
\bibliographystyle{unsrt}
%\pagestyle{empty}
\newcommand{\twofig}[2]
 {
   \begin{figure}
     \begin{minipage}[t]{\minitwocolumn}
         \begin{center}   #1
         \end{center}
     \end{minipage}
         \hspace{\columnsep}
     \begin{minipage}[t]{\minitwocolumn}
         \begin{center} #2
         \end{center}
     \end{minipage}
   \end{figure}
 }
%%%%%%%%%%%%%%%%%%%%%%%%%%%%%%%%%
%\vspace*{\baselineskip}
%%%%%%%%%%%%%%%%%%%%%%%%%%%%%%%%%%%%%%%%%%%%%%%%%%%%%%%%%%%%%%%%
\setcounter{chapter}{8}
\chapter{静定骨組み構造}
\section{はじめに}
複数の部材を平面的あるいは立体的に結合して作られた構造は"{\bf 骨組み構造}"と呼ばれる.
これまでに学んだ軸力問題と曲げ問題に関する知識は,骨組み構造の解析にも適用することができる.
骨組み構造では,互いに向きのことなる部材同士が連結されていることと,
部材軸に対して斜め方向に外力が作用することに起因し,曲げモーメントと軸力が
同時に発生する.そのため,変形量や断面力を計算する際, 曲げと軸力が同時に
発生する曲げ-軸力問題として考える必要がある.
骨組み構造の変形解析は,部材数の増加に伴い計算が非常に煩雑となり,
手計算で解くことのできる問題はかなり限定されてしまう.
一方,静帝構造の断面力計算に関しては,骨組み構造の場合もこれまでとほとんど同じ
方法で行うことができる.以下では,その手順を具体的に示すために,単径間梁の曲げ-
軸力問題について述べ,次に,平面骨組み構造の断面力計算を具体的な例題を用いて
説明する.最後に,静定トラスの軸力計算方法について,例題を用いて説明する.
\section{曲げ-軸力問題}
\subsection{例1}
図\ref{fig:fig12_1}-(a)に示すような,単純支持された梁ACが,支間中央の点B
で大きさ$P$の集中荷重を受けるとする.
荷重の向きは,梁の長手方向から反時計回りの方向に$\theta$であるとする.
ここで,支点反力の正方向を図\ref{fig:fig12_1}-(b)のように定め,
構造全体の釣り合い条件式を立てれば,
\begin{eqnarray}
	P\cos\theta-H_A=0 & \Rightarrow & H_A=P\cos\theta \\
	P\sin\theta \times \frac{l}{2}-R_C\times l =0 & \Rightarrow & R_C=\frac{P}{2}\sin\theta \\
	R_A+R_C=P\sin\theta & \Rightarrow & R_A=\frac{P}{2}\sin\theta
\end{eqnarray}
となり,水平反力$H_A$と鉛直反力$R_A,R_C$が求められる.
そこで,図\ref{fig:fig12_1}-(b)のa-a'の位置で梁を切断して
自由物体図を描くと,同図(c)のようになる.これより,区間ABにおける
軸力$N(x)$, せん断力$Q(x)$, 曲げモーメント$M(x)$が, 次のように求められる.
\begin{eqnarray}
	N(x)+P\cos\theta=0 & \Rightarrow & N(x)=-P\cos\theta \\
	Q(x)-\frac{P}{2}\sin\theta=0 & \Rightarrow & Q(x)=\frac{P}{2}\sin\theta \\
	M(x)-\frac{P}{2}\sin\theta\times x =0 & \Rightarrow & M(x)=\frac{P}{2}x\sin\theta
\end{eqnarray}
このように,水平反力$H_A$がゼロでないことに起因して,区間AB
では,せん断力と曲げモーメントに加え,軸力が同時に発生する.
一方,右半分の区間BCは,図\ref{fig:fig12_1}-(c)の自由物体図をもとに
水平力,鉛直力,モーメントの釣り合いを考えることで,断面力が次のように求まる.
\begin{eqnarray}
	N(s)&=& 0 \\
	Q(s)&=&-\frac{P}{2}\sin\theta \\
	M(s)&=&\frac{P}{2}s \sin\theta
\end{eqnarray}
以上の結果を断面力図として示すと,図\ref{fig:fig12_1_1}のようになる.

\subsection{例2}
単径間梁における曲げ-軸力問題のもう一つの例として,図\ref{fig:fig12_1}-(d)に示すような
鉛直方向の等布荷重を受ける梁ABを考える.
ここで,等分布荷重の大きさは,梁長手方向の単位長さあたり$q_0$,
梁は水平方向から$\theta$だけ傾いた状態で単純支持されているとする.
支点Aはピン支点のため,水平,鉛直方向に変位が拘束されることから,
水平反力$H_A$,鉛直反力$R_A$が発生する.一方,支点Bは,ローラー支点で,
水平方向と回転に関する変位の拘束がなく,鉛直変位のみが拘束される.
よって,支点Bでは鉛直反力$R_B$のみが生じる.
これらの反力と外力に対して,水平力,鉛直力,モーメントの釣り合い条件を
立てれば,3つの反力が次のように決まる.
\begin{equation}
	R_A=R_B=\frac{q_0l}{2}, \ \ H_A=0
\end{equation}
と求まる.
次に,図\ref{fig:fig12_1}-(d)のa-a'の位置で梁を長手方向に垂直に切断して
自由物体図を描くと,図\ref{fig:fig12_1}-(e)のようになる.
そこで,梁の長手方向($N$方向),長手直角方向($Q$方向)の力の釣り合い式を立てると,
\begin{eqnarray}
	N(x)+H_A\cos\theta +R_A\sin\theta - q_0x\sin \theta =0 
	& \Rightarrow  & 
	N(x)=q_0\left(x-\frac{l}{2}\right)\sin\theta
	%N(x)=-\frac{q_0l}{2}\sin\theta -q_0x \sin \theta
	\\
	Q(x)+H_A\sin\theta -R_A\cos\theta +q_0x\cos \theta =0 
	& \Rightarrow  &  
	Q(x)=-q_0\left(x-\frac{l}{2}\right) \cos\theta	
	%Q(x)=\frac{q_0l}{2}\cos\theta -q_0x \cos \theta
\end{eqnarray}
と,軸力$N$,せん断力$Q$が求められ,a-a'断面に関するモーメントの釣り合い式からは,
\begin{equation}
	M(x)+H_Ax\sin\theta -R_Ax\cos\theta +q_0x\times \frac{x}{2}\cos \theta =0 
	\Rightarrow    
	%M(x)=\frac{q_0lx}{2}\cos\theta -\frac{q_0x^2}{2} \cos \theta
	M(x)=-\frac{q_0x}{2}\left(x-l\right)\cos\theta
\end{equation}
と曲げモーメントが求められる.以上の結果を断面力図として示すと,図\ref{fig:fig12_1_2}のようになる.
%--------------------
\begin{figure}[h]
	\begin{center}
	\includegraphics[width=0.85\linewidth]{./fig1.eps} 
	\end{center}
	\caption{
		単径間梁に関する曲げ-軸力問題.
	} 
	\label{fig:fig12_1}
\end{figure}
%--------------------
\begin{figure}[h]
	\begin{center}
	\includegraphics[width=1.00\linewidth]{./fig11.eps} 
	\end{center}
	\caption{
		単径間梁の曲げ-軸力問題に対する断面力図(例1).
	} 
	\label{fig:fig12_1_1}
\end{figure}
%--------------------
\begin{figure}[h]
	\begin{center}
	\includegraphics[width=1.00\linewidth]{./fig10.eps} 
	\end{center}
	\caption{
		単径間梁の曲げ-軸力問題に対する断面力図(例2).
	} 
	\label{fig:fig12_1_2}
\end{figure}
%%%%%%%%%%%%%%%%%%%%%%%%%%%%%%%%%%%%%%%%%%%%%%%%%
\section{静定骨組み構造の断面力計算}
\subsection{例題 1}
図\ref{fig:fig12_2}-(a)に示すように,2つの部材ABとBCを直角に剛結して作られた
平面骨組み構造が単純支持された状態で,水平方向の等分布荷重を部材ABに受けるとする.
以下ではABを部材1, BCを部材2と呼ぶ.

はじめに,支点AとCにおける支点反力の正方向を図\ref{fig:fig12_3}-(a)のように定める.
この図を元に,構造全体の力とモーメントの釣り合い式を立てれば,支点反力が次のように求められる.
\begin{eqnarray}
	q_0l-H_A=0 & \Rightarrow & H_A=q_0l \\ 
	R_C\times l -q_0l \times \frac{l}{2}
	 & \Rightarrow & R_C=\frac{q_0l}{2} \\ 
	R_A+R_C=0 &\Rightarrow& R_A=-\frac{q_0l}{2}
\end{eqnarray}
次に,a-a'の位置で部材1を,b-b'の位置で部材2をそれぞれ切断して断面力を求める.
ここで,部材$i(=1,2)$の軸力を$N_i$,せん断力を$Q_i$,曲げモーメント$M_i$と表し,
これらの正方向を図\ref{fig:fig12_3}-(b)および(c)のように定める.
断面力の正方向は,軸力は引張が正となるようにとる.せん断力と曲げモーメントの
の正方向は任意に定めてよいが,ここでは,A$\rightarrow$B$\rightarrow$Cの方向を
正断面と考え,正断面における曲げモーメントは反時計周りを正方向とするようにした.
また,正断面におけるせん断力の正方向は,軸力の正方向を時計回りに90度回転させた方向としている.
これらの図\ref{fig:fig12_3}-(c)と(d)に示された自由物体図を参照して力とモーメントの
釣り合いを考えれば,部材1と2の断面力は,以下に示すように求められる.
\begin{eqnarray}
	N_1(x_1) &= & \frac{q_0l}{2}  
	\\
	Q_1(x_1) &= & q_0(l-x_1)
	\\
	M_1(x_1) &= & q_0x_1\left(l-\frac{x_1}{2}\right)
\end{eqnarray}
\begin{eqnarray}
	N_2(s_2) &= & 0
	\\
	Q_2(s_2) &= & -\frac{q_0l}{2}
	\\
	M_2(s_2) &= & \frac{1}{2}q_0ls_2
\end{eqnarray}
図\ref{fig:fig12_3}-(c)から(e)は,以上の結果を断面力図として示したものである.
骨組み構造の断面力計算では,断面力の正方向を各自で定義し,断面力図には
断面力の正負を明記することが必要である.
\begin{figure}[h]
	\begin{center}
	\includegraphics[width=1.0\linewidth]{./fig2.eps} 
	\end{center}
	\caption{
		剛結された2つの部材から構成される静定平面骨組み構造(a)〜(e). 
		分布荷重の大きさは,梁の単位長さあたり$q_0$で,図中の数字1と2は部材番号を表す.
	} 
	\label{fig:fig12_2}
\end{figure}
%%%%%%
%--------------------
\begin{figure}[h]
	\begin{center}
	\includegraphics[width=0.9\linewidth]{./fig3.eps} 
	\end{center}
	\caption{
		図\ref{fig:fig12_2}-(a)の構造の断面力計算に用いる自由物体図(a)-(c)と,
		断面力図(d)-(f).
	} 
	\label{fig:fig12_3}
\end{figure}
%%%%%%%%%%%%%%%%%%%%%%%%%%%%%%%%%%
\subsection{例題 2}
図\ref{fig:fig12_2}-(d)に示した骨組み構造の断面力を求める.
はじめに,構造全体の釣り合い条件から支点反力を求める.
次に,部材毎に構造を切断して自由物体図を描き,部分構造毎にたてた
釣り合い条件式から断面力を決定する.以下では,
図\ref{fig:fig12_4}-(a)に示すように支点反力$R_A,H_A,M_A$の正方向をとり,
部材$i$の断面力を$N_i, Q_i, M_i$と書く.

この問題では,水平方向に作用する外力は存在しないため$H_A=0$は明らかである.
また,鉛直方向の力の釣り合いと点Aに関するモーメントの釣り合いより,
\begin{equation}
	R_A=q_0l, \ \ M_A=-\frac{3}{2}q_0l^2
\end{equation}
と残りの支点反力も決定できる.
以上を踏まえ,a-a'の箇所で構造を切断して自由物体図を描けば,図\ref{fig:fig12_4}-(b)のようになる.
そこで,支点反力を部材軸方向と部材軸直角方向の力に分解し,それぞれの方向で釣り合い条件式を立てれば,
部材1の断面力が次のように求められる.
\begin{eqnarray}
	N_1(x_1) &= & -\frac{q_0l}{\sqrt{2}} 
	\\
	Q_1(x_1) &= & \frac{q_0l}{\sqrt{2}} 
	\\
	M_1(x_1) &= & q_0l \left( \frac{x_1}{\sqrt{2}}-\frac{3}{2}l\right) 
\end{eqnarray}
部材2の断面力については,図\ref{fig:fig12_4}-(a)のb-b'において構造を切断して
自由物体図を描けば,同図(c)のようになる.この場合,水平に支持された片持梁
と同様にして断面力を得ることができる.その結果は次の通りで,
一連の結果を断面力図として示せば,図\ref{fig:fig12_5}のようになる.
\begin{eqnarray}
	N_2(s_2) &= & 0
	\\
	Q_2(s_2) &= & q_0s_2
	\\
	M_2(s_2) &= & -\frac{q_0}{2}s_2^2
\end{eqnarray}
%--------------------
\begin{figure}[h]
	\begin{center}
	\includegraphics[width=0.9\linewidth]{./fig4.eps} 
	\end{center}
	\caption{
		図\ref{fig:fig12_2}-(d)の構造の断面力計算に用いる自由物体図.
	} 
	\label{fig:fig12_4}
\end{figure}
%--------------------
\begin{figure}[h]
	\begin{center}
	\includegraphics[width=1.0\linewidth]{./fig5.eps} 
	\end{center}
	\caption{
		図\ref{fig:fig12_2}-(d)の構造の断面力図.
	} 
	\label{fig:fig12_5}
\end{figure}
\subsection{問題}
図\ref{fig:fig12_2}-(b),(c),(e)に示された構造の断面力図を描け.
%%%%%%%%%%%%%%%%%%%%%%%%%%%%%%%%%%%%%%5
\section{静定トラス構造の軸力計算}
\subsection{トラス構造の部材力}
図\ref{fig:fig12_9}-(a)に示すような,三角形の部材組を基本単位とした構造を"{\bf トラス構造}"と呼ぶ.
トラス構造を構成する部材の中で,構造の上側,下側に位置する水平材はそれぞれ"{\bf 上弦材}",
"{\bf 下弦材}"と呼ばれ,斜めに配置された部材は斜材と呼ばれる.
部材どうしの結合部は,"{\bf 節点}"あるいは格点と呼ばれ,節点では複数の部材がピンで結合されていると考える.
トラス構造では,部材が外力を直接受けることはなく,トラス構造上に設置された床版や桁を介して,
間接的に節点へ集中荷重が加えられる.従って,各トラス部材に作用する力は部材端部にのみ働く.
従って,単一のトラス部材の自由物体図を描くと,図\ref{fig:fig12_9}-(b)のようになる.
ここで,部材端部は,他の部材や支点にピンで結合されていることから曲げモーメントは加わらない.
また,力とモーメントの釣り合いを考えれば,
\begin{equation}
	N_a=N_b, \ \ Q_a=Q_b=0
\end{equation}
でなければならない.さらに,軸力が部材内部で一定値をとることも,軸力の釣り合い条件から容易に示される.
以上より,ピン結合で作られたトラス構造は,各部材が軸力だけを伝達する構造となっており,
断面力計算では,部材毎に一定となる軸力値を求めればよいことになる.
%--------------------
\begin{figure}[h]
	\begin{center}
	\includegraphics[width=0.9\linewidth]{./fig9.eps} 
	\end{center}
	\caption{
		(a)トラス構造の一例(ワーレントラス), (b)トラス部材に作用する力. 
	} 
	\label{fig:fig12_9}
\end{figure}
%--------------------
\begin{figure}[h]
	\begin{center}
	\includegraphics[width=0.9\linewidth]{./fig6.eps} 
	\end{center}
	\caption{
		分布荷重を受ける骨組み構造.数字は部材番号を表す.
	} 
	\label{fig:fig12_6}
\end{figure}
%--------------------
\subsection{例題1}
図\ref{fig:fig12_6}-(a)に示すような,水平荷重を受ける正三角形の部材組について
部材1から3に生じる軸力を求める.はじめに,支点反力の正方向を図
\ref{fig:fig12_7}-(a)のように定める.これらは,構造全体の釣り合い条件から
\begin{equation}
	H_A=F, \ \ 
	R_A=-\frac{\sqrt{3}}{2}F, \ \ 
	R_B= \frac{\sqrt{3}}{2}F
\end{equation}
と求まる.次に,図\ref{fig:fig12_7}-(a)のa-a'にそって構造を切断し,
節点Cに加わる力を描き込むと,同図(b)のようになる.
ここで,$N_i$は部材$i(=1,2,3)$の軸力を意味し,$N_i>0$は引張,
$N_i<0$は圧縮の軸力を表す.$N_2$と$N_3$は,節点Cに作用する力の釣り合いより,
\begin{equation}
	N_2=F, \ \ N_3=-F
\end{equation}
決まる.  残る$N_1$は節点AあるいはBにおける力の釣り合いから
求めることができる.あるいは,図\ref{fig:fig12_7}-(a)のb-b'の位置で
構造を切断した結果として得られる部分構造を利用して$N_1$を決めることもできる
(図\ref{fig:fig12_7}-(c)).この場合,力の釣り合いだけでなく,
モーメントの釣り合い条件を利用することができる.
例えば,点Cに関するモーメントの釣り合い式を立てれば,
\begin{equation}
	N_1\times \frac{\sqrt{3}}{2}l -R_B\times \frac{l}{2}=0 \Rightarrow
	N_1=\frac{F}{2}
\end{equation}
となり,軸力$N_2$が未知でも$N_1$を直接求めることができる.
\begin{figure}[h]
	\begin{center}
	\includegraphics[width=0.9\linewidth]{./fig7.eps} 
	\end{center}
	\caption{
		図\ref{fig:fig12_6}-(a)の構造に対する軸力計算に用いた自由物体図.
	} 
	\label{fig:fig12_7}
\end{figure}
%--------------------
\subsection{例題2}
図\ref{fig:fig12_6}-(b)のようなワーレントラスに発生する軸力を, 
図\ref{fig:fig12_8}に示す自由物体図を用いて求める.
図\ref{fig:fig12_8}-(a)は,支点反力の正方向と,軸力計算に用いる自由物体図(b)$\sim$(e)を
描くための切断位置を示したものである.
これらの図を参照し,はじめに構造全体の釣り合い条件を用いれば
\begin{equation}
	H_A=0, \ \ R_A=R_C=\frac{F}{2}
\end{equation}
と反力が求まる.次に,a-a'で切断を行い,節点Aに関する力の釣り合いを考えれば,軸力$N_1,N_3$が
\begin{eqnarray}
	N_3\sin\frac{\pi}{3}+\frac{F}{2}=0 
	&\Rightarrow  & 
	N_3=-\frac{F}{\sqrt{3}} \\
	N_3\cos\frac{\pi}{3}+N_1=0
	&\Rightarrow  & 
	N_1=\frac{F}{2\sqrt{3}} 
\end{eqnarray}
と求められる.c-c'での切断に基づき,節点Cにおける力の釣り合いを考えれば,
ほとんど同じ計算で$N_2$と$N_6$が
%\end{document}
\begin{equation}
	N_6=-\frac{F}{\sqrt{3}}, \ \ N_2=\frac{F}{2\sqrt{3}} 
	\label{eqn:}
\end{equation}
と求まる.次に,切断b-b'によって得られる部分構造(図\ref{fig:fig12_8}-(d))に
着目すれば,節点Bに関するモーメントの釣り合いから
\begin{equation}
	N_7 \times l\sin\frac{\pi}{3}+\frac{F}{2}\times l =0 \Rightarrow 
	N_7=-\frac{F}{\sqrt{3}}
\end{equation}
と$N_7$が求まる.続いて,節点Cに関するモーメントの釣り合いを考えると,
\begin{equation}
	N_4 \times l\sin\frac{\pi}{3}- N_7 \times l\sin\frac{\pi}{3} - F\times l =0 \Rightarrow 
	N_4=\frac{F}{\sqrt{3}}
\end{equation}
となり,$N_4$が求められる.
最後に,d-d'での切断では,構造と載荷条件の対称性より,$N_4$と$N_7$を求めた直前の計算とほとんど同じ
計算により,$N_7$と$N_5$が次のように求められる.
\begin{equation}
	N_7=-\frac{F}{\sqrt{3}}
	, \ \ 
	N_5=\frac{F}{\sqrt{3}}
	\label{eqn:}
\end{equation}
勿論,ここで得られた$N_7$は,切断b-b'で既に明らかになった結果と一致する.\\

節点に作用する力の釣り合い条件は,水平方向と鉛直方向の2つの条件式を与える,
従って,その節点に作用する未知の荷重が2つ以下であれば,
節点に関する釣り合いからそれらの未知荷重を決定できる.
このことを踏まえて節点DとEを見ると,これらの節点には3本の部材が,
節点Bには4本の部材が結合されている.そのため,最初にこれらの節点の釣り合い式を
立てるだけでは未知の軸力を決定できない.
一方,切断b-b'やd-d'の結果として得られる部分構造では,力の釣り合い式に
モーメントの釣り合い式を加えた,合計3つの条件式が得られる.
従って,切断の結果として現れる未知の軸力の数が3以下であればそれらの軸力を決定できる.
さらに,モーメントの基準点を適切に選べば,いくつかの未知の軸力が
つりあい方程式に含まれないようにすることができる.その結果,未知数の合計
を3以下にすることができれば,切断面に現れる軸力の一部が決定できる.
上で示した軸力$N_7$の求め方は,このような方針に基づいて選択したものである.
\begin{figure}[h]
	\begin{center}
	\includegraphics[width=0.9\linewidth]{./fig8.eps} 
	\end{center}
	\caption{
	図\ref{fig:fig12_6}-(b)の構造に対する軸力計算に用いた自由物体図.
	} 
	\label{fig:fig12_8}
\end{figure}
\subsection{問題}
図\ref{fig:fig12_6}-(c)と(d)のトラス構造において発生する軸力を全てを求めよ.
\end{document}

\documentclass[10pt,a4j]{jbook}
%\usepackage{graphicx,wrapfig}
\usepackage{graphicx,amsmath}
\setlength{\topmargin}{-1.5cm}
%\setlength{\textwidth}{16.5cm}
\setlength{\textheight}{25.2cm}
\newlength{\minitwocolumn}
\setlength{\minitwocolumn}{0.5\textwidth}
\addtolength{\minitwocolumn}{-\columnsep}
%\addtolength{\baselineskip}{-0.1\baselineskip}
%
\def\Mmaru#1{{\ooalign{\hfil#1\/\hfil\crcr
\raise.167ex\hbox{\mathhexbox 20D}}}}
%
\begin{document}
\newcommand{\fat}[1]{\mbox{\boldmath $#1$}}
\newcommand{\D}{\partial}
\newcommand{\w}{\omega}
\newcommand{\ga}{\alpha}
\newcommand{\gb}{\beta}
\newcommand{\gx}{\xi}
\newcommand{\gz}{\zeta}
\newcommand{\vhat}[1]{\hat{\fat{#1}}}
\newcommand{\spc}{\vspace{0.7\baselineskip}}
\newcommand{\halfspc}{\vspace{0.3\baselineskip}}
\bibliographystyle{unsrt}
%\pagestyle{empty}
\newcommand{\twofig}[2]
 {
   \begin{figure}
     \begin{minipage}[t]{\minitwocolumn}
         \begin{center}   #1
         \end{center}
     \end{minipage}
         \hspace{\columnsep}
     \begin{minipage}[t]{\minitwocolumn}
         \begin{center} #2
         \end{center}
     \end{minipage}
   \end{figure}
 }
%%%%%%%%%%%%%%%%%%%%%%%%%%%%%%%%%
%\vspace*{\baselineskip}
%%%%%%%%%%%%%%%%%%%%%%%%%%%%%%%%%%%%%%%%%%%%%%%%%%%%%%%%%%%%%%%%
\setcounter{chapter}{6}
\chapter{複数径間に渡る梁の曲げ問題}
\section{張出し梁のたわみと断面力}
図\ref{fig:fig10_1}に示すような,支点から張出した部分を持つ梁を{\bf 張出し梁}と呼ぶ.
張出し梁のたわみ,たわみ角,断面力は,これまでに学んだ方法を用いて求めることができる.
その手順を例題を用いて以下に説明する.
\subsection{例題}
図\ref{fig:fig10_1}-(a)に示す張出し梁のたわみと断面力を求める.
点Aを原点として右向きを正とする座標$x$を用いるとき,たわみを求めるべき
区間は$0<x<l$である.この区間で梁に作用する外力は,等分布荷重と
支点Bからの反力$R_B$である.従って,図\ref{fig:fig10_1_1}のように
支点反力の正方向をとれば,外力項$q(x)$は
\begin{equation}
	q(x)=q_0-R_B\delta\left(x-\frac{l}{2}\right)
	\label{eqn:qx}
\end{equation}
と表される.この4回の不定積分は
\begin{equation}
	\iiiint q(x) dx^4= \frac{q_0}{24}x^4-\frac{R_B}{6}\left<x-\frac{l}{2}\right>^3
	\label{eqn:int_qx4}
\end{equation}
だから,$\xi=\frac{x}{l}$として,たわみを
\begin{equation}
	v(x)=\frac{q_0l^4}{24EI}\left( \xi^4+C_1\xi^3 +C_2\xi^2+C_3\xi+C_4\right)
	-
	\frac{R_Bl^3}{6EI}\left< \xi-\frac{1}{2}\right>^3
	\label{eqn:vx_gen}
\end{equation}
と書くことができる.これに,支点AとBおよびCにおける条件:
\begin{eqnarray}
	x=0 &;& v(0)=0, \, M(0)=0 
	\\
	x=\frac{l}{2} &;& v\left( \frac{l}{2} \right)=0
	\\
	x=l &;& M(l)=0,\, Q(l)=0
\end{eqnarray}
を考慮することで,積分定数$C_1\sim C_4$と支点反力$R_B$が以下のように求められる.
\begin{equation}
	R_B=q_0l, \ \ C_1=C_2=C_4=0, \ \ C_3=-\frac{1}{8}
\end{equation}
その結果,たわみが
\begin{equation}
	v(x)=\frac{q_0l^4}{24EI}\left( \xi^4-\frac{1}{8}\xi\right)
	-
	\frac{q_0l^4}{6EI}\left< \xi-\frac{1}{2}\right>^3
	\label{eqn:vx}
\end{equation}
となり,これを微分して曲げモーメントとせん断力が次のように得られる.
以上より,せん断力図と曲げモーメント図は図\ref{fig:fig10_1_1}-(b)と(c)のようになる.
\begin{eqnarray}
	M(x)&=&
	q_0l^2 \left\{ -\frac{1}{2}\xi^2 + \left< \xi -\frac{1}{2}\right> \right\}
	\\
	Q(x)&=&
	q_0l \left\{-\xi+H\left(\xi-\frac{1}{2}\right)\right\}
\end{eqnarray}
なお,ここで考えている問題は静定問題であり,釣り合い条件から支点反力を求め,
その結果($R_A=0,\, R_B=q_0l$)から断面力を決定することもできる.
たわみを求める必要が無い場合,釣り合い条件から断面力を求めることが多くの場合容易である.
また,たわみを求める必要がある場合も,支点反力$R_B$を釣り合い条件から最初に決定し,
式(\ref{eqn:vx_gen})にその結果を予め代入しておけば,残る作業は4つの積分定数を決定することに
なり計算の手間は軽減できる.
\subsection{問題}
図\ref{fig:fig10_1}-(b)$\sim$(f)に示した張出し梁について,たわみと断面力,支点反力を求めよ.
%--------------------
\begin{figure}[h]
	\begin{center}
	\includegraphics[width=0.7\linewidth]{fig10_1.eps} 
	\end{center}
	\caption{
	鉛直荷重を受ける張出し梁(a)$\sim$(f).
	} 
	\label{fig:fig10_1}
\end{figure}
%--------------------
\begin{figure}[h]
	\begin{center}
	\includegraphics[width=0.4\linewidth]{fig10_1_1.eps} 
	\end{center}
	\caption{
	(a)支点反力の正方向と(b),(c)断面力図.
	} 
	\label{fig:fig10_1_1}
\end{figure}
%
%%%%%%%%%%%%%%%%%%%%%%%%%%%%%%%%%%%%%%%%%%%%%%%%%%%%
%
%
\section{連続梁のたわみと断面力}
互いに隣接した支点間のことを{\bf 径間}あるいは{\bf 支間}と呼ぶ.例えば,
図\ref{fig:fig10_2}-(a)に示す梁は,径間ABと径間BCの二径間で構成される構造である.
2つの径間は連続した単一の梁で架橋されているため,このような構造は{\bf 二径間連続梁}と呼ばれる.
同様に,連続した梁で二つ以上の径間が架橋された構造は{\bf 多径間連続梁}と呼ばれる.
多径間連続梁は不静定構造だが,張出し梁の曲げ問題と同様にして,たわみや断面力,
支点反力を求めることができる.以下では,そのことを例題の解答を通じて示す.
\subsection{例題}
図\ref{fig:fig10_2}-(a)に示すような二径間連続梁が受ける支点反力と断面力分布を求め,断面力図を描く.
\begin{enumerate}
\item
荷重条件は前節の例題と同じであるため,たわみを式(\ref{eqn:vx_gen})で表すことができる.
ただしこの場合,\underline{力とモーメントの釣り合いから予め反力$R_B$を決めておくことはできない}.
支持条件は
\begin{eqnarray}
	x=0 &;& v(0)=0, \, M(0)=0 
	\label{eqn:BC_0}
	\\
	x=\frac{l}{2} &;& v\left( \frac{l}{2} \right)=0
	\label{eqn:BC_l2}
	\\
	x=l &;& v(l)=0, \, M(l)=0
	\label{eqn:BC_l}
\end{eqnarray}
であるから,これら5つの条件から$R_B$と積分定数を決定すれば,
\begin{equation}
	R_B=\frac{5}{8}q_0l, \ \ C_1=-\frac{3}{4}, \ \ C_2=C_4=0, \ \ C_3=\frac{1}{16}
\end{equation}
となる.これを式(\ref{eqn:vx_gen})に代入すると,たわみは
\begin{equation}
	v(x)=\frac{q_0l^4}{24EI}\left( \xi^4-\frac{3}{4}\xi^3 +\frac{1}{16}\xi \right)
	-
	\frac{5q_0l^4}{48EI}\left< \xi-\frac{1}{2}\right>^3
\end{equation}
となる.よって,曲げモーメントとせん断力は
\begin{eqnarray}
	M(x) &= & 
		-\frac{q_0l^2}{8}\left(4\xi^2 -\frac{3}{2}\xi \right)
		+
		\frac{5}{8}q_0l^2\left< \xi-\frac{1}{2}\right>
	\\
	Q(x) &= & 
		-\frac{q_0l}{8}\left(8\xi -\frac{3}{2} \right)
		+
		\frac{5}{8}q_0lH\left( \xi-\frac{1}{2}\right)
\end{eqnarray}
と得られ,これらを断面力図として示すと図\ref{fig:fig10_2_1}のようになる.
\item
{\bf (別解)}
本質的には上の解法と同じだが,以下のようにして反力やたわみの計算をしてもよい.

図\ref{fig:fig10_2_2}に示すように,与えられた梁(a)を二つの静定梁(b)と(c)の重ね合わせで表現する.
すなわち,梁(a),(b),(c)のたわみをそれぞれ,$v_a(x),\,v_b(x),\,v_c(x)$と書き,
\begin{equation}
	v_a(x)=v_b(x)+v_c(x)
	\label{eqn:v_sum}
\end{equation}
と表す.ここで,$R_B$を梁(a)の支点Bにおける鉛直反力とし,$R_B$は式(\ref{eqn:BC_l2})で与えられる
拘束条件を満足するように決まると考える. 
$v_b$や$v_c$は,基本的な単径間梁のたわみであり,比較的簡単に求めることができる.
実際に計算を行うと
\begin{eqnarray}
	v_b(x)&= & 
		\frac{q_0l^4}{24EI} \left( \xi^4-2\xi^3 +\xi \right)
	\label{eqn:vbx}
	\\
	v_c(x)&= & 
		-\frac{R_Bl^3}{6EI} \left\{ \left< \xi-\frac{1}{2}\right>^3 
		-\frac{1}{2}\xi^3 +\frac{3}{8}\xi \right\}
	\label{eqn:vcx}
\end{eqnarray}
となる.そこで,式(\ref{eqn:vbx}),(\ref{eqn:vcx})を利用して,
$v_a(x)=v_b(x)+v_c(x)$で表される梁(a)のたわみが満足すべき
支持条件(\ref{eqn:BC_0})-(\ref{eqn:BC_l})から$R_B$を決定する.
梁(b)と梁(c)が支点AとCで満足すべき条件は,梁(a)と同様,
式(\ref{eqn:BC_0})と式(\ref{eqn:BC_l})で表されるので,式(\ref{eqn:vsum})は
任意の$R_B$に対して点AとCにおける条件を満足する.
一方,支点Bにおける条件は特別な$R_B$についてのみ満たされるため,
\begin{equation}
	v_a\left(\frac{l}{2}\right)
	=
	v_b\left(\frac{l}{2}\right)
	+
	v_c\left(\frac{l}{2}\right)
	=0
	\label{eqn:constraint}
\end{equation}
から$R_B$を決めることができる.そこで,式(\ref{eqn:constraint})に,
式(\ref{eqn:vbx})と(\ref{eqn:vcx})で与えられる
$v_b\left(\frac{l}{2}\right)$と$v_c\left(\frac{l}{2}\right)$を代入すると,
\begin{equation}
	R_B=\frac{5}{8}q_0l
\end{equation}
と,上で求めた結果と同じ$R_B$が得られる.この結果を$v_c$の$R_B$に代入して,$v_b$と$v_c$の
和をとれば,求めるべきたわみ$v_a(x)$得られる.
なお,$R_B$が既知であれば,他の反力$R_A$と$R_C$は釣り合い式から求められる.
このようなケースでは$R_B$を得た後,釣り合い式を用いて断面力を計算する方が,
式(\ref{eqn:vbx})と式(\ref{eqn:vcx})をそれぞれ微分して断面力を求め,マッコーレー括弧を含む式を解釈して
断面力図を描くよりも容易な場合があることを覚えておくとよい.
\end{enumerate}
\subsection{問題}
図\ref{fig:fig10_2}-(b)$\sim$(d)に示す構造について全ての支点反力を求め,断面力図を描け.
%--------------------
\begin{figure}[h]
	\begin{center}
	\includegraphics[width=0.75\linewidth]{fig10_2.eps} 
	\end{center}
	\caption{
		二径間連続梁を始めとする単純な不静定梁.
	} 
	\label{fig:fig10_2}
\end{figure}
\begin{figure}[h]
	\begin{center}
	\includegraphics[width=0.45\linewidth]{fig10_2_1.eps} 
	\end{center}
	\caption{
		等分布荷重を受ける二径間連続梁の断面力図.
	} 
	\label{fig:fig10_2_1}
\end{figure}
\begin{figure}[h]
	\begin{center}
	\includegraphics[width=0.45\linewidth]{fig10_2_2.eps} 
	\end{center}
	\caption{
		静定系の重ね合わせによる二径間連続梁の表現.
	} 
	\label{fig:fig10_2_2}
\end{figure}
%%%%%%%%%%%%%%%%%%%%%%%%%%%%%%%%%%%%%%%%
\section{集中モーメント}
図\ref{fig:fig10_4}のような大きさが等しく向きが反対の集中荷重の対(偶力)について考える.
各々の集中荷重の大きさ$F$は等しく,作用点は$x=a$の位置を中心として互いに$\varepsilon$だけ
離れているとする.このとき偶力は反時計周りの方向に$M_0=F\varepsilon$のモーメント
を生じさせる.ここで,$M_0$を一定に保ちながら$\varepsilon \rightarrow 0$の極限をとる.
そのような極限では,方向が反対で大きさ無限大の力が同じ位置に作用する.そのため,
集中荷重を区別して扱うことはできない.
このようにして加えられたモーメントは{\bf 集中モーメント}と呼ばれる
.集中荷重はディラクのデルタ関数を使って表すことができる.
そこで,偶力を構成する力を大きさ$M_0/\varepsilon$,作用点位置$x=a\pm\frac{\varepsilon}{2}$の
デルタ関数で表し$\varepsilon \rightarrow 0$の極限をとれば,
\begin{equation}
	\lim_{\varepsilon \rightarrow 0} 
	\left\{
	M_0\frac{
		\delta\left(x-a+\frac{\varepsilon}{2}\right)
		-
		\delta\left(x-a-\frac{\varepsilon}{2}\right)
	}{\varepsilon}
	\right\}
	=M_0 \delta '\left( x-a \right)
\end{equation}
と,集中モーメントをデルタ関数の微分で表現できることが分かる.従って,梁の曲げ問題において外力項を
$q(x)=M_0\delta'\left(x-a\right)$と置けば,$x=a$に大きさ$M_0$の反時計回りの集中モーメントが
加えられたときのたわみや断面力を調べることができる.
\subsection{例題}
図\ref{fig:fig10_3}-(a)に示すような支間中央部で集中モーメントを受ける単純支持梁について支点反力と
断面力分布を求め,断面力図を描く.外力は
\begin{equation}
	q(x)=M_0\delta'\left(x-\frac{l}{2}\right)
\end{equation}
と表され,その4回の積分は
\begin{equation}
	\iiiint q(x)dx^4 =\frac{1}{2}\left< x-\frac{l}{2}\right>^2
\end{equation}
であることから,たわみを
\begin{equation}
	v(x)=\frac{q_0l^2}{2EI}\left\{ \left< \xi -\frac{1}{2} \right>^2+C_1\xi^3 +C_2\xi^2+C_3\xi+C_4\right\}, \ \ 
	\left(\xi=\frac{x}{l}\right)
\end{equation}
と表すことができる.これに単純支持条件:
\begin{equation}
	v(0)=v(l)=0, \ \ M(0)=M(l)=0
\end{equation}
を課せば,積分定数が
\begin{equation}
	C_1=-\frac{1}{3}, \ \ C_2=C_4=0, \ \ C_3=\frac{1}{12}
\end{equation}
と決まる.その結果,たわみ$v(x)$が次のように求められる.
\begin{equation}
	v(x)=\frac{q_0l^2}{2EI}\left\{ \left< \xi -\frac{1}{2} \right>^2-\frac{1}{3}\xi^3 +\frac{1}{12}\xi \right\}
\end{equation}
梁は単純支持されているため,反力,断面力とも力とモーメントの釣り合いから決定する
ことができるが,ここでは,上で求めたたわみを微分して,曲げモーメントとせん断力を
求めると,次に示すようになる.
\begin{eqnarray}
	M(x) &= & 
		M_0\left\{ \xi -H \left(\xi-\frac{1}{2}\right) \right\} \\
	Q(x) &= & 
		\frac{M_0}{l}\left\{1 -\delta\left(\xi-\frac{1}{2}\right)  \right\}
\end{eqnarray}
せん断力の表現にはデルタ関数が含まれているが,
この場合$x=\frac{l}{2}$では$\delta\left( \xi-\frac{1}{2}\right)$の関数値は定義されず,
$x\neq \frac{l}{2}$ではゼロである.
また,このような点が生じるのは,曲げモーメント$M(x)$が$x=\frac{l}{2}$で不連続
となり微分できないためである.以上のことに注意して,断面力図を描くと図\ref{fig:fig10_3_1}
のようになる.
\subsection{問題}
\begin{enumerate}
\item
図\ref{fig:fig10_3}-(b)$\sim$(d)に示す構造について,たわみ,断面力および支点反力を求めよ.
断面力図を描け.
\item
図\ref{fig:fig10_3}-(e)$\sim$(f)に示す構造について,点Aと点Bにおけるたわみ角を求めよ.
\end{enumerate}
\begin{figure}[h]
	\begin{center}
	\includegraphics[width=0.4\linewidth]{fig10_4.eps} 
	\end{center}
	\caption{
		大きさ$M_0$の偶力.$\varepsilon\rightarrow 0$の極限において
		大きさ$M_0$の集中モーメントの生成する.
	} 
	\label{fig:fig10_4}
\end{figure}
\begin{figure}[h]
	\begin{center}
	\includegraphics[width=0.7\linewidth]{fig10_3.eps} 
	\end{center}
	\caption{
		集中モーメントを受ける単径間梁.
	} 
	\label{fig:fig10_3}
\end{figure}
\begin{figure}[h]
	\begin{center}
	\includegraphics[width=0.5\linewidth]{fig10_3_1.eps} 
	\end{center}
	\caption{
		支間中央の点において集中モーメントを受ける単純支持梁の断面力図.
	} 
	\label{fig:fig10_3_1}
\end{figure}
%--------------------
\end{document}

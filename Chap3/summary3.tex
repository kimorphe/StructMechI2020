\documentclass[10pt,a4j]{jbook}
%\usepackage{graphicx,wrapfig}
\usepackage{showkeys}
\usepackage{graphicx}
\setlength{\topmargin}{-1.5cm}
%\setlength{\textwidth}{16.5cm}
\setlength{\textheight}{25.2cm}
\newlength{\minitwocolumn}
\setlength{\minitwocolumn}{0.5\textwidth}
\addtolength{\minitwocolumn}{-\columnsep}
%\addtolength{\baselineskip}{-0.1\baselineskip}
%
\def\Mmaru#1{{\ooalign{\hfil#1\/\hfil\crcr
\raise.167ex\hbox{\mathhexbox 20D}}}}
%
\begin{document}
\newcommand{\fat}[1]{\mbox{\boldmath $#1$}}
\newcommand{\D}{\partial}
\newcommand{\w}{\omega}
\newcommand{\ga}{\alpha}
\newcommand{\gb}{\beta}
\newcommand{\gx}{\xi}
\newcommand{\gz}{\zeta}
\newcommand{\vhat}[1]{\hat{\fat{#1}}}
\newcommand{\spc}{\vspace{0.7\baselineskip}}
\newcommand{\halfspc}{\vspace{0.3\baselineskip}}
\bibliographystyle{unsrt}
%\pagestyle{empty}
\newcommand{\twofig}[2]
 {
   \begin{figure}
     \begin{minipage}[t]{\minitwocolumn}
         \begin{center}   #1
         \end{center}
     \end{minipage}
         \hspace{\columnsep}
     \begin{minipage}[t]{\minitwocolumn}
         \begin{center} #2
         \end{center}
     \end{minipage}
   \end{figure}
 }
%%%%%%%%%%%%%%%%%%%%%%%%%%%%%%%%%
%\vspace*{\baselineskip}
%%%%%%%%%%%%%%%%%%%%%%%%%%%%%%%%%%%%%%%%%%%%%%%%%%%%%%%%%%%%%%%%
\setcounter{chapter}{2}
\chapter{ひずみテンソルと一般化されたフックの法則}
以下, 太字は$o-x_1x_2x_3$直角直交座標系による成分表示を意図としたベクトルや行列を表す.
また,$(\cdot)^T$は行列やベクトルの転置を表す.
\section{変位ベクトル}
変位とは,物体の変形に伴う物質点の移動量を意味する.
2次元あるいは3次元的な形状をした物体や一般の荷重条件の元で生じる変形では,
変位は大きさと向きをもつベクトル量として記述する必要がある.
そこで,物体内の点$\fat{x}=(x_1,x_2,x_3)^T$における変位を表すベクトルを
\begin{equation}
	\fat{u}(\fat{x})=\left( u_1, u_2, u_3 \right)^T
	\label{eqn:uvec}
\end{equation}
と書く.変位ベクトルの成分$u_i,(i=1,2,3)$も位置に依存する.
そのことを明示する場合には
\begin{equation}
	u_i=u_i(\fat{x})=u_i\left(x_1, x_2, x_3\right), \ \ (i=1,2,3)
	\label{eqn:uvec_compo}
\end{equation}
と表す.ただし,引数は明示する必要があるものだけを書くこととし,例えば,
第三引数$x_3$に関する変位の分布を問題にしない場合には,
式(\ref{eqn:uvec_compo})を$u_i=u_i(x_1,x_2)$のように,引数から$x_3$を省略した形で書く.
以下では,変位ベクトル成分$u_i,\,(i=1,2,3)$を用いてひずみテンソルを定義する.
なお,ひずみとは物体内各点の変形量を記述する無次元の量である.
%
\section{微小ひずみテンソル}
ここでは,3次元問題における微小ひずみテンソルを定義する.ただし,説明のための図が
煩雑になることを避けるために,以下では2次元的な図を参照しながらひずみテンソル
を成分毎に定義する.
\subsection{直ひずみ}
物体中の着目点$\fat{x}$を含む,幅$\Delta x_1$,高さ$\Delta x_2$の微小矩形領域をとる(図\ref{fig:fig0}).
物体が外力を受けるとき,この微小領域の各面には何らかの
応力$\fat{\sigma}=\{\sigma_{ij}\}$が作用する.その結果として,この微小矩形領域では
応力成分$\sigma_{ij}$それぞれに起因した変形が発生する.
\begin{figure}[h]
	\begin{center}
	\includegraphics[width=0.7\linewidth]{fig0.eps} 
	\end{center}
	\caption{物体内部の着目点$\fat{x}$のまわりにとった微小矩形領域.} 
	\label{fig:fig0}
\end{figure}
例えば,$\sigma_{11}$が作用することにより,$x_1$軸方向に図\ref{fig:fig3_1}-(a)のような伸びが生じる.
伸び(縮み)に関するひずみは,軸力問題の場合と同様に無限小領域の伸び率として与えることができる.
ただし,この場合の伸びは$x_1$軸方向の変位$u_1(x_1,x_2)$で表されるため,ひずみは
\begin{equation}
	\varepsilon_{11}=\lim_{\Delta x_1,\Delta x_2 \rightarrow 0}
	\frac{u_1(x_1+\Delta x_1,x_2)-u_1(x_1,x_2)}{\Delta x_1}
	=\frac{\partial u_1}{\partial x_1}
	\label{eqn:def_e11}
\end{equation}
と$x_1$に関する偏微分として定義される.ここで,$\varepsilon_{11}$のインデックス11は,
応力テンソル成分$\sigma_{11}$との対応を想起させるためのものである.
同じ観点から,$x_2$軸と$x_3$軸それぞれの方向への伸び縮みを表すひずみは
$\varepsilon_{22},\varepsilon_{33}$と書くことにする.
図\ref{fig:fig3_1}-(b)に示すように,$\varepsilon_{22}$には変位$u_2(x_1,x_2)$が関係するため,
$\varepsilon_{22}$は以下のように定義する.
\begin{equation}
	\varepsilon_{22}=\lim_{\Delta x_1,\Delta x_2 \rightarrow 0}
	\frac{u_2(x_1,x_2+\Delta x_2)-u_2(x_1,x_2)}{\Delta x_2}
	=\frac{\partial u_2}{\partial x_2}
	\label{eqn:def_e22}
\end{equation}
以上の考え方に従えば,$x_3$軸方向のひずみが
\begin{equation}
	\varepsilon_{33}=\frac{\partial u_3}{\partial x_3}
\end{equation}
で定義されるべきであることは明らかであろう.
\subsection{せん断ひずみ}
せん断応力成分は$\sigma_{12}=\sigma_{21}$であることから,$\sigma_{12}$と$\sigma_{21}$を
対にして考える必要がある.しかしながら,もし一方だけが作用するならば,その際に
微小矩形領域に生じる変形は図\ref{fig:fig3_1}-(c)と(d)のようなものと考えてよい.
このような変形のモード(様式)は{\bf せん断変形}あるいは{\bf ずり変形}と呼ばれる.
せん断変形量は,図\ref{fig:fig3_1}-(c),(d)に示した角度$\alpha$と$\beta$を
用いれば,無次元の量で数値化できる.
ただし,実際には(c)と(d)に示した変形が同時に生ずるため,せん断変形を表すひずみ
$\varepsilon_{12}=\varepsilon_{21}$を
$\alpha$と$\beta$の平均:
\begin{equation}
	\varepsilon_{12}=\varepsilon_{21}=\frac{\alpha+\beta}{2}
	\label{eqn:e12_ab}
\end{equation}
で定義する.ここでも,インデックス12と21は対応するせん断応力成分$\sigma_{12}=\sigma_{21}$を
想起させることを意図したものである.
$\alpha$と$\beta$は,変位ベクトル成分を用いて
\begin{equation}
	\alpha=
	\lim_{\Delta x_1,\Delta x_2 \rightarrow 0}
	\frac{u_2(x_1+\Delta x_1,x_2)-u_2(x_1,x_2)}{\Delta x_1}
	=\frac{\partial u_2}{\partial x_1}
	\label{eqn:alpha}
\end{equation}
\begin{equation}
	\beta=
	\lim_{\Delta x_1,\Delta x_2 \rightarrow 0}
	\frac{u_1(x_1,x_2+\Delta x_2)-u_1(x_1,x_2)}{\Delta x_2}
	=\frac{\partial u_1}{\partial x_2}
	\label{eqn:beta}
\end{equation}
と表される.$\Delta x_1,\Delta x_2\rightarrow 0$の極限をとる理由は,
局所量としてせん断ひずみを定義するためである.式(\ref{eqn:alpha})と(\ref{eqn:beta})を
式(\ref{eqn:e12_ab})に代入すれば,
\begin{equation}
	\varepsilon_{12}=
	\varepsilon_{21}=
	\frac{1}{2}
	\left(
	\frac{\partial u_2}{\partial x_1}
	+
	\frac{\partial u_1}{\partial x_2}
	\right)
	\label{eqn:def_e12}
\end{equation}
となる.\\

以上の議論は物体の$x_1-x_2$平面内における断面を考えて行ったものである.そこで,
同様な議論を$x_2-x_3$平面と$x_3-x_1$平面内それぞれで行えば,
それらの平面内におけるせん断ひずみ$\varepsilon_{23}(=\varepsilon_{32})$と
$\varepsilon_{31}(=\varepsilon_{13})$を
\begin{equation}
	\varepsilon_{23}=
	\varepsilon_{32}=
	\frac{1}{2}
	\left(
	\frac{\partial u_2}{\partial x_3}
	+
	\frac{\partial u_3}{\partial x_2}
	\right)
	\label{eqn:def_e23}
\end{equation}
\begin{equation}
	\varepsilon_{31}=
	\varepsilon_{13}=
	\frac{1}{2}
	\left(
	\frac{\partial u_3}{\partial x_1}
	+
	\frac{\partial u_1}{\partial x_3}
	\right)
	\label{eqn:def_e31}
\end{equation}
と定義すればよいことは明らかである.
\subsection{ひずみテンソル}
これまでに定義したひずみ$\varepsilon_{ij}$は,一括して
\begin{equation}
	\varepsilon_{ij}=\frac{1}{2}
	\left(
	\frac{\partial u_i}{\partial x_j}
	+
	\frac{\partial u_j}{\partial x_i}
	\right), \ \ (i,j=1,2,3)
	\label{eqn:def_eij}
\end{equation}
と表すことができる.これらの$\varepsilon_{ij}$の全てを$3\times 3$行列として並べた
\begin{equation}
	\fat{\varepsilon}=\left\{ \varepsilon_{ij}\right\}
	=\left(
	\begin{array}{ccc}
		\varepsilon_{11} & \varepsilon_{12} & \varepsilon_{13}\\
		\varepsilon_{21} & \varepsilon_{22} & \varepsilon_{23}\\
		\varepsilon_{31} & \varepsilon_{32} & \varepsilon_{33}
	\end{array}
	\right)
	\label{eqn:def_emat}
\end{equation}
は{\bf ひずみテンソル}と呼ばれる.$\fat{\varepsilon}$は定義上対称行列となり,
対角項成分$\varepsilon_{11},\varepsilon_{22},\varepsilon_{33}$は
伸び縮みに関するひずみを表し{\bf 軸ひずみ}あるいは{\bf 直ひずみ}と総称される.
これに対して,非対角項成分$\varepsilon_{12}=\varepsilon_{21},\varepsilon_{23}=\varepsilon_{32},
\varepsilon_{31}=\varepsilon_{13}$は{\bf せん断ひずみ}と総称される.

なお,せん断ひずみ$\varepsilon_{ij},(i\neq j)$を二倍した
\begin{equation}
	\gamma_{ij}=2\varepsilon_{ij}
	=
	\left(
	\frac{\partial u_2}{\partial x_1}
	+
	\frac{\partial u_1}{\partial x_2}
	\right), \ \ (i\neq j)
	\label{eqn:def_g12}
\end{equation}
は,{\bf 工学せん断ひずみ}と呼ばれる.
\subsection{座標変換法則}
これまでの議論で用いた$o-x_1x_2x_3$座標系に加え,もう一つの直交座標系$o-x_1'x_2'x_3'$を
導入する.$o-x_1'x_2'x_3'$座標系におけるひずみテンソル$\fat{\varepsilon}'$は,
\begin{equation}
	\varepsilon_{ij}'=\frac{1}{2}
	\left(
	\frac{\partial u_i'}{\partial x_j'}
	+
	\frac{\partial u_j'}{\partial x_i'}
	\right), \ \ (i,j=1,2,3)
	\label{eqn:def_eijd}
\end{equation}
を成分とする3$\times$3行列となる.ここで$o-x_1x_2x_3$から$o-x_1'x_2'x_3'$への
座標変換行列を$\fat{Q}$とすれば,$\fat{\varepsilon}$と$\fat{\varepsilon}'$の間で
成り立つの座標変換則は
\begin{equation}
	\fat{\varepsilon}'=\fat{Q}\fat{\varepsilon}\fat{Q}^T, \ \ 
	\fat{\varepsilon}=\fat{Q}^T\fat{\varepsilon}'\fat{Q} 
	\label{eqn:eij2eijd}
\end{equation}
で与えられ,応力テンソルの場合と同じ形式となる.
式(\ref{eqn:eij2eijd})の座標変換則は,変位ベクトルに関する座標変換法則と偏微分に関する
連鎖公式を用いて証明することができる.
%%%%%%%%%%%%%%%%%%%%%%
\begin{figure}[h]
	\begin{center}
	\includegraphics[width=0.8\linewidth]{strain.eps} 
	\end{center}
	\caption{応力成分に応じて微小矩形領域に発生する変形の状態.破線は変形前の状態を,
	実線は変形後の形状を表す.} 
	\label{fig:fig3_1}
\end{figure}
\newpage
\subsection{問題}
2次元空間中の物体がある応力を受けて変形し, そのときの変位ベクトル$\fat{u}=(u_1, \, u_2)$が
$o-x_1x_2$直交座標系において,位置ベクトル$\fat{x}=(x_1, x_2)$の関数として次のように与えられたとする.
\begin{equation}
	\fat{u}
	=
	\fat{F}\fat{x}
	, \ \ 
	\fat{F}=
	\frac{1}{5}
	\left( 
		\begin{array}{cc}
		 0 & 1\\
		 1 & 0 
		\end{array}
	\right)
	\label{eqn:defo}
\end{equation}
このとき以下の問に答えよ.
\begin{enumerate}
\item
	ひずみテンソル$\fat{\varepsilon}$の各成分を求めよ.
\item
	変形以前に図\ref{fig:fig6_1}-(a)の正方形領域ABCDにあった物体は,
	変形後どのような領域に移るか図示せよ.
\item
	式(\ref{eqn:defo})における行列$\fat{F}$の固有値$\lambda_1, \lambda_2$と, 
	それに対応する固有ベクトル$\fat{n}_1$および$\fat{n}_2$を求めよ.	
	なお,固有ベクトルは単位ベクトルとして表し,$\fat{n}_1$は第1象限を,$\fat{n}_2$は
	第2象限を向くように選ぶこと.
\item
	図\ref{fig:fig6_1}-(b)に示すように,固有ベクトル$\fat{n}_1$と$\fat{n}_2$を正方向
	とする$o-x_1'x_2'$座標系を考える.$o-x_1x_2$から$o-x_1'x_2'$への座標変換マトリクス
	$\fat{Q}$を求めよ.
\item
	$o-x_1'x_2'$座標系における変位ベクトルを$\fat{u}'$, 位置ベクトルを$\fat{x}'$とする.
	これらのベクトルは2$\times$2行列$\fat{F}'$を用いて
	\[
		\fat{u}'=\fat{F}'\fat{x}'
	\]
	と表すことができる.このような行列$\fat{F}'$を求めよ.
\item
	変形前に$o-x_1'x_2'$座標系で, 図\ref{fig:fig6_1}-(b)のように表される正方形PQRSは,
	物体の変形に伴いどのような領域に移るか図示せよ.
\item
	$o-x_1'x_2'$座標系におけるひずみテンソル$\fat{\varepsilon}'$の成分
	$\varepsilon'_{ij}, \, (i,j=1,2)$を求めよ.
\end{enumerate}
%--------------------
\begin{figure}[h]
	\begin{center}
	\includegraphics[width=0.8\linewidth]{fig1.eps} 
	\end{center}
	\caption{(a)$o-x_1x_2$座標系と, (b)固有ベクトル$\fat{n}_1,\fat{n}_2$を基底ベクトルにもつ$o-x_1'x_2'$座標系.} 
	\label{fig:fig6_1}
\end{figure}
%--------------------
\subsubsection{解答}
{\small
\begin{enumerate}
\item
変位ベクトと位置ベクトルの関係は
\begin{equation}
	\left( 
		\begin{array}{cc}
		 u_1 \\
		 u_2 
		\end{array}
	\right)
	=
	\frac{1}{5}
	\left( 
		\begin{array}{cc}
		 0 & 1\\
		 1 &0 
		\end{array}
	\right)
	\left( 
		\begin{array}{c}
		 x_1 \\
		 x_2 
		\end{array}
	\right)
	=
	\frac{1}{5}
	\left( 
		\begin{array}{cc}
		 x_2 \\
		 x_1
		\end{array}
	\right)
\end{equation}
より,ひずみテンソルの各成分は,
\begin{eqnarray}
	\varepsilon_{11} &=& \frac{\partial u_1}{\partial x_1} = 0 \\
	\varepsilon_{12} &=& \varepsilon_{21} = \frac{1}{2} \left({ \frac{\partial u_1}{\partial x_2} 
		+ \frac{\partial u_2}{\partial x_1}  }\right) 
		= \frac{1}{5}\\
	\varepsilon_{22} &=& \frac{\partial u_2}{\partial x_2} =0
\end{eqnarray}
となる.
\item
	正方形ABCDの各頂点の座標は,
\begin{equation}
	A = (0,0) ,\  B = ( 1,0) ,\  C = (1, 1) ,\  D = (0, 1)
\end{equation}
である.これらの点における変位ベクトルを
$\fat{u}_A ,\  \fat{u}_B, \  \fat{u}_C, \  \fat{u}_D$とすると,
\begin{eqnarray}
 \left[\fat{u}_A \  \fat{u}_B \  \fat{u}_C \  \fat{u}_D \right] 
 &=& 
 	\frac{1}{5}
	\left( 
		\begin{array}{cc}
		 0 & 1 \\
		 1 & 0 
		\end{array}
	\right)
	\left( 
		\begin{array}{cccc}
		 0 &  1  &  1  &  0  \\
		 0 &  0  &  1  &  1
		\end{array}
	\right) \\
\nonumber
	&=&
	\frac{1}{5}
	\left( 
		\begin{array}{cccc}
		 0 &  0  &  1  &  1  \\
		 0 &  1  &  1  &  0 
		\end{array}
	\right)
\end{eqnarray}
である.よって, 変形後の各頂点の座標をA',B',C',D'とすれば,
\begin{equation}
	A' = \left( 0, 0\right) ,\  
	B' = \left(1, \frac{1}{5}\right) ,\  
	C' = \left(\frac{6}{5},\frac{6}{5} \right) ,\  
	D' = \left(\frac{1}{5}, 1 \right)
\end{equation}
となる.以上より,正方形ABCDは図\ref{fig:fig1}(a)のような四辺形A'B'C'D'へ移される.
%--------------------
\item
行列$\fat{F}$の固有値$\lambda$は
\begin{equation}
	det\ (\fat{F} - \lambda \fat{I}) =
	\left|
		\begin{array}{cc}
		 -\lambda & \frac{1}{5} \\
		 \frac{1}{5} & -\lambda 
		\end{array}
	\right|
	= \left( -\lambda\right)^2- \left(\frac{1}{5}\right)^2=0
\end{equation}
より$\lambda=\frac{1}{5},-\frac{1}{5}$である.$\lambda=\frac{1}{5}$のときの固有ベクトル$\fat{n}$は,
$k$を任意の実数として$\fat{n}=k(1,\,1)$と表される.
一方$\lambda=-\frac{1}{5}$のときの固有ベクトルは$\fat{n}=k(-1,\, 1)$である.
よって,
\begin{equation}
	\lambda_1=\frac{2}{5}, \ \  
	\fat{n}_1 
	=
	\frac{1}{\sqrt{2}}
	\left\{ 
		\begin{array}{c}
			1 \\
			1
		\end{array}
	\right\}	
\end{equation}
\begin{equation}
\lambda_2=-\frac{1}{5}, \ \ 
\fat{n}_2 =
	\frac{1}{\sqrt{2}}
	\left\{ 
		\begin{array}{c}
		-1 \\
		 1 
		\end{array}
	\right\}.	
\end{equation}
\item
\[
	\fat{Q} = 
	\left( 
		\begin{array}{cc}
		\fat{n}_1 \cdot \fat{e}_1 & \fat{n}_1 \cdot \fat{e}_2 \\
		\fat{n}_2 \cdot \fat{e}_1 & \fat{n}_2 \cdot \fat{e}_2
		\end{array}
	\right)
	= 
	\frac{1}{\sqrt{2}}
	\left( 
		\begin{array}{cc}
		 1 & 1 \\
		-1 & 1
		\end{array}
	\right)
\]
\item
前問で求めた座標変換マトリクス$\fat{Q}$を用いて変位ベクトル$\fat{u}$及び位置ベクトル$\fat{x}$が,
\begin{equation}
	\fat{u} = \fat{Q}^T \fat{u}' ,\  \fat{x} = \fat{Q}^T \fat{x}'
\end{equation}
と書ける.したがって,
\begin{equation}
	\fat{Q}^T \fat{u}' = \fat{F} \fat{Q}^T \fat{x}'
\end{equation}
より,
\begin{equation}
	\fat{u}' = \left({ \fat{Q} \fat{F} \fat{Q}^T }\right) \fat{x}'.
\end{equation}
だから,$\fat{F}'=\fat{Q}\fat{F}\fat{Q}^T$となる.よって,
\begin{eqnarray}
 \fat{F}'=\fat{Q} \fat{F} \fat{Q}^T 
 &=&
\frac{1}{10}
	\left( 
		\begin{array}{cc}
		 1 & 1 \\
		-1 & 1
		\end{array}
	\right)
 	\left( 
		\begin{array}{cc}
		 0 & 1 \\
		 1 & 0
		\end{array}
	\right)
	\left( 
		\begin{array}{cc}
		 1 & -1 \\
		 1 & 1
		\end{array}
	\right) \\
 &=&
\nonumber
	\frac{1}{5}
	\left( 
		\begin{array}{cc}
		 1 &  0 \\
		 0 &-1
		\end{array}
	\right)
\end{eqnarray}
\item
正方形PQRSの各頂点がP',Q',R',S'へ移動したとする.
これらの点における変位ベクトルを$\fat{u}'_P, \  \fat{u}'_Q, \  \fat{u}'_R, \  \fat{u}'_S$とすると,
\begin{eqnarray}
	\left[
		\fat{u}'_P \  \fat{u}'_Q \  \fat{u}'_R \  \fat{u}'_S 
	\right] 
	&=& 
	\frac{1}{5}
	\left( 
		\begin{array}{cc}
		 1 &  0 \\
		 0 & -1 
		\end{array}
	\right)
	\left( 
		\begin{array}{cccc}
		 0 & 1 & 1 & 0  \\
		 0 & 0 & 1 & 1
		\end{array}
	\right) \\
	&=&
	\frac{1}{5}
	\left( 
		\begin{array}{cccc}
		 0 &  1  & 1  & 0  \\
		 0 &  0  & -1  & -1 
		\end{array}
	\right)
\end{eqnarray}
である.よって, $o-x_1'x_2'$座標系における変形後の頂点の座標は,
\begin{equation}
	P' = \left( 0,0 \right) ,\  Q' = \left(  \frac{6}{5},0 \right) ,\  
	R' = \left(\frac{6}{5}, \frac{4}{5}\right) ,\  S' = \left(0,\frac{4}{5}\right)
\end{equation}
となる.以上より,正方形領域PQRSは, 変形後, 図\ref{fig:fig1}(b)
に示すような長方形P'Q'R'S'に移されることがわかる.
\item
\begin{eqnarray}
	\varepsilon_{11}' &=& \frac{\partial u_1'}{\partial x_1'}=\frac{1}{5} \\
	\varepsilon_{12}' =\varepsilon_{21}' 
	&=& \frac{1}{2}\left(
		\frac{\partial u_1'}{\partial x_2'}
		+
		\frac{\partial u_2'}{\partial x_1'}
		\right) 
		=0
		\\
	\varepsilon_{22}' &=& \frac{\partial u_2'}{\partial x_2'}=-\frac{1}{5}
\end{eqnarray}
\end{enumerate}
}
%--------------------
\begin{figure}[h]
	\begin{center}
	\includegraphics[width=0.85\linewidth]{fig1ans.eps} 
	\end{center}
	\vspace{-5mm}
	\caption{正方形領域ABCDおよびPQRSの変形状況.} 
	\label{fig:fig1}
\end{figure}

%--------------------
%%%%%%%%%%%%%%%%%%%%%%%%%%%%%%%%%%%%%%%%%%%%
\section{一般化されたフック則}
物体に発生しているひずみや応力が小さいとき,多くの材料でひずみは応力に比例すると仮定できる.
その際,軸ひずみ$\varepsilon_{ii}$と軸応力$\sigma_{ii}$を関係付ける比例係数は,
軸力問題と同様に{\bf ヤング率}と呼ばれ$E$で表す.
いま,物体内のある点$\fat{x}$で$\sigma_{11}$のみが作用していたとする.
このとき,$\sigma_{11}$に比例した軸ひずみ$\varepsilon_{11}$が発生し,
両者の関係はヤング率を用いて
\begin{equation}
	\varepsilon_{11}=\frac{\sigma_{11}}{E}
\end{equation}
と表される.通常,物体をある方向に引張ると,それとは直角方向には
若干の縮みが生じる(図\ref{fig:fig3_1}-(a)を参照).
従って,$\sigma_{11}$だけが加えられた状況においても,
$\varepsilon_{22}$と$\varepsilon_{33}$が同時に発生する.
これらのひずみは$\varepsilon_{11}>0$のときに負,$\varepsilon_{11}<0$のときには正となる.
また,その大きさは$\varepsilon_{11}=\frac{\sigma_{11}}{E}$に比例すると考えられる.
そのため,$\varepsilon_{22}$と$\varepsilon_{33}$は,
ある正の定数$\nu$を用いて$\sigma_{11}$と次のように関係づけることができる.
\begin{equation}
	\varepsilon_{22}=-\nu \frac{\sigma_{11}}{E}, \ \
	\varepsilon_{33}=-\nu \frac{\sigma_{11}}{E}
\end{equation}
ここで,係数$\nu$は{\bf ポアソン比}と呼ばれる無次元の材料定数で,物理的には
引張軸方向のひずみ(縦ひずみ)と引張軸直角方向のひずみ(横ひずみ)の
比を表わす. 次に,$\sigma_{22}$だけが零でない状況を考えると,
$\varepsilon_{22}$に加え,それと逆符号の直ひずみが$x_2$および$x_3$軸方向に生じる.
同様なことは$\sigma_{33}$のみ加えられた状況下でも発生する.
従って,一般に$\sigma_{11},\,\sigma_{22}$および$\sigma_{33}$が同時に作用するとき,
直ひずみと直応力の関係は次のように表されることになる.
\begin{eqnarray}
	\varepsilon_{11} &=& \frac{\sigma_{11}}{E}-\frac{\nu}{E}\left( \sigma_{22}+\sigma_{33} \right) \\
	\varepsilon_{22} &=& \frac{\sigma_{22}}{E}-\frac{\nu}{E}\left( \sigma_{33}+\sigma_{11} \right) \\
	\varepsilon_{33} &=& \frac{\sigma_{33}}{E}-\frac{\nu}{E}\left( \sigma_{11}+\sigma_{22} \right)
	\label{eqn:eii_sjj}
\end{eqnarray}
これらの関係をまとめれば,
\begin{equation}
	\left\{ 
	\begin{array}{*{20}{c}}
	\varepsilon _{11}\\
	\varepsilon _{22}\\
	\varepsilon _{33}
	\end{array}
	\right\} 
	= 
	\frac{1}{E}\left( 
	\begin{array}{*{20}{c}}
	1& - \nu & - \nu \\
	 - \nu &1& - \nu \\
	 - \nu & - \nu &1
	\end{array}
	\right)
	\left\{ 
	\begin{array}{*{20}{c}}
	\sigma _{11}\\
	\sigma _{22}\\
	\sigma _{33}
	\end{array} 
	\right\}
	\label{eqn:eii_sjj_mat}
\end{equation}
と表すことができる.なお,この逆の関係は
\begin{equation}
	\left\{ 
	\begin{array}{*{20}{c}}
		\sigma _{11}\\
		\sigma _{22}\\
		\sigma _{33}
	\end{array} 
	\right\} 
	= 
	\frac{E}{\left( 1 + \nu  \right)\left( 1 - 2\nu  \right)}
	\left( 
		\begin{array}{*{20}{c}}
		1 - \nu & \nu &\nu \\
		\nu & 1 - \nu &\nu \\
		\nu &\nu &1 - \nu 
		\end{array}
	\right)
	\left\{ 
		\begin{array}{*{20}{c}}
		\varepsilon _{11}\\
		\varepsilon _{22}\\
		\varepsilon _{33}
		\end{array}
	\right\}
	\label{eqn:sii_ejj_mat}
\end{equation}
となる.式(\ref{eqn:sii_ejj_mat})より,正の$\varepsilon_{11}$に
対して正の$\sigma_{11}$が発生するためには,
\begin{equation}
	0< \nu < \frac{1}{2} 
\end{equation}
でなければならないことが分かる.
一方,せん断ひずみ$\varepsilon_{12}$は,せん断応力$\sigma_{12}$だけに起因
し,他の応力成分には関係しないと期待される.このことは, $\varepsilon_{23}$や
$\varepsilon_{31}$についても同様である.そこで,せん断応力$\sigma_{ij}$
と工学せん断ひずみ$\gamma_{ij}$の関係におけ比例係数を$G$とすれば,
\begin{equation}
	\left\{ 
		\begin{array}{*{20}{c}}
		\gamma_{23}\\
		\gamma_{13}\\
		\gamma_{12}
		\end{array}
	\right\} 
	=
	\left\{ 
		\begin{array}{*{20}{c}}
		\gamma_{32}\\
		\gamma_{31}\\
		\gamma_{21}
		\end{array}
	\right\} 
	= \frac{1}{G}
	\left\{
		\begin{array}{*{20}{c}}
		\sigma _{23}\\
		\sigma _{13}\\
		\sigma _{12}
		\end{array}
	\right\}
	\label{eqn:gij_sij}
\end{equation}
あるいは,
\begin{equation}
	\left\{ 
		\begin{array}{*{20}{c}}
		\sigma_{23}\\
		\sigma_{31}\\
		\sigma_{12}
		\end{array}
	\right\} 
	=
	G
	\left\{
		\begin{array}{*{20}{c}}
		\gamma_{23}\\
		\gamma_{31}\\
		\gamma_{12}
		\end{array}
	\right\}
	=
	2G
	\left\{
		\begin{array}{*{20}{c}}
		\varepsilon_{23}\\
		\varepsilon_{31}\\
		\varepsilon_{12}
		\end{array}
	\right\}
	\label{eqn:sij_gij}
\end{equation}
と書くことができる.
比例係数$G$は,{\bf せん断剛性}あるいは{\bf せん断弾性係数}と呼ばれ,
ヤング率と同様に単位面積あたりの力の次元を持つ.
直応力-直ひずみ関係(\ref{eqn:sii_ejj_mat})
と,せん断応力-せん断ひずみ関係(\ref{eqn:sij_gij})をまとめて,
{\bf 一般化されたフックの法則}と呼ぶ.
応力とひずみの関係が一般化されたフック則に従うような固体をフック固体と呼ぶ.
本講義では部材をつくる材料は全てフック固体であると仮定する.
\subsection{問題}
式(\ref{eqn:eii_sjj_mat})から式(\ref{eqn:sii_ejj_mat})の関係を導く過程を示せ. 
%%%%%%%%%%%%%%%%%%%%%%%%%%%%%%%%%%%%%%%%%%%%%%%%%%%%%%%%%%%%
\section{2次元問題}
連続体(固体や流体)における2次元問題には,{\bf 平面ひずみ問題}と{\bf 平面応力問題}の
2種類がある.
\subsection{平面ひずみ問題}
図\ref{fig:fig6_2}-(a)に示すように,一様な断面を持つ非常に長い物体があるとする.
ここで,物体の長手方向に第3軸が向くよう$o-x_1x_2x_3$直角直交座標系を取り,
物体に働く力や物体表面の拘束条件も$x_3$軸方向には変化しない場合を考える.
このとき,$x_3$軸に垂直な物体断面内は,どの断面も同じ力学的な状態にあり互いに
区別がつかない.このような力学的状態は{\bf 平面ひずみ状態}と呼ばれ,応力や
ひずみ、変位等の力学量は$x_1$と$x_2$の2変数のみの関数となり,$x_3$に依らない.
そのため,$x_3$に関する偏微分は恒等的に零となり$\varepsilon_{33}=
\frac{\partial u_3}{\partial x_3}=0$が言える.
従って,直応力と直ひずみの関係は,式(\ref{eqn:sii_ejj_mat})より,
\begin{equation}
	\left\{ 
	\begin{array}{*{20}{c}}
		\sigma _{11}\\
		\sigma _{22}
	\end{array} 
	\right\} 
	= 
	\frac{E}{\left( 1 + \nu  \right)\left( 1 - 2\nu  \right)}
	\left( 
		\begin{array}{*{20}{c}}
		1 - \nu & \nu \\
		\nu & 1 - \nu \\
		\end{array}
	\right)
	\left\{ 
		\begin{array}{*{20}{c}}
		\varepsilon _{11}\\
		\varepsilon _{22}\\
		\end{array}
	\right\}
	\label{eqn:Hooke_pstrain}
\end{equation}
となり,この逆は
\begin{equation}
	\left\{ 
	\begin{array}{*{20}{c}}
		\varepsilon _{11}\\
		\varepsilon_{22}
	\end{array} 
	\right\} 
	= 
	\frac{1+\nu}{E}
	\left( 
		\begin{array}{*{20}{c}}
		1 - \nu & -\nu \\
		-\nu & 1 - \nu \\
		\end{array}
	\right)
	\left\{ 
		\begin{array}{*{20}{c}}
		\sigma_{11}\\
		\sigma_{22}\\
		\end{array}
	\right\}
	\label{eqn:Hooke_pstrain}
\end{equation}
で与えられる.ただし, $\sigma_{33}$はゼロでなく,
\begin{equation}
	\sigma_{33}
	=\nu \left(\sigma_{11}+\sigma_{22}\right)
	=\frac{\nu E}{(1+\nu)(1-2\nu)}\left(\varepsilon_{11}+\varepsilon_{22}\right)
	\label{eqn:s33_pstrain}
\end{equation}
であることが式(\ref{eqn:eii_sjj_mat})と(\ref{eqn:sii_ejj_mat})から示される.
なお,せん断ひずみと応力の関係は式(\ref{eqn:gij_sij})の通りであり,
$u_3$が定数でない限り,せん断ひずみは零でなく,
\begin{equation}
	\varepsilon_{12}=\frac{1}{2}\left(
		\frac{\partial u_1}{\partial x_2} 
		+
		\frac{\partial u_2}{\partial x_1} 
	\right), \ \ 
	\varepsilon_{23}=\frac{1}{2}\frac{\partial u_3}{\partial x_2}, \ \  
	\varepsilon_{31}=\frac{1}{2}\frac{\partial u_3}{\partial x_1}
\end{equation}
で与えられる.
\subsection{平面応力問題}
図\ref{fig:fig6_2}-(b)のように,$x_3$軸を法線とするような平板があったとする.
外力は平板側面において$x_1x_2$面内の方向に作用し,上下面に外力は作用しないとする.
平板の厚みが十分に小さいとき,変位やひずみ,応力の厚み方向($x_3$軸方向)への変化は
無視でき,これらの力学量は$x_1$と$x_2$のみに依存する.
このような意味での2次元問題は{\bf 平面応力問題}と呼ばれ,物体は{\bf 平面応力状態}にあるという.
平面応力問題では平板上下面に作用する外力がゼロであることから, 
\begin{equation}
 \sigma_{31}=\sigma_{32}=\sigma_{33}=0
\end{equation}
としてよい.また, 式(\ref{eqn:gij_sij})から$\varepsilon_{32}=\varepsilon_{31}=0$が,
式(\ref{eqn:eii_sjj_mat})から次のことが言える.
\begin{equation}
	\left\{ 
	\begin{array}{*{20}{c}}
	\varepsilon _{11}\\
	\varepsilon _{22}
	\end{array}
	\right\} 
	= 
	\frac{1}{E}\left( 
	\begin{array}{*{20}{c}}
	1& - \nu \\
	 - \nu &1
	\end{array}
	\right)
	\left\{ 
	\begin{array}{*{20}{c}}
	\sigma _{11}\\
	\sigma _{22}
	\end{array} 
	\right\}
	\label{eqn:Cmpl_pstress}
\end{equation}
が従う.また,式(\ref{eqn:Cmpl_pstress})のを直ひずみ成分について解くと,
\begin{equation}
	\left\{ 
	\begin{array}{*{20}{c}}
	\sigma _{11}\\
	\sigma _{22}
	\end{array}
	\right\} 
	= 
	\frac{E}{1-\nu^2}\left( 
	\begin{array}{*{20}{c}}
	1&  \nu \\
	  \nu &1
	\end{array}
	\right)
	\left\{ 
	\begin{array}{*{20}{c}}
	\varepsilon _{11}\\
	\varepsilon _{22}
	\end{array} 
	\right\}
	\label{eqn:Hooke_pstress}
\end{equation}
となる.最後に,$\sigma_{33}=0$であることから,式(\ref{eqn:eii_sjj_mat})と(\ref{eqn:sii_ejj_mat})より,
\begin{equation}
	\varepsilon_{33}=-\frac{\nu}{E} \left( \sigma_{11}+\sigma_{22} \right)
	= -\frac{\nu}{1-\nu}\left(\varepsilon_{11}+\varepsilon_{22}\right)
\end{equation}
が言え,平面応力問題では$\varepsilon_{33}$がゼロでないことに注意が必要となる.
\begin{figure}[h]
	\begin{center}
	\includegraphics[width=0.9 \linewidth]{2D_problems.eps} 
	\end{center}
	\caption{
	(a)平面ひずみ状態と,(b)平面応力状態のイメージ.
	 } 
	\label{fig:fig6_2}
\end{figure}
\end{document}
%%%%%%%%%%%%%%%%%%%%%%

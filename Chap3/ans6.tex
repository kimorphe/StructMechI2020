\subsubsection{解答}
{\small
\begin{enumerate}
\item
変位ベクトと位置ベクトルの関係は
\begin{equation}
	\left( 
		\begin{array}{cc}
		 u_1 \\
		 u_2 
		\end{array}
	\right)
	=
	\frac{1}{5}
	\left( 
		\begin{array}{cc}
		 0 & 1\\
		 1 &0 
		\end{array}
	\right)
	\left( 
		\begin{array}{c}
		 x_1 \\
		 x_2 
		\end{array}
	\right)
	=
	\frac{1}{5}
	\left( 
		\begin{array}{cc}
		 x_2 \\
		 x_1
		\end{array}
	\right)
\end{equation}
より,ひずみテンソルの各成分は,
\begin{eqnarray}
	\varepsilon_{11} &=& \frac{\partial u_1}{\partial x_1} = 0 \\
	\varepsilon_{12} &=& \varepsilon_{21} = \frac{1}{2} \left({ \frac{\partial u_1}{\partial x_2} 
		+ \frac{\partial u_2}{\partial x_1}  }\right) 
		= \frac{1}{5}\\
	\varepsilon_{22} &=& \frac{\partial u_2}{\partial x_2} =0
\end{eqnarray}
となる.
\item
	正方形ABCDの各頂点の座標は,
\begin{equation}
	A = (0,0) ,\  B = ( 1,0) ,\  C = (1, 1) ,\  D = (0, 1)
\end{equation}
である.これらの点における変位ベクトルを
$\fat{u}_A ,\  \fat{u}_B, \  \fat{u}_C, \  \fat{u}_D$とすると,
\begin{eqnarray}
 \left[\fat{u}_A \  \fat{u}_B \  \fat{u}_C \  \fat{u}_D \right] 
 &=& 
 	\frac{1}{5}
	\left( 
		\begin{array}{cc}
		 0 & 1 \\
		 1 & 0 
		\end{array}
	\right)
	\left( 
		\begin{array}{cccc}
		 0 &  1  &  1  &  0  \\
		 0 &  0  &  1  &  1
		\end{array}
	\right) \\
\nonumber
	&=&
	\frac{1}{5}
	\left( 
		\begin{array}{cccc}
		 0 &  0  &  1  &  1  \\
		 0 &  1  &  1  &  0 
		\end{array}
	\right)
\end{eqnarray}
である.よって, 変形後の各頂点の座標をA',B',C',D'とすれば,
\begin{equation}
	A' = \left( 0, 0\right) ,\  
	B' = \left(1, \frac{1}{5}\right) ,\  
	C' = \left(\frac{6}{5},\frac{6}{5} \right) ,\  
	D' = \left(\frac{1}{5}, 1 \right)
\end{equation}
となる.以上より,正方形ABCDは図\ref{fig:fig1}(a)のような四辺形A'B'C'D'へ移される.
%--------------------
\item
行列$\fat{F}$の固有値$\lambda$は
\begin{equation}
	det\ (\fat{F} - \lambda \fat{I}) =
	\left|
		\begin{array}{cc}
		 -\lambda & \frac{1}{5} \\
		 \frac{1}{5} & -\lambda 
		\end{array}
	\right|
	= \left( -\lambda\right)^2- \left(\frac{1}{5}\right)^2=0
\end{equation}
より$\lambda=\frac{1}{5},-\frac{1}{5}$である.$\lambda=\frac{1}{5}$のときの固有ベクトル$\fat{n}$は,
$k$を任意の実数として$\fat{n}=k(1,\,1)$と表される.
一方$\lambda=-\frac{1}{5}$のときの固有ベクトルは$\fat{n}=k(-1,\, 1)$である.
よって,
\begin{equation}
	\lambda_1=\frac{2}{5}, \ \  
	\fat{n}_1 
	=
	\frac{1}{\sqrt{2}}
	\left\{ 
		\begin{array}{c}
			1 \\
			1
		\end{array}
	\right\}	
\end{equation}
\begin{equation}
\lambda_2=-\frac{1}{5}, \ \ 
\fat{n}_2 =
	\frac{1}{\sqrt{2}}
	\left\{ 
		\begin{array}{c}
		-1 \\
		 1 
		\end{array}
	\right\}.	
\end{equation}
\item
\[
	\fat{Q} = 
	\left( 
		\begin{array}{cc}
		\fat{n}_1 \cdot \fat{e}_1 & \fat{n}_1 \cdot \fat{e}_2 \\
		\fat{n}_2 \cdot \fat{e}_1 & \fat{n}_2 \cdot \fat{e}_2
		\end{array}
	\right)
	= 
	\frac{1}{\sqrt{2}}
	\left( 
		\begin{array}{cc}
		 1 & 1 \\
		-1 & 1
		\end{array}
	\right)
\]
\item
前問で求めた座標変換マトリクス$\fat{Q}$を用いて変位ベクトル$\fat{u}$及び位置ベクトル$\fat{x}$が,
\begin{equation}
	\fat{u} = \fat{Q}^T \fat{u}' ,\  \fat{x} = \fat{Q}^T \fat{x}'
\end{equation}
と書ける.したがって,
\begin{equation}
	\fat{Q}^T \fat{u}' = \fat{F} \fat{Q}^T \fat{x}'
\end{equation}
より,
\begin{equation}
	\fat{u}' = \left({ \fat{Q} \fat{F} \fat{Q}^T }\right) \fat{x}'.
\end{equation}
だから,$\fat{F}'=\fat{Q}\fat{F}\fat{Q}^T$となる.よって,
\begin{eqnarray}
 \fat{F}'=\fat{Q} \fat{F} \fat{Q}^T 
 &=&
\frac{1}{10}
	\left( 
		\begin{array}{cc}
		 1 & 1 \\
		-1 & 1
		\end{array}
	\right)
 	\left( 
		\begin{array}{cc}
		 0 & 1 \\
		 1 & 0
		\end{array}
	\right)
	\left( 
		\begin{array}{cc}
		 1 & -1 \\
		 1 & 1
		\end{array}
	\right) \\
 &=&
\nonumber
	\frac{1}{5}
	\left( 
		\begin{array}{cc}
		 1 &  0 \\
		 0 &-1
		\end{array}
	\right)
\end{eqnarray}
\item
正方形PQRSの各頂点がP',Q',R',S'へ移動したとする.
これらの点における変位ベクトルを$\fat{u}'_P, \  \fat{u}'_Q, \  \fat{u}'_R, \  \fat{u}'_S$とすると,
\begin{eqnarray}
	\left[
		\fat{u}'_P \  \fat{u}'_Q \  \fat{u}'_R \  \fat{u}'_S 
	\right] 
	&=& 
	\frac{1}{5}
	\left( 
		\begin{array}{cc}
		 1 &  0 \\
		 0 & -1 
		\end{array}
	\right)
	\left( 
		\begin{array}{cccc}
		 0 & 1 & 1 & 0  \\
		 0 & 0 & 1 & 1
		\end{array}
	\right) \\
	&=&
	\frac{1}{5}
	\left( 
		\begin{array}{cccc}
		 0 &  1  & 1  & 0  \\
		 0 &  0  & -1  & -1 
		\end{array}
	\right)
\end{eqnarray}
である.よって, $o-x_1'x_2'$座標系における変形後の頂点の座標は,
\begin{equation}
	P' = \left( 0,0 \right) ,\  Q' = \left(  \frac{6}{5},0 \right) ,\  
	R' = \left(\frac{6}{5}, \frac{4}{5}\right) ,\  S' = \left(0,\frac{4}{5}\right)
\end{equation}
となる.以上より,正方形領域PQRSは, 変形後, 図\ref{fig:fig1}(b)
に示すような長方形P'Q'R'S'に移されることがわかる.
\item
\begin{eqnarray}
	\varepsilon_{11}' &=& \frac{\partial u_1'}{\partial x_1'}=\frac{1}{5} \\
	\varepsilon_{12}' =\varepsilon_{21}' 
	&=& \frac{1}{2}\left(
		\frac{\partial u_1'}{\partial x_2'}
		+
		\frac{\partial u_2'}{\partial x_1'}
		\right) 
		=0
		\\
	\varepsilon_{22}' &=& \frac{\partial u_2'}{\partial x_2'}=-\frac{1}{5}
\end{eqnarray}
\end{enumerate}
}
%--------------------
\begin{figure}[h]
	\begin{center}
	\includegraphics[width=0.85\linewidth]{fig1ans.eps} 
	\end{center}
	\vspace{-5mm}
	\caption{正方形領域ABCDおよびPQRSの変形状況.} 
	\label{fig:fig1}
\end{figure}

\documentclass[10pt,a4j]{jbook}
%\usepackage{graphicx,wrapfig}
\usepackage{graphicx}
\setlength{\topmargin}{-1.5cm}
%\setlength{\textwidth}{16.5cm}
\setlength{\textheight}{25.2cm}
\newlength{\minitwocolumn}
\setlength{\minitwocolumn}{0.5\textwidth}
\addtolength{\minitwocolumn}{-\columnsep}
%\addtolength{\baselineskip}{-0.1\baselineskip}
%
\def\Mmaru#1{{\ooalign{\hfil#1\/\hfil\crcr
\raise.167ex\hbox{\mathhexbox 20D}}}}
%
\begin{document}
\newcommand{\fat}[1]{\mbox{\boldmath $#1$}}
\newcommand{\D}{\partial}
\newcommand{\w}{\omega}
\newcommand{\ga}{\alpha}
\newcommand{\gb}{\beta}
\newcommand{\gx}{\xi}
\newcommand{\gz}{\zeta}
\newcommand{\vhat}[1]{\hat{\fat{#1}}}
\newcommand{\spc}{\vspace{0.7\baselineskip}}
\newcommand{\halfspc}{\vspace{0.3\baselineskip}}
\bibliographystyle{unsrt}
%\pagestyle{empty}
\newcommand{\twofig}[2]
 {
   \begin{figure}
     \begin{minipage}[t]{\minitwocolumn}
         \begin{center}   #1
         \end{center}
     \end{minipage}
         \hspace{\columnsep}
     \begin{minipage}[t]{\minitwocolumn}
         \begin{center} #2
         \end{center}
     \end{minipage}
   \end{figure}
 }
%%%%%%%%%%%%%%%%%%%%%%%%%%%%%%%%%
\setcounter{chapter}{1}
\chapter{応力テンソル}
以下, 太字は{\bf 成分表示することを前提とした}ベクトルあるいは行列を表す.
\section{トラクションベクトル}
位置ベクトル$\fat{x}$で表される物体内の点において, 単位ベクトル$\fat{n}$を
法線にもつような面$\Pi(\fat{n};\fat{x})$を考える(図\ref{fig:fig3_1}-(a),(b)).
物体を$\Pi(\fat{n};\fat{x})$で仮想的に切断したとき,
点$\fat{x}$において面$\Pi(\fat{n};\fat{x})$に作用する単位面積あたりの
力のベクトルを$\fat{t}^{(n)}(\fat{x})$と書き, これを{\rm \bf トラクションベクトル}と呼ぶ.

\begin{figure}[h]
	\begin{center}
	\includegraphics[width=0.9\linewidth]{fig3_1.eps} 
	\end{center}
	\caption{(a)物体中の着目点$\fat{x}$と仮想的な切断面$\Pi(\fat{n};\fat{x})$.
	(b),(c) 着目点周辺の拡大図.
	$x_3$軸は紙面に垂直手前側に伸びる座表軸を,
	$\fat{e}_3$は$x_3$軸方向の基底ベクトルを表す.(c)は面$\Pi(-\fat{n};\fat{x})$
	に作用するトラクション$\fat{t}^{(-n)}$を表す.} 
	\label{fig:fig3_1}
\end{figure}
3次元空間におけるベクトルを表すために$o-x_1x_2x_3$直角直交座標系を
とり, その基底ベクトルを$\fat{e}_i,\, (i=1,2,3)$と表す.
ここでの基底ベクトルとは,各座標軸方向を向く単位ベクトルのことと考えて良い.
このとき, $\fat{e}_i\, (i=1,2,3)$は互いに次の関係を満たす.
\begin{equation}
	\fat{e}_i\cdot \fat{e}_j=
	\delta_{ij}=
	\left\{
	\begin{array}{cc}
		1 & (i=j) \\
		0 & (i \neq j) \\
	\end{array}
	\right.
	\label{eqn:ortho}
\end{equation}
ここで,$\fat{a} \cdot \fat{b}$は,ベクトル$\fat{a}$と$\fat{b}$の内積を意味する.
$\fat{a}$と$\fat{b}$のなす角を$\theta_{ab}$とすれば,内積は次の式で定義される.
\begin{equation}
	\fat{a}\cdot\fat{b}=
	\left|\fat{a}\right|
	\left|\fat{b}\right|
	\cos \theta_{ab}
\end{equation}
従って,式(\ref{eqn:ortho})の関係は, 基底ベクトルが互いに直交してその長さは1であることを意味する.
式(\ref{eqn:ortho})を満たす基底ベクトルを, 正規直交基底と呼ぶ.
いま, $\fat{n}=\fat{e}_i,\, (i=1,2,3)$としたときのトラクション
ベクトルを$\fat{t}^{(i)}$と表し, $\fat{t}^{(i)}$の第$j$成分を$\sigma_{ij}$
と書く.すなわち
\begin{eqnarray}
	\fat{n}=\fat{e}_1 &;& \fat{t}^{(n)}=\fat{t}^{(1)}
		=\left( \sigma_{11},\, \sigma_{12},\, \sigma_{13}\right)^T 
		\label{eqn:t1}
		\\
	\fat{n}=\fat{e}_2 &;& \fat{t}^{(n)}=\fat{t}^{(2)} 
		=\left( \sigma_{21},\, \sigma_{22},\, \sigma_{23}\right)^T 
		\label{eqn:t2}
		\\
	\fat{n}=\fat{e}_3 &;& \fat{t}^{(n)}=\fat{t}^{(3)} 
		=\left( \sigma_{31},\, \sigma_{32},\, \sigma_{33}\right)^T 
		\label{eqn:t3}
\end{eqnarray}
と表現する.ここに$(\cdot)^T$は, $(\cdot)$の転置を意味する.
従って,式(\ref{eqn:t1})-(\ref{eqn:t3})はいずれも縦ベクトルを表す.
図\ref{fig:fig3_3}はこれら特別なトラクションベクトルとそれが作用する面の様子を
図示したものである.この図で第3軸($x_3$軸)とその基底ベクトル$\fat{e}_3$は紙面鉛直方向としている.
また,$\fat{n}=-\fat{e}_i$に対するトラクションベクトルを$\fat{t}^{(-i)}$と表し, 
それらが作用する面との関係も示している.
$\fat{t}^{(-i)}=-\fat{t}^{(i)}$となることの理由は第\ref{equib}節で説明する.
\begin{figure}[h]
	\begin{center}
	\includegraphics[width=0.7\linewidth]{fig3_3.eps} 
	\end{center}
	\caption{基底ベクトル$\fat{e}_i$を法線ベクトルとする
	面(座標面)と,各々の面に作用するトラクションベクトル$\fat{t}^{(\pm i)}$.}
	\label{fig:fig3_3}
\end{figure}
\section{応力テンソル}
\subsection{応力テンソルの定義}
トラクションベクトル$\fat{t}^{(1)},\, \fat{t}^{(2)},\,\fat{t}^{(3)}$の成分
$\sigma_{ij},(i,j=1,2,3)$を並べた
\begin{equation}
	\fat{\sigma}=\left\{ \sigma_{ij}\right\}
	=\left(
	\begin{array}{ccc}
		\sigma_{11} & \sigma_{12} & \sigma_{13} \\
		\sigma_{21} & \sigma_{22} & \sigma_{23} \\
		\sigma_{31} & \sigma_{32} & \sigma_{33} 
	\end{array}
	\right)
\end{equation}
の行列を{\rm \bf 応力テンソル}, その成分$\sigma_{ij}$を応力成分と呼ぶ.
9つの応力成分のうち, $\sigma_{11},\sigma_{22},\sigma_{33}$を{\rm\bf 直応力}, 
それ以外のものを{\bf \rm せん断応力成分}と呼ぶ. $\sigma_{ij}$の
第一インデックス$i$は面の向きが$\fat{n}=\fat{e}_i$であることを表す.
一方, 第二インデックス$j$は, 第$j$方向の単位面積あたりの力を意味している.
また,図\ref{fig:fig3_5}のような図
を描くことで, 応力成分の意味を視覚的に表現することがしばしば行われる.
%%%%%%%%%%%%%%%%%%%%%%%%
\begin{figure}[h]
	\begin{center}
	\includegraphics[width=0.7\linewidth]{fig3_5.eps} 
	\end{center}
	\caption{着目点$\fat{x}$における応力成分の視覚的な表現.
	矩形領域は, 応力成分が作用する面を表すために描かれたものであり
	実際には領域の大きさは無限小であることに注意する必要がある.
	この図では, 以上の点について注意を喚起することを意図し, 
	矩形領域の幅と高さを$\Delta x_1, \Delta x_2$でなく
	$dx_1, dx_2$と表している. }
	\label{fig:fig3_5}
\end{figure}
%%%%%%%%%%%%%%%%%%%%%%%%
\subsection{釣り合い条件からの帰結}\label{equib}
トラクションベクトルと応力テンソルはいずれも力に関する量である.
そのためこれらは物体が静止しているならば釣り合い条件を満足しなければならない.
釣り合い条件は,物体全体としてのみならず,物体内部の各点や物体内部の任意の部分領域
で成り立たなければならない.もしそうでなければ,力の釣り合いが成立しない箇所では
物体は運動状態にあることになる.
このことを利用すれば,トラクションや応力が満たすべきいくつかの有用な関係を以下に
示すようにして導くことができる.
\begin{itemize}
\item
	{\bf 表裏面それぞれに作用するトラクションの関係:}\\
	図\ref{fig:fig3_2}のように,物体中の着目点$\fat{x}$を含むように微小な薄層領域ABCDをとり
	その釣り合い条件式を立てる.トラクションが単位面積あたりの力であることに注意し,
	$\Delta S$や$\Delta h$は十分小さいとすれば,釣り合い式は
	\begin{equation}
		\left(\fat{t}^{(n)}+\fat{t}^{(-n)}\right)\Delta S 
		+
		\left(\fat{b}^{l}+\fat{b}^{r}\right)\Delta h 
		=\fat{0}
	\end{equation}
	となる.この式の両辺を$\Delta S$で除し,$\Delta h \rightarrow 0$と$\Delta S\rightarrow 0$の極限を
	順にとれば,
	\begin{equation}
		\fat{t}^{(n)}+\fat{t}^{(-n)}=\fat{0}
		\label{eqn:t_pmn}
	\end{equation}
	となることが示される.これは, 仮想切断面の表裏$\Pi(\pm \fat{n};\fat{x})$に働く
	トラクションは, 大きさが等しく方向が反対であることを意味する.
	\begin{figure}[h]
	\begin{center}
	\includegraphics[width=0.7\linewidth]{fig3_2.eps} 
	\end{center}
	\caption{物体中の着目点$\fat{x}$を含む微小薄膜層ABCDと,その表面に
	作用するトラクションベクトル. } 
	\label{fig:fig3_2}
	\end{figure}
\item
	{\bf トラクションベクトルと法線ベクトルの関係:}\\
	図\ref{fig:fig3_4}のような,着目点$\fat{x}$を含む微小三角形領域ABC
	(3次元問題の場合は四面体領域)に対する釣り合い式は
	\begin{equation}
		\fat{t}^{(n)}\Delta S +\fat{t}^{(-1)}\Delta x_2 + \fat{t}^{(-2)}\Delta x_1 =\fat{0}
		\label{eqn:equib_tri}
	\end{equation}
	となる.式(\ref{eqn:t_pmn})と,
	\begin{equation}
		\fat{n}= \left(n_1,n_2\right)=\left( \frac{\Delta x_2}{\Delta S}, \frac{\Delta x_1}{\Delta S}\right)^T
	\end{equation}
	であることを踏まえれば,式(\ref{eqn:equib_tri})は
	\begin{equation}
		\fat{t}^{(n)}=n_1\fat{t}^{(1)}+n_2\fat{t}^{(2)} 
	\end{equation}
	よって, 一般のトラクションベクトル$\fat{t}^{(n)}(\fat{x})$と法線ベクトル$\fat{n}$が, 
	\begin{equation}
		\fat{t}^{(n)}=\fat{\sigma}^{T}\fat{n}
		\label{eqn:t_as_sign}
	\end{equation}
	と, 応力テンソル$\fat{\sigma}$を介して結び付けられることが示される.
	応力テンソル成分は(座標系には依存するが)法線ベクトル$\fat{n}$, 
	すなわち切断面の向きに依存しない. このことは, 応力テンソルが内力を表す
	本質的な量であることを意味する.
	\begin{figure}[h]
		\begin{center}
		\includegraphics[width=0.7\linewidth]{fig3_4.eps} 
		\end{center}
		\caption{着目点$\fat{x}$の近傍にとった微小三角形領域の自由物体図.}
		\label{fig:fig3_4}
									\end{figure}
\item
	{\bf 応力テンソルの対称性:}\\
	図\ref{fig:fig4_1}に示すように, 着目点$\fat{x}$を含む微小な矩形領域
	(3次元問題の場合は立方体領域)に対してモーメントの釣り合いを考える.
	いま,図\ref{fig:fig4_1}-(b)を参照し,モーメントの基準点を$\fat{x}$,反時計回りの方向を正とすれば,
	モーメントの釣り合い式は次のようになる.
	\begin{eqnarray}
		&& \left. \sigma_{12}\right|_{x_1+\frac{\Delta x_1}{2},x_2}\Delta x_2  \times \frac{\Delta x_1}{2} 
		-
		\left. \sigma_{21}\right|_{x_1,x_2+\frac{\Delta x_2}{2}}\Delta x_1  \times \frac{\Delta x_2}{2} \nonumber \\
		&&
		+
		\left. \sigma_{12}\right|_{x_1-\frac{\Delta x_1}{2},x_2}\Delta x_2  \times \frac{\Delta x_1}{2} 
		-
		\left. \sigma_{21}\right|_{x_1,x_2-\frac{\Delta x_2}{2}}\Delta x_1  \times \frac{\Delta x_2}{2}
		=0
	\end{eqnarray}
	この式の両辺を$\Delta x_1\Delta x_2$で割,$\Delta x_1, \Delta x_2\rightarrow 0$の極限を取れば,
	\begin{equation}
		\sigma_{12}=\sigma_{21}
	\end{equation}
	となることが示される.
	$x_2x3$および$x_3x_1$平面内でも同じようにしてモーメントの釣り合い式を立てれば,
	$\sigma_{23}=\sigma_{32}, \sigma_{31}=\sigma_{13}$となることが言え,最終的に
	\begin{equation}
		\fat{\sigma}^{T}=\fat{\sigma}
		\label{eqn:sig_sym}
	\end{equation}
	と,応力テンソルは対称であることが示される.
	すなわち, 全ての$i,j$の組に対して
	\begin{equation}
		\sigma_{ij}=\sigma_{ji}, \ \ (i,j=1,2,3)
		\label{eqn:sig_sym_comp}
	\end{equation}
	の関係があり,このことから独立な応力成分は, 直応力成分が3つ, せん断応力成分3つの計6つ
	であることが分かる.平面内でのモーメント(トルク)の
	計算方法は図\ref{fig:fig4_2}を参照.
\begin{figure}[h]
	\begin{center}
	\includegraphics[width=1.0\linewidth]{fig4_2.eps} 
	\end{center}
	\caption{着目点$\fat{x}$近傍の微小矩形領域に対する自由物体図.
	(a)トラクションベクトルを用いて表現した場合の図.(b)応力成分を用いて表現した場合の図.} 
	\label{fig:fig4_1}
\end{figure}
\item
	{\bf 応力成分の釣り合い方程式(参考):}\\
	図\ref{fig:fig4_1}に示すような,着目点$\fat{x}$を含む微小な矩形領域
	(3次元問題の場合は立方体領域)に対して, 力の釣り合いを考えることで, 
	応力成分が満足すべき偏微分方程式が次のように得られる.
	\begin{equation}
		\sum_{j=1}^3 \frac{\partial \fat{t}^{(j)}}{\partial x_j}+\fat{b}=\fat{0}
		\label{eqn:stress_equib}
	\end{equation}
	ただし, $\fat{b}$は単位体積にあたりの力として与えられる分布外力を意味し, 一般に
	物体力と呼ばれる.重力は典型的な物体力の例である.
	式(\ref{eqn:stress_equib})を成分表記すると, 
	\begin{equation}
		\sum_{j=1}^3 \frac{\partial \sigma_{ji}}{\partial x_j}+b_i=0, \ \ (i=1,2,3)
		\label{eqn:stress_equib_compo}
	\end{equation}
	と表すこともできる.ただし$\fat{b}=(b_1,\,b_2,\, b_3)$である.
	式(\ref{eqn:stress_equib})あるいは(\ref{eqn:stress_equib_compo})で表される
	連立偏微分方程式は, 応力の釣り合い条件式と呼ばれる.
\end{itemize}
\begin{figure}[h]
	\begin{center}
	\includegraphics[width=0.5\linewidth]{fig4_3.eps} 
	\end{center}
	\caption{
	位置$\fat{x}$に作用する荷重ベクトル$\fat{F}$.
	$h$は位置ベクトルの原点$\fat{o}$と荷重作用線の距離を,
	$F_n$は$\fat{F}$の$\fat{x}$と直交する方向の成分を表す.
	これらの量を用いて荷重$\fat{F}$による点$o$に関するモーメント(トルク)$T$を,
	$T=|\fat{x}|\times F_n=|\fat{F}|\times h$で計算することができる.
	どちらも同じ結果$T=|\fat{F}| |\fat{x}|\sin\alpha$を与える.
	網掛けした平行四辺形は, トルク(モーメント)の大きさに相当する.
	 } 
	\label{fig:fig4_2}
\end{figure}
%%%%%%%%%%%%%%%%%%%%%%%%
\subsection{問題}
ニュートンの第二法則によれば,静止状態にある物体に働く力の和(合力)は零となる.
また,静止状態にある物体に働くモーメントの和(合モーメント)も零でなければならない
ことを,第二法則から導くことができる.
これら,合力と合モーメントが零の条件,すなわち力とモーメントの釣り合い条件を
利用すれば,物体に作用する未知の力が求められる場合がある.
図\ref{fig:ex5_Torque}に関する以下の問題は,
このことを理解するとともにモーメントの計算方法を学ぶためのものである.
\begin{enumerate}
\item
図(a)に示すような大きさ$F$の力を受ける棒部材について, 点A,B,CおよびDに関する
モーメントを求めよ. ただし, モーメントは反時計回りの方向を正とする. 
\item
図(b)に示すような力を受けて棒部材ACが静止している. このとき, 水平力$H_A$, 
鉛直力$V_A$および$V_C$を$F$を用いて表わせ. 
\item
図(c)に示すような荷重を受けるL字型部材について, 点A,B,C,Dに関するモーメントを求めよ. 
ただし, モーメントは反時計回りの方向を正とする. 
\item
図(d)のように, 長さ$l$,質量$m$の棒部材ACが, 壁に立てかけられて静止している. 
摩擦力が 壁側では無視できると仮定するとき, 部材が床と
壁から受ける力$H_A, V_A$と$H_C$を求めよ. また, $\alpha\rightarrow 0$と$\alpha \rightarrow \frac{\pi}{2}$
においてこれらの力が取る値を答えよ. なお, 部材には, 壁面と床面からの力に加え, 
重力$mg$が鉛直下向きに重心位置Bに集中して作用するものと考えてよい. 
\end{enumerate}
\begin{figure}[h]
	\begin{center}
	\includegraphics[width=0.8\linewidth]{ex5_Torque.eps} 
	\end{center}
	\caption{力とモーメント(トルク)の釣り合いに関する例題.} 
	\label{fig:ex5_Torque}
\end{figure}
%%%%%%%%%%%%%%%%%%%%%%%%%%%%%%%%%%%%%
\section{ベクトルとテンソルの座標変換}
\subsection{座標変換マトリクス}
原点$o$を共有する2つの直交座標系$o-x_1x_2x_3$と$o-x_1'x_2'x_3'$の間で成り立つ
座標変換法則を導く.図\ref{fig:fig4_3}は原点を共有する2つの座標系の一例を示したもので,
この例は第3軸は共有($x_3=x_3'$)された特別なケースである.
以下では断りの無い限り,2つの座標系は各々が直交座標系で,原点位置が互いに一致している
ということだけを前提として議論を行う.\\
はじめに, 実体としてのベクトルを$\vec{u},\vec{x}$, それらを特定の座標系で成分表示する
場合には$\fat{u},\fat{x}$や$\fat{u}',\fat{x}'$のように太字を用いる. 
なお$()'$を付したものは$o-x_1'x_2'x_3'$系におけるベクトルを表す.
それ以外のものは$o-x_1x_2x_3$座標系における表現であると解釈し, 
座標成分を用いてベクトルやテンソルを表する場合,使用する座標系に応じて
\[
	\fat{u}=(u_1, u_2, u_3)^T, \ \ \fat{x}'=(x_1', x_2', x_3')^T
\]
等と書く.
いま, $o-x_1x_2x_3$座標系の基底ベクトルを$\vec{e}_1, \vec{e}_2, \vec{e}_3$, 
$o-x_1'x_2'x_3'$におけるそれを$\vec{e}_1\,', \vec{e}_2\,', \vec{e}_3 \,'$, 
とする. これらは正規直交系をなし,
\begin{equation}
	\vec{e}_i\cdot \vec{e}_j=\delta_{ij}, \ \ 
	\vec{e}_i\,' \cdot \vec{e}_j\,'=\delta_{ij}, \ \ (i,j=1,2,3)
	\label{eqn:ortho_norm}
\end{equation}
を満たす.ただし$\delta_{ij}$はクロネッカーデルタ
\begin{equation}
	\delta _{ij}=\left\{
	\begin{array}{cc}
		1 & (i=j) \\
		0 & (i\neq j)
	\end{array}
	\right. , \ \ (i,j=1,2,3)
	\label{eqn:dij}
\end{equation}
を意味する.2つの座標系に対する基底ベクトルを用いれば, 任意のベクトル$\vec{u}$は
\begin{eqnarray}
	\vec{u} &=& \sum_{i=1}^3 u_i\vec{e}_i 
	\label{eqn:u_ei}
	\\
	&=& \sum_{i=1}^3 u'_i\vec{e}_i\,' 
	\label{eqn:u_eid}
\end{eqnarray}
と二通りに表すことができる. ここで,式(\ref{eqn:u_ei})と$\fat{e}_i$の内積,
式(\ref{eqn:u_eid})と$\fat{e}'_i$の内積を取れば,ベクトル成分$u_i$と$u_i'$が
次のように与えられることが示される.
\begin{eqnarray}
	u_i &=& \vec{u}\cdot \vec{e}_i \label{eqn:ui} , \ \ (i=1,2,3) \\
	u'_i&=& \vec{u}\cdot \vec{e}_i\,' \label{eqn:uid}, \ \ (i=1,2,3)
\end{eqnarray}
と書ける. ここで, 式(\ref{eqn:uid})の右辺に式(\ref{eqn:u_ei})を代入すれば, 
\begin{equation}
	u_i'=\sum_{j=1}^3 (\vec{e}_i\,'\cdot \vec{e}_j) u_j , \ \ (i=1,2,3)
	\label{eqn:i2id}
\end{equation}
が得られる.一方, 式(\ref{eqn:ui})の右辺に, 式(\ref{eqn:u_eid})を代入すれば, 
\begin{equation}
	u_i=\sum_{j=1}^3 (\vec{e}_i\cdot \vec{e}_j\, ') u_j', \ \ (i=1,2,3) 
	\label{eqn:id2i}
\end{equation}
の関係が得られる.そこで,3$\times$3行列$\fat{Q}$を
\begin{equation}
	\fat{Q}=
	\left\{
		\vec{e}_i\,'\cdot \vec{e}_j
	\right\}
	=
	\left( 
	\begin{array}{ccc}
		\vec{e}_1\,'\cdot \vec{e}_1 & \vec{e}_1\,'\cdot \vec{e}_2 & \vec{e}_1\,'\cdot \vec{e}_3 \\
		\vec{e}_2\,'\cdot \vec{e}_1 & \vec{e}_2\,'\cdot \vec{e}_2 & \vec{e}_2\,'\cdot \vec{e}_3 \\
		\vec{e}_3\,'\cdot \vec{e}_1 & \vec{e}_3\,'\cdot \vec{e}_2 & \vec{e}_3\,'\cdot \vec{e}_3 
	\end{array}
	\right)
	\label{eqn:defQ}
\end{equation}
とすれば,式(\ref{eqn:i2id})と式(\ref{eqn:id2i})の関係は,それぞれ 
\begin{equation}
	\fat{u}'=\fat{Q}\fat{u}, \ \ 
	\fat{u}=\fat{Q}^T\fat{u}'
	\label{eqn:u2ud}
\end{equation}
と表すことができる.このことは同時に
\begin{equation}
	\fat{Q}^{-1}=\fat{Q}^{T} \ \ \Rightarrow \ \ \fat{QQ}^T=\fat{Q}^T\fat{Q}=\fat{I}
	\label{eqn:OrthQ}
\end{equation}
であることを意味する.式(\ref{eqn:u2ud})は2つの座標系の間の座標変換法則を与えることから,
$\fat{Q}$は{\bf 座標変換行列}と呼ばれる.
数学的には,式(\ref{eqn:OrthQ})の関係を満たす行列は直交行列と呼ばれる.
2次空間におけるベクトルに対する座標変換マトリクスは, 第3軸に関する情報を省略して,
次の$2\times 2$行列で与えられる.
\begin{equation}
	\fat{Q}=
	\left( 
	\begin{array}{cc}
		\vec{e}_1\,'\cdot \vec{e}_1 & \vec{e}_1\,'\cdot \vec{e}_2 \\
		\vec{e}_2\,'\cdot \vec{e}_1 & \vec{e}_2\,'\cdot \vec{e}_2 \
	\end{array}
	\right)
\end{equation}
特に, 
$o-x_1'x_2'$座標が $o-x_1x_2$座標を反時計回りの方向に$\theta$だけ回転させて得られる
図\ref{fig:fig4_3}のようなものであるとき,
座標変換行列は
\begin{equation}
	\fat{Q}=
	\left( 
	\begin{array}{cc}
		 \cos \theta & \sin \theta \\
		-\sin \theta & \cos \theta 
	\end{array}
	\right)
	\label{eqn:Qth}
\end{equation}
と与えられる.
\subsection{問題}
2つの座標系$o-x_1x_2$と$o-x_1'x_2'$が, 互いに, 図\ref{fig:Qth_example}
のような関係にあるとき, 座標変換マトリクス$\fat{Q}$を求めよ. 
%%%%%%%%%%%%%%%%%%%%%%%%%%%%%%%%%%%%%%%%%%%%%%%%%%%%%%%%%%%%%%%%
\begin{figure}[h]
	\begin{center}
	\includegraphics[width=0.45\linewidth]{fig4_1.eps} 
	\end{center}
	\caption{原点$o$を共有する二つの直交座標系$o-x_1x_2$および$o-x_1'x_2'$. 
	$\fat{e}_1, \fat{e}_2$は前者の,$\fat{e}_1', \fat{e}_2'$は後者の座標系に
	おける基底ベクトルを表す.一般のベクトル$\fat{u}$は,$o-x_1x_2$座標系では
	座標$(u_1, u_2)$により,$o-x_1'x_2'$系では座標$(u_1', u_2')$によって
	表現される.}
	\label{fig:fig4_3}
\end{figure}
\begin{figure}[h]
	\begin{center}
	\includegraphics[width=0.7\linewidth]{Qth_example.eps} 
	\end{center}
	\caption{原点$o$を共有する二つの直交座標系$o-x_1x_2$および$o-x_1'x_2'$ 
	(座標変換行列を計算するための例題).}
	\label{fig:Qth_example}
\end{figure}
%%%%%%%%%%%%%%%%%%%%%%%%%%%%%%%%%%%%%%%%%%%%%%%%%%%%%%%%%%%%%%%%
\section{応力テンソルの座標変換法則}
\subsection{座標変換法則の導出}
応力テンソルを表す行列を,$o-x_1x_2x_3$座標系では$\fat{\sigma}$, 
$o-x_1'x_2'x_3'$座標系では$\fat{\sigma}'$と表す.このとき,
トラクションと法線ベクトルの関係は,各々の座標系において
\begin{eqnarray}
	\fat{t}^{(n)} &=& \fat{\sigma}\fat{n} 
		\label{eqn:n2tn} \\
	\fat{t}'^{(n')} &=& \fat{\sigma}'\fat{n}'
		\label{eqn:nd2tnd} 
\end{eqnarray}
と表される($\fat{\sigma}=\fat{\sigma}^T$を使った).
ここで式(\ref{eqn:n2tn})にベクトルの座標変換法則を適用すれば,
\begin{equation}
	\fat{t}^{(n)}=\fat{Q}^T \fat{t}'^{(n')}, \ \ 
	\fat{n} = \fat{Q}^T \fat{n}'
\end{equation}
だから,これらを式(\ref{eqn:n2tn})に代入して両辺に左側から$\fat{Q}$をかければ, 
次の式が得られる.
\begin{equation}
	\fat{t}'^{(n')}=\fat{Q} \fat{\sigma} \fat{Q}^T \fat{n}'
\end{equation}
これを式(\ref{eqn:nd2tnd})と比較することで, {\bf 応力テンソルに対する
座標変換則}:
\begin{equation}
	\fat{\sigma}'=\fat{Q}\fat{\sigma}\fat{Q}^T
	\label{eqn:s2sd}
\end{equation}
が導かれる.ここで, 2次元問題における応力テンソルを
\begin{equation}
	\fat{\sigma}=\left(
		\begin{array}{cc}
		 \sigma_{11} & \sigma_{12} \\
		 \sigma_{21} & \sigma_{22} 
		\end{array}
	\right)
	, \ \ 
	\fat{\sigma}'=\left(
		\begin{array}{cc}
	 \sigma'_{11} & \sigma'_{12} \\
	 \sigma'_{21} & \sigma'_{22} 
	\end{array}
	\right)
\end{equation}
と表し, 式(\ref{eqn:Qth})を式(\ref{eqn:s2sd})に代入して各成分を具体的に計算すれば, 
\begin{eqnarray}
	\sigma'_{11} &=& \bar{\sigma} + \frac{\Delta \sigma}{2} \cos 2\theta + \tau \sin 2\theta
		\label{eqn:s11d} \\
	\sigma'_{12}=\sigma'_{21} &=& - \frac{\Delta \sigma}{2} \sin 2\theta + \tau \cos 2\theta
		\label{eqn:s12d} \\
	\sigma'_{22} &=& \bar{\sigma} - \frac{\Delta \sigma}{2} \cos 2\theta - \tau \sin 2\theta 
		\label{eqn:s22d}
\end{eqnarray}
となることが示される.ただし, 
\begin{equation}
	\bar \sigma = \frac{\sigma_{11}+\sigma_{22}}{2}, \ \ 
	\Delta \sigma = \sigma_{11} -\sigma_{22}, \ \ 
	\tau = \sigma_{12}=\sigma_{21}
	\label{eqn:def_sig_bar}
\end{equation}
である. さらに,
\begin{equation}
	R=\sqrt{\left(\frac{\Delta \sigma}{2}\right)^2 + \tau ^2 }, \ \ 
	\tan \phi = \frac{\tau}{\Delta\sigma /2}
	\label{eqn:def_R}
\end{equation}
とすれば, 式(\ref{eqn:s11d})-(\ref{eqn:s22d})は以下のように書くことができる.
\begin{eqnarray}
	\sigma'_{11} &=& \bar{\sigma}+R \cos \left( \phi -2\theta \right) 
		\label{eqn:s11d_2} \\
	\sigma'_{12}=\sigma'_{21} &=&  R \sin \left( \phi -2 \theta \right)
		\label{eqn:s12d_2} \\
	\sigma'_{22} &=& \bar{\sigma} - R \cos \left( \phi-2\theta \right) 
		\label{eqn:s22d_2}
\end{eqnarray}
\subsection{モールの応力円}
式(\ref{eqn:s11d_2})と式(\ref{eqn:s12d_2})から$\theta$を消去すれば, 
\begin{equation}
	\left( \sigma'_{11} -\bar{\sigma}\right)^2 + \left(\sigma'_{12}\right)^2 = R^2
\end{equation}
が得られる.従って,$\sigma_{11}'$を横軸に, $\sigma'_{12}$を縦軸とした平面では,
$(\sigma'_{11}, \sigma'_{12})$は中心$(\bar{\sigma}, 0)$, 半径$R$の円上の点で
あることが分かる(図\ref{fig:fig5_1}).このように, 座標系の選び方によるによる応力成分の変化を
表す円を{\bf モールの応力円}と呼ぶ.

$\theta=0$のとき, $o-x_1x_2$座標系と$o-x_1'x_2'$座標系は一致する.
従って, $(\sigma_{11}',\,\sigma_{12}')= (\sigma_{11},\,\sigma_{12})$である.
図\ref{fig:fig5_1}ではこの位置を点Aと表している.
なお,
\begin{equation}
	\angle {\rm APQ} 
	= \tan^{-1}\left(\frac{\overline{\rm AQ}}{\overline{\rm PQ}}\right)
	=\frac{\sigma_{12}}{\Delta\sigma /2}
\end{equation}
だから,式(\ref{eqn:def_R})より
\begin{equation}
	\angle {\rm APB} =
	\angle {\rm APQ} =\phi
\end{equation}
であることが分かる.さらに,式(\ref{eqn:s11d_2})と式(\ref{eqn:s12d_2})より,
$\theta$が零から次第に大きくなるとき,$(\sigma_{11}',\sigma_{12}')$は点Aを
始点としてモールの応力円上を反時計回るの方向に移動することも分かる.
特に,点Bと点Cでは$\sigma_{12}'=0$で,それぞれに対応する方向$\theta$と$\sigma_{11}'$は,
\begin{eqnarray}
	{\rm B}点 &:& \theta=\frac{\phi}{2}, \ \ \sigma_{11}'=\bar\sigma +R 
	\label{eqn:th_max}
	\\
	{\rm C}点 &:& \theta=\frac{\phi+\pi}{2}, \ \ \sigma_{11}'=\bar\sigma -R
	\label{eqn:th_min}
\end{eqnarray}
となり,これらは直応力$\sigma_{11}'$の最大および最小値となることも明らかである.
以上より,モールの応力円上で$\angle$APBを図り$\phi$を決定すれば,
直応力の最大,最小値とそれらを与える方向が求められる.
最後に,一般の方向$\theta$における$\sigma_{11}'$と$\sigma_{12}'$の値を
モールの応力円から読み取れば,$\sigma_{22}'$は
\begin{equation}
	\sigma_{11}'+\sigma_{22}'=2\bar\sigma \ \ 
	\Rightarrow 
	\ \ 
	\sigma_{22}'=2\bar \sigma- \sigma_{11}'
\end{equation}
から得られ,2次元の応力テンソル成分全てを決定することができる.
この意味で,モールの応力円は2次元応力テンソルが$\theta$に応じてどのように変化する
かについての情報が全て表現されている.
%%%%%%
\begin{figure}[h]
	\begin{center}
	\includegraphics[width=0.9\linewidth]{fig5_1.eps} 
	\end{center}
	\caption{
		(a)モールの応力円.
	(b) 二つの座標系における各応力成分が作用する面とその方向
	} 
	\label{fig:fig5_1}
\end{figure}
\subsection{問題}
$o-x_1x_2$座標系における応力テンソル$\fat{\sigma}$が, 次のように与えられるとき, 
$o-x_1'x_2'$における応力テンソル$\fat{\sigma}'$を, 式(\ref{eqn:Qth})の座標変換行列
を用いて求めよ. 
\begin{enumerate}
\item
\begin{enumerate}
\item
\begin{equation}
	\fat{\sigma}
	=
	\left(
	\begin{array}{cc}
		p & 0 \\
		0 & p
	\end{array}
	\right)
	\label{eqn:sig_iso}
\end{equation}
\item
\begin{equation}
	\fat{\sigma}=
	\left(
	\begin{array}{cc}
		q & 0 \\
		0 & -q
	\end{array}
	\right)
	\label{eqn:sig_diff}
\end{equation}
\item
\begin{equation}
	\fat{\sigma}=
	\left(
	\begin{array}{cc}
		0 & r \\
		r & 0 
	\end{array}
	\right)
	\label{eqn:sig_shear}
\end{equation}
\end{enumerate}
\item
$\bar \sigma,\, \Delta \sigma, \,\tau$および$R$を, 式(\ref{eqn:def_sig_bar})-(\ref{eqn:def_R})
のように定義するとき, 任意の応力テンソル$\fat{\sigma}=\left\{ \sigma_{ij}\right\}$について
以下の関係が成り立つことを示せ. 
\begin{equation}
	\fat{\sigma}=
	\left(
	\begin{array}{cc}
		\sigma_{11} & \sigma_{12} \\
		\sigma_{21} & \sigma_{22} 
	\end{array}
	\right)
	=
	\left(
	\begin{array}{cc}
		\bar {\sigma} & 0 \\
		0 & \bar{\sigma}
	\end{array}
	\right)
	+
	\left(
	\begin{array}{cc}
		\frac{\Delta \sigma}{2}  & 0 \\
		0 & -\frac{\Delta \sigma}{2}  
	\end{array}
	\right)
	+
	\left(
	\begin{array}{cc}
		0 & \tau  \\
		\tau & 0 
	\end{array}
	\right)
\end{equation}
\item
式(\ref{eqn:sig_iso})から(\ref{eqn:sig_shear})の3つの応力テンソル
について,それぞれモールの応力円を描け. 
\item
	\begin{enumerate}
	\item
		モールの応力円が図\ref{fig:ex6_Mohr}のように与えられるとき, 
		モールの応力円上の点Aで表される, 応力成分の値を全てを答えよ. 
	\item
		応力テンソル:
		\begin{equation}
		\fat{\sigma}=\left(
		\begin{array}{cc}
			0 & \sqrt{3}\\
			\sqrt{3} & -2
		\end{array}
		\right)
		\label{eqn:sig_num}
		\end{equation}
		に対するモールの応力円を描け. 	
	\end{enumerate}
\end{enumerate}
\begin{figure}[h]
	\begin{center}
	\includegraphics[width=0.4\linewidth]{ex6_Mohr.eps} 
	\end{center}
	\caption{
	モールの応力円. 点Aは, $o-x_1x_2$座標と$o-x_1'x_2'$座標が
	一致する, すなわち$\theta=0$のときの応力値を示す. 
	 } 
	\label{fig:ex6_Mohr}
\end{figure}
%%%%%%
\section{主応力と主応力方向}
図\ref{fig:fig5_2}-(a)に示すように,一般的な状況ではトラクションベクトル$\fat{t}^{(n)}$と
法線ベクトル$\fat{n}$は一致しない.
別の言い方をすれば,$\fat{t}^{(n)}$は垂直成分$\sigma_p$と零でないせん断成分(接線方向成分)$\tau$をもつ.
しかしながら,特別な法線ベクトルの方向を選べば$t_m=0$とすることができる.
そのような法線ベクトルを$\fat{n}_p$とすれば,
図\ref{fig:fig5_2}-(b)に示すように$\fat{t}^{n_p}$は$\fat{n}_p$と同じ方向を向くため,
トラクションベクトルを
\begin{equation}
	\fat{t}^{(n_p)}=\lambda \fat{n}_p
	\label{eqn:def_np}
\end{equation}
と書くことができる.
%--------------------
\begin{figure}[h]
	\begin{center}
	\includegraphics[width=0.7\linewidth]{fig5_2.eps} 
	\end{center}
	\caption{
	同一の点$\fat{x}$におけるトラクションベクトル. (a)一般の面.(b)主応力面($\fat{n}_p$は主応力方向を
	表す単位ベクトル).} 
	\label{fig:fig5_2}
\end{figure}
%--------------------
このような$\lambda$を{\bf 主応力}, $\fat{n}_p$を{\bf 主応力方向}と呼ぶ.
主応力とその方向は$\fat{t}^{(np)}=\fat{\sigma}\fat{n}_p$を式\ref{eqn:def_np})に代入して得られる
次の関係を用いて決定することができる.
\begin{equation}
	\fat{\sigma}\fat{n_p} 
	=\lambda \fat{n}_p, \ \ (|\fat{n}_p|=1)
	\label{eqn:eig_prb}
\end{equation}
これは,与えられた行列$\fat{\sigma}$に対して,その固有値$\lambda$と
固有ベクトル$\fat{n}_p$を求める問題である.
固有値と固有ベクトルは数学的な概念だが,ここではそれらが主応力と
主応力方向という物理的な対象を表現している.
応力テンソルは実数値対称行列であることから, 固有値は実数で固有ベクトルは
互いに直交する.また,固有値は
$\lambda$は固有方程式:
\begin{equation}
	{\rm det} \left( \fat{\sigma}- \lambda \fat{I} \right) =\fat{0} 
	\label{eqn:eig_eq}
\end{equation}
を$\lambda$について解くことで得られる.ただし, $\fat{I}$は単位行列を表す.
以上の議論は2次元,3次元問題ともに当てはまるものである.
2次元問題の場合は, 式(\ref{eqn:eig_eq})は次のような2次方程式になり,
2つの固有値すなわち主応力値を簡単に求めることができる.
\begin{equation}
	\lambda^2 -\left( \sigma_{11}+\sigma_{22}\right)\lambda + 
	\sigma_{11}\sigma_{22}-\sigma_{12}\sigma_{21}=0
	\label{eqn:eig_eq_2d}
\end{equation}
この方程式の根を求め,式(\ref{eqn:def_sig_bar})と式(\ref{eqn:def_R})を用いて
整理すれば,その結果は
\begin{equation}
	\lambda=\bar \sigma \pm R
	\label{eqn:lmb_solved}
\end{equation}
となり,2つの主応力(固有値)は直応力の最大値と最小値に一致する.
このことから,$\bar \sigma +R$を最大主応力,$\bar \sigma-R$を最小主応力と呼ぶ.
主応力方向は式(\ref{eqn:lmb_solved})を
\begin{equation}
	\left( \fat{\sigma} -\lambda \fat{I} \right) \fat{n}_p=\fat{0}
	\label{eqn:find_np}
\end{equation}
に代入して対応する固有ベクトルを求めることで決定できる.
このようにして得た主応力方向は,必然的に式(\ref{eqn:th_max}),(\ref{eqn:th_min})に一致する.
ただし,式(\ref{eqn:find_np})は式(\ref{eqn:th_max})と(\ref{eqn:th_min})に含まれる
$\phi$が陽に求められない場合にも主応力方向を求めることができるという利点がある.

最後に,2次元問題において最大主応力方向に$x_1'$軸を, 最小主応力方向に$x_2'$軸を
とれば, $o-x_1'x_2'$座標系での主応力方向は
\begin{equation}
	\fat{n}_p=(1,0)^T \quad {\rm or} \quad (0,1)^T
\end{equation}
で,$\fat{\sigma}'\fat{n}_p'=\lambda \fat{n}_p'$より
応力テンソルが
\begin{equation}
	\fat{\sigma}'
	=
	\left(
	\begin{array}{cc}
		\sigma_{11}' & \sigma_{12}' \\
		\sigma_{21}' & \sigma_{22}' 
	\end{array}
	\right)
	=
	\left(
	\begin{array}{cc}
		\sigma_1 & 0 \\
		0 & \sigma_2
	\end{array}
	\right)
\end{equation}
と対角行列になることが分かる.ただし,$\sigma_1$は最大主応力,$\sigma_2$は最小主応力である.
%ただし, $\sigma_1$と$\sigma_2$は2つの
%主応力値を意味する.モールの応力円上において, せん断応力$\sigma_{12}'=0$と
%なるのは, 図\ref{fig:fig5_1}-(a)の点Bと点Cである.
%点Bは最大主応力値$\bar{\sigma}+R$を与え, そのときの$x_1'$軸の方向, すなわち
%最大主応力方向は, $\phi -2 \theta =0 $より$x_1$軸から反時計回りに$\frac{\phi}{2}$
%の方向である.一方, 点Cは最小主応力値$\bar{\sigma}-R$を与え, そのときの$x_1'$軸の方向, 
%すなわち最小主応力方向は$\phi -2 \theta =- \pi $より$x_1$軸から反時計回りに
%$\frac{\phi+\pi}{2}$の方向である.
\subsection{問題}
式(\ref{eqn:sig_iso})-(\ref{eqn:sig_shear})と, 式(\ref{eqn:sig_num})
で与えられるそれぞれの応力テンソルについて, 最大および最小主応力と対応する主応力方向を求めよ. 
\end{document}

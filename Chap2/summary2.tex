\documentclass[10pt,a4j]{jbook}
%\usepackage{graphicx,wrapfig}
\usepackage{graphicx}
\setlength{\topmargin}{-1.5cm}
%\setlength{\textwidth}{16.5cm}
\setlength{\textheight}{25.2cm}
\newlength{\minitwocolumn}
\setlength{\minitwocolumn}{0.5\textwidth}
\addtolength{\minitwocolumn}{-\columnsep}
%\addtolength{\baselineskip}{-0.1\baselineskip}
%
\def\Mmaru#1{{\ooalign{\hfil#1\/\hfil\crcr
\raise.167ex\hbox{\mathhexbox 20D}}}}
%
\begin{document}
\newcommand{\fat}[1]{\mbox{\boldmath $#1$}}
\newcommand{\D}{\partial}
\newcommand{\w}{\omega}
\newcommand{\ga}{\alpha}
\newcommand{\gb}{\beta}
\newcommand{\gx}{\xi}
\newcommand{\gz}{\zeta}
\newcommand{\vhat}[1]{\hat{\fat{#1}}}
\newcommand{\spc}{\vspace{0.7\baselineskip}}
\newcommand{\halfspc}{\vspace{0.3\baselineskip}}
\bibliographystyle{unsrt}
%\pagestyle{empty}
\newcommand{\twofig}[2]
 {
   \begin{figure}
     \begin{minipage}[t]{\minitwocolumn}
         \begin{center}   #1
         \end{center}
     \end{minipage}
         \hspace{\columnsep}
     \begin{minipage}[t]{\minitwocolumn}
         \begin{center} #2
         \end{center}
     \end{minipage}
   \end{figure}
 }
%%%%%%%%%%%%%%%%%%%%%%%%%%%%%%%%%
%\vspace*{\baselineskip}
%\begin{center}
%	{\Large \bf 構造力学I及び演習A 講義メモ2} \\
%\end{center}
%%%%%%%%%%%%%%%%%%%%%%%%%%%%%%%%%%%%%%%%%%%%%%%%%%%%%%%%%%%%%%%%
\setcounter{chapter}{1}
\chapter{応力テンソル}
以下, 太字は{\bf 成分表示することを前提とした}ベクトルあるいは行列を表す.
\section{トラクションベクトル}
位置ベクトル$\fat{x}$で表される物体内の点において, 単位ベクトル$\fat{n}$を
法線にもつような面$\Pi(\fat{n};\fat{x})$を考える(図\ref{fig:fig3_1}-(a),(b)).
物体を$\Pi(\fat{n};\fat{x})$で仮想的に切断したとき,
点$\fat{x}$において面$\Pi(\fat{n};\fat{x})$に作用する単位面積あたりの
力のベクトルを$\fat{t}^{(n)}(\fat{x})$と書き, これを{\rm \bf トラクションベクトル}と呼ぶ.

\begin{figure}[h]
	\begin{center}
	\includegraphics[width=0.9\linewidth]{fig3_1.eps} 
	\end{center}
	\caption{(a)物体中の着目点$\fat{x}$と仮想的な切断面$\Pi(\fat{n};\fat{x})$.
	(b),(c) 着目点周辺の拡大図.
	$x_3$軸は紙面に垂直手前側に伸びる座表軸を,
	$\fat{e}_3$は$x_3$軸方向の基底ベクトルを表す.(c)は面$\Pi(-\fat{n};\fat{x})$
	に作用するトラクション$\fat{t}^(-n)$を表す.} 
	\label{fig:fig3_1}
\end{figure}
3次元空間におけるベクトルを表すために$o-x_1x_2x_3$直角直交座標系を
とり, その基底ベクトルを$\fat{e}_i,\, (i=1,2,3)$と表す.
ここで,基底ベクトルとは,各座標軸方向を向く単位ベクトルのことと考えて良い.
このとき, $\fat{e}_i\, (i=1,2,3)$は互いに次の関係を満たす.
\begin{equation}
	\fat{e}_i\cdot \fat{e}_j=
	\delta_{ij}=
	\left\{
	\begin{array}{cc}
		1 & (i=j) \\
		0 & (i \neq j) \\
	\end{array}
	\right.
	\label{eqn:ortho}
\end{equation}
ここで,$\fat{a} \cdot \fat{b}$は,ベクトル$\fat{a}$と$\fat{b}$の内積を意味する.
$\fat{a}$と$\fat{b}$のなす角を$\theta_{ab}$とすれば,内積は次の式で定義される.
\begin{equation}
	\fat{a}\cdot\fat{b}=
	\left|\fat{a}\right|
	\left|\fat{b}\right|
	\cos \theta_{ab}
\end{equation}
従って,式(\ref{eqn:ortho})の関係は, 基底ベクトルが互いに直交してその長さは1であることを意味する.
式(\ref{eqn:ortho})を満たす基底ベクトルを, 正規直交基底と呼ぶ.
いま, $\fat{n}=\fat{e}_i,\, (i=1,2,3)$としたときのトラクション
ベクトルを$\fat{t}^{(i)}$と表し, $\fat{t}^{(i)}$の第$j$成分を$\sigma_{ij}$
と書く.すなわち
\begin{eqnarray}
	\fat{n}=\fat{e}_1 &;& \fat{t}^{(n)}=\fat{t}^{(1)}
		=\left( \sigma_{11},\, \sigma_{12},\, \sigma_{13}\right)^T 
		\label{eqn:t1}
		\\
	\fat{n}=\fat{e}_2 &;& \fat{t}^{(n)}=\fat{t}^{(2)} 
		=\left( \sigma_{21},\, \sigma_{22},\, \sigma_{23}\right)^T 
		\label{eqn:t2}
		\\
	\fat{n}=\fat{e}_3 &;& \fat{t}^{(n)}=\fat{t}^{(3)} 
		=\left( \sigma_{31},\, \sigma_{32},\, \sigma_{33}\right)^T 
		\label{eqn:t3}
\end{eqnarray}
と表現する.ここに$(\cdot)^T$は, $(\cdot)$の転置を意味する.
従って,式(\ref{eqn:t1})-(\ref{eqn:t3})はいずれも縦ベクトルを表す.
図\ref{fig:fig3_3}はこれら特別なトラクションベクトルとそれが作用する面の様子を
2次元空間において図示したものである.
$\fat{n}=-\fat{e}_i$に対するトラクションベクトルを$\fat{t}^{(-i)}$と表し, 
それらが作用する面との関係も示している.
$\fat{t}^{(-i)}=-\fat{t}^{(i)}$となることの理由は, 第\ref{equib}節で述べる.
\begin{figure}[h]
	\begin{center}
	\includegraphics[width=0.7\linewidth]{fig3_3.eps} 
	\end{center}
	\caption{基底ベクトル$\fat{e}_i$を法線ベクトルとする
	面(座標面)と,各々の面に作用するトラクションベクトル$\fat{t}^{(\pm i)}$.}
	\label{fig:fig3_3}
\end{figure}
\section{応力テンソル}
\subsection{応力テンソルの定義}
トラクションベクトル$\fat{t}^{(1)},\, \fat{t}^{(2)},\,\fat{t}^{(3)}$の成分
$\sigma_{ij},(i,j=1,2,3)$を並べた
\begin{equation}
	\fat{\sigma}=\left\{ \sigma_{ij}\right\}
	=\left(
	\begin{array}{ccc}
		\sigma_{11} & \sigma_{12} & \sigma_{13} \\
		\sigma_{21} & \sigma_{22} & \sigma_{23} \\
		\sigma_{31} & \sigma_{32} & \sigma_{33} 
	\end{array}
	\right)
\end{equation}
の行列を{\rm \bf 応力テンソル}, その成分$\sigma_{ij}$を応力成分と呼ぶ.
9つの応力成分のうち, $\sigma_{11},\sigma_{22},\sigma_{33}$を{\rm\bf 直応力}, 
それ以外のものを{\bf \rm せん断応力成分}と呼ぶ. $\sigma_{ij}$の
第一インデックス$i$は面の向きが$\fat{n}=\fat{e}_i$であることを表す.
一方, 第二インデックス$j$は, 第$j$方向の単位面積あたりの力を意味している.
また,図\ref{fig:fig3_5}のような図
を描くことで, 応力成分の意味を視覚的に表現することがしばしば行われる.
%%%%%%%%%%%%%%%%%%%%%%%%
\begin{figure}[h]
	\begin{center}
	\includegraphics[width=0.4\linewidth]{fig3_5.eps} 
	\end{center}
	\caption{着目点$\fat{x}$における応力成分の視覚的な表現.
	矩形領域は, 応力成分が作用する面を表すために描かれたものであり
	実際には領域の大きさは無限小であることに注意する必要がある.
	この図では, 以上の点について注意を喚起することを意図し, 
	矩形領域の幅と高さを$\Delta x_1, \Delta x_2$でなく
	$dx_1, dx_2$と表している. }
	\label{fig:fig3_5}
\end{figure}
%%%%%%%%%%%%%%%%%%%%%%%%
\subsection{釣り合い条件からの帰結}\label{equib}
\begin{itemize}
\item
	{\bf 表裏面それぞれに作用するトラクションの関係:}\\
	図\ref{fig:fig3_2}のように,物体中の着目点$\fat{x}$を含むように微小な薄層領域ABCDをとり
	その釣り合い条件式を立てる.$\Delta h \rightarrow 0$と$\Delta S\rightarrow 0$の極限を
	順にとれば,釣り合い式は
	\begin{equation}
		\fat{t}^{(n)}+\fat{t}^{(-n)}=\fat{0}
		\label{eqn:t_pmn}
	\end{equation}
	となることが示される.これは, 仮想切断面の表裏$\Pi(\pm \fat{n};\fat{x})$に働く
	トラクションは, 大きさが等しく方向が反対であることを意味する.
\begin{figure}[h]
	\begin{center}
	\includegraphics[width=0.6\linewidth]{fig3_2.eps} 
	\end{center}
	\caption{物体中の着目点$\fat{x}$を含む微小薄膜層ABCDと,その表面に
	作用するトラクションベクトル. } 
	\label{fig:fig3_2}
\end{figure}
\item
	図\ref{fig:fig3_4}のように,着目点$\fat{x}$を含む微小な三角形領域
	(3次元問題の場合は四面体領域)に対して釣り合い条件を考える.
	その結果, 一般のトラクションベクトル$\fat{t}^{(n)}(\fat{x})$と法線ベクトル$\fat{n}$が, 
	\begin{equation}
		\fat{t}^{(n)}=\fat{\sigma}^{T}\fat{n}
		\label{eqn:t_as_sign}
	\end{equation}
	と, 応力テンソル$\fat{\sigma}$を介して結び付けられることが示される.
	応力テンソル成分は(座標系には依存するが)法線ベクトル$\fat{n}$, 
	すなわち切断面の向きに依存しない. このことは, 応力テンソルが内力を表す
	本質的な量であることを意味する.
\item
	図\ref{fig:fig4_1}に示すように, 着目点$\fat{x}$を含む微小な矩形領域
	(3次元問題の場合は立方体領域)に対して, トルクの釣り合いを考えることで, 
	\begin{equation}
		\fat{\sigma}^{T}=\fat{\sigma}
		\label{eqn:sig_sym}
	\end{equation}
	となることが示される.すなわち, 応力テンソルは対称行列で表され, せん断応力成分
	の間には
	\begin{equation}
		\sigma_{ij}=\sigma_{ji}, \ \ (i,j=1,2,3)
		\label{eqn:sig_sym_comp}
	\end{equation}
	の関係がある.よって,独立な応力成分は, 直応力成分3つ, せん断応力成分3つの計6つ
	であることが分かる.平面内でのモーメント(トルク)の
	計算方法は図\ref{fig:fig4_2}を参照.
\item
	図\ref{fig:fig4_1}に示すような,着目点$\fat{x}$を含む微小な矩形領域
	(3次元問題の場合は立方体領域)に対して, 力の釣り合いを考えることで, 
	応力成分が満足すべき偏微分方程式が次のように得られる.
	\begin{equation}
		\sum_{j=1}^3 \frac{\partial \fat{t}^{(j)}}{\partial x_j}+\fat{b}=\fat{0}
		\label{eqn:stress_equib}
	\end{equation}
	ただし, $\fat{b}$は単位体積にあたりの力として与えられる分布外力を意味し, 一般に
	物体力と呼ばれる.重力は典型的な物体力の例である.
	式(\ref{eqn:stress_equib})を成分表記すると, 
	\begin{equation}
		\sum_{j=1}^3 \frac{\partial \sigma_{ji}}{\partial x_j}+b_i=0, \ \ (i=1,2,3)
		\label{eqn:stress_equib_compo}
	\end{equation}
	と表すこともできる.ただし$\fat{b}=(b_1,\,b_2,\, b_3)$である.
	式(\ref{eqn:stress_equib})あるいは(\ref{eqn:stress_equib_compo})で表される
	連立偏微分方程式は, 応力の釣り合い条件式と呼ばれる.
\end{itemize}
\begin{figure}[h]
	\begin{center}
	\includegraphics[width=0.4\linewidth]{fig3_4.eps} 
	\end{center}
	\caption{着目点$\fat{x}$の近傍にとった微小三角形領域の自由物体図.}
	\label{fig:fig3_4}
\end{figure}
\begin{figure}[h]
	\begin{center}
	\includegraphics[width=1.0\linewidth]{fig4_2.eps} 
	\end{center}
	\caption{着目点$\fat{x}$近傍の微小矩形領域に対する自由物体図.
	(a)トラクションベクトルを用いて表現した場合の図.(b)応力成分を用いて表現した場合の図.} 
	\label{fig:fig4_1}
\end{figure}
\begin{figure}[h]
	\begin{center}
	\includegraphics[width=0.5\linewidth]{fig4_3.eps} 
	\end{center}
	\caption{
	位置$\fat{x}$に作用する荷重ベクトル$\fat{F}$.
	$h$は位置ベクトルの原点$\fat{o}$と荷重作用線の距離を,
	$F_n$は$\fat{F}$の$\fat{x}$と直交する方向の成分を表す.
	これらの量を用いて荷重$\fat{F}$による点$o$に関するモーメント(トルク)$T$を,
	$T=|\fat{x}|\times F_n=|\fat{F}|\times h$で計算することができる.
	どちらも同じ結果$T=|\fat{F}| |\fat{x}|\sin\alpha$を与える.
	網掛けした平行四辺形は, トルク(モーメント)の大きさに相当する.
	 } 
	\label{fig:fig4_2}
\end{figure}
%%%%%%%%%%%%%%%%%%%%%%%%
\subsection{問題}
図\ref{fig:ex5_Torque}に関する以下の問に答えよ.
\begin{enumerate}
\item
図(a)に示すような大きさ$F$の力を受ける棒部材について, 点A,B,CおよびDに関する
モーメントを求めよ. ただし, モーメントは反時計回りの方向を正とする. 
\item
図(b)に示すような力を受けて, 棒部材ACが静止している. このとき, 水平力$H_A$, 
鉛直力$V_A$および$V_C$を$F$を用いて表わせ. 
\item
図(c)に示すような荷重を受ける, L型の部材について, 点A,B,C,Dに関するモーメントを求めよ. 
ただし, モーメントは反時計回りの方向を正とする. 
\item
図(d)のように, 長さ$l$,質量$m$の棒部材ACが, 壁に立てかけられて静止している. 
摩擦力が 壁側では無視できると仮定するとき, 部材が床と
壁から受ける力$H_A, V_A$と$H_C$を求めよ. また, $\alpha\rightarrow 0$と$\alpha \rightarrow \infty$
においてこれらの力が取る値を答えよ. なお, 部材には, 壁面と床面からの力に加え, 
重心位置Bにおいて, 重力$mg$が鉛直下向きに作用するものとする. 
\end{enumerate}
\begin{figure}[h]
	\begin{center}
	\includegraphics[width=0.8\linewidth]{ex5_Torque.eps} 
	\end{center}
	\caption{力とモーメント(トルク)の釣り合いに関する例題.} 
	\label{fig:ex5_Torque}
\end{figure}
%%%%%%%%%%%%%%%%%%%%%%%%%%%%%%%%%%%%%
\section{ベクトルとテンソルの座標変換}
\subsubsection{座標変換マトリクス}
原点$o$を共有する2つの直交座標系$o-x_1x_2x_3$と$o-x_1'x_2'x_3'$の間の座標変換について考える.
ここで, 実体としてのベクトルを$\vec{u},\vec{x}$, それらの特定の座標系における表現を
$\fat{u},\fat{x}$や$\fat{u}',\fat{x}'$のように太字で表す. 
なお$()'$を付したものは$o-x_1'x_2'x_3'$系におけるベクトルを表す.
それ以外のものは$o-x_1x_2x_3$座標系における表現であると解釈し, 
座標成分を用いてベクトルやテンソルを表する場合,使用する座標系に応じて
\[
	\fat{u}=(u_1, u_2, u_3)^T, \ \ \fat{x}'=(x_1', x_2', x_3')^T
\]
等と書くこととする.
いま, $o-x_1x_2x_3$座標系の基底ベクトルを$\vec{e}_1, \vec{e}_2, \vec{e}_3$, 
$o-x_1'x_2'x_3'$におけるそれを$\vec{e}_1\,', \vec{e}_2\,', \vec{e}_3 \,'$, 
とする. これらは
\begin{equation}
	\vec{e}_i\cdot \vec{e}_j=\delta_{ij}, \ \ 
	\vec{e}_i\,' \cdot \vec{e}_j\,'=\delta_{ij}
	\label{eqn:ortho_norm}
\end{equation}
で表されるように,正規直交系をなす.ただし, $\delta_{ij}$はクロネッカーデルタ
\begin{equation}
	\delta _{ij}=\left\{
	\begin{array}{cc}
		1 & (i=j) \\
		0 & (i\neq j)
	\end{array}
	\right.
	\label{eqn:dij}
\end{equation}
を意味する.2つの座標系に対する基底ベクトルを用いれば, 任意のベクトル$\vec{u}$は
\begin{eqnarray}
	\vec{u} &=& \sum_{i=1}^3 u_i\vec{e}_i 
	\label{eqn:u_ei}
	\\
	&=& \sum_{i=1}^3 u'_i\vec{e}_i\,' 
	\label{eqn:u_eid}
\end{eqnarray}
と二通りに表すことができる. 座標成分$u_i$や$u_i'$は式(\ref{eqn:ortho_norm})-(\ref{eqn:u_eid})より
\begin{eqnarray}
	u_i &=& \vec{u}\cdot \vec{e}_i \label{eqn:ui}\\
	u'_i&=& \vec{u}\cdot \vec{e}_i\,' \label{eqn:uid}
\end{eqnarray}
と書ける. ここで, 式(\ref{eqn:uid})の右辺に式(\ref{eqn:u_ei})を代入すれば, 
\begin{equation}
	u_i'=\sum_{j=1}^3 (\vec{e}_i\,'\cdot \vec{e}_j) u_j 
\end{equation}
が得られる.一方, 式(\ref{eqn:ui})の右辺に, 式(\ref{eqn:u_eid})を代入すれば, 
\begin{equation}
	u_i=\sum_{j=1}^3 (\vec{e}_i\cdot \vec{e}_j\, ') u_j' 
\end{equation}
の関係が得られる.ここで,
\begin{equation}
	\fat{Q}=
	\left\{
		\vec{e}_i\,'\cdot \vec{e}_j
	\right\}
	=
	\left( 
	\begin{array}{ccc}
		\vec{e}_1\,'\cdot \vec{e}_1 & \vec{e}_1\,'\cdot \vec{e}_2 & \vec{e}_1\,'\cdot \vec{e}_3 \\
		\vec{e}_2\,'\cdot \vec{e}_1 & \vec{e}_2\,'\cdot \vec{e}_2 & \vec{e}_2\,'\cdot \vec{e}_3 \\
		\vec{e}_3\,'\cdot \vec{e}_1 & \vec{e}_3\,'\cdot \vec{e}_2 & \vec{e}_3\,'\cdot \vec{e}_3 
	\end{array}
	\right)
\end{equation}
とすれば, 
\begin{equation}
	\fat{u}'=\fat{Q}\fat{u}, \ \ 
	\fat{u}=\fat{Q}^T\fat{u}'
\end{equation}
となることが示され, これは同時に
\begin{equation}
	\fat{Q}^{-1}=\fat{Q}^{T} \ \ \Rightarrow \ \ \fat{QQ}^T=\fat{Q}^T\fat{Q}=\fat{I}
	\label{eqn:OrthQ}
\end{equation}
であることを意味する.$\fat{Q}$は座標変換行列と呼ばれる.
なお,数学的には,式(\ref{eqn:OrthQ})の関係を満たす行列は直交行列と呼ばれる.
2次空間におけるベクトルに対する座標変換マトリクスは, 次の$2\times 2$行列で与えられる.
\begin{equation}
	\fat{Q}=
	\left( 
	\begin{array}{cc}
		\vec{e}_1\,'\cdot \vec{e}_1 & \vec{e}_1\,'\cdot \vec{e}_2 \\
		\vec{e}_2\,'\cdot \vec{e}_1 & \vec{e}_2\,'\cdot \vec{e}_2 \
	\end{array}
	\right)
\end{equation}
特に, 図\ref{fig:fig4_3}に示すように, $o-x_1'x_2'$座標が, $o-x_1x_2$座標を反時計回りの方向に$\theta$だけ回転させることで
得られる場合, 座標変換マトリクスは
\begin{equation}
	\fat{Q}=
	\left( 
	\begin{array}{cc}
		 \cos \theta & \sin \theta \\
		-\sin \theta & \cos \theta 
	\end{array}
	\right)
	\label{eqn:Qth}
\end{equation}
と与えられる.
\subsection{問題}
2つの座標系$o-x_1x_2$と$o-x_1'x_2'$が, 互いに, 図\ref{fig:Qth_example}
のような関係にあるときの, 座標変換マトリクス$\fat{Q}$を求めよ. 
%%%%%%%%%%%%%%%%%%%%%%%%%%%%%%%%%%%%%%%%%%%%%%%%%%%%%%%%%%%%%%%%
\begin{figure}[h]
	\begin{center}
	\includegraphics[width=0.45\linewidth]{fig4_1.eps} 
	\end{center}
	\caption{原点$o$を共有する二つの直交座標系$o-x_1x_2$および$o-x_1'x_2'$. 
	$\fat{e}_1, \fat{e}_2$は前者の,$\fat{e}_1', \fat{e}_2'$は後者の座標系に
	おける基底ベクトルを表す.一般のベクトル$\fat{u}$は,$o-x_1x_2$座標系では
	座標$(u_1, u_2)$により,$o-x_1'x_2'$系では座標$(u_1', u_2')$によって
	表現される.}
	\label{fig:fig4_3}
\end{figure}
\begin{figure}[h]
	\begin{center}
	\includegraphics[width=0.7\linewidth]{Qth_example.eps} 
	\end{center}
	\caption{原点$o$を共有する二つの直交座標系$o-x_1x_2$および$o-x_1'x_2'$ 
	(座標変換行列を計算するための例題).}
	\label{fig:Qth_example}
\end{figure}
%%%%%%%%%%%%%%%%%%%%%%%%%%%%%%%%%%%%%%%%%%%%%%%%%%%%%%%%%%%%%%%%
\subsection{応力テンソルの座標変換法則}
応力テンソルを表す行列を,$o-x_1x_2x_3$座標系では$\fat{\sigma}$, 
$o-x_1'x_2'x_3'$座標系では$\fat{\sigma}'$と表す.このとき,
トラクションと法線ベクトルの関係は,各々の座標系において
\begin{eqnarray}
	\fat{t}^{(n)} &=& \fat{\sigma}\fat{n} 
		\label{eqn:n2tn} \\
	\fat{t}'^{(n')} &=& \fat{\sigma}'\fat{n}'
		\label{eqn:nd2tnd} 
\end{eqnarray}
と表される.ここで式(\ref{eqn:n2tn})にベクトルの座標変換法則を適用すれば,
\begin{equation}
	\fat{t}^{(n)}=\fat{Q}^T \fat{t}'^{(n')}, \ \ 
	\fat{n} = \fat{Q}^T \fat{n}'
\end{equation}
だから,これらを式(\ref{eqn:n2tn})に代入して両辺に左側から$\fat{Q}$をかければ, 
次の式が得られる.
\begin{equation}
	\fat{t}'^{(n')}=\fat{Q} \fat{\sigma} \fat{Q}^T \fat{n}'
\end{equation}
これを式(\ref{eqn:nd2tnd})と比較することで, 応力テンソルに対する
座標変換則:
\begin{equation}
	\fat{\sigma}'=\fat{Q}\fat{\sigma}\fat{Q}^T
	\label{eqn:s2sd}
\end{equation}
が導かれる.ここで, 2次元問題における応力テンソルを
\begin{equation}
	\fat{\sigma}=\left(
		\begin{array}{cc}
		 \sigma_{11} & \sigma_{12} \\
		 \sigma_{21} & \sigma_{22} 
		\end{array}
	\right)
	, \ \ 
	\fat{\sigma}'=\left(
		\begin{array}{cc}
	 \sigma'_{11} & \sigma'_{12} \\
	 \sigma'_{21} & \sigma'_{22} 
	\end{array}
	\right)
\end{equation}
と表し, 式(\ref{eqn:Qth})を式(\ref{eqn:s2sd})に代入して各成分を具体的に計算すれば, 
\begin{eqnarray}
	\sigma'_{11} &=& \bar{\sigma} + \frac{\Delta \sigma}{2} \cos 2\theta + \tau \sin 2\theta
		\label{eqn:s11d} \\
	\sigma'_{12}=\sigma'_{21} &=& - \frac{\Delta \sigma}{2} \sin 2\theta + \tau \cos 2\theta
		\label{eqn:s12d} \\
	\sigma'_{22} &=& \bar{\sigma} - \frac{\Delta \sigma}{2} \cos 2\theta - \tau \sin 2\theta 
		\label{eqn:s22d}
\end{eqnarray}
となることが示される.ただし, 
\begin{equation}
	\bar \sigma = \frac{\sigma_{11}+\sigma_{22}}{2}, \ \ 
	\Delta \sigma = \sigma_{11} -\sigma_{22}, \ \ 
	\tau = \sigma_{12}=\sigma_{21}
	\label{eqn:def_sig_bar}
\end{equation}
である. さらに,
\begin{equation}
	R=\sqrt{\left(\frac{\Delta \sigma}{2}\right)^2 + \tau ^2 }, \ \ 
	\tan \phi = \frac{\tau}{\Delta\sigma /2}
	\label{eqn:def_R}
\end{equation}
とすれば, 式(\ref{eqn:s11d})-(\ref{eqn:s22d})は以下のように書くことができる.
\begin{eqnarray}
	\sigma'_{11} &=& \bar{\sigma}+R \cos \left( \phi -2\theta \right) 
		\label{eqn:s11d_2} \\
	\sigma'_{12}=\sigma'_{21} &=&  R \sin \left( \phi -2 \theta \right)
		\label{eqn:s12d_2} \\
	\sigma'_{22} &=& \bar{\sigma} - R \cos \left( \phi-2\theta \right) 
		\label{eqn:s22d_2}
\end{eqnarray}
式(\ref{eqn:s11d_2})と式(\ref{eqn:s12d_2})から$\theta$を消去すれば, 
\begin{equation}
	\left( \sigma'_{11} -\bar{\sigma}\right)^2 + \left(\sigma'_{12}\right)^2 = R^2
\end{equation}
が得られる.その結果,$\sigma_{11}'$を横軸に, $\sigma'_{12}$を縦軸とした平面では,
$(\sigma'_{11}, \sigma'_{12})$は中心$(\bar{\sigma}, 0)$, 半径$R$の円上の点で
あることが分かる(図\ref{fig:fig5_1}).このように, 座標系の選び方によるによる応力成分の変化を
表す円を, モールの応力円と呼ぶ.
なお, $\theta=0$のとき, $o-x_1x_2$座標系と$o-x_1'x_2'$座標系は一致する.従って, 
$(\sigma_{11}',\,\sigma_{12}')= (\sigma_{11},\,\sigma_{12})$である.
図\ref{fig:fig5_1}ではこの位置を点Aと表している.
%%%%%%
\begin{figure}[h]
	\begin{center}
	\includegraphics[width=0.8\linewidth]{fig5_1.eps} 
	\end{center}
	\caption{
		(a)モールの応力円.
	(b) 二つの座標系における各応力成分が作用する面とその方向
	} 
	\label{fig:fig5_1}
\end{figure}
\subsection{問題}
$o-x_1x_2$座標系における応力テンソル$\fat{\sigma}$が, 次のように与えられるとき, 
$o-x_1'x_2'$における応力テンソル$\fat{\sigma}'$を, 式(\ref{eqn:Qth})の座標変換行列
を用いて求めよ. 
\begin{enumerate}
\item
\begin{enumerate}
\item
\begin{equation}
	\fat{\sigma}
	=
	\left(
	\begin{array}{cc}
		p & 0 \\
		0 & p
	\end{array}
	\right)
	\label{eqn:sig_iso}
\end{equation}
\item
\begin{equation}
	\fat{\sigma}=
	\left(
	\begin{array}{cc}
		q & 0 \\
		0 & -q
	\end{array}
	\right)
	\label{eqn:sig_diff}
\end{equation}
\item
\begin{equation}
	\fat{\sigma}=
	\left(
	\begin{array}{cc}
		0 & r \\
		r & 0 
	\end{array}
	\right)
	\label{eqn:sig_shear}
\end{equation}
\end{enumerate}
\item
$\bar \sigma,\, \Delta \sigma, \,\tau$および$R$を, 式(\ref{eqn:def_sig_bar})-(\ref{eqn:def_R})
のように定義するとき, 任意の応力テンソル$\fat{\sigma}=\left\{ \sigma_{ij}\right\}$について
以下の関係が成り立つことを示せ. 
\begin{equation}
	\fat{\sigma}=
	\left(
	\begin{array}{cc}
		\sigma_{11} & \sigma_{12} \\
		\sigma_{21} & \sigma_{22} 
	\end{array}
	\right)
	=
	\left(
	\begin{array}{cc}
		\bar {\sigma} & 0 \\
		0 & \bar{\sigma}
	\end{array}
	\right)
	+
	\left(
	\begin{array}{cc}
		\frac{\Delta \sigma}{2}  & 0 \\
		0 & -\frac{\Delta \sigma}{2}  
	\end{array}
	\right)
	+
	\left(
	\begin{array}{cc}
		0 & \tau  \\
		\tau & 0 
	\end{array}
	\right)
\end{equation}
\item
式(\ref{eqn:sig_iso})から(\ref{eqn:sig_shear})の3つの応力テンソル
それぞれについて, モールの応力円を描け. 
\item
	\begin{enumerate}
	\item
		モールの応力円が図\ref{fig:ex6_Mohr}のように与えられるとき, 
		モールの応力円上の点Aで表される, 応力成分の値を全てを答えよ. 
	\item
		応力テンソル:
		\begin{equation}
		\fat{\sigma}=\left(
		\begin{array}{cc}
			0 & \sqrt{3}\\
			\sqrt{3} & -2
		\end{array}
		\right)
		\label{eqn:sig_num}
		\end{equation}
		に対するモールの応力円を描け. 	
	\end{enumerate}
\end{enumerate}
\begin{figure}[h]
	\begin{center}
	\includegraphics[width=0.4\linewidth]{ex6_Mohr.eps} 
	\end{center}
	\caption{
	モールの応力円. 点Aは, $o-x_1x_2$座標と$o-x_1'x_2'$座標が
	一致する, すなわち$\theta=0$のときの応力値を示す. 
	 } 
	\label{fig:ex6_Mohr}
\end{figure}
%%%%%%
\section{主応力と主応力方向}
トラクションベクトル$\fat{t}^{(n)}$のせん断成分がゼロとなるような方向を
$\fat{n}=\fat{n}_p$とする. このとき, $\fat{t}^{(n_p)}$は法線方向性分のみを持ち, 
\begin{equation}
	\fat{t}^{(n_p)}=\lambda \fat{n}_p
	\label{eqn:def_np}
\end{equation}
が成り立つ(図\ref{fig:fig5_2}).このとき$\lambda$を主応力, $\fat{n}_p$を主応力方向と呼ぶ.
主応力とその方向は, 式(\ref{eqn:t_as_sign})と(\ref{eqn:def_np})より, 
\begin{equation}
	\fat{t}^{(n)}
	=\fat{\sigma}^T \fat{n_p} 
	=\fat{\sigma}\fat{n_p} 
	=\lambda \fat{n}_p
\end{equation}
であることから, それぞれ, 応力テンソル$\fat{\sigma}$の固有値, 
固有ベクトルとして求めることができる.応力テンソルは実数値対称行列であること
から, 固有値は実数で, 固有ベクトルは互いに直交する.
$\lambda$は固有方程式:
\begin{equation}
	{\rm det} \left( \fat{\sigma}- \lambda \fat{I} \right) =\fat{0} 
	\label{eqn:eig_eq}
\end{equation}
を解くことで得られる.ただし, $\fat{I}$は単位行列を表す.
2次元問題の場合, 式(\ref{eqn:eig_eq})は次のような2次方程式になり,
2つの固有値すなわち主応力値が存在する.
\begin{equation}
	\lambda^2 -\left( \sigma_{11}+\sigma_{22}\right)\lambda + 
	\sigma_{11}\sigma_{22}-\sigma_{12}\sigma_{21}=0
	\label{eqn:eig_eq_2d}
\end{equation}
この方程式の根を求め, その結果を
\begin{equation}
	\left( \fat{\sigma} -\lambda \fat{I} \right) \fat{n}=\fat{0}
\end{equation}
に代入して対応する固有ベクトルを求めれば, 主応力と主応力方向が決定できる.

2次元問題において, 主応力方向$\fat{n}_p$の向きに$x_1'$軸を, それに直交するように
$x_2$軸をとれば, $o-x_1'x_2'$座標系での応力テンソルは
\begin{equation}
	\fat{\sigma}'
	=
	\left(
	\begin{array}{cc}
		\sigma_{11}' & \sigma_{12}' \\
		\sigma_{21}' & \sigma_{22}' 
	\end{array}
	\right)
	=
	\left(
	\begin{array}{cc}
		\sigma_1 & 0 \\
		0 & \sigma_2
	\end{array}
	\right)
\end{equation}
となりせん断応力成分を持たない.ただし, $\sigma_1$と$\sigma_2$は2つの
主応力値を意味する.モールの応力円上において, せん断応力$\sigma_{12}'=0$と
なるのは, 図\ref{fig:fig5_1}-(a)の点Bと点Cである.
点Bは最大主応力値$\bar{\sigma}+R$を与え, そのときの$x_1'$軸の方向, すなわち
最大主応力方向は, $\phi -2 \theta =0 $より$x_1$軸から反時計回りに$\frac{\phi}{2}$
の方向である.一方, 点Cは最小主応力値$\bar{\sigma}-R$を与え, そのときの$x_1'$軸の方向, 
すなわち最小主応力方向は$\phi -2 \theta =- \pi $より$x_1$軸から反時計回りに
$\frac{\phi+\pi}{2}$の方向である.
\newpage
\subsection{問題}
式(\ref{eqn:sig_iso})-(\ref{eqn:sig_shear})と, 式(\ref{eqn:sig_num})
で与えられるそれぞれの応力テンソルについて, 最大および最小主応力と対応する主応力方向を
求めよ. 
%%%%%%%%%%%%%%%%%%%%%%
%--------------------
\begin{figure}[h]
	\begin{center}
	\includegraphics[width=0.7\linewidth]{fig5_2.eps} 
	\end{center}
	\caption{
	同一の点$\fat{x}$におけるトラクションベクトル.
	(a)一般の面.(b)主応力面($\fat{n}_p$は主応力方向を
	表す単位ベクトル).
	 } 
	\label{fig:fig5_2}
\end{figure}
\end{document}

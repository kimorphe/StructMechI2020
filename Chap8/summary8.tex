\documentclass[10pt,a4j]{jbook}
%\usepackage{graphicx,wrapfig}
\usepackage{graphicx,amsmath}
\setlength{\topmargin}{-1.5cm}
%\setlength{\textwidth}{16.5cm}
\setlength{\textheight}{25.2cm}
\newlength{\minitwocolumn}
\setlength{\minitwocolumn}{0.5\textwidth}
\addtolength{\minitwocolumn}{-\columnsep}
%\addtolength{\baselineskip}{-0.1\baselineskip}
%
\def\Mmaru#1{{\ooalign{\hfil#1\/\hfil\crcr
\raise.167ex\hbox{\mathhexbox 20D}}}}
%
\begin{document}
\newcommand{\fat}[1]{\mbox{\boldmath $#1$}}
\newcommand{\D}{\partial}
\newcommand{\w}{\omega}
\newcommand{\ga}{\alpha}
\newcommand{\gb}{\beta}
\newcommand{\gx}{\xi}
\newcommand{\gz}{\zeta}
\newcommand{\vhat}[1]{\hat{\fat{#1}}}
\newcommand{\spc}{\vspace{0.7\baselineskip}}
\newcommand{\halfspc}{\vspace{0.3\baselineskip}}
\bibliographystyle{unsrt}
%\pagestyle{empty}
\newcommand{\twofig}[2]
 {
   \begin{figure}
     \begin{minipage}[t]{\minitwocolumn}
         \begin{center}   #1
         \end{center}
     \end{minipage}
         \hspace{\columnsep}
     \begin{minipage}[t]{\minitwocolumn}
         \begin{center} #2
         \end{center}
     \end{minipage}
   \end{figure}
 }
%%%%%%%%%%%%%%%%%%%%%%%%%%%%%%%%%
%\vspace*{\baselineskip}
%%%%%%%%%%%%%%%%%%%%%%%%%%%%%%%%%%%%%%%%%%%%%%%%%%%%%%%%%%%%%%%%
\setcounter{chapter}{7}
\chapter{断面係数}
たわみの支配微分方程式に含まれる曲げ剛性$EI$のうち,ヤング率$E$は
実験によって求めるべき物性値である.一方,断面2次モーメント$I$は,
断面の幾何学的な形状によって決まる{\bf 断面係数}と呼ばれる量
の一つである.なお,ここでいう断面2次モーメントとは,
断面$S$の水平方向の中立軸$z$からの距離を$y$として,
\begin{equation}
	I=\int_S y^2 dS
	\label{eqn:Iz_xy}
\end{equation}
で与えられる.
従ってより正確には,$I$は"断面$S$の水平方向の中立軸($z$軸)に関する断面2次モーメント"を指す.
\section{中立軸と断面1次モーメント}
断面2次モーメント$I$の計算に先立ち,$S$の中立軸($z$軸)位置を決定する必要がある.
%中立面は,梁に軸力が作用せず曲げのみが生じるとき,伸びも縮みも生じない面が当初(変形前)
%に占める平面として定義されていた.
そのために,軸力$N$に関する条件を用いる.ここで,梁の長手方向に$x$軸をとり,$x$方向の直応力を
$\sigma_{xx}$と書く.軸力$N$と曲げモーメント$M$は$\sigma_{xx}$を用いて次の式で定義される.
\begin{eqnarray}
	N &= & \int_S \sigma_{xx}dS 
	\label{eqn:def_N}
	\\
	M &= & \int_S y\sigma_{xx}dS 
	\label{eqn:def_M}
\end{eqnarray}
また,梁に軸力が作用せず($N\equiv 0$)曲げ変形のみが生じるとき,直応力分布が
\begin{equation}
	\sigma_{xx}(x,y)=\frac{M(x)}{I(x)}y
	\label{eqn:sig_xx}
\end{equation}
で与えられることは既に学んだ通りである.従って,曲げ問題において軸力が零となるための条件は
\begin{equation}
	N =  \int_S \sigma_{xx}dS = \frac{M(x)}{I(x)} \int_S ydS = 0 \ \ 
	\Rightarrow \ \ 
	\int_S y dS=0
	\label{eqn:N_is_0}
\end{equation}
と表すことができる.ただし,現時点で中立軸位置($y$軸の原点位置)は未知のため,
式(\ref{eqn:N_is_0})の積分をこのままの形で実行することはできない.
そこで,図\ref{fig:fig11_1}のような既知の$YZ$直交座標系を積分計算のために導入する.
ただし,$YZ$座標系の原点は任意とし,$Y$軸と$Z$軸は,それぞれ$y$軸と$z$軸と同じ方向に取る.
ここで,$z$軸は$Z$軸から$\bar{Y}$だけ離れているとすれば,2つの座標系間の関係は
%--------------------
\begin{figure}[h]
	\begin{center}
	\includegraphics[width=0.4\linewidth]{./fig11_1.eps} 
	\end{center}
	\caption{
		断面係数計算のために定めた$ZY$座標系と中立軸位置$Y=\bar Y$.
	} 
	\label{fig:fig11_1}
\end{figure}
%--------------------
\begin{equation}
	y=Y-\bar Y, \ \ Z=z
	\label{eqn:y_shift}
\end{equation}
となる.これを式(\ref{eqn:N_is_0})に用いれば,
\begin{equation}
	N=\frac{M}{I}\int_S \left(Y-\bar Y \right)dS=0 
	\ \ \Rightarrow \ \
	\int_S Y dS =
	\bar Y \int_S dS
	=
	\bar Y \left| S \right|
	\label{eqn:as_Nis0}
\end{equation}
の関係が得られる.ただし,$\left| S\right|$は$S$の面積:
\begin{equation}
	\left| S \right| = \int_S dS
	\label{eqn:area}
\end{equation}
を表す.さらに,
\begin{equation}
	G_Z=\int_S YdS 
	\label{eqn:GZ}
\end{equation}
と定義すれば,式(\ref{eqn:as_Nis0})より
\begin{equation}
	\bar Y
	= \frac{\int_S YdS }{\int dS}
	= \frac{G_Z}{\left| S \right|}
	\label{eqn:Ybar}
\end{equation}
となることが示される. 
$G_Z$は断面$S$の$Z$軸に関する{\bf 断面1次モーメント}と呼ばれ,
中立軸位置を求めるには$\left|S\right|$と$G_Z$を求めて式(\ref{eqn:Ybar})に
用いればよいことが分かる.
\section{断面2次モーメント}
以上の方法によって中立軸位置が決定できれば,式(\ref{eqn:Iz_xy})の積分を$yz$座標系で
実行して断面2次モーメント$I$を求めることができる.
あるいは,$y=Y-\bar{Y}$を式(\ref{eqn:Iz_xy})に代入して$I$を次のように書き直し,
$YZ$座標系で積分計算を行うこともできる.
\begin{eqnarray}
	I &= &
	\int_S y^2 dS 
	\nonumber \\
	 &= &
	\int_S \left( Y-\bar{Y}\right)^2 dS 
	\nonumber \\
	 &= &
	\int_S Y^2dS -2\bar{Y}\int_S Y dS +\bar Y^2 \int_S dS
	\nonumber \\
	 &= &
	 I_Z-2\bar{Y}G_Z+\bar{Y}^2\left| S \right|
	\nonumber \\
	 &= &
	 I_Z-2\bar{Y}\times \bar{Y}\left| S \right|+\bar{Y}^2 \left| S \right|
	\nonumber \\
	 &= &
	 I_Z-\bar{Y}^2\left| S \right|
	\label{eqn:IZ2I}
\end{eqnarray}
ここで
\begin{equation}
	I_Z=\int_S Y^2 dS
	\label{eqn:def_IZ}
\end{equation}
で定義され,$S$の{\bf $Z$軸に関する}断面2次モーメントと呼ばれる.
式(\ref{eqn:def_IZ})のように,基準となる軸($Z$軸)を添字で示すならば,
$z$軸に関する断面2次モーメントである式(\ref{eqn:Iz_xy})の$I$を
$I_z$と書くことが自然である.
そこで,$I=I_z$と書き,式(\ref{eqn:IZ2I})を
\begin{equation}
	I_z=I_Z-\bar Y^2 \left| S \right|
	\label{eqn:IZ2Iz}
\end{equation}
とすることで,$z$軸あるいは$Z$軸,いずれに関する2次モーメントであるかを区別する.
式(\ref{eqn:IZ2Iz})は,$YZ$座標系で$I_z$を求めるための式として用いることができる.
式(\ref{eqn:IZ2Iz})の右辺に現れる$\bar{Y}$と
$\left| S\right|$は, $I_z$を$yz$座標系で計算する際にも必要とされる.
従って,$I_z$を計算する際に必要とされる計算に関して,2つの座標系の間での違いは,
$I_z$を直接$yz$座標系で積分を行って求めるか,$I_Z$を$YZ$座標系で求めるかの違いにある.
$YZ$座標系は断面1次モーメント$G_Z$を計算する上で最も都合の座標を選ぶことを考えると,
$I_Z$の計算が$I_z$を直接求めるより煩雑になることはない.このことを踏まえれば式(\ref{eqn:IZ2Iz})
に基づいて$I_z$を得ることが得策であると言える.

式(\ref{eqn:IZ2Iz})は自明な変形により,
\begin{equation}
	I_Z=I_z+\bar Y^2 \left| S \right|
	\label{eqn:Iz2IZ}
\end{equation}
とすることができる.式(\ref{eqn:Iz2IZ})は,$I_z$から任意の位置にとった別の軸($Z$軸)に関する
断面2次モーメント$I_Z$を求めるために利用することのできる関係である.
また,式(\ref{eqn:Iz2IZ})において$\bar{Y}$を変数としてみれば,
$\bar{Y}=0$で$I_Z$は最小値をとることを示している.言い換えれば,
あらゆる軸に関する断面2次モーメントの中で最小となるのは,$\bar Y=0 (Z=z)$すなわち$S$の
中立軸に関するものであることを示している.
%%%%%%%%%%
\section{逐次積分による二重積分の計算}
断面係数や中立軸位置を求めるには,面積積分(2重積分)の計算が必要となる.
そこで本節では,逐次積分法による2重積分の計算方法について述べる.
図\ref{fig:fig10_1}に示すように,$z=f(x,y)$を$xy$直角直交座標系を持つ2次元平面上で
定義された2変数関数
とし,$f(x,y)$の二重積分:
\begin{equation}
	F(S)=\int_S f(x,y) dS
	\label{eqn:Int2D}
\end{equation}
を考える.ここで,$S$は積分範囲を,$dS=dxdy$は微小面積要素を表す.
積分範囲$S$は$xy$平面内の点$(x,y)$の集合(領域)として指定される.
2重積分$F(S)$の計算は,幾何学的には領域$S$を底面とし曲面$z=f(x,y)$を上面とする
柱状領域の体積を求めることに相当する.
与えられた積分範囲$S$に対して積分$F(S)$を計算するには,積分変数$x$と$y$それぞれについて,
1変数関数の場合と同様に順次積分を実行すればよい.
ただしその際には,いずれの変数についての積分を先に行うかを決め,それぞれの変数が動きうる範囲を
書き下す必要がある.この作業の煩雑さは,積分範囲$S$の形状によってかなり異なる.
%--------------------
\begin{figure}[h]
	\begin{center}
	\includegraphics[width=0.8\linewidth]{./fig10_1.eps} 
	\end{center}
	\caption{
		(a) 2変数関数$z=f(x,y)$が矩形領域上に作る曲面. 
		(b) 矩形および(c)任意の積分領域と, $xy$直角直交座標系における微小面積要素. 
	} 
	\label{fig:fig10_1}
\end{figure}
\subsection{長方形の積分範囲}
例えば,$S$として図\ref{fig:fig10_1}-(b)に示すような長方形領域:
\begin{equation}
	S=\left\{(x,y):x_{min}<x<x_{max}, \, y_{min}<y<y_{max} \right\}
	\label{eqn:Sa}
\end{equation}
をとる.この場合,$x$と$y$に関する積分範囲の上限と下限は互いに独立かつ一定で,
積分範囲を次のように指定することができる.
\begin{eqnarray}
	F(S)
	&=&
	\int_{x=x_{min}}^{x_{max}}\left\{\int_{y=y_{min}}^{y_{max}} f(x,y)dy\right\} dx
	\label{eqn:y_then_x}
	\\
	&=&
	\int_{y=y_{min}}^{y_{max}}\left\{\int_{x=x_{min}}^{x_{max}} f(x,y)dx\right\} dy
	\label{eqn:x_then_y}
\end{eqnarray}
ここに,式(\ref{eqn:y_then_x})では最初に$y$で,式(\ref{eqn:x_then_y})では最初に$x$で
積分を行うことを意味する.なお,一方の変数に関する積分実行時には,もう一方の変数は一定値
に固定されていると考え,1変数関数の積分と同様に計算を行えばよい.
\subsection{任意の積分領域}
一方,$S$が図\ref{fig:fig10_1}-(c)に示すような任意形状の場合,積分変数$x$と$y$の上限と下限は互いに
独立でなく,$y$の上下限は$x$に依存し,$x$の上下限は$y$に依存する.
そこで,図\ref{fig:fig10_2}に示すように,$x$を固定したときの$y$の下限,上限をそれぞれ$y_1(x),y_2(x)$とする.
同様に,$y$を固定したときの$x$の下限と上限を,それぞれ$x_1(y),x_2(y)$と表す.
このとき,$S$上での2重積分は次のように書くことができる.
\begin{eqnarray}
	F(S) &=& 
	\int_{x=x_{min}}^{x_{max}}\ \left( \int_{y=y_1(x)}^{y_2(x)}f(x,y)dy\right) dx
	\label{eqn:iint_xy}
	\\
	&=& 
	\int_{y=y_{min}}^{y_{max}}\ \left( \int_{x=x_1(y)}^{x_2(y)}f(x,y)dx\right) dy
	\label{eqn:iint_yx}
\end{eqnarray}
ただし,$x_{min},x_{max}$は$S$における$x$の最小値と最大値を,
$y_{min},y_{max}$は$S$における$y$の最小値と最大値を表す.
%-----------------------------
\begin{figure}[h]
	\begin{center}
	\includegraphics[width=0.8\linewidth]{./fig10_2.eps} 
	\end{center}
	\caption{
		領域$S$において
		(a) 固定された$y$に対して$x$の取りうる範囲$\left(x_1(y),x_2(y)\right)$, 
		(b) 固定された$x$に対して$y$の取りうる範囲$\left(y_1(x),y_2(x)\right)$. 
	} 
	\label{fig:fig10_2}
\end{figure}
\begin{figure}[h]
	\begin{center}
	\includegraphics[width=0.5\linewidth]{./fig10_3.eps} 
	\end{center}
	\caption{
		簡単な形状をした積分領域の例. 
	} 
	\label{fig:fig10_3}
\end{figure}
%--------------------
\subsection{三角形領域}
以上の手順を,図\ref{fig:fig10_3}-(b)に示すような三角形領域$S_b$に適用する.
この場合,$S$における$x$と$y$の下限($x_{min},y_{min}$),上限($x_{max},y_{max}$)は
\begin{equation}
	x_{min}=y_{min}=0, \ \ x_{max}=b, \ \ y_{max}=h
\end{equation}
である.また,$x$を固定したときに$y$が動きうる範囲は,
\begin{equation}
	\frac{h}{b}x < y < h
	\label{eqn:ybnd_Sb}
\end{equation}
で,$y$を固定したときに$x$が動きうる範囲は,
\begin{equation}
	0 < x < \frac{b}{h}y
	\label{eqn:xbnd_Sb}
\end{equation}
である.よって,領域$S_b$における二変数関数$f(x,y)$の面積積分は
\begin{eqnarray}
	\int_{S_b} f(x,y) dS
	&=&
	\int_{x=0}^b \left\{ \int_{y=\frac{h}{b}x}^h f(x,y)dy\right\} dx 
	\label{eqn:int_Sb_yx}
	\\
	&=&
	\int_{y=0}^h \left\{ \int_{x=0}^{\frac{b}{h}y} f(x,y)dx\right\}dy
	\label{eqn:int_Sb_xy}
\end{eqnarray}
と表すことができる.

一方,図\ref{fig:fig10_3}-(c)に示すような三角形領域$S_c$であれば,
$x,y$座標のとりうる範囲を調べれば,$S_c$上での積分が
\begin{eqnarray}
	\int_{S_c} f(x,y) dS
	&=&
	\int_{x=0}^b \left\{ \int_{y=0}^{h\left(1-\frac{x}{b}\right)} f(x,y)dy\right\}dx 
	\label{eqn:int_Sc_yx}
	\\
	&=&
	\int_{y=0}^h \left\{ \int_{x=0}^{b\left(1-\frac{y}{h}\right)} f(x,y)dx\right\} dy
	\label{eqn:int_Sc_xy}
\end{eqnarray}
と書けることが分かる.

\subsection{四分円領域}
最後に,図\ref{fig:fig10_3}-(d)の領域$S_d$については,極座標:
\begin{equation}
	\left(x,\,y\right) = \left(\ r\cos\theta,\, r\sin\theta \right)
	\label{eqn:cart2pol}
\end{equation}
を用いて積分を行えばよい.
この場合は,微小面積要素が
\begin{equation}
	dS=r dr d\theta
	\label{eqn:dS_rth}
\end{equation}
となる.
$S$において$r$と$\theta$の動きうる範囲を調べれば,
$r$の動く範囲は$\theta$に依らず$0<r<a$で,
$\theta$の動く範囲は$r$に依らず$0<\theta<\frac{\pi}{2}$である.
よって,
\begin{equation}
	\int_{S_d} f(x,y) dS
	=
	\int_{r=0}^a \int_{\theta=0}^{\frac{\pi}{2}} 
	f(r\cos\theta,r\sin\theta)rd\theta dr
	\label{eqn:int_pol}
\end{equation}
となる.なお,$r$と$\theta$それぞれの積分範囲は$r$と$\theta$の積分順序を交換しても同じである.

\subsection{計算例}
以上を踏まえて,$f(x,y)=x^my^n$($m,n$は非負整数)の$S_a, S_b$および$S_c$における
面積積分を行ってみる.その結果は以下のようである.
\begin{itemize}
\item 長方形領域$S_a$:
\begin{eqnarray}
	F(S_a) &= & 
	\int_{S_a}x^my^ndS \nonumber \\
	&= & 
	\int_0^h \left( \int_0^b x^m y^n dx \right) dy \nonumber \\
	&= & 
	\int_0^h y^n \left[ \frac{x^{m+1}}{m+1} \right]_0^b dy \nonumber \\
	&= & 
	\frac{b^{m+1}}{m+1}
	\left[
		\frac{y^{n+1}}{n+1}
	\right]_0^h
	\nonumber
	\\
	&= & 
	\frac{ b^{m+1} h^{n+1}}{(m+1)(n+1)}
	\label{eqn:int_Sa_xmyn}
\end{eqnarray}
\item 三角形領域$S_b$:
\begin{eqnarray}
	F(S_b) 
	&= & 
	\int_0^h \left(\int_0^{\frac{b}{h}y} x^my^ndx \right) dy
	\nonumber
	\\
	&= & 
	\int_0^h y^n \left[\frac{x^{m+1}}{m+1}\right]_0^{\frac{b}{h}y}dy 
	\nonumber
	\\
	&= & 
	\frac{1}{m+1}\left(\frac{b}{h}\right)^{m+1} \int_0^h y^{m+n+1}dy 
	\nonumber
	\\
	&= & 
	\frac{b^{m+1}h^{n+1}}{(m+1)(m+n+2)}
	\label{eqn:int_Sb_xmyn}
\end{eqnarray}
\item 三角形領域$S_c$:
\begin{eqnarray}
	F(S_c)&= & 
	\int_0^h \left(\int_0^{b\left( 1-\frac{y}{h}\right) } x^my^ndx \right) dy
	\nonumber
	\\
	&=&
	\int_0^h \left[ \frac{x^{m+1}}{m+1}\right]_0^{b\left(1-\frac{y}{h}\right)} dy \nonumber
	\\
	&=&
	\frac{b^{m+1}}{m+1} \int_0^h \left( 1-\frac{y}{h} \right)^{m+1} y^n dy 
	\nonumber
	\\
	&=&
	\frac{b^{m+1}}{m+1}h^{n+1} \int_0^1 \left( 1-\xi \right)^{m+1} \xi^n d\xi, \ \ \left(\xi=\frac{y}{h}\right) 
	\label{eqn:int_Sc_via}
	\\
	&=&
	\frac{m!n!}{(m+n+2)!}b^{m+1}h^{n+1}
	\label{eqn:int_Sc_xmyn}
\end{eqnarray}
なお,式(\ref{eqn:int_Sc_via})の積分は,次に示すように部分積分を$n$回繰り返して行うことで計算することができる.
\begin{eqnarray}
	\int_0^1 \left( 1-\xi \right)^{m+1} \xi^n d\xi
	&=&
	\left[-\frac{(1-\xi)}{m+2}\xi^n\right]_0^1 + \frac{n}{m+2}\int_0^1 (1-\xi)^{m+2}\xi^{n-1}d\xi
	\nonumber
	\\
	&=&
	\cdots
	\nonumber
	\\
	&=&
	\frac{n(n-1)\cdots 1}{(m+2)\cdots (m+n+1)} \int_0^1 (1-\xi)^{m+n+1}d\xi 
	\nonumber
	\\
	&=&
	\frac{(m+1)! n!}{(m+n+2)!}
	\label{eqn:int_by_part}
\end{eqnarray}
以上は全て$x$に関する積分を先に行った場合の計算過程を示しているが,
$y$から先に積分を行っても結果に変わりはない.

\item 四分円領域$S_d$\\
最後に,$S_d$における積分を$f(x,y)=y$,$f(x,y)=y^2$の場合について計算すると
以下のようになる.
\begin{eqnarray}
	F(S_d) 
	&=& \int_{S_d} y dS \nonumber \\
	&=& \int_0^\frac{\pi}{2} \left( \int_0^a r\sin\theta rdr \right)d\theta \nonumber \\
	&=& \frac{a^3}{3}\int_0^\frac{\pi}{2} \sin\theta  d\theta \nonumber \\
	&=& \frac{a^3}{3}
	\label{eqn:int_Sd_y}
\end{eqnarray}
\begin{eqnarray}
	F(S_d) 
	&=& \int_{S_d} y^2 dS \nonumber \\
	&=& \int_0^\frac{\pi}{2} \left( \int_0^a r^2\sin^2\theta rdr \right)d\theta \nonumber \\
	&=& \frac{a^4}{4}\int_0^\frac{\pi}{2} \sin^2\theta d\theta \nonumber \\
	&=& \frac{a^4}{4} \int_0^\frac{\pi}{2} \frac{1-\cos 2\theta}{2} d\theta 
	\nonumber
	\\
	&=& \frac{\pi}{16}a^4
	\label{eqn:int_Sd_y2}
\end{eqnarray}
\end{itemize}
この後で見るように,以上に示した面積積分の結果は,四角形や三角形,円形断面の断面係数を計算するために利用することができる.
\begin{figure}[h]
	\begin{center}
	\includegraphics[width=0.8\linewidth]{./fig11_3.eps} 
	\end{center}
	\caption{
		基本的な断面形と断面係数計算のための座標系.
	} 
	\label{fig:fig11_3}
\end{figure}
\section{長方形,三角形,円の断面係数}
図\ref{fig:fig11_3}-(1)から(3)に示す3つの断面について断面係数を求める.
これらの結果は基本的なものであるため,その都度計算するのでなく記憶しておくべきものである.
\subsection{長方形}
断面積は$\left|S_1\right|=bh$,断面1次モーメントは
\begin{equation}
	G_Z=\int_{S_1} Y dS=\int_0^h\int_0^b Y dZdY=\int_0^h bYdY=\frac{bh^2}{2}
	\label{eqn:GZ_S1}
\end{equation}
であることから,中立軸の位置が
\begin{equation}
	\bar Y= \frac{G_Z}{\left| S_1 \right|}=\frac{bh^2/2}{bh}=\frac{h}{2}
\end{equation}
と求まる.$Z$軸に関する断面2次モーメントを計算すると
\begin{equation}
	I_Z=\int_{S_1} Y^2dZdY
	=\int_0^h \int_0^b Y^2dZdY 
	=\int_0^h bY^2dY 
	=\frac{bh^3}{3}
	\label{eqn:IZ_S1}
\end{equation}
となるので,式(\ref{eqn:IZ2Iz})を用いて,中立軸周りの断面2次モーメントが
\begin{equation}
	I_z= I_Z-\bar Y^2 \left| S_1 \right| = \frac{bh^3}{3}-\left(\frac{h}{2}\right)^2\times bh=\frac{bh^3}{12}
	\label{eqn:Iz_S1}
\end{equation}
と得られる.
\subsection{三角形}
断面積は$\left|S_2\right|=\frac{bh}{2}$,$Z$軸に関する断面1次および2次モーメント
\begin{equation}
	G_Z=\int_{S_2} Y dS, \ \ I_Z=\int_{S_2}Y^2 dS
\end{equation}
は,式(\ref{eqn:int_Sb_xmyn})を利用して計算することができる.
具体的には,式(\ref{eqn:int_Sb_xmyn})において$m=0, n=1$と
した場合$G_Z$が,$m=0, n=2$とした場合に$I_Z$が得られる.
その結果,
\begin{equation}
	G_Z=\frac{bh^2}{3}, \ \ 
	I_Z=\frac{bh^3}{4}
	\label{eqn:GI_S2}
\end{equation}
となり,式(\ref{eqn:Ybar})より
\begin{equation}
	\bar{Y}=\frac{2h}{3}
\end{equation}
が,式(\ref{eqn:IZ2Iz})より
\begin{equation}
	I_z=\frac{bh^3}{4}-\left(\frac{2h}{3}\right)^2\times \frac{bh}{2}=\frac{bh^3}{36}
	\label{eqn:Iz_S2}
\end{equation}
が得られる.
\subsection{円}
円の面積は$\left|S_3\right|=\pi a^2$で与えられ,$G_Z$は,$S_d$が$Z$軸に関して上下対称,
被積分関数である$y$が$Z$軸に関して奇関数であることから$G_Z=0$である.
従って,中立軸の位置は$\bar Y=0$で,この場合$I_Z=I_z$となる.
また,$I_Z$は式(\ref{eqn:int_Sd_y2})の積分の丁度4倍であることから
\begin{equation}
	I_z=I_Z=\int_{S_3}Y^2dS=4\times \frac{\pi a^4}{16}=\frac{\pi a^4}{4}
	\label{eqn:Iz_S3}
\end{equation}
となる.
\subsubsection{問題}
図\ref{fig:fig11_3}-(4)から(6)に示す断面$S_4,S_5,S_6$について以下の問に答えよ.
\begin{enumerate}
\item
三角形$S_4$に関する断面係数と中立軸位置を求めよ.
\item
平行四辺形$S_5$に関する断面係数と中立軸の位置は,長方形$S_1$の場合と同じになることを示せ.
\item
三角形$S_6$に関する断面係数と中立軸の位置は,三角形$S_4$の場合と同じになることを示せ.
\end{enumerate}
%%%
\section{部分断面への分割を利用した断面係数の計算方法}
断面$S$の$n$個部分断面$\{ S_i, (i=1,\dots, n)\}$への分割:
\begin{equation}
	S=S_1\cup S_2 \cup \dots \cup S_n=\cup_{i=1}^n S_i
	\label{eqn:cup_Si}
\end{equation}
を考える.ただし,部分断面どうしは互いに共有部分を持たない,すなわち
\begin{equation}
	S_i \cap S_j =\phi(空集合), \ \ (i\neq j,\; i,j=1,2,\dots ,n ) 
	\label{eqn:cap_Si}
\end{equation}
とする.いま,各々の部分断面$S_i,(i=1,\dots n)$は比較的簡単な形状をしており,
$S_i$の中立軸位置や断面2次モーメントは既知であるか容易に計算が可能と仮定する.
図\ref{fig:fig11_2}-(a)は,この状況を$n=2$の場合について示したものである.


断面$S$の断面積$\left| S \right|$は,$S_i$が互いに共通部分を持たないことから
部分断面の断面積の和

\begin{equation}
	\left|S\right| = \sum_{i=1}^n \left| S_i \right|
	\label{eqn:Stot}
\end{equation}
で与えられる.ここで,断面係数を求める際の積分計算を行う座標を$(Y,\,Z)$とし,
$S_i$の中立軸を$z_i$, $Z$軸と$z_i$軸の距離を$\bar Y_i$と表す(図\ref{fig:fig11_2}-(b)).
式(\ref{eqn:Ybar})より$\bar Y_i$は
\begin{equation}
	\bar{Y}_i = \frac{G_Z(S_i)}{\left| S_i \right|}, 
	\ \ G_Z(S_i)=\int_{S_i}YdS
\ \ (i=1,\dots,n)
	\label{eqn:Y_i}
\end{equation}
と,各々の断面に関する断面1次モーメント$G_Z(S_i)$から求められる.
一方,全断面$S$の断面1次モーメント$G_Z(S)$は,
\begin{equation}
	G_Z(S)=\int_SYdS
	=\sum_{i=1}^n \int_{S_i} YdS
	=\sum_{i=1}^n G_Z(S_i)
%	= \sum_{i=1}^n \bar{Y}_i\left| S_i \right|
	\label{eqn:GZ_Stot}
\end{equation}
で,$G_Z(S_i)$の和で与えられる.よって,式(\ref{eqn:GZ_Stot})と式(\ref{eqn:Y_i})より
\begin{equation}
	\bar Y 
	= \frac{G_Z(S)}{\left| S \right|}
	= \sum_{i=1}^n \frac{G(S_i)}{\left| S \right|}
	= \sum_{i=1}^n \frac{\left| S_i\right|}{\left| S \right|}\bar{Y}_i
	\label{eqn:Yi2Y}
\end{equation}
が得られる.式(\ref{eqn:Yi2Y})は,部分断面の断面積と中立軸位置から,全断面$S$の
中立軸位置$\bar Y$を計算するために用いることができる.

同様に,$S$の$z$軸に関する断面2次モーメント$I_z(S)$は,部分断面の断面2次モーメント$I_z(S_i)$の
和として,
\begin{equation}
	I_z(S)=\int_S y^2 dS=\sum_{i=1}^n\int_{S_i} y^2dS=\sum_{i=1}^n I_z(S_i)
	\label{eqn:sum_of_Izi}
\end{equation}
と表される.ここで,
\[
	I_z(S_i): 部分断面S_iのz軸に関する断面2次モーメント
\]
と,
\[
	I_{z_i}(S_i): S_i自身の中立軸z_iに関する断面2次モーメント
\]
の関係は,式(\ref{eqn:Iz2IZ})より
\begin{equation}
	I_z(S_i)=I_{z_i}(S_i)+\left( \bar Y_i-\bar{Y} \right)^2 \left|S_i\right|
	\label{eqn:Iz_Si}
\end{equation}
となる.よって,式(\ref{eqn:sum_of_Izi})と式(\ref{eqn:Iz_Si})より
\begin{equation}
	I_z(S)=
	\sum_{i=1}^n\left\{
		I_{z_i}(S_i)+\left( \bar Y_i-\bar{Y} \right)^2 \left|S_i\right|
	\right\}
	\label{eqn:IzS_sum}
\end{equation}
の関係が得られる.
式(\ref{eqn:Stot})と式(\ref{eqn:Yi2Y})を用いるとき,式(\ref{eqn:IzS_sum})右辺の量は,
全て部分断面ごとに計算できる.従って,部分断面が簡単な形状をしている場合,式(\ref{eqn:IzS_sum})を
用いるれば,比較的簡単な計算の繰り返しで,より複雑な断面$S$に関する断面2次モーメントを求めることができる.
%--------------------
\begin{figure}[h]
	\begin{center}
	\includegraphics[width=1.0\linewidth]{./fig11_2.eps} 
	\end{center}
	\caption{
		(a) 複合断面$S$の部分断面$S_1$および$S_2$への分割.
		(b) $ZY$座標系における複合断面とその部分断面の中立軸$z_1,z_2$.
	} 
	\label{fig:fig11_2}
\end{figure}
%--------------------
\begin{figure}[h]
	\begin{center}
	\includegraphics[width=0.8\linewidth]{./fig11_4.eps} 
	\end{center}
	\caption{
		部分断面への分割を利用した断面係数計算の例題
	} 
	\label{fig:fig11_4}
\end{figure}
%--------------------
\subsection{部分断面分割を利用した断面係数の計算例}
\subsubsection{長方形断面}
図\ref{fig:fig11_4}-(a)のような長方形断面を,2つの長方形$S_1$と$S_2$に分割して計算を行う.
長方形断面の中立軸位置や断面係数は前節で求めた通りだが,
部分断面分割に基づく断面係数の計算過程を簡単な計算で示し,
既に答えが分かっている問題に対して正しい結果が得られることを見る.

はじめに,断面$S_1$と$S_2$の断面積は
\begin{equation}
	\left| S_1 \right|
	=
	\left| S_2 \right|
	=
	\frac{bh}{2}
	\label{eqn:area_S1S2}
\end{equation}
で,全断面$S=S_1\cup S_2$の断面積はこれらの和として$\left|S\right|=\frac{bh}{2}+\frac{bh}{2}=bh$と
求められる.これより,部分断面の全断面に対する面積比$\xi_i$は,
\begin{equation}
	\xi_1=\frac{\left| S_1 \right|}{\left|S\right|} =\frac{1}{2}
	, \ \ 
	\xi_2=\frac{\left| S_2 \right|}{\left|S\right|} =\frac{1}{2}
	\label{eqn:ratio_area}
\end{equation}
となる. $i=1,2$に対し,$S_i$の中立軸$z_i$の位置を,$Z$軸からの距離$\bar{Y}_i$で
表すと,$\bar{Y}_i$は
\begin{equation}
	\bar{Y}_1=\frac{h}{4}
	, \  \
	\bar{Y}_2=\frac{3h}{4}
	\label{eqn:}
\end{equation}
である.よって,
\begin{equation}
	\bar{Y}=\sum_{i=1}^2 \xi_i \bar{Y}_i
	=\frac{h}{8}+\frac{3h}{8}=\frac{h}{2}
	\label{eqn:}
\end{equation}
となり,$S$の中立軸位置$\bar Y$が正しく求められる.この結果から
\begin{equation}
	\sum_{i=1}^2 \left( \bar{Y}_i-\bar{Y}\right) \left| S_i\right|
	=
	\left(\frac{h}{4}\right)^2\times \frac{bh}{2}
	+
	\left(\frac{h}{4}\right)^2\times \frac{bh}{2}
	=\frac{bh^3}{16}
\end{equation}
となる.最後に,$S_i$のそれ自身の中立軸$z_i$に関する断面2次モーメント$I_{z_i}$の
和を求めると,
\begin{equation}
	\sum_{i=1}^2 I_{z_i}=
	\frac{b}{12}\left(\frac{h}{2}\right)^3
	+
	\frac{b}{12}\left(\frac{h}{2}\right)^3
	=
	\frac{bh^3}{48}
\end{equation}
となるので,これらを式(\ref{eqn:IzS_sum})に代入すれば,$S$の$z$に関する
断面2次モーメント$I_z(S)$が
\begin{equation}
	I_z(S)=
	\frac{bh^3}{16}
	+
	\frac{bh^3}{48}
	=
	\frac{bh^3}{12}
\end{equation}
となり,期待した通りの結果が得られる.
表\ref{tbl0}は,以上の計算過程をまとめたものである.
一部,"ー"を記入した箇所は合計値を計算して記入してもよいが,その値には特に
意味が無いために未記入としている.
部分断面分割に基づく計算を行う際にはこのような表を作成して計算を進めるとよく,
3つ以上の部分断面に分割される場合は,適宜3つ目以後の部分断面に関する行を追加して,
最下行に全ての部分断面に関する和を記入するようにすればよい.
\subsubsection{台形断面}
図\ref{fig:fig11_4}-(b)のような台形断面を,長方形$S_1$と三角形$S_2$に分割する.
$S_1$と$S_2$の中立軸位置,断面係数を既知として,全断面$S=S_1 \cup S_2$の中立軸位置と
断面2次モーメントを計算すると,その過程は表\ref{tbl1}のようにまとめられ,
最終的に次の結果が得られる.
\begin{equation}
	\bar{Y}=\frac{4h}{9}, \ \ 
	I_z=\frac{bh^3}{108}+\frac{bh^3}{9}=\frac{13}{108}bh^3
\end{equation}
\subsubsection{L字型断面}
最後に,図\ref{fig:fig11_4}-(d)のようなL字型断面$S$を,2つの長方形部分断面
$S_1,S_2$に分割して断面係数計算を行った過程を表\ref{tbl2}に示す.
$S$の中立軸位置と断面2次モーメント$I_z$は,この表に示された計算結果より
\begin{equation}
	\bar{Y}=\frac{19}{5}t, \ \ 
	I_z=\frac{216}{5}t^4+\frac{44}{3}t^4 = \frac{868}{15}t^4
\end{equation}
となる.
\subsubsection{問題}
図\ref{fig:fig11_4}-(c),(e),(f)に示された断面形と部分断面分割について,以下の問に答えよ.
\begin{enumerate}
\item
図\ref{fig:fig11_4}-(c)のように,台形断面を2つの3角形に分割して
中立軸位置$\bar{Y}$と断面2次モーメント$I_z$を求め,その結果が
同図-(b)の断面分割に基づく計算結果と一致することを示せ.
\item
図\ref{fig:fig11_4}-(e)の断面分割に基づいてL字型断面$S=S_1\cup S_2$の中立軸位置
$\bar{Y}$と断面2次モーメント$I_z$を求め,その結果が同図-(d)の断面分割による計算結果
と一致することを示せ.
\item
図\ref{fig:fig11_4}-(f)に示す半径$a$の3つの部分断面$S_i,(i=1,2,3)$から構成される断面$S=\cup_{i=1}^3S_i$ついて,
中立軸位置$\bar{Y}$と断面2次モーメント$I_z$を求めよ.
\end{enumerate}
%--------------------
\begin{table}
\begin{center}
	\caption{部分断面への分割に基づく長方形の断面係数計算の過程}
	\begin{tabular}{c||c|c|c|c|c|c|c}
		&
		$\left| S_i \right|$ & 
		$ \xi_i=\frac{\left| S_i \right|}{\left| S\right|} $  &
		$ \bar{Y}_i $ & 
		$ \xi_i\bar{Y}_i $ & 
		$\bar{Y}_i -\bar Y$ & 
		$ \left(\bar{Y}_i -\bar Y\right)^2\left| S_i \right|$ & 
		$ I_{z_i}$  
		\\
		\hline 
		断面1&	
		$\frac{bh}{2}$ & 
		$\frac{1}{2}$  &
		$\frac{3h}{4}$ & 
		$\frac{3h}{8}$ & 
		$\frac{h}{4}$ & 
		$\left(\frac{h}{4}\right)^2 \times \frac{bh}{2}$ & 
		$\frac{b}{12}\times \left(\frac{h}{2}\right)^3$  
		\\
		\hline
		断面2&	
		$\frac{bh}{2}$ & 
		$\frac{1}{2}$  &
		$\frac{h}{4}$ & 
		$\frac{h}{8}$ & 
		$-\frac{h}{4}$ & 
		$\left(\frac{h}{4}\right)^2 \times \frac{bh}{2}$ & 
		$\frac{b}{12}\times \left(\frac{h}{2}\right)^3$  
		\\
		\hline 
		合計&	
		$bh$ & 
		$1$  &
		$-$ & 
		$\bar Y =\frac{h}{2}$ & 
		$-$ & 
		$\frac{bh^3}{16}$ & 
		$\frac{bh^3}{48}$ 
	\end{tabular}
\label{tbl0}
\end{center}
\end{table}
\begin{table}
\begin{center}
	\caption{部分断面への分割に基づく台形の断面係数計算の過程}
	\caption{複合断面の断面係数計算(台形)}
	\begin{tabular}{c||c|c|c|c|c|c|c}
		&
		$\left| S_i \right|$ & 
		$ \xi_i=\frac{\left| S_i \right|}{\left| S\right|} $  &
		$ \bar{Y}_i $ & 
		$ \xi_i\bar{Y}_i $ & 
		$\bar{Y}_i -\bar Y$ & 
		$ \left(\bar{Y}_i -\bar Y\right)^2\left| S_i \right|$ & 
		$ I_{z_i}$  
		\\
		\hline 
		断面1&	
		$bh$ & 
		$\frac{2}{3}$  &
		$\frac{h}{2}$ & 
		$\frac{h}{3}$ & 
		$\frac{h}{18}$ & 
		$\left(\frac{h}{18}\right)^2 \times bh $ & 
		$\frac{bh^3}{12}$  
		\\
		\hline
		断面2&	
		$\frac{bh}{2}$ & 
		$\frac{1}{3}$  &
		$\frac{h}{3}$ & 
		$\frac{h}{9}$ & 
		$-\frac{h}{9}$ & 
		$\left(-\frac{h}{9}\right)^2 \times \frac{bh}{2}$ & 
		$\frac{bh^3}{36}$  
		\\
		\hline 
		合計&	
		$\frac{3}{2}bh$ & 
		$1$  &
		$-$ & 
		$\bar Y=\frac{4}{9}h$ & 
		$-$ & 
		$\frac{bh^3}{108}$ & 
		$\frac{bh^3}{9}$ 
	\end{tabular}
	\label{tbl1}
\end{center}
\end{table}
\begin{table}
\begin{center}
	\caption{部分断面への分割に基づくL字型断面の断面係数計算の過程}
	\begin{tabular}{c||c|c|c|c|c|c|c}
		&
		$\left| S_i \right|$ & 
		$ \xi_i=\frac{\left| S_i \right|}{\left| S\right|} $  &
		$ \bar{Y}_i $ & 
		$ \xi_i\bar{Y}_i $ & 
		$\bar{Y}_i -\bar Y$ & 
		$ \left(\bar{Y}_i -\bar Y\right)^2\left| S_i \right|$ & 
		$ I_{z_i}$  
		\\
		\hline 
		断面1&	
		$8t^2$ & 
		$\frac{2}{5}$  &
		$2t$ & 
		$\frac{4}{5}t$ & 
		$-\frac{9}{5}t$ & 
		$\left(\frac{9}{5}t\right)^2 \times 8t^2 $ & 
		$\frac{2t\times (4t)^3}{12}=\frac{32}{3}t^4$  
		\\
		\hline
		断面2&	
		$12t^2$ & 
		$\frac{3}{5}$  &
		$5t$ & 
		$3t $ & 
		$\frac{6}{5}t$ & 
		$\left(\frac{6t}{5}\right)^2 \times 12t^2 $ & 
		$\frac{6t\times (2t)^3}{12}=4t^4$  
		\\
		\hline 
		合計&	
		$20t^2$ & 
		$1$  &
		$-$ & 
		$\bar Y=\frac{19}{5}t$ & 
		$-$ & 
		$\frac{216}{5}t^4$ & 
		$\frac{44}{3}t^4$ 
	\end{tabular}
\label{tbl2}
\end{center}
\end{table}
\end{document}

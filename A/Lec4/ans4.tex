\documentclass[10pt,a4j]{jarticle}
%\usepackage{graphicx,wrapfig}
\usepackage{graphicx}
\setlength{\topmargin}{-1.5cm}
\setlength{\textwidth}{15.5cm}
\setlength{\textheight}{25.2cm}
\newlength{\minitwocolumn}
\setlength{\minitwocolumn}{0.5\textwidth}
\addtolength{\minitwocolumn}{-\columnsep}
%\addtolength{\baselineskip}{-0.1\baselineskip}
%
\def\Mmaru#1{{\ooalign{\hfil#1\/\hfil\crcr
\raise.167ex\hbox{\mathhexbox 20D}}}}
%
\begin{document}
\newcommand{\fat}[1]{\mbox{\boldmath $#1$}}
\newcommand{\D}{\partial}
\newcommand{\w}{\omega}
\newcommand{\ga}{\alpha}
\newcommand{\gb}{\beta}
\newcommand{\gx}{\xi}
\newcommand{\gz}{\zeta}
\newcommand{\vhat}[1]{\hat{\fat{#1}}}
\newcommand{\spc}{\vspace{0.7\baselineskip}}
\newcommand{\halfspc}{\vspace{0.3\baselineskip}}
\bibliographystyle{unsrt}
\pagestyle{empty}
\newcommand{\twofig}[2]
 {
   \begin{figure}[here]
     \begin{minipage}[t]{\minitwocolumn}
         \begin{center}   #1
         \end{center}
     \end{minipage}
         \hspace{\columnsep}
     \begin{minipage}[t]{\minitwocolumn}
         \begin{center} #2
         \end{center}
     \end{minipage}
   \end{figure}
 }
%%%%%%%%%%%%%%%%%%%%%%%%%%%%%%%%%
%\vspace*{\baselineskip}
\begin{center}
{\Large \bf 2019年度 構造力学I及び演習A 演習問題4 解答} \\
\end{center}
%%%%%%%%%%%%%%%%%%%%%%%%%%%%%%%%%%%%%%%%%%%%%%%%%%%%%%%%%%%%%%%%
\vspace{15mm}
\begin{enumerate}
\item
$o-x_1x_2$から$o-x_1'x_2'$への座標変換マトリクス$\fat{Q}$は
\[
	\fat{Q}=\left(
		\begin{array}{cc}
			\cos \theta & \sin \theta \\
			-\sin\theta & \cos \theta
		\end{array}
	\right)
\]
で与えられる.これを用いて$\fat{\sigma}'$を計算すれば
\begin{eqnarray}
	\fat{\sigma} ' 
	&=& 
	\fat{Q \sigma Q}^T\\
	&=&
	2
 	\left( 
 	\begin{array}{cc}
		-1- \cos 2\theta+\sqrt{3} \sin 2 \theta & 
		\sin 2\theta +\sqrt{3} \cos 2\theta \\
		\sin 2\theta +\sqrt{3} \cos 2\theta &
		-1+ \cos 2\theta-\sqrt{3} \sin 2 \theta 
	\end{array}
	\right) \\
 &=&
 2
 \left(
 	\begin{array}{cc}
		-1+ 2 \cos \left( \frac{2\pi}{3} -2\theta\right) & 
		2\sin \left(\frac{2\pi}{3} -2 \theta\right) \\
		2\sin \left(\frac{2\pi}{3} -2 \theta\right) &
		-1- 2 \cos \left( \frac{2\pi}{3} -2\theta\right)   \\
 	\end{array}
 \right)
	\label{eqn:sigd}
\end{eqnarray}
\item
	式(\ref{eqn:sigd})より,$\sigma'_{11}$は$\frac{2\pi}{3} -2\theta=0$, すなわち
	$\theta_1=\frac{\pi}{3}$で, 最大値$\sigma_1=2$を取る.
\item
	$\theta=\theta_1=\frac{\pi}{3}$のとき,$\sigma_{12}'=0,\sigma_{22}'=-6$.
\item
	式(\ref{eqn:sigd})より,$\frac{2\pi}{3}-2\theta=\frac{\pi}{2}$, すなわち
	$\theta_\tau=\frac{1}{6}\pi$のとき, $\sigma'_{12}$は,最大値$\tau=4$を取る.
\item
	$\theta=\theta_{\tau}=\frac{1}{6}\pi$のとき,$\sigma_{11}'=0,\sigma_{22}'=-4$.
\item
	指定された法線ベクトルは, $o-x_1'x_2'$座標系では
	$\fat{n}'=(0,\, 1)$と表される. よって,
	トラクションベクトル$\fat{t}^{(n')}$は, $o-x_1'x_2'$座標系において
	\begin{equation}
		\fat{t}^{(n')}= \fat{\sigma}' \cdot \fat{n}'
		= 
		\left( 
			\begin{array}{c} 
				\sigma_{12}' \\ 
				\sigma_{22}'
			\end{array}
		\right)
		=
		\left(
			\begin{array}{c}
				2\sin 2\theta +2\sqrt{3} \cos 2\theta \\
				-2+ 2\cos 2\theta-2\sqrt{3} \sin 2 \theta 
			\end{array}
		\right)
		=
		\left(
			\begin{array}{c}
			4\sin \left(\frac{2\pi}{3} -2 \theta\right)\\
			-2- 4 \cos \left( \frac{2\pi}{3} -2\theta\right)
 		\end{array}
		\right) 
	\end{equation}
	である.
\end{enumerate}
%--------------------
\end{document}

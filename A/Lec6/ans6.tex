\documentclass[10pt,a4j]{jarticle}
\usepackage{graphicx,wrapfig}
\setlength{\topmargin}{-1.5cm}
\setlength{\textwidth}{15.5cm}
\setlength{\textheight}{25.2cm}
\newlength{\minitwocolumn}
\setlength{\minitwocolumn}{0.5\textwidth}
\addtolength{\minitwocolumn}{-\columnsep}
%\addtolength{\baselineskip}{-0.1\baselineskip}
%
\def\Mmaru#1{{\ooalign{\hfil#1\/\hfil\crcr
\raise.167ex\hbox{\mathhexbox 20D}}}}
%
\begin{document}
\newcommand{\fat}[1]{\mbox{\boldmath $#1$}}
\newcommand{\D}{\partial}
\newcommand{\w}{\omega}
\newcommand{\ga}{\alpha}
\newcommand{\gb}{\beta}
\newcommand{\gx}{\xi}
\newcommand{\gz}{\zeta}
\newcommand{\vhat}[1]{\hat{\fat{#1}}}
\newcommand{\spc}{\vspace{0.7\baselineskip}}
\newcommand{\halfspc}{\vspace{0.3\baselineskip}}
\bibliographystyle{unsrt}
\pagestyle{empty}
\newcommand{\twofig}[2]
 {
   \begin{figure}[h]
     \begin{minipage}[t]{\minitwocolumn}
         \begin{center}   #1
         \end{center}
     \end{minipage}
         \hspace{\columnsep}
     \begin{minipage}[t]{\minitwocolumn}
         \begin{center} #2
         \end{center}
     \end{minipage}
   \end{figure}
 }
%%%%%%%%%%%%%%%%%%%%%%%%%%%%%%%%%
%\vspace*{\baselineskip}
\begin{center}
{\Large \bf  2020年度 構造力学I及び演習A 演習問題6 解答} 
\end{center}
%%%%%%%%%%%%%%%%%%%%%%%%%%%%%%%%%%%%%%%%%%%%%%%%%%%%%%%%%%%%%%%%
\begin{enumerate}
\item
$\fat{e}_1=(1,0)^T, \fat{e}_2=(0,1)^T$とすると,
\[
	\fat{y}_1=\fat{A}\fat{e}_1=
	\left(
		\begin{array}{cc}
			3 & 3 \\
			1 & -1 
		\end{array}
	\right)
	\left(
		\begin{array}{c}
			1  \\
			0 
		\end{array}
	\right)
	=
	\left(
		\begin{array}{c}
			3  \\
			1
		\end{array}
	\right), 
\]
\[
	\fat{y}_2=\fat{A}\fat{e}_2=
	\left(
		\begin{array}{cc}
			3 & 3 \\
			1 & -1 
		\end{array}
	\right)
	\left(
		\begin{array}{c}
			0  \\
			1 
		\end{array}
	\right)
	=
	\left(
		\begin{array}{c}
			3  \\
			-1 
		\end{array}
	\right)
\]
となる.
\item
	指定された正方形領域内の任意の点
	$\fat{x}=(x_1,x_2)^T$は,単位ベクトル$\fat{e}_1,\fat{e}_2$を用いて
	\[
		\fat{x}=x_1\fat{e}_1+ x_2 \fat{e}_2, \ \ (0\leq x_1,x_2 \leq 1)
	\]
	と表されるので,
	\[
		\fat{y}=\fat{A}\left(x_1\fat{e}_1+ x_2\fat{e}_2\right)
		=x_1\fat{A}\fat{e}_1 + x_2\fat{A}\fat{e}_2
		=x_1\fat{y}_1 + x_2\fat{y}_2
	\]
	である.よって,正方形領域:
	\[
		\left\{ (x_1,x_2) : 0\leq x_1 \leq 1, \ , 0 \leq x_2 \leq 1 \right\}
	\]
	は,図\ref{fig:fig1}$に示すような,\fat{y}_1$と$\fat{y}_2$で張られる
	平行四辺形の内部および境界に移されることが分かる.
\item
	$\fat{u}=\fat{y}-\fat{x}=\left(\fat{A}-\fat{I}\right)\fat{x}$より,
	\[
		\fat{u}=
		\left(
		\begin{array}{c}
			u_1  \\
			u_2 
		\end{array}
		\right)
		=
		\left(
			\begin{array}{cc}
				2 & 3  \\
				1 & -2  
			\end{array}
		\right)
		\left(
		\begin{array}{c}
			x_1  \\
			x_2 
		\end{array}
		\right)
		=
		\left(
		\begin{array}{c}
			2x_1+3x_2  \\
			x_1 -2x_2
		\end{array}
		\right)
	\]
\item
	\begin{eqnarray}
		\varepsilon_{11} &=&\frac{\partial u_1}{\partial x_1}=2 
		\label{eqn:e11_val}	
		\\
		\varepsilon_{12}=
		\varepsilon_{21}&=&
		\frac{1}{2}\left(
			\frac{\partial u_1}{\partial x_2}
			+
			\frac{\partial u_2}{\partial x_1}
		\right)
		=2
		\label{eqn:e12_val}	
		\\ 
		\varepsilon_{22}&=&\frac{\partial u_2}{\partial x_2}=-2
		\label{eqn:e22_val}
	\end{eqnarray}
\item
変位とひずみの関係は
	\begin{equation}
		\varepsilon_{ij}=\frac{1}{2} 
		\left(
		\frac{\partial u_i}{\partial x_j}
		+
		\frac{\partial u_j}{\partial x_i}
		\right), \, (i,j=1,2)
	\end{equation}
	であるから,
	\begin{equation}
		\fat{D}=
		\left(
		\begin{array}{cc}
			\frac{\partial u_1}{\partial x_1} &
			\frac{\partial u_1}{\partial x_2}  \\
			\frac{\partial u_2}{\partial x_1} &
			\frac{\partial u_2}{\partial x_2} 
		\end{array}
		\right)
	\end{equation}
	とすれば,
	\begin{equation}
		\fat{\varepsilon}=\frac{1}{2}
		\left( \fat{D} +\fat{D}^T\right)
	\end{equation}
	と表すことができる.また,$\fat{u}=(\fat{A}-\fat{I})\fat{x}$より
	$\fat{D}=\fat{A}-\fat{I}$が言え,
	\[
		\fat{\varepsilon}=
		\frac{1}{2}
		\left\{
		\left(
			\fat{A}-\fat{I}
		\right)
		+
		\left(
			\fat{A}-\fat{I}
		\right)^T
		\right\}
		=\frac{\fat{A}+\fat{A}^T}{2}-\fat{I}
	\]
	となることが示される.
\item
	ひずみテンソルは応力テンソルと同じ座標変換法則に従う.よって,
	\begin{equation}
		\bar \varepsilon = \frac{\varepsilon_{11}+ \varepsilon_{22}}{2} , \ \ 
		\Delta  \varepsilon = \varepsilon_{11}- \varepsilon_{22}, \ \ 
		\eta = \varepsilon_{12}=\varepsilon_{21}
	\end{equation}
	とすれば,
	\begin{eqnarray}
		\varepsilon_{11}' &=& 
			\bar \varepsilon + \frac{\Delta \varepsilon}{2} \cos 2\theta + \eta \sin 2\theta 
			\label{eqn:e11d}
			\\
		\varepsilon_{12}' &=& 
			-\frac{\Delta \varepsilon}{2} \sin 2\theta + \eta \cos 2\theta 
			\label{eqn:e12d}
			\\
		\varepsilon_{22}' &=& 
			\bar \varepsilon - \frac{\Delta \varepsilon}{2} \cos 2\theta - \eta \sin 2\theta 
			\label{eqn:e22d}
	\end{eqnarray}
	の関係が成り立つことは明らかである.
	式(\ref{eqn:e11_val})$\sim$(\ref{eqn:e22_val})の計算結果より,
	\[
		\bar\varepsilon = 0, \ \ \Delta \varepsilon =2, \ \ \eta =2
	\]
	だから,
	\[
		\varepsilon_{11}'=
		\cos 2\theta +2 \sin 2\theta 
	\]
	\[
		\varepsilon_{12}'=\varepsilon_{21}'
		=
		 -\sin 2\theta +2 \cos 2\theta 
	\]
	となる.
\item
	\begin{equation}
		L=\sqrt{\left(\frac{\Delta \varepsilon}{2}\right)^2 + \eta ^2 }, 
		\ \ 
		\psi=\tan^{-1}\frac{\eta}{\Delta \varepsilon/2}
	\end{equation}
	とすれば,式(\ref{eqn:e11d})$\sim$式(\ref{eqn:e22d})は
	\begin{eqnarray}
		\varepsilon_{11}' &=& 
			\bar \varepsilon + 
			L \cos \left( \psi-2\theta  \right)	
			\label{eqn:e11d2}
			\\
		\varepsilon_{12}' &=& 
			L \sin \left( \psi-2\theta  \right)	
			\label{eqn:e12d2}
			\\
		\varepsilon_{22}' &=& 
			\bar \varepsilon - 
			L \cos \left( \psi-2\theta  \right)	
			\label{eqn:e22d2}
	\end{eqnarray}
	と書くことができる.$\psi-2\theta=0$のとき$\varepsilon_{12}'=0$で,
	このとき,$\varepsilon_{11}'=L=\sqrt{5}$となる.
	このように,せん断ひずみがゼロになるときの軸ひずみは主ひずみと呼ばれ,主ひずみを与える
	方向は主ひずみ方向と呼ばれる.
\end{enumerate}
%--------------------
\begin{figure}[h]
	\begin{center}
	\includegraphics[width=0.5\linewidth]{fig1.eps} 
	\end{center}
	\vspace{-5mm}
	\caption{単位正方形領域が変形後に占める平行四辺形領域.} 
	\label{fig:fig1}
\end{figure}
%--------------------
\end{document}

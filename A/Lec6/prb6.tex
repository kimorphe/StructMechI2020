\documentclass[10pt,a4j]{jarticle}
\usepackage{graphicx,wrapfig}
\setlength{\topmargin}{-1.5cm}
%\setlength{\textwidth}{15.5cm}
\setlength{\textheight}{25.2cm}
\newlength{\minitwocolumn}
\setlength{\minitwocolumn}{0.5\textwidth}
\addtolength{\minitwocolumn}{-\columnsep}
%\addtolength{\baselineskip}{-0.1\baselineskip}
%
\def\Mmaru#1{{\ooalign{\hfil#1\/\hfil\crcr
\raise.167ex\hbox{\mathhexbox 20D}}}}
%
\begin{document}
\newcommand{\fat}[1]{\mbox{\boldmath $#1$}}
\newcommand{\D}{\partial}
\newcommand{\w}{\omega}
\newcommand{\ga}{\alpha}
\newcommand{\gb}{\beta}
\newcommand{\gx}{\xi}
\newcommand{\gz}{\zeta}
\newcommand{\vhat}[1]{\hat{\fat{#1}}}
\newcommand{\spc}{\vspace{0.7\baselineskip}}
\newcommand{\halfspc}{\vspace{0.3\baselineskip}}
\bibliographystyle{unsrt}
\pagestyle{empty}
\newcommand{\twofig}[2]
 {
   \begin{figure}[h]
     \begin{minipage}[t]{\minitwocolumn}
         \begin{center}   #1
         \end{center}
     \end{minipage}
         \hspace{\columnsep}
     \begin{minipage}[t]{\minitwocolumn}
         \begin{center} #2
         \end{center}
     \end{minipage}
   \end{figure}
 }
%%%%%%%%%%%%%%%%%%%%%%%%%%%%%%%%%
%\vspace*{\baselineskip}
\begin{center}
{\Large \bf2020年度 構造力学I及び演習A 演習問題6}
\end{center}
%%%%%%%%%%%%%%%%%%%%%%%%%%%%%%%%%%%%%%%%%%%%%%%%%%%%%%%%%%%%%%%%
以下で,$(\cdot)^T$は$(\cdot)$の転置を意味する.例えば,横ベクトル$(a,b)$の
転置$(a,b)^T$は次のような縦ベクトルと解釈する.
%\begin{equation}
\[
	(a,b)^T=\left(
	\begin{array}{c}
		a \\
		b
	\end{array}
	\right)
\]
%\end{equation}
\subsubsection*{問題}
2次元空間内の物体が外力を受けて変形し, 当初$\fat{x}$にあった点が
変形後$\fat{y}$へ移動したとする.ここで,$\fat{y}$が,
\begin{equation}
	\fat{y}=\fat{A}\fat{x}
	\label{eqn:disp_fld}
\end{equation}
のように$\fat{x}$の関数として与えられる場合を考える.
ただし,$\fat{x}=(x_1,\, x_2)^T, \fat{y}=(y_1, y_2)^T$で,$\fat{A}$は
\begin{equation}
	\fat{A}=
	\left(
	\begin{array}{cc}
		 3 & 3 \\
	 	 1 & -1 
	\end{array}
	\label{eqn:Amat}
	\right)
\end{equation}
とする.このとき物体に発生している変位$\fat{u}$とひずみ$\fat{\varepsilon}$を,
それぞれ
\begin{equation}
	\fat{u}=\left( 
	\begin{array}{c}
	u_1,\\
	u_2 
	\end{array}
	\right)
	, \ \ 
	\fat{\varepsilon}
	=
	\left(
	\begin{array}{cc}
		\varepsilon_{11} & \varepsilon_{12} \\
		\varepsilon_{21} & \varepsilon_{22} 
	\end{array}
	\right)
\end{equation}
と表す.なお,以上は全て$o-x_1x_2$直交座標系による表現とする.
このとき,以下の問1$\sim$5に答えよ.
\begin{enumerate}
\item
	変形前に$(1,0)^T$と$(0,1)^T$にあった2点が,変形後に移る位置をそれぞれ答えよ.
\item
	$o-x_1x_2$座標系において,
	\[
		\left\{ (x_1,x_2):
		0\leq x_1 \leq 1, 
		\ \
		0\leq x_2 \leq 1
		\right\}
	\]
	で表される正方形領域が,変形後$o-x_1x_2$平面内でどのような領域を占めるか調べ,
	その結果を図示せよ.
\item
	変位ベクトル成分$u_1,u_2$を求めよ.
\item
	ひずみテンソルの成分
	$\varepsilon_{11},\varepsilon_{12}(=\varepsilon_{21}), \varepsilon_{22}$
	を求めよ.
\item
	ひずみテンソルは,行列$\fat{A}$と次の関係にあることを示せ.
	\begin{equation}
		\fat{\varepsilon}=\frac{\fat{A}+\fat{A}^T}{2}-\fat{I}
		\label{eqn:def_eps}
	\end{equation}
	ただし,$\fat{I}$は$2\times 2$の単位行列を意味する.
\end{enumerate}


次に,$o-x_1x_2$座標系を原点を中心に$\theta$だけ反時計回りの方向へ
回転させて得られる直交座標系$o-x_1'x_2'$を導入する.
$o-x_1'x_2'$座標系では,式(\ref{eqn:disp_fld})の関係を表すときに,
ベクトルやマトリクスとその成分に$(\cdot)'$をつけ,
それらの諸量が$o-x_1'x_2'$座標系におけるものであることを示す.
すなわち,$o-x_1'x_2'$座標系では式(\ref{eqn:disp_fld})の関係を
\begin{equation}
	\fat{y}'=\fat{A}'\fat{x}'
	\label{eqn:disp_fld2}
\end{equation}
と表す.ここで,
\begin{equation}
	\fat{u}'=(u_1',\, u_2')^T
\end{equation}
は変位ベクトルを,
\begin{equation}
	\fat{x}'=(x_1',\, x_2')^T, \ \ \fat{y}'=(y_1',\, y_2')^T
\end{equation}
は$\fat{x}$と$\fat{y}$と同じ位置を指す位置ベクトルを,
$\fat{A}'$は式(\ref{eqn:disp_fld})と同じ変形状態を記述する行列である.
いま,$o-x_1x_2$から$o-x_1'x_2'$への座標変換マトリクスを$\fat{Q}$とすれば,
$\fat{A}$と$\fat{A}'$の間には
\begin{equation}
	\fat{A}'=\fat{Q}\fat{A}\fat{Q}^T
	\label{eqn:QAQ}
\end{equation}
の関係が成り立つ.なぜなら,
\[
	\fat{x}'=\fat{Qx}, \ \ \fat{y}'=\fat{Qy}
\]
だから,これらを式(\ref{eqn:disp_fld})に代入し,その結果を式(\ref{eqn:disp_fld2})と比べれば,
式(\ref{eqn:QAQ})が成り立たなければならないことが示されるためでる.
このことを踏まえれば,$o-x_1'x_2'$座標系におけるひずみテンソル
\begin{equation}
	\fat{\varepsilon}'=\frac{\fat{A}'+\left(\fat{A}'\right)^T}{2}-\fat{I}
	\label{eqn:epsd_def}
\end{equation}
を得るための座標変換則は
\begin{equation}
	\fat{\varepsilon}'=\fat{Q}\fat{\varepsilon}\fat{Q}^T
	\label{eqn:QeQ}
\end{equation}
となることも分かる.以上の結果を用いて,下記6と7の問に答えよ.
\begin{enumerate}
\setcounter{enumi}{5}
\item
	$o-x_1'x_2'$座標系におけるひずみテンソルを
	\[
	\fat{\varepsilon}'
	=
	\left(
	\begin{array}{cc}
		\varepsilon_{11}' & \varepsilon_{12}' \\
		\varepsilon_{21}' & \varepsilon_{22}'
	\end{array}
	\right)
	\]
	と表すとき,$\varepsilon_{11}'$と$\varepsilon_{12}'$を
	$\cos 2\theta, \, \sin 2\theta$を使って表わせ.
\item
	$\varepsilon_{12}'=0$となる$\theta$における,直ひずみ$\varepsilon_{11}'$を求めよ.
\end{enumerate}
\end{document}

%--------------------
\begin{figure}[h]
	\begin{center}
	\includegraphics[width=0.8\linewidth]{fig1.eps} 
	\end{center}
	\caption{(a)$o-x_1x_2$座標系と, (b)固有ベクトル$\fat{n}_1,\fat{n}_2$を基底ベクトルにもつ$o-x_1'x_2'$座標系.} 
	\label{fig:fig1}
\end{figure}
%--------------------

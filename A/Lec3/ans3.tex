\documentclass[10pt,a4j]{jarticle}
\usepackage{graphicx,wrapfig}
\setlength{\topmargin}{-1.5cm}
\setlength{\textwidth}{15.5cm}
\setlength{\textheight}{25.2cm}
\newlength{\minitwocolumn}
\setlength{\minitwocolumn}{0.5\textwidth}
\addtolength{\minitwocolumn}{-\columnsep}
%\addtolength{\baselineskip}{-0.1\baselineskip}
%
\def\Mmaru#1{{\ooalign{\hfil#1\/\hfil\crcr
\raise.167ex\hbox{\mathhexbox 20D}}}}
%
\begin{document}
\newcommand{\fat}[1]{\mbox{\boldmath $#1$}}
\newcommand{\D}{\partial}
\newcommand{\w}{\omega}
\newcommand{\ga}{\alpha}
\newcommand{\gb}{\beta}
\newcommand{\gx}{\xi}
\newcommand{\gz}{\zeta}
\newcommand{\vhat}[1]{\hat{\fat{#1}}}
\newcommand{\spc}{\vspace{0.7\baselineskip}}
\newcommand{\halfspc}{\vspace{0.3\baselineskip}}
\bibliographystyle{unsrt}
\pagestyle{empty}
\newcommand{\twofig}[2]
 {
   \begin{figure}[h]
     \begin{minipage}[t]{\minitwocolumn}
         \begin{center}   #1
         \end{center}
     \end{minipage}
         \hspace{\columnsep}
     \begin{minipage}[t]{\minitwocolumn}
         \begin{center} #2
         \end{center}
     \end{minipage}
   \end{figure}
 }
%%%%%%%%%%%%%%%%%%%%%%%%%%%%%%%%%
%\vspace*{\baselineskip}
%\begin{flushright}
%	更新, 修正:2017年10月30日
%\end{flushright}
\begin{center}
	{\Large \bf 2019年度 構造力学I及び演習A 演習問題3 解答} \\
\end{center}
%%%%%%%%%%%%%%%%%%%%%%%%%%%%%%%%%%%%%%%%%%%%%%%%%%%%%%%%%%%%%%%%
\vspace{15mm}
\begin{enumerate}
\item
行列$\fat{N}$は,
\begin{equation}
	\fat{N}=\left(	
	\begin{array}{cc}
		\cos\frac{\pi}{6} & \cos\frac{\pi}{3} \\
		\sin\frac{\pi}{6} & \sin\frac{\pi}{3}
	\end{array}
	\right)
	=
	\frac{1}{2}
	\left(
	\begin{array}{cc}
		\sqrt{3} & \sqrt{1} \\
		1 & \sqrt{3}
	\end{array}
	\right)
\end{equation}
で与えられ、この逆行列を計算すると
\begin{equation}
	\fat{N}^{-1}=
	\left(	
	\begin{array}{cc}
		\sqrt{3} & -1 \\ 
		-1 & \sqrt{3} 
	\end{array}
	\right)
\end{equation}
となる.
\item
$\fat{t}^{(n)}=\fat{\sigma}^T \fat{n}$より,
\begin{equation}
	\fat{t}_1=\fat{\sigma}\fat{n}_1, \ \ 
	\fat{t}_2=\fat{\sigma}\fat{n}_2 
\end{equation}
だから、これらの式を連立して$\fat{\sigma}$を求めれば良い.
ここで、2つの列ベクトル$\fat{t}_1$と
		$\fat{t}_2$を並べた行列を
\begin{equation}
	\fat{T}=\left\{ \fat{t}_1\,\, \fat{t}_2 \right\}
	=
	\left(
	\begin{array}{cc}
		-\sqrt{3} & 1 \\
		3 & \sqrt{3} 
	\end{array}
	\right)
\end{equation}
とすれば、$\fat{\sigma}\fat{N}=\fat{T}$
より、
		\begin{equation}
			\fat{\sigma}=\fat{T}\fat{N}^{-1}=
			\left(
			\begin{array}{cc}
				-4 & 2\sqrt{3} \\
				2\sqrt{3} & 0
			\end{array}
			\right)
		\end{equation}
となる.
\item
\[
	\fat{t}^{(n)}=
	\fat{\sigma}^T\fat{n}(\theta)
	=\left(
		\begin{array}{c}
			-4\cos\theta + 2\sqrt{3}\sin\theta \\
			2\sqrt{3}\cos\theta
		\end{array}
	\right)
\]
\item
表面力$\fat{t}^{(n)}$と単位接ベクトル$\fat{n}(\theta)$の内積をとればよい.
\begin{eqnarray}
	\sigma_n &=& \fat{t}^{(n)}\cdot \fat{n} \\
	&=&
	-4\cos ^2\theta +4\sqrt{3}\sin\theta\cos\theta\\
	&=&
	-2-2\cos 2\theta +2\sqrt{3}\sin 2\theta \\
	&=& 
	-2 + 4\cos\left(2\theta -\frac{2\pi}{3}\right)
	\label{eqn:sign}
\end{eqnarray}
\item
表面力$\fat{t}^{(n)}$と単位接ベクトル$\fat{m}$の内積をとればよい.
$\fat{m}$は,
\begin{equation}
	\fat{m} =
	\left(
	\begin{array}{c}
		-\sin{\theta} \\
		 \cos{\theta}
	\end{array}
	\right)
\end{equation}
だから,
\begin{eqnarray}
	\tau &=& \fat{t}^{(n)} \cdot \fat{m}  \\
	&=&
	4\sin\theta\cos\theta
	+
	2\sqrt{3} ( \cos^2\theta  -\sin^2\theta ) 
	\\
	&=&
	2\sin 2\theta + 2\sqrt{3} \cos 2\theta \\
	&=&
	-4\sin \left( 2\theta -\frac{2\pi}{3}\right)
	\label{eqn:tau1}
\end{eqnarray}
となる
\item
式(\ref{eqn:sign})より, $\sigma_n$は$\theta=\frac{\pi}{3}$で最大値2をとり、
このとき式(\ref{eqn:tau1})から$\tau=0$となることが分かる。
\item
式(\ref{eqn:sign})と式(\ref{eqn:tau1})から$\theta$を消去すれば,
\begin{equation}
	\left( \sigma_n +2 \right)^2+\tau^2 = 4^2
\end{equation}
の関係が得られ,$(\sigma_n, \, \tau)$は中心$(-2,\, 0)$,
半径$4$の円を描くことが分かる.
\end{enumerate}

%--------------------
\end{document}
\begin{figure}[h]
	\begin{center}
	\includegraphics[width=0.7\linewidth]{Morl.eps} 
	\end{center}
	\caption{直応力$\sigma_n$とせん断応力$\tau$の関係.} 
	\label{fig:fig1}
\end{figure}

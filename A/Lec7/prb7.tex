\documentclass[10pt,a4j]{jarticle}
\usepackage{graphicx,wrapfig,amsmath}
\setlength{\topmargin}{-1.5cm}
%\setlength{\textwidth}{15.5cm}
\setlength{\textheight}{25.2cm}
\newlength{\minitwocolumn}
\setlength{\minitwocolumn}{0.5\textwidth}
\addtolength{\minitwocolumn}{-\columnsep}
%\addtolength{\baselineskip}{-0.1\baselineskip}
%
\def\Mmaru#1{{\ooalign{\hfil#1\/\hfil\crcr
\raise.167ex\hbox{\mathhexbox 20D}}}}
%
\begin{document}
\newcommand{\fat}[1]{\mbox{\boldmath $#1$}}
\newcommand{\D}{\partial}
\newcommand{\w}{\omega}
\newcommand{\ga}{\alpha}
\newcommand{\gb}{\beta}
\newcommand{\gx}{\xi}
\newcommand{\gz}{\zeta}
\newcommand{\vhat}[1]{\hat{\fat{#1}}}
\newcommand{\spc}{\vspace{0.7\baselineskip}}
\newcommand{\halfspc}{\vspace{0.3\baselineskip}}
\bibliographystyle{unsrt}
\pagestyle{empty}
\newcommand{\twofig}[2]
 {
   \begin{figure}[here]
     \begin{minipage}[t]{\minitwocolumn}
         \begin{center}   #1
         \end{center}
     \end{minipage}
         \hspace{\columnsep}
     \begin{minipage}[t]{\minitwocolumn}
         \begin{center} #2
         \end{center}
     \end{minipage}
   \end{figure}
 }
%%%%%%%%%%%%%%%%%%%%%%%%%%%%%%%%%
%\vspace*{\baselineskip}
\begin{center}
	{\Large \bf 2020年度 構造力学I及び演習A 演習問題7}
\end{center}
%%%%%%%%%%%%%%%%%%%%%%%%%%%%%%%%%%%%%%%%%%%%%%%%%%%%%%%%%%%%%%%%
\vspace{15mm}
図\ref{fig:fig1}-(a)のような梁に,同図(b)に示す直線的に変化する鉛直方向の分布力
$q(x)$が作用している. 梁のヤング率$E$と断面2次モーメント$I$は全ての
位置(断面)で一定とするとき,以下の問に答えよ.
\begin{enumerate}
\item
	$q(x)$を$q_0,\,l$と$x$を使って表わせ.
\item
	$q(x)$の二階の不定積分$\iint q(x)dx^2$を求めよ.
\item
	部材端A,B点における曲げモーメントが
	$M\left(\pm\frac{l}{2}\right)=0$のとき,
	区間AB$\left( -\frac{l}{2}<x < \frac{l}{2}\right)$
	における曲げモーメント分布$M(x)$を求めよ.
\item
	区間ABにおけるせん断力分布$Q(x)$を求めよ.
\item
	上で求めた曲げモーメント$M(x)$の,二階の不定積分$\iint M(x) dx^2$を求めよ.
\item
	部材端A,B点におけるたわみが$v\left(\pm\frac{l}{2}\right)=0$のとき,
	区間AB$\left( -\frac{l}{2}<x < \frac{l}{2}\right)$
	におけるたわみ分布$v(x)$を求めよ.
\item
	区間ABにおけるたわみ角分布$\theta(x)$を求めよ.
\item
	位置$x$の断面における,直応力$\sigma_{xx}$の$y$方向分布を求めよ.     
\end{enumerate}
%--------------------
%\begin{figure}[here]
\begin{figure}[h]
	\begin{center}
	\includegraphics[width=0.8\linewidth]{fig1.eps} 
	\end{center}
	\caption{(a)長さ$l$の梁AB. (b)梁に加えられた鉛直方向の分布荷重$q(x)$.(c)梁の断面形状と座標系.
        たわみ$v(x)$と荷重$q(x)$は鉛直下向きを正とする.}
	\label{fig:fig1}
\end{figure}
%--------------------
\end{document}

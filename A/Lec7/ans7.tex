\documentclass[10pt,a4j]{jarticle}
\usepackage{graphicx,wrapfig,amsmath}
\setlength{\topmargin}{-1.5cm}
%\setlength{\textwidth}{15.5cm}
\setlength{\textheight}{25.2cm}
\newlength{\minitwocolumn}
\setlength{\minitwocolumn}{0.5\textwidth}
\addtolength{\minitwocolumn}{-\columnsep}
%\addtolength{\baselineskip}{-0.1\baselineskip}
%
\def\Mmaru#1{{\ooalign{\hfil#1\/\hfil\crcr
\raise.167ex\hbox{\mathhexbox 20D}}}}
%
\begin{document}
\newcommand{\fat}[1]{\mbox{\boldmath $#1$}}
\newcommand{\D}{\partial}
\newcommand{\w}{\omega}
\newcommand{\ga}{\alpha}
\newcommand{\gb}{\beta}
\newcommand{\gx}{\xi}
\newcommand{\gz}{\zeta}
\newcommand{\vhat}[1]{\hat{\fat{#1}}}
\newcommand{\spc}{\vspace{0.7\baselineskip}}
\newcommand{\halfspc}{\vspace{0.3\baselineskip}}
\bibliographystyle{unsrt}
\pagestyle{empty}
\newcommand{\twofig}[2]
 {
   \begin{figure}[h]
     \begin{minipage}[t]{\minitwocolumn}
         \begin{center}   #1
         \end{center}
     \end{minipage}
         \hspace{\columnsep}
     \begin{minipage}[t]{\minitwocolumn}
         \begin{center} #2
         \end{center}
     \end{minipage}
   \end{figure}
 }
%%%%%%%%%%%%%%%%%%%%%%%%%%%%%%%%%
%\vspace*{\baselineskip}
\begin{center}
{\Large \bf 2020年度 構造力学I及び演習A 演習問題7 解答}
\end{center}
%%%%%%%%%%%%%%%%%%%%%%%%%%%%%%%%%%%%%%%%%%%%%%%%%%%%%%%%%%%%%%%%
\vspace{15mm}
\begin{enumerate}
\item
	$q(x)=\frac{q_0x}{l}$
\item
	$\iint q(x)dx^2 =\frac{q_0l^2}{6}\left(\frac{x}{l}\right)^3$
\item
	$M''(x)=-q(x)$より,
	\begin{equation}
		M(x)= 
		-\frac{q_0l^2}{6}
		\left\{	
		\left(\frac{x}{l}\right)^3
		+
		C_1
		\left(\frac{x}{l}\right)
		+
		C_2
		\right\}
	\end{equation}
	と書くことができる.積分定数$C_1,C_2$は,
	$M\left(\pm\frac{l}{2}\right)=0$より$C_1=-\frac{1}{4},C_2=0$と決まる.
	よって,
	\begin{equation}
		M(x)= 
		-\frac{q_0l^2}{6}
		\left\{	
		\left(\frac{x}{l}\right)^3
		-
		\frac{1}{4}	
		\left(\frac{x}{l}\right)
		\right\}.
		\label{eqn:Mx}
	\end{equation}
\item
	\begin{equation}
		Q(x)=M'(x)=-\frac{q_0l}{6} \left\{ 
		3\left(\frac{x}{l}\right)^2
		-
		\frac{1}{4}	
		\right\} 
	\end{equation}
\item
	式(\ref{eqn:Mx})より
	\begin{equation}
		\iint M(x)dx^2
		=
		-\frac{q_0l^4}{6}
		\left\{	
		\frac{1}{20}
		\left(\frac{x}{l}\right)^5
		-
		\frac{1}{24}	
		\left(\frac{x}{l}\right)^3
		\right\}
		\label{eqn:vx0}
	\end{equation}
	となる.
\item
	$EIv''=-M$より,$v$は式(\ref{eqn:vx0})を利用して
	\begin{equation}
		v(x)=\frac{q_0l^4}{120EI}
		\left\{
			\left(\frac{x}{l}\right)^5
			-
			\frac{5}{6}
			\left(\frac{x}{l}\right)^3
			+
			C_3
			\left(\frac{x}{l}\right)
			+
			C_4
		\right\}
	\end{equation}
	と書くことができる.ただし$C_3,C_4$は積分定数を表す.
	部材端におけるたわみの条件$v\left(\pm \frac{l}{2}\right)=0$より,
	$C_3=\frac{7}{48},C_4=0$となるので,たわみは次のように決まる.
	\begin{equation}
		v(x)=\frac{q_0l^4}{120EI}
		\left\{
			\left(\frac{x}{l}\right)^5
			-
			\frac{5}{6}
			\left(\frac{x}{l}\right)^3
			+
			\frac{7}{48}
			\left(\frac{x}{l}\right)
		\right\}
	\end{equation}
\item
	\begin{equation}
		\theta(x)=v'(x)=\frac{q_0l^3}{120EI}
		\left\{
			5
			\left(\frac{x}{l}\right)^4
			-
			\frac{5}{2}
			\left(\frac{x}{l}\right)^2
			+
			\frac{7}{48}
		\right\}
	\end{equation}
\item
曲げ応力$\sigma_{xx}$は, たわみ$v$あるいは曲げモーメント$M$により
\begin{equation}
	\sigma_{xx}=-Ev''y= \frac{M}{I}y 
	\label{eqn:Sxx_M}
\end{equation}
と表される.よって,式(\ref{eqn:Sxx_M})に,式(\ref{eqn:Mx})を代入すれば,
\begin{equation}
	\sigma_{xx}=
	-\frac{q_0l^2}{6I}
	\left\{	
	\left(\frac{x}{l}\right)^3
	-
	\frac{1}{4}	
	\left(\frac{x}{l}\right)
	\right\}y
\end{equation}
	となる.
\end{enumerate}
\end{document}

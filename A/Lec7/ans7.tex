\documentclass[10pt,a4j]{jarticle}
\usepackage{graphicx,wrapfig}
\setlength{\topmargin}{-1.5cm}
\setlength{\textwidth}{15.5cm}
\setlength{\textheight}{25.2cm}
\newlength{\minitwocolumn}
\setlength{\minitwocolumn}{0.5\textwidth}
\addtolength{\minitwocolumn}{-\columnsep}
%\addtolength{\baselineskip}{-0.1\baselineskip}
%
\def\Mmaru#1{{\ooalign{\hfil#1\/\hfil\crcr
\raise.167ex\hbox{\mathhexbox 20D}}}}
%
\begin{document}
\newcommand{\fat}[1]{\mbox{\boldmath $#1$}}
\newcommand{\D}{\partial}
\newcommand{\w}{\omega}
\newcommand{\ga}{\alpha}
\newcommand{\gb}{\beta}
\newcommand{\gx}{\xi}
\newcommand{\gz}{\zeta}
\newcommand{\vhat}[1]{\hat{\fat{#1}}}
\newcommand{\spc}{\vspace{0.7\baselineskip}}
\newcommand{\halfspc}{\vspace{0.3\baselineskip}}
\bibliographystyle{unsrt}
\pagestyle{empty}
\newcommand{\twofig}[2]
 {
   \begin{figure}[h]
     \begin{minipage}[t]{\minitwocolumn}
         \begin{center}   #1
         \end{center}
     \end{minipage}
         \hspace{\columnsep}
     \begin{minipage}[t]{\minitwocolumn}
         \begin{center} #2
         \end{center}
     \end{minipage}
   \end{figure}
 }
%%%%%%%%%%%%%%%%%%%%%%%%%%%%%%%%%
%\vspace*{\baselineskip}
\begin{center}
{\Large \bf 2019年度 構造力学I及び演習A 演習問題7 解答} \\
\end{center}
%%%%%%%%%%%%%%%%%%%%%%%%%%%%%%%%%%%%%%%%%%%%%%%%%%%%%%%%%%%%%%%%
\vspace{15mm}
\begin{enumerate}
\item
せん断力$Q$と外力$q$の関係は,
\begin{equation}
	\frac{dQ(x)}{dx} = -q(x) = -\frac{q_0}{l}x
	\label{eqn:Q_q}
\end{equation}
で与えられる.よって, $Q$は, $C_1$を積分定数として
\begin{equation}
	Q(x) =	-\frac{q_0}{2l}x^2+C_1
	\label{eqn:Q_gen}
\end{equation}
と書くことができる.ここで,$Q\left(0 \right)=0$より$C=0$である.
よって,
\begin{equation}
	Q(x) = -\frac{q_0}{2l}x^2
	\label{eqn:Q_par}
\end{equation}
となる.
\item
$\frac{dM}{dx}=Q$より,曲げモーメント$M$は,積分定数を$C_2$として
\begin{equation}
	M(x) =
	-\frac{q_0}{6l}x^3+C_2
	\label{eqn:M_gen}
\end{equation}
と表すことができる.ここで, $M\left(0\right)=0$より$C_2=0$である.
よって
\begin{equation}
	M(x) = 
	-\frac{q_0}{6l}x^3.
	\label{eqn:M_par}
\end{equation}
\item
たわみ角$\theta$と曲げモーメント$M$の関係,$\frac{d\theta}{dx}=-\frac{M}{EI}$より,
\begin{equation}
	EI \theta(x) = 
	\frac{q_0}{24l}x^4+C_3' =
	\frac{q_0l^3}{24}\left\{
		\left(\frac{x}{l}\right)^4+C_3
	\right\}
	\label{eqn:th_gen}
\end{equation}
と書くことができる.積分定数$C_3$は, $\theta\left(l\right)=0$より,
$C_3=-1$となる.以上より,たわみ角は
\begin{equation}
	\theta(x) = 
	\frac{q_0l^3}{24EI}\left\{
		\left(\frac{x}{l}\right)^4-1
	\right\}
	\label{eqn:th_par}
\end{equation}
と求められる.
\item
たわみ$v$とたわみ角の関係$\frac{dv}{dx}=\theta$と,式(\ref{eqn:th_par})
から,$v$は積分定数$C_4'$あるいは$C_4$を用いて
\begin{equation}
	EIv(x) = 
	\frac{q_0l^3}{24}\left\{
		\frac{1}{5}\left(\frac{x}{l}\right)^5-x + C_4'
	\right\}
	=
	\frac{q_0l^4}{120}\left\{
		\left(\frac{x}{l}\right)^5
		-
		5
		\left(\frac{x}{l}\right) 
		+ C_4
	\right\}
	\label{eqn:v_gen}
\end{equation}
と表すことができる.よって, $v\left(l \right)=0$
を考慮すれば,積分定数は$C_4=4$となり,
\begin{equation}
	v(x) = 
	\frac{q_0l^4}{120EI}\left\{
		\left(\frac{x}{l}\right)^5
		-
		5
		\left(\frac{x}{l}\right) 
		+ 
		4
	\right\}
	\label{eqn:v_par}
\end{equation}
と決まる.
%--------------------
%\begin{figure}[h]
%	\begin{center}
%	\includegraphics[width=0.45\linewidth]{fig1ans.eps} 
%	\end{center}
%	\caption{(a) せん断力図, (b)曲げモーメント図.}
%	\label{fig:fig1}
%\end{figure}
%--------------------
\item
曲げ応力$\sigma_{xx}$は, たわみ$v$あるいは曲げモーメント$M$により
\begin{equation}
	\sigma_{xx}=-Ev''y= \frac{M}{I}y 
	\label{eqn:Sxx_M}
\end{equation}
と表される.よって,式(\ref{eqn:Sxx_M})に,式(\ref{eqn:M_par})を代入すれば,
\begin{equation}
	\sigma_{xx}=
	-\frac{q_0}{6Il}x^3y
	\label{eqn:Sxx_xy}
\end{equation}
が得られる.
\item
応力のつり合い式に式(\ref{eqn:Sxx_xy})を代入すれば, 
\begin{equation}
	\frac{\partial \sigma_{xy}}{\partial y}
	=
	-\frac{\partial \sigma_{xx}}{\partial x}
	=
	\frac{q_0}{2Il}x^2y
	\label{eqn:ODE}
\end{equation}
となることから,せん断応力$\sigma_{xy}$は,任意定数$C$を用いて,
\begin{equation}
	\sigma_{xy}=
	\frac{q_0}{4Il}x^2(y^2+C)
	\label{eqn:Sxy_gen}
\end{equation}
と表すことができる.$y=\pm \frac{h}{2}$で$\sigma_{xy}=0$
となることを考慮すれば,この定数は$C=-\frac{h^2}{4}$と決まる.
以上より,せん断応力分布は
\begin{equation}
	\sigma_{xy}(x,y) 
	= 
	\frac{q_0}{4Il}x^2\left( 
		y^2 -\frac{h^2}{4}
	\right)
\end{equation}
となる.
\end{enumerate}
\end{document}

\documentclass[10pt,a4j]{jarticle}
\usepackage{graphicx,wrapfig}
\setlength{\topmargin}{-1.5cm}
\setlength{\textwidth}{15.5cm}
\setlength{\textheight}{25.2cm}
\newlength{\minitwocolumn}
\setlength{\minitwocolumn}{0.5\textwidth}
\addtolength{\minitwocolumn}{-\columnsep}
%\addtolength{\baselineskip}{-0.1\baselineskip}
%
\def\Mmaru#1{{\ooalign{\hfil#1\/\hfil\crcr
\raise.167ex\hbox{\mathhexbox 20D}}}}
%
\begin{document}
\newcommand{\fat}[1]{\mbox{\boldmath $#1$}}
\newcommand{\D}{\partial}
\newcommand{\w}{\omega}
\newcommand{\ga}{\alpha}
\newcommand{\gb}{\beta}
\newcommand{\gx}{\xi}
\newcommand{\gz}{\zeta}
\newcommand{\vhat}[1]{\hat{\fat{#1}}}
\newcommand{\spc}{\vspace{0.7\baselineskip}}
\newcommand{\halfspc}{\vspace{0.3\baselineskip}}
\bibliographystyle{unsrt}
\pagestyle{empty}
\newcommand{\twofig}[2]
 {
   \begin{figure}[h]
     \begin{minipage}[t]{\minitwocolumn}
         \begin{center}   #1
         \end{center}
     \end{minipage}
         \hspace{\columnsep}
     \begin{minipage}[t]{\minitwocolumn}
         \begin{center} #2
         \end{center}
     \end{minipage}
   \end{figure}
 }
%%%%%%%%%%%%%%%%%%%%%%%%%%%%%%%%%
%\vspace*{\baselineskip}
\begin{center}
{\Large \bf 2019年度 構造力学I及び演習A 演習問題5 解答} \\
\end{center}
%%%%%%%%%%%%%%%%%%%%%%%%%%%%%%%%%%%%%%%%%%%%%%%%%%%%%%%%%%%%%%%%
\vspace{15mm}
\begin{enumerate}
\item
与えられた応力テンソルの成分を
\begin{equation}
	\fat{\sigma}
	=
	\left( 
		\begin{array}{cc}
		\sigma_{11} & \sigma_{12} \\
		\sigma_{12} & \sigma_{22} 
		\end{array}
	\right)
	=
	\left( 
		\begin{array}{cc}
			1-\sqrt{3} & 1\\
			1 & 1+\sqrt{3} 
		\end{array}
	\right)
	\label{eqn:sigma_given}
\end{equation}
と表すとき,平均応力$\bar{\sigma}$, 偏差応力$\Delta \sigma$および,
モール円の半径$R$は以下のように求められる.
\begin{equation}
	\bar{\sigma} 
	= \frac{1}{2}\left({ \sigma_{11} + \sigma_{22} }\right) = 1,
	\quad
	\Delta \sigma = \sigma_{11} - \sigma_{22} = -2\sqrt{3},
	\quad
	R = \sqrt{\left({ \frac{\Delta \sigma}{2} }\right)^2 + \tau^2 } 
	=2 
\end{equation}
となる.以上より,モールの応力円は図\ref{fig:fig1}のように,
中心(1,0)で半径が2の円となる.
\item
	$\left(\sigma_{11}, \sigma_{12}\right)=\left(1-\sqrt{3}, \, 1 \right)$
は,応力円上の点Aに相当する.
		なお,$\tan \phi = \frac{\sigma_{12}}{\Delta \sigma/2}=-\frac{1}{\sqrt{3}}$より, 
$\phi=\frac{5\pi}{6}$である.
\end{enumerate}
\begin{figure}[h]
	\begin{center}
	\includegraphics[width=.50\linewidth]{fig1ans.eps} 
	\end{center}
	\caption{モールの応力円.} 
	\label{fig:fig1}
\end{figure}
%
\begin{enumerate}
\setcounter{enumi}{2}
\item
最大主応力とその方向は,モールの応力円上の点Bで表される. 従って,
\begin{equation}
	\sigma_{max}=\bar\sigma+R=3
\end{equation}
\begin{equation}
	2\theta_{max}=\phi 
	\ \ \Rightarrow 	
	\theta_{max}=\frac{\phi}{2} =\frac{5\pi}{12} 
\end{equation}
\item
最小主応力とその方向は,モールの応力円上の点Cで表され, 
\begin{equation}
	\sigma_{min}=\bar\sigma-R=-1
\end{equation}
\begin{equation}
	2\theta_{min}=\phi+\pi 
	\ \ \Rightarrow 	
	\theta_{min}=\frac{\phi+\pi}{2} =\frac{11\pi}{12} 
\end{equation}
となる.
\item
最大せん断応力とそれが作用する面の方向は, 応力円上の点Dで表され,
\begin{equation}
	\tau_{max}=R=2
\end{equation}
\begin{equation}
	2\theta_{\tau}=\frac{\pi}{3}  \ \ \Rightarrow \ \ 
	\theta_{\tau}=\frac{\pi}{6}.
\end{equation}
である.
\item
応力テンソル$\fat{\sigma}$の固有値$\lambda_1,\lambda_2$は
\begin{equation}
	\det \left( \fat{\sigma}-\lambda I  \right) =0
\end{equation}
の根として求められる($\fat{I}$は単位行列を表す).
これを具体的に計算すれば
\begin{equation}
	\left| 
		\begin{array}{cc}
		1-\sqrt{3}-\lambda & 1 \\
		1 & 1+\sqrt{3}-\lambda 
		\end{array}
	\right|
	=
		(1-\sqrt{3}-\lambda)(1+\sqrt{3}-\lambda )
	=(1-\lambda)^2-4=0
\end{equation}
より,
\begin{equation}
	\lambda_1 = 3, \ \ \lambda_2 = -1
\end{equation}
となり,主応力値が求められる.
\item
次に,$\lambda_1, \lambda_2$に対応する固有ベクトル
$\fat{n}_1,\fat{n}_2$を, 
\begin{equation}
	\left( \fat{\sigma}-\lambda \fat{I} \right)=\fat{0}	
\end{equation}
に$\lambda=\lambda_1$あるいは$\lambda_2$を代入して求めれば,それぞれ
\begin{equation}
	\fat{n}_1= 
	\pm \frac{1}{2\sqrt{2+\sqrt{3}}}
	\left( 
		\begin{array}{c}
			1 \\
			2+\sqrt{3} 
		\end{array}
	\right)
\end{equation}
\begin{equation}
	\fat{n}_2= 
	\pm \frac{1}{2\sqrt{2-\sqrt{3}}}
	\left( 
		\begin{array}{c}
			-1 \\
			2-\sqrt{3} 
		\end{array}
	\right)
\end{equation}
となる.
\item
$\fat{t}^{(n)}=\fat{\sigma}^T \fat{n}$において,  
$\fat{n}=\fat{n}_1=(\cos\theta_{max},\,\sin\theta_{max})^T$
とすればよい. この$\fat{n}$は,$\fat{\sigma}$の固有ベクトル
で,対応する固有値は$\sigma_{max}$だから,
\begin{equation}
	\fat{t}^{(n)}=\sigma_{max}
		\left(
		\begin{array}{c}
			\cos \theta_{max} \\	
			\sin \theta_{max} 
		\end{array}
		\right)
		=
		\pm \frac{3}{2\sqrt{2+\sqrt{3}}}
		\left(
		\begin{array}{c}
			1 \\
			2+\sqrt{3}
		\end{array}
		\right)
\end{equation}
となる($+,-$の符号はどちらを選んでも良い).
\end{enumerate}
\end{document}



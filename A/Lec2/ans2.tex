\documentclass[10pt,a4j]{jarticle}
\usepackage{graphicx,wrapfig}
\setlength{\topmargin}{-1.5cm}
\setlength{\textwidth}{16.5cm}
\setlength{\textheight}{25.2cm}
\newlength{\minitwocolumn}
\setlength{\minitwocolumn}{0.5\textwidth}
\addtolength{\minitwocolumn}{-\columnsep}
%\addtolength{\baselineskip}{-0.1\baselineskip}
%
\def\Mmaru#1{{\ooalign{\hfil#1\/\hfil\crcr
\raise.167ex\hbox{\mathhexbox 20D}}}}
%
\begin{document}
\newcommand{\fat}[1]{\mbox{\boldmath $#1$}}
\newcommand{\D}{\partial}
\newcommand{\w}{\omega}
\newcommand{\ga}{\alpha}
\newcommand{\gb}{\beta}
\newcommand{\gx}{\xi}
\newcommand{\gz}{\zeta}
\newcommand{\vhat}[1]{\hat{\fat{#1}}}
\newcommand{\spc}{\vspace{0.7\baselineskip}}
\newcommand{\halfspc}{\vspace{0.3\baselineskip}}
\bibliographystyle{unsrt}
\pagestyle{empty}
\newcommand{\twofig}[2]
 {
   \begin{figure}
     \begin{minipage}[t]{\minitwocolumn}
         \begin{center}   #1
         \end{center}
     \end{minipage}
         \hspace{\columnsep}
     \begin{minipage}[t]{\minitwocolumn}
         \begin{center} #2
         \end{center}
     \end{minipage}
   \end{figure}
 }
%%%%%%%%%%%%%%%%%%%%%%%%%%%%%%%%%
%\vspace*{\baselineskip}
\begin{center}
{\Large \bf 2020年度 構造力学I及び演習A 演習問題2(解答)} \\
\end{center}
%%%%%%%%%%%%%%%%%%%%%%%%%%%%%%%%%%%%%%%%%%%%%%%%%%%%%%%%%%%%%%%%
\vspace{15mm}
\begin{enumerate}
\item
部材軸方向に働く単位長さあたりの力$p(x)$は
\begin{equation}
	p(x)= p_0\frac{x}{l}
	\label{eqn:body_force}
\end{equation}
だから,支配方程式は
\begin{equation}
	\frac{d}{dx  }\left( EA\frac{du}{dx}\right)
	+
	p_0\frac{x}{l}
	=0
	\label{eqn:equiv}
\end{equation}
と表される($EA$は一定であることから, 式(\ref{eqn:equiv})の
第一項は, $EA\frac{d^2u}{dx^2}$としてもよい).
支持条件は, 両端固定のため,
\begin{equation}
	u\left(-\frac{l}{3}\right)=0, \ \ u\left(\frac{2l}{3}\right)=0
	\label{eqn:bcon}
\end{equation}
となる.
\item
微分方程式(\ref{eqn:equiv})を境界条件(\ref{eqn:bcon})の元で解けば,
\begin{equation}
	u(x)=
	-
	\frac{p_0l^2}{6EA}
	\left\{ 
		\left(\frac{x}{l}\right)^3
		-
		\frac{1}{3}
		\left(\frac{x}{l}\right)
		-\frac{2}{27}
	\right\}
	\label{eqn:disp}
\end{equation}
となる.
\item
\begin{equation}
	N(x)= A \sigma = AE\varepsilon=EA u'(x)
	=
	-\frac{p_0l}{2}
		\left\{
		\left( \frac{x}{l} \right)^2 
		-\frac{1}{9}
		\right\}
\end{equation}
\item
図\ref{fig:fig1}-(a)に示す通り.
\item
図\ref{fig:fig1}-(b)に示す通り.
\item
	$N\left(-\frac{l}{3}\right)=0$より,左端点における支点反力は0.
		右端点における支点反力は,$N\left(\frac{2l}{3}\right)=-\frac{p_0l}{6}$より,
		左向き,大きさ$\frac{p_0l}{6}$の力となる.
\end{enumerate}
%%%%%%%%%%%%%%%%%%%%%%%%%%%%%%%%%%%%%%%%%%%%
\begin{figure}[h]
	\vspace{-3mm}
	\begin{center}
	\includegraphics[width=0.45\linewidth]{fig1ans.eps} 
	\end{center}
	\vspace{-5mm}
	\caption{変位及び軸力分布を表すグラフ.} 
	\label{fig:fig1}
\end{figure}
%%%%%%%%%%%%%%%%%%%%%%%%%%%%%%%%%%%%%%%%%%%%
\end{document}



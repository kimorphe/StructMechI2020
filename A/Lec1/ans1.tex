\documentclass[10pt,a4j]{jarticle}
\usepackage{graphicx,wrapfig}
\setlength{\topmargin}{-1.5cm}
%\setlength{\textwidth}{16.5cm}
\setlength{\textheight}{25.2cm}
\newlength{\minitwocolumn}
\setlength{\minitwocolumn}{0.5\textwidth}
\addtolength{\minitwocolumn}{-\columnsep}
%\addtolength{\baselineskip}{-0.1\baselineskip}
%
\def\Mmaru#1{{\ooalign{\hfil#1\/\hfil\crcr
\raise.167ex\hbox{\mathhexbox 20D}}}}
%
\begin{document}
\newcommand{\fat}[1]{\mbox{\boldmath $#1$}}
\newcommand{\D}{\partial}
\newcommand{\w}{\omega}
\newcommand{\ga}{\alpha}
\newcommand{\gb}{\beta}
\newcommand{\gx}{\xi}
\newcommand{\gz}{\zeta}
\newcommand{\vhat}[1]{\hat{\fat{#1}}}
\newcommand{\spc}{\vspace{0.7\baselineskip}}
\newcommand{\halfspc}{\vspace{0.3\baselineskip}}
\bibliographystyle{unsrt}
\pagestyle{empty}
\newcommand{\twofig}[2]
 {
   \begin{figure}
     \begin{minipage}[t]{\minitwocolumn}
         \begin{center}   #1
         \end{center}
     \end{minipage}
         \hspace{\columnsep}
     \begin{minipage}[t]{\minitwocolumn}
         \begin{center} #2
         \end{center}
     \end{minipage}
   \end{figure}
 }
%%%%%%%%%%%%%%%%%%%%%%%%%%%%%%%%%
\begin{center}
{\Large \bf 2020年度 構造力学I及び演習A 演習問題1解答} \\
\end{center}
%%%%%%%%%%%%%%%%%%%%%%%%%%%%%%%%%%%%%%%%%%%%%%%%%%%%%%%%%%%%%%%%
\subsection*{問題1.}
\begin{enumerate}
\item
変位$u(x)$とひずみ$\varepsilon(x)$の関係は
\begin{equation}
	u(x)-u(0)=\int_0^x \varepsilon(x')dx'
	\label{eqn:eq1}
\end{equation}
で, ここでは$u(0)=0$である.また,
\begin{equation}
	\varepsilon(x) = \left\{
	\begin{array}{cc}
		\varepsilon_0 \left(1-\frac{6x}{l}\right)& (0<x \leq \frac{l}{3}) \\
		-\varepsilon_0 & \left(\frac{l}{3}< x\leq \frac{2l}{3} \right) \\
		  2\varepsilon_0 & \left(\frac{2l}{3} <x \leq l\right) 
	\end{array}
	\right.
\end{equation}
であることから,
\begin{enumerate}
\item 
$0<x \leq \frac{l}{3}$のとき, 
\begin{equation}
	u(x) = 
	\int_0^x 
		\varepsilon_0 \left(1-\frac{6x'}{l}\right)
	dx '=
	\varepsilon_0 l
	\left\{
		\frac{x}{l}
		-
		3
		\left(\frac{x}{l}\right)^2
	\right\}
	\label{eqn:u1}
\end{equation}
\item
$\frac{l}{3}<x\leq \frac{2l}{3}$のとき, 
\begin{equation}
	u(x) =
		\int_0^{\frac{l}{3}} 
		\varepsilon_0 \left(1-\frac{6x'}{l}\right)
		dx '
		+
		\int_{\frac{l}{3}}^x (-\varepsilon_0 )dx '
		=
		-\varepsilon_0 l
		\left\{ \left(\frac{x}{l}\right)-\frac{1}{3}\right\}
	\label{eqn:u2}
\end{equation}
\item
$\frac{2l}{3}<x\leq l$のとき, 
\begin{equation}
	u(x) =
		\int_0^{\frac{l}{3}} 
		\varepsilon_0 \left(1-\frac{6x'}{l}\right)
		dx '
		+
		\int_{\frac{l}{3}}^{\frac{2l}{3}} (-\varepsilon_0 )dx '
		+
		\int_{\frac{2l}{3}}^x 2\varepsilon_0 dx '
		=
		\varepsilon_0 l
		\left\{ 2\left(\frac{x}{l}\right)-\frac{5}{3}\right\}
	\label{eqn:u3}
\end{equation}
\end{enumerate}
以上から,変位分布は図\ref{fig:fig1}のようなグラフとなる.
%
\begin{figure}[h]
	\begin{center}
	\includegraphics[width=0.6\linewidth]{fig1ans.eps} 
	\end{center}
	\caption{変位分布を表すグラフ.} 
	\label{fig:fig1}
\end{figure}
\item
	棒全体の伸び$\Delta l$は, 変位から$\Delta l=u(l)-u(0)=\frac{\varepsilon_0l}{3}$.
\item
	区間XYの伸びを$\Delta l(X,Y)$とすれば,区間AB, BC, CDにおける伸びは
	式(\ref{eqn:u1})$\sim$(\ref{eqn:u3})を用いて,次のように与えられる.
	\begin{eqnarray*}
		\Delta l(A,B) 
		&=& 
		u\left( \frac{l}{3}\right)-u(0)=0
		\\
		\Delta l(B,C)
		&=&
		u\left( \frac{2l}{3}\right)-u\left(\frac{l}{3}\right)
		=-\frac{\varepsilon_0l}{3}
		\\	
		\Delta l(C,D)
		&=&u(l)-u\left( \frac{2l}{3}\right)=\frac{2}{3}\varepsilon_0l
	\end{eqnarray*}	
\end{enumerate}
%
%
\subsubsection*{問題2.}
図\ref{fig:fig2}-(a)に示すような位置$a-a'$と$b-b'$において部材を仮想的に切断したときの自由物体図を描くと,
それぞれ, 同図の(c)と(d)のようになる.いずれのケースでも,区間$(x,l)$に作用する外力の合計を
$P(x,l)$とすれば,釣り合い条件は
\begin{equation}
	N-P(x,l)=0
\end{equation}
より$N=P(x,l)$となる.そこで,分布荷重$p(x)$を座標$x$を用いて表すと,
\begin{equation}
	p(x)=\left\{
		\begin{array}{cc}
			2p_0 \frac{x}{l} & \left(0 < x \leq \frac{l}{2}\right) \\
			p_0  & \left(\frac{l}{2}< x \leq l \right) 
		\end{array}
	\right.
\end{equation}
となるので,これを用いて合力$P(x,l$を具体的に計算すれば,次のような結果が得られる.
\begin{equation}
	P(x,l)=
	\int_x^{l/2} 2p_0 \frac{x'}{l}dx'
	+
	\int_{l/2}^l  p_0 dx'
	=
	-p_0l\left\{ \left( \frac{x}{l}\right)^2 -\frac{3}{4}\right\} , \ \ 
	 \left(0 < x \leq \frac{l}{2}\right) 
\end{equation}
\begin{equation}
	P(x,l)=
	\int_{l/2}^l  p_0 dx'=p_0(l-x), \ \  \left(\frac{l}{2}< x \leq l \right) 
\end{equation}
従って,軸力は
\begin{eqnarray}
	N(x) &= & -p_0l\left\{ \left( \frac{x}{l}\right)^2 -\frac{3}{4}\right\} , 
	\ \ \left(0 < x \leq \frac{l}{2}\right)  \\
	N(x) &= & 
	p_0(l-x), \ \  \left(\frac{l}{2}< x \leq l \right) 
\end{eqnarray}
と求められ,軸力図は図\ref{fig:fig2}-(b)のようになる.
\begin{figure}[h]
	\vspace{-3mm}
	\begin{center}
	\includegraphics[width=0.8\linewidth]{fig2ans.eps} 
	\end{center}
	\vspace{-5mm}
	\caption{(a)自由物体図を描くための部材切断位置, (b)軸力図,
	および(c),(d)自由物体図.} 
	\label{fig:fig2}
\end{figure}
\end{document}

\documentclass[10pt,a4j]{jarticle}
\usepackage{graphicx,wrapfig}
\setlength{\topmargin}{-1.5cm}
%\setlength{\textwidth}{16.5cm}
\setlength{\textheight}{25.2cm}
\newlength{\minitwocolumn}
\setlength{\minitwocolumn}{0.5\textwidth}
\addtolength{\minitwocolumn}{-\columnsep}
%\addtolength{\baselineskip}{-0.1\baselineskip}
%
\def\Mmaru#1{{\ooalign{\hfil#1\/\hfil\crcr
\raise.167ex\hbox{\mathhexbox 20D}}}}
%
\begin{document}
\newcommand{\fat}[1]{\mbox{\boldmath $#1$}}
\newcommand{\D}{\partial}
\newcommand{\w}{\omega}
\newcommand{\ga}{\alpha}
\newcommand{\gb}{\beta}
\newcommand{\gx}{\xi}
\newcommand{\gz}{\zeta}
\newcommand{\vhat}[1]{\hat{\fat{#1}}}
\newcommand{\spc}{\vspace{0.7\baselineskip}}
\newcommand{\halfspc}{\vspace{0.3\baselineskip}}
\bibliographystyle{unsrt}
\pagestyle{empty}
\newcommand{\twofig}[2]
 {
   \begin{figure}
     \begin{minipage}[t]{\minitwocolumn}
         \begin{center}   #1
         \end{center}
     \end{minipage}
         \hspace{\columnsep}
     \begin{minipage}[t]{\minitwocolumn}
         \begin{center} #2
         \end{center}
     \end{minipage}
   \end{figure}
 }
%%%%%%%%%%%%%%%%%%%%%%%%%%%%%%%%%
%\vspace*{\baselineskip}
\begin{center}
	{\Large \bf 2020年度 構造力学I及び演習A 演習問題1} \\
\end{center}
%%%%%%%%%%%%%%%%%%%%%%%%%%%%%%%%%%%%%%%%%%%%%%%%%%%%%%%%%%%%%%%%
\subsubsection*{問題1.}
長さ$l$の棒部材ADに図\ref{fig:fig1}のようなひずみが生じているとき,以下の問に答えよ.
\begin{enumerate}
\item
	変位分布$u(x)$を求めそのグラフを描け.ただし,棒部材は$x=0$において固定されているとする.
\item
	棒部材全体の伸びを求めよ.
\item
	区間AB, BCおよびCDそれぞれの区間における伸びを求めよ.
\end{enumerate}
\begin{figure}[h]
	\vspace{-3mm}
	\begin{center}
	\includegraphics[width=0.45\linewidth]{fig1.eps} 
	\end{center}
	\vspace{-5mm}
	\caption{一端を固定壁に支持された棒部材ADにおけるひずみの分布.} 
	\label{fig:fig1}
\end{figure}
%%%%%%%%%%%%%%%%%%%%%%%%%%%%%%%%%%%%%%%%%%%%
	\vspace{-5mm}
\subsubsection*{問題2.} 
長さ$l$の棒部材ACに図\ref{fig:fig2}のような分布荷重(単位長さあたりの力)が作用している.
このとき,棒部材に生じる軸力分布を求め,その結果をグラフとして示せ.
なお,分布荷重は右方向を正,軸力は引張りを正とする.
\begin{figure}[h]
	\begin{center}
	\includegraphics[width=0.5\linewidth]{fig2.eps} 
	\end{center}
	\vspace{-5mm}
	\caption{分布荷重を受ける棒部材AC.} 
	\label{fig:fig2}
\end{figure}
\end{document}

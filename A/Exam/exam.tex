\documentclass[10pt,a4j]{jarticle}
\usepackage{graphicx,wrapfig}
\setlength{\topmargin}{-1.5cm}
\setlength{\textwidth}{15.5cm}
\setlength{\textheight}{25.2cm}
\newlength{\minitwocolumn}
\setlength{\minitwocolumn}{0.5\textwidth}
\addtolength{\minitwocolumn}{-\columnsep}
%\addtolength{\baselineskip}{-0.1\baselineskip}
%
\def\Mmaru#1{{\ooalign{\hfil#1\/\hfil\crcr
\raise.167ex\hbox{\mathhexbox 20D}}}}
%
\begin{document}
\newcommand{\fat}[1]{\mbox{\boldmath $#1$}}
\newcommand{\D}{\partial}
\newcommand{\w}{\omega}
\newcommand{\ga}{\alpha}
\newcommand{\gb}{\beta}
\newcommand{\gx}{\xi}
\newcommand{\gz}{\zeta}
\newcommand{\vhat}[1]{\hat{\fat{#1}}}
\newcommand{\spc}{\vspace{0.7\baselineskip}}
\newcommand{\halfspc}{\vspace{0.3\baselineskip}}
\bibliographystyle{unsrt}
%\pagestyle{empty}
\newcommand{\twofig}[2]
 {
   \begin{figure}[here]
     \begin{minipage}[t]{\minitwocolumn}
         \begin{center}   #1
         \end{center}
     \end{minipage}
         \hspace{\columnsep}
     \begin{minipage}[t]{\minitwocolumn}
         \begin{center} #2
         \end{center}
     \end{minipage}
   \end{figure}
 }
%%%%%%%%%%%%%%%%%%%%%%%%%%%%%%%%%
\begin{center}
{\Large \bf 2019年度 構造力学I及び演習A 期末試験} \\
\end{center}
\begin{flushright}
	2019年11月26日(火)
\end{flushright}
%%%%%%%%%%%%%%%%%%%%%%%%%%%%%%%%%%%%%%%%%%%%%%%%%%%%%%%%%%%%%%%%
解答用紙は片面のみ使用してください(各問題2枚). 
下書き用紙と問題用紙は, 試験終了後, 持ち帰って下さい.
%
%
%
\subsubsection*{問題1.}
図\ref{fig:fig1}-(a)のような両端が固定された棒部材
ACついて以下の問に答えよ.なお,棒部材の断面積$A$, ヤング率$E$は
全断面で一定で,部材軸方向に加えられた分布荷重の大きさ$p$は図\ref{fig:fig1}-(b)に示す通りとする.
\begin{enumerate}
\item
	棒部材に発生する軸方向変位を$u(x)$とするとき,$u(x)$が満足すべき
	微分方程式(支配方程式)と境界条件(支持条件)を$E, A, l$および$p_0$を用いて表わせ.
	ただし,座標$x$はBを原点とし,BからCの方向を正とする.
\item
	区間ACにおける変位分布を求めよ.
\item	
	区間ACにおける軸力分布を求めよ.
\item	
	支点AとCにおいて,棒部材が固定壁から受ける反力の大きさと方向を答えよ.
\item
	部材内部で伸びが最大となる区間を示し,その区間に生じる伸びを求めよ.
\end{enumerate}
\begin{figure}[h]
	\begin{center}
	\includegraphics[width=0.5\linewidth]{fig1.eps} 
	\end{center}
	\caption{分布力を受ける両端が固定された棒部材AC.}
	\label{fig:fig1}
\end{figure}
%--------------------
\newpage
\subsubsection*{問題2.}
以下では,$o-x_1x_2$直角直交座標において,トラクションベクトルを$\fat{t}^{(n)}$, 
応力テンソルを$\fat{\sigma}$,単位法線ベクトルを$\fat{n}$と表す.
物体中のある着目点$\fat{x}$において$\fat{\sigma}$が
\[
	\fat{\sigma}=
	\left(
	\begin{array}{cc}
		q &  r \\
		r & -q 
	\end{array}
	\right), \ \ (q,r \geq 0)
\]
で与えられるとき,以下の問に答えよ.
ただし,$q$と$r$の間には$\tan\phi =r/q$の関係があるとし,解答には角度$\phi$を用いてよい.
\begin{enumerate}
\item
	応力テンソル$\fat{\sigma}$に対するモールの応力円を描け.
\item
	応力テンソル$\fat{\sigma}$について, 最大および最小主応力と主応力方向を求めよ.
\item
	$\fat{n}=(\cos\alpha,\, \sin\alpha)^T$を法線ベクトルとする面に
	おけるトラクションベクトル$\fat{t}^{(n)}$を求めよ.
\item
	問3で求めたトラクションベクトルの$\fat{n}$方向成分を$t^{(n)}_n$と表す.
	$0\leq \alpha < 2\pi$における$t^{(n)}_n$の最大値を$t_{max}$,その時の
	法線ベクトルの方向$\alpha$を$\alpha_{max}$とする.このとき,$t_{max}$と
	$\alpha_{max}$を求めよ.
\item
	物体のヤング率を$E$, ポアソン比を$\nu$とする.物体が平面応力状態にあるとき,
	応力$\fat{\sigma}$によって生じるひずみテンソルの,
	直ひずみ成分$\varepsilon_{11}$と$\varepsilon_{22}$を求めよ.
\item
	図\ref{fig:fig2}に示すような$o-x_1'x_2'$直交座標系におけるひずみテンソルの
	直ひずみの成分を$\varepsilon_{11}',\varepsilon_{22}'$と表す.
	$\theta=\alpha_{max}$のときの,$\varepsilon_{11}'$と$\varepsilon_{22}'$を求めよ.
\item
	問6で求めた$\varepsilon_{11}'$は,$0 \leq \theta < 2\pi$における$\varepsilon_{11}'$
		の最大値であることを示せ.
\end{enumerate}
%--------------------
\begin{figure}[h]
	\begin{center}
	\includegraphics[width=0.45\linewidth]{fig2.eps} 
	\end{center}
	\caption{物体内部の着目点$\fat{x}$と,二つの直角直交座標系$o-x_1x_2$と$o-x_1'x_2'$.} 
	\label{fig:fig2}
\end{figure}
%--------------------
%%%%%%%%%%%%%%%%%%%%%%%%%%%%%%%%%%%%%%%%%%%%
%%%%%%%%%%%%%%%%%%%%%%%%%%%%%%%%%%%%%%%%%%%%
\end{document}

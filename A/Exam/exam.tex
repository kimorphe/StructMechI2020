\documentclass[10pt,a4j]{jarticle}
\usepackage{graphicx,wrapfig}
\setlength{\topmargin}{-1.5cm}
%\setlength{\textwidth}{15.5cm}
\setlength{\textheight}{25.2cm}
\newlength{\minitwocolumn}
\setlength{\minitwocolumn}{0.5\textwidth}
\addtolength{\minitwocolumn}{-\columnsep}
%\addtolength{\baselineskip}{-0.1\baselineskip}
%
\def\Mmaru#1{{\ooalign{\hfil#1\/\hfil\crcr
\raise.167ex\hbox{\mathhexbox 20D}}}}
%
\begin{document}
\newcommand{\fat}[1]{\mbox{\boldmath $#1$}}
\newcommand{\D}{\partial}
\newcommand{\w}{\omega}
\newcommand{\ga}{\alpha}
\newcommand{\gb}{\beta}
\newcommand{\gx}{\xi}
\newcommand{\gz}{\zeta}
\newcommand{\vhat}[1]{\hat{\fat{#1}}}
\newcommand{\spc}{\vspace{0.7\baselineskip}}
\newcommand{\halfspc}{\vspace{0.3\baselineskip}}
\bibliographystyle{unsrt}
%\pagestyle{empty}
\newcommand{\twofig}[2]
 {
   \begin{figure}[here]
     \begin{minipage}[t]{\minitwocolumn}
         \begin{center}   #1
         \end{center}
     \end{minipage}
         \hspace{\columnsep}
     \begin{minipage}[t]{\minitwocolumn}
         \begin{center} #2
         \end{center}
     \end{minipage}
   \end{figure}
 }
%%%%%%%%%%%%%%%%%%%%%%%%%%%%%%%%%
\begin{center}
{\Large \bf 2020年度 構造力学I及び演習A 期末試験} \\
\end{center}
\begin{flushright}
	2020年11月24日(火)
\end{flushright}
%%%%%%%%%%%%%%%%%%%%%%%%%%%%%%%%%%%%%%%%%%%%%%%%%%%%%%%%%%%%%%%%
%解答用紙は片面のみ使用してください(各問題2枚). 
%下書き用紙と問題用紙は, 試験終了後, 持ち帰って下さい.
%
%
%
\subsubsection*{問題1.}
図\ref{fig:fig1}-(a)のような左端が固定され,右端に水平方向の大きさ$F$の
集中荷重を受ける棒部材ABがある.棒部材の断面積$A$とヤング率$E$は全断面で一定で,
部材軸方向に加えられた分布荷重の大きさ$p$は図\ref{fig:fig1}-(b)に示す通りとする.
このとき,以下の問に答えよ.
\begin{enumerate}
\item
	棒部材に発生する軸方向変位を$u(x)$とするとき,$u(x)$が満足すべき
	微分方程式(支配方程式)と境界条件(支持条件)を$E, A, l$および$p_0$を用いて表わせ.
\item
	区間ABにおける変位分布を求めよ.
\item	
	区間ABにおける軸力分布を求めよ.
\item	
	支点Aにおいて棒部材が固定壁から受ける反力の大きさと方向を答えよ.
\item
	棒部材全体での伸びを求めよ.
\item
	ひずみが0となる点をCとする.C点の座標を$x=x_C$とするとき$x_C$を求めよ.
\end{enumerate}
集中荷重の大きさ$F$が$x_C=\frac{l}{2}$となるように与えられているとするとき,以下の問に答えよ.
\begin{enumerate}
\setcounter{enumi}{6}
\item
	区間ABにおける軸力分布を表すグラフを描け.
\item
	区間ABにおける変位$u$の分布を表すグラフを描け.
\end{enumerate}
\begin{figure}[h]
	\begin{center}
	\includegraphics[width=0.5\linewidth]{fig1.eps} 
	\end{center}
	\caption{左端部で固定された棒部材AB.}
	\label{fig:fig1}
\end{figure}
%--------------------
\newpage
\subsubsection*{問題2.}
以下では,$o-x_1x_2$直角直交座標において,トラクションベクトルを$\fat{t}^{(n)}$, 
応力テンソルを$\fat{\sigma}$,単位法線ベクトルを$\fat{n}$と表す.
平面応力状態にある物体中の着目点$\fat{x}$において$\fat{\sigma}$が
\begin{equation}
	\fat{\sigma}=
	p
	\left(
	\begin{array}{cc}
		\sqrt{3} &  1 \\
		1 & -\sqrt{3} 
	\end{array}
	\right), \ \ (p \geq 0)
	\label{eqn:sig}
\end{equation}
で与えられるとき,以下の問に答えよ.
\begin{enumerate}
\item
	式(\ref{eqn:sig})の応力テンソル$\fat{\sigma}$に対するモールの応力円を描け.
\item
	式(\ref{eqn:sig})の応力テンソル$\fat{\sigma}$について, 最大および最小主応力と主応力方向を求めよ.
\item
	$\fat{n}=(\cos\alpha,\, \sin\alpha)^T$を法線ベクトルとする面に
	作用するトラクションベクトル$\fat{t}^{(n)}$を求めよ.
\item 
	$\fat{t}^{(n)}$の大きさ,すなわち$\left| \fat{t}^{(n)}\right|$は$\alpha$に依らないことを示せ.
\item
	図\ref{fig:fig2}に示すような$o-x_1'x_2'$直交座標系における応力テンソルを
	\begin{equation}
	\fat{\sigma}'=
	\left(
	\begin{array}{cc}
		\sigma_{11}' &  \sigma_{12}' \\
		\sigma_{21}' &  \sigma_{22}' 
	\end{array}
	\right)
	\label{eqn:sigd}
	\end{equation}
		と表す.このとき,$\sigma_{11}',\sigma_{22}'$と$\sigma_{12}'$を$\theta$の関数として求めよ.
\item
	$o-x_1'x_2'$直交座標系におけるひずみテンソルを
	\begin{equation}
	\fat{\varepsilon}'=
	\left(
	\begin{array}{cc}
		\varepsilon_{11}' &  \varepsilon_{12}' \\
		\varepsilon_{21}' &  \varepsilon_{22}' 
	\end{array}
	\right)
	\label{eqn:epsd}
	\end{equation}
	と表す.このとき,$\varepsilon_{11}',\varepsilon_{22}'$と$\varepsilon_{12}'$を$\theta$の関数として求めよ.
	ただし,物体のヤング率を$E$, ポアソン比を$\nu$,せん断剛性を$G$とする.
\item
	$\theta(x_1$軸)の方向を問2で求めた最大主応力方向にとったときの,$\varepsilon_{11}',\varepsilon_{22}'$と$\varepsilon_{12}'$
		を求めよ.
\item
	$\varepsilon_{11}'$を横軸に,$\varepsilon_{12}'$を縦軸とする平面内で$(\varepsilon_{11}',\varepsilon_{12}')$
	がどのような軌跡(図形)を描くか答えよ.なお,$G,E,\nu$の間に
	\begin{equation}
		G=\frac{E}{2(1+\nu)}
	\end{equation}
	の関係があることを利用してよい.
\end{enumerate}
%--------------------
\begin{figure}[h]
	\begin{center}
	\includegraphics[width=0.45\linewidth]{fig2.eps} 
	\end{center}
	\caption{物体内部の着目点$\fat{x}$と,二つの直角直交座標系$o-x_1x_2$と$o-x_1'x_2'$.} 
	\label{fig:fig2}
\end{figure}
%--------------------
%%%%%%%%%%%%%%%%%%%%%%%%%%%%%%%%%%%%%%%%%%%%
%%%%%%%%%%%%%%%%%%%%%%%%%%%%%%%%%%%%%%%%%%%%
\end{document}

\documentclass[10pt,a4j]{jarticle}
\usepackage{graphicx,wrapfig}
\setlength{\topmargin}{-1.5cm}
%\setlength{\textwidth}{15.5cm}
\setlength{\textheight}{25.2cm}
\newlength{\minitwocolumn}
\setlength{\minitwocolumn}{0.5\textwidth}
\addtolength{\minitwocolumn}{-\columnsep}
%\addtolength{\baselineskip}{-0.1\baselineskip}
%
\def\Mmaru#1{{\ooalign{\hfil#1\/\hfil\crcr
\raise.167ex\hbox{\mathhexbox 20D}}}}
%
\begin{document}
\newcommand{\fat}[1]{\mbox{\boldmath $#1$}}
\newcommand{\D}{\partial}
\newcommand{\w}{\omega}
\newcommand{\ga}{\alpha}
\newcommand{\gb}{\beta}
\newcommand{\gx}{\xi}
\newcommand{\gz}{\zeta}
\newcommand{\vhat}[1]{\hat{\fat{#1}}}
\newcommand{\spc}{\vspace{0.7\baselineskip}}
\newcommand{\halfspc}{\vspace{0.3\baselineskip}}
\bibliographystyle{unsrt}
\pagestyle{empty}
\newcommand{\twofig}[2]
 {
   \begin{figure}[h]
     \begin{minipage}[t]{\minitwocolumn}
         \begin{center}   #1
         \end{center}
     \end{minipage}
         \hspace{\columnsep}
     \begin{minipage}[t]{\minitwocolumn}
         \begin{center} #2
         \end{center}
     \end{minipage}
   \end{figure}
 }
%%%%%%%%%%%%%%%%%%%%%%%%%%%%%%%%%
%\vspace*{\baselineskip}
\begin{center}
{\Large \bf 2020年度 構造力学I及び演習A(第8回) 期末試験 解答} \\
\end{center}
%%%%%%%%%%%%%%%%%%%%%%%%%%%%%%%%%%%%%%%%%%%%%%%%%%%%%%%%%%%%%%%%
%%%%%%%%%%%%%%%%%%%%%%%%%%%%%%%%%%%%%%%%%%%%%%%%%%%%%%%%%%%%%%%%%%%%%%%%%%%%%%%%%%%%%%%%%%
\subsubsection*{問題1.}
\begin{enumerate}
\item
	外力を$p(x)$とする. $p(x)$は
	\begin{equation}
		p(x)=\frac{x}{l}p_0 
		\ \ \left(0 < x < l \right)
		\label{eqn:px}
	\end{equation}
	と表される.よって,軸方向変位$u(x)$の支配方程式は,
	\begin{equation}
		\left( EAu' \right)'=-\frac{x}{l}p_0
		\label{eqn:gv_eq}
	\end{equation}
	となる.境界条件は,左端($x=0l$)部が固定,右端部で軸力が与えられていることから,
	\begin{equation}
		u(0)=0, \ \  N(l)=EAu'(l)=-F
		\label{eqn:bcon}
	\end{equation}
	と表される.
\item
	$EA$一定のため$u''=-p/EA$である.また,式(\ref{eqn:gv_eq})より,
	積分定数を$C_1, C_2$とすれば,
	\begin{equation}
		u(x)=-\frac{p_0l^2}{6EA}\left\{
			\left( \frac{x}{l}\right)^3
			+
			C_1\left( \frac{x}{l}\right)
			+
			C_2
		\right\}
	\end{equation}
		と表すことができる.積分定数は式(\ref{eqn:bcon})より, $C_1=\frac{6F}{p_0l}, C_2=0$となる.
	\begin{equation}
		u(x)=-\frac{p_0l^2}{6EA}\left\{
			\left( \frac{x}{l}\right)^3
			-
			3
			\left( \frac{x}{l}\right)
		\right\}
		-\frac{Fl}{EA}\frac{x}{l}
		\label{eqn:ux}
	\end{equation}
	と求められる.
\item
	軸力を$N$は,変位を微分して断面剛性$EA$を乗じ,
	\begin{equation}
		N(x)=EAu'=-\frac{p_0l}{2}
			\left\{
			\left(\frac{x}{l}\right)^2 -1
			\right\}
			-F
		\label{eqn:Nx}
	\end{equation}
	と得られる.
\item
	式(\ref{eqn:Nx})より,$N(0)=\frac{p_0l}{2}-F$(左向き)である.
\item
	$u(l)-u(0)=\frac{p_0l^2}{3EA}-\frac{Fl}{EA}$.
\item
	$\varepsilon=u'=\frac{N}{EA}=0$より$x_C=l\sqrt{1-\frac{2F}{p_0l}}$.
\item
	$x_C=\frac{l}{2}$より,$F=\frac{3p_0l}{8}$. このとき,
	\[
		N(0)=\frac{p_0l}{8}, \ \ N(l)=-F=-\frac{3}{8}p_0l
	\]
	だから,軸力分布は図\ref{fig:fig1}-(a)のようになる.
\item
	\[
		u\left(\frac{l}{2}\right)=\frac{p_0l^2}{24EI}, \ \ 
		u\left(l \right)=-\frac{p_0l^2}{24EI}
	\]
	で,$u'(\frac{l}{2})$だから,変位のグラフは図\ref{fig:fig1}(b)のようになる.
\end{enumerate}
\begin{figure}
	\begin{center}
	\includegraphics[width=0.6\linewidth]{fig1ans.eps} 
	\end{center}
	\caption{軸力および変位分布.} 
	\label{fig:fig1}
\end{figure}
\newpage
%%%%%%%%%%%%%%%%%%%%%%%%%%%%%%%%%%%%%%%%%%%%%%%%%%%%%%%%%%%%%%%%%%%%%%%%%%%%%%%%%%%%%%%%%%
\subsubsection*{問題2.}
\begin{enumerate}
\item
	$\bar{\sigma}=0, \Delta \sigma=2\sqrt{3}p, \tau = p$より,モールの応力円の中心は
	$(\sigma_{11}',\, \sigma_{12}')=(0,\, 0)$, 半径$R$は
	\begin{equation}
		R=\sqrt{\left(\frac{\Delta \sigma}{2}\right)^2+\tau ^2}=2p
		\label{neq:Radi}
	\end{equation}
	である.従って,モールの応力円は図\ref{fig:fig2}のようになる.
\item
	最大主応力を$\sigma_{max}$, $x_1$軸方向から反時計回りの方向に測った最大
	主応力方向を$\theta_{max}$とする.同様に,最小主応力を$\sigma_{min}$, 
	その方向を$\theta_{min}$とする.
	与えられた応力テンソル$\fat{\sigma}$は,モールの応力円において図\ref{fig:fig2}の
	Aの点で表される.一方,最大および最小主応力は,それぞれ,モールの応力円上の
	点BとCに相当することから,
	\begin{equation}
		\sigma_{max}=2p, \ \ \theta_{max}=\frac{\pi}{12}
	\end{equation}
	\begin{equation}
		\sigma_{min}=-2p, \ \ \theta_{min}=\frac{7\pi}{12}
	\end{equation}
	となる.
\item
	\begin{equation}
		\fat{t}^{(n)}
		=\fat{\sigma}^T\fat{n}
		=
		p
		\left( 
		\begin{array}{c}
			\sqrt{3} \cos\alpha +r\sin\alpha \\
			\cos\alpha - \sqrt{3} \sin\alpha 
		\end{array}
		\right)
		=
		2p
		\left( 
		\begin{array}{c}
			\cos\left(\frac{\pi}{6}-\alpha\right) \\
			\sin\left(\frac{\pi}{6}-\alpha\right) 
		\end{array}
		\right)
		\label{eqn:tn}
	\end{equation}
\item
	式(\ref{eqn:tn})より$\left| \fat{t}^{(n)}\right|^2=4p^2$で$\alpha$に依存しない.
\item
	\begin{eqnarray}
		\sigma_{11}'
		&=&\bar \sigma +R \cos \left( \phi- 2\theta \right)
		= 2p \cos \left( \frac{\pi}{6}-2\theta\right) 
		\label{eqn:s11d}
		\\
		\sigma_{12}'
		&=& R \sin \left( \phi- 2\theta \right)
		= 2p \sin \left( \frac{\pi}{6}-2\theta\right) 
		\label{eqn:s12d}
		\\
		\sigma_{22}'
		&=&\bar \sigma -R \cos \left( \phi- 2\theta \right)
		= -2p \cos \left( \frac{\pi}{6}-2\theta\right) 
		\label{eqn:s22d}
	\end{eqnarray}
\item
	$o-x_1'x_2'$座標系における直ひずみと直応力の関係
	\begin{equation}
		\left(
		\begin{array}{c}
			\varepsilon_{11}' \\
			\varepsilon_{22}' \\
			\varepsilon_{33}'
		\end{array}
		\right)
		=
		\frac{1}{E}
		\left(
		\begin{array}{ccc}
			1 & -\nu & -\nu \\
			-\nu & 1 & -\nu \\
			-\nu & -\nu & 1 
		\end{array}
		\right)
		\left(
		\begin{array}{c}
			\sigma_{11}' \\
			\sigma_{22}' \\
			\sigma_{33}'
		\end{array}
		\right)
		\label{eqn:Hooke_d}
	\end{equation}
	に式(\ref{eqn:s11d})と(\ref{eqn:s22d})を代入し,$\sigma_{33}'=0$とすれば,
	\begin{eqnarray}
		\varepsilon_{11}'&=&\frac{2(1+\nu)}{E}p\cos\left(\frac{\pi}{6}-2\theta\right) 
		\label{eqn:e11d}
		\\ 
		\varepsilon_{22}'&=&-\frac{2(1+\nu)}{E}p\cos\left(\frac{\pi}{6}-2\theta\right) 
		\label{eqn:e22d}
	\end{eqnarray}
	が得られる.
	また,せん断ひずみは
	\begin{equation}
		\varepsilon_{12}'=\frac{\sigma_{12}'}{2G}=\frac{p}{G}\sin\left(\frac{\pi}{6}-2\theta \right)
		\label{eqn:e12d}
	\end{equation}
	と求められる.
\item
	式(\ref{eqn:e11d})-(\ref{eqn:e12d})に$\theta=\theta_{max}=\frac{\pi}{12}$を代入すれば
	\[
		\varepsilon_{11}'=\frac{2(1+\nu)}{E} p,
		\ \ 
		\varepsilon_{22}'=-\frac{2(1+\nu)}{E} p
		\ \ 
		\varepsilon_{12}'=0
	\]
\item
	式(\ref{eqn:e11d})と(\ref{eqn:e12d})より
	\[
		(\varepsilon_{11}')^2 + (\varepsilon_{22}')^2  = \frac{p^2}{G^2}
	\]
	より,中心$(0,0)$, 半径$\frac{p}{G}$の円となる.
\end{enumerate}
%--------------------
\begin{figure}
	\begin{center}
	\includegraphics[width=0.46\linewidth]{fig2ans.eps} 
	\end{center}
	\caption{モールの応力円.} 
	\label{fig:fig2}
\end{figure}
\end{document}
